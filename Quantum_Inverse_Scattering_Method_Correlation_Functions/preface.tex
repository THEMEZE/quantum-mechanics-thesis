Ce livre est consacré aux solutions exactes des modèles de la théorie quantique des champs (dans une dimension spatiale et une dimension temporelle). Nous étudions également des modèles bidimensionnels de physique statistique classique, qui sont naturellement liés à ces problèmes. Les descriptions complètes des modèles solubles sont données par le Bethe Ansatz, découvert par H. Bethe en 1931 \cite{??}, lors de l'étude de l'antiferromagnétisme de Heisenberg. Le Bethe Ansatz a été très utile pour la résolution de divers problèmes \cite{??, ??, ??, ??}, \cite{??}, \cite{??}, et \cite{??}.

Certains modèles solubles par Bethe Ansatz ont des applications physiques directes. Un problème célèbre résolu par le Bethe Ansatz est le problème de Kondo (voir \cite{??} et \cite{??}). Un autre modèle est le modèle de Hubbard \cite{??}, qui est lié à la supraconductivité à haute température. Une application importante du Bethe Ansatz se trouve en optique non linéaire, où l'émission spontanée coopérative de radiation peut être décrite par un modèle quantique exactement soluble \cite{??}. Le Bethe Ansatz est très utile en physique théorique moderne \cite{??, ??}. Les fonctions de corrélation nous fournissent des informations dynamiques sur le modèle. Elles sont décrites en détail dans ce livre.

Les modèles solubles par Bethe Ansatz ne sont pas libres ; ils généralisent les modèles libres de la théorie quantique des champs dans le sens suivant. La dynamique à plusieurs corps des modèles libres peut être réduite à une dynamique à un corps. Avec le Bethe Ansatz, la dynamique à plusieurs corps peut être réduite à une dynamique à deux corps. La matrice de diffusion pour plusieurs particules est égale au produit de celles à deux particules. Cela conduit à la relation d'auto-consistance pour la matrice de diffusion à deux particules, connue sous le nom de l'équation de Yang-Baxter (un aperçu des articles peut être trouvé dans \cite{??}), qui est le concept central des modèles exactement solubles. Le rôle de l'équation de Yang-Baxter va au-delà de la théorie des systèmes dynamiques. Elle est très importante dans la théorie des nœuds \cite{??} et des groupes quantiques \cite{??}.

Les modèles quantiques exactement solubles sont étroitement liés à la théorie des équations différentielles complètement intégrables. La relation la plus simple est fournie par la limite quasiclassique. Les fonctions de corrélation quantiques sont décrites par des équations différentielles classiques. La méthode moderne pour résoudre ces équations, la méthode de diffusion inverse, a été fondée en 1967 par Gardner, Greene, Kruskal et Miura \cite{??}. Ils ont étudié l'équation de Korteweg-de Vries. P. Lax a montré que cette équation peut être représentée comme une condition de commutativité pour deux opérateurs différentiels linéaires \cite{??}. Il est intéressant de noter que la représentation de Lax est liée algébriquement à l'équation de Yang-Baxter \cite{??}.

La méthode de diffusion inverse a permis de résoudre une large classe d'équations différentielles non linéaires. Le réseau de Toda en est un exemple (voir \cite{??}, \cite{??} et \cite{??}). Ces équations ont des applications dans divers domaines de la physique : la physique des plasmas, l'optique non linéaire, les vagues océaniques non linéaires, et d'autres. Il existe un certain nombre de livres très intéressants sur la méthode de diffusion inverse \cite{??, ??, ??, ??, ??, ??, ??, ??}. On doit également mentionner qu'il existe des équations différentielles complètement intégrables multidimensionnelles. La plus célèbre est l'équation de Yang-Mills autoduale \cite{??}. D'autres exemples sont les équations de Davey-Stewartson \cite{??} et de Kadomtsev-Petviashvili \cite{??}. Les équations différentielles ordinaires complètement intégrables sont également extrêmement importantes ; par exemple, les célèbres équations transcendantes de Painlevé (voir \cite{??} et les références qui y sont citées). Elles apparaissent également en gravitation bidimensionnelle et dans les modèles matriciels (voir \cite{??}, \cite{??}, et \cite{??}) ainsi que dans la description des fonctions de corrélation quantiques \cite{??, ??, ?? et ??}.

Le développement ultérieur de la méthode de diffusion inverse est lié à \cite{??}, où l'interprétation hamiltonienne a été comprise ; L.D. Faddeev et V.E. Zakharov ont montré que la solution d'un modèle par la méthode de diffusion inverse peut être considérée comme une transformation en variables action-angle. Cela offre une opportunité pour la quantification quasiclassique. La théorie quantique des solitons a été construite dans \cite{??, ??, ??, et ??}, où il a été montré qu'après quantification, les solitons apparaissent comme des particules élémentaires dans le spectre du Hamiltonien.

La méthode quantique de diffusion inverse a été découverte dans \cite{??}. Elle offre une vue unifiée de la solution exacte des modèles classiques et quantiques. Elle combine les idées du Bethe Ansatz et de la méthode de diffusion inverse. Le premier modèle à être résolu par la méthode quantique de diffusion inverse était l'équation de Schrödinger non linéaire.


$$i\partial_t \Psi = -\partial_x^2 \Psi + 2c \Psi^\dag \Psi \Psi .$$

La représentation de Lax pour cette équation a été construite dans la référence \cite{??}. Le Bethe Ansatz pour la version quantique de cette équation a été construit dans \cite{??} et \cite{??}. La méthode quantique de diffusion inverse a permis de reproduire les résultats du Bethe Ansatz en partant de la représentation de Lax. Un développement important de la méthode quantique de diffusion inverse est lié à l'étude des équations différentielles pour les fonctions de corrélation quantiques. Il a été montré dans \cite{??} que les équations différentielles pour les fonctions de corrélation quantiques sont simplement reliées à l'équation différentielle originale qui a été quantifiée. Le langage correct pour la description des fonctions de corrélation est celui des fonctions $\tau$. Cela est décrit dans notre livre. Nous devons mentionner les articles \cite{??} et \cite{??} où les équations différentielles ont été obtenues pour la première fois pour les fonctions de corrélation du modèle d'Ising.

La méthode quantique de diffusion inverse est une méthode bien développée (voir les revues \cite{??}, \cite{??}, \cite{??}, \cite{??}, et \cite{??}). Elle a permis de résoudre une large classe d'équations d'évolution non linéaires. Elle explique la nature algébrique du Bethe Ansatz. Notre livre explique cette méthode en détail. Un exemple important résolu par la méthode quantique de diffusion inverse est l'équation de sine-Gordon \cite{??}.


$$\partial_t^2 u - \partial_x^2 u + \frac{m^2}{\beta} \sin \beta u = 0 .$$

En relation avec ce modèle, nous souhaitons mentionner le nouveau livre de Smirnov \cite{Smirnov53}. L'Ansatz de Bethe algébrique est lié aux groupes quantiques \cite{QuantumGroups17}. Il est également profondément lié à la théorie des matrices S factorisées de Zamolodchikov \cite{Zamolodchikov62} et à la théorie des modèles de réseaux exactement solvables en physique statistique classique (la meilleure revue de ces modèles est le livre de Baxter \cite{Baxter5}) ainsi qu'à la théorie conforme des champs \cite{CFT6,CFT30}.

Dans notre livre, nous essayons d'illustrer les énoncés généraux à travers quelques modèles simples. Notre principal exemple est l'équation de Schrödinger non linéaire (dans le cas quantique, il s'agit du modèle d'un gaz de Bose unidimensionnel avec répulsion $\delta$). Nous considérons également le modèle de sine-Gordon, l'antiferromagnétisme de Heisenberg, et le modèle de Hubbard.

Notre livre est divisé en quatre parties, chacune ayant une introduction. La première partie explique l'Ansatz de Bethe en coordonnées. Nous évaluons l'énergie et l'impulsion des excitations ainsi que la matrice de diffusion dans la limite thermodynamique. Normalement, l'état fondamental du modèle est une sphère de Fermi (ou mer de Dirac). La thermodynamique du modèle est construite explicitement.

La deuxième partie explore la méthode de diffusion inverse quantique et l'Ansatz de Bethe algébrique. La classification des modèles exactement solvables y est donnée. Le concept important du déterminant quantique y est introduit (il est lié à l'antipode dans les groupes quantiques). La fonction de partition du modèle à six sommets est représentée comme un déterminant pour le réseau fini avec des conditions aux limites de paroi. Le principe de Pauli pour les bosons en interaction unidimensionnelle est également discuté. Des versions discrètes de modèles continus sont construites de manière à préserver les caractéristiques dynamiques les plus importantes.

Les troisième et quatrième parties décrivent la théorie des fonctions de corrélation. Dans la troisième partie, les fonctions de corrélation quantiques sont représentées comme des déterminants d'opérateurs intégraux (d'une forme très spéciale). La troisième partie commence par une étude algébrique des produits scalaires. Par exemple, il est prouvé que le carré de la norme de la fonction d'onde de Bethe est égal au déterminant de matrices très simples. Cela peut être obtenu par linéarisation des conditions aux limites périodiques (en forme logarithmique) près de la solution. Les fonctions de corrélation sont représentées comme des déterminants d'opérateurs intégraux spéciaux (de type Fredholm).

Dans la quatrième partie, des équations différentielles pour les fonctions de corrélation sont dérivées. Tout d'abord, nous représentons les fonctions de corrélation comme des déterminants de Fredholm d'un certain opérateur intégral. Cet opérateur intégral a une structure très spéciale qui nous permet de le considérer comme un opérateur de Gel'fand-Levitan-Marchenko d'une autre équation différentielle. Cette dernière équation gouverne les fonctions de corrélation. Les asymptotiques des fonctions de corrélation sont évaluées explicitement, même les fonctions de corrélation dépendant du temps et de la température.

Une autre approche des fonctions de corrélation est également discutée. Elle est liée à la théorie conforme des champs quantiques et permet d'évaluer les asymptotiques à longue distance des fonctions de corrélation à température nulle. Les modèles considérés ici sont sans gap, et les fonctions de corrélation décroissent en fonction de la distance ; la température nulle est le point de transition de phase. Ces puissances sont appelées exposants critiques. Elles dépendent de tous les paramètres du modèle et nous les évaluons explicitement. La méthode d'évaluation est liée aux corrections de taille finie. La charge centrale de l'algèbre de Virasoro décrivant les asymptotiques de nos modèles est généralement égale à un.

Les parties du livre sont divisées en chapitres. Les objectifs de chaque chapitre sont expliqués dans une introduction, et à la fin de chaque chapitre, il y a une conclusion qui donne un résumé et contient des commentaires bibliographiques. Les références sont listées par chapitre à la fin du livre.

Il y a une double numérotation des formules dans le livre. Le premier nombre correspond au numéro de la section et le second est le numéro de la formule dans la section. Lorsqu'on fait référence à une formule dans différents chapitres, on précède le numéro de l'équation par le numéro du chapitre. Si nous faisons référence à une section d'un autre chapitre, le numéro de la section sera précédé du numéro du chapitre. Les théorèmes sont numérotés séparément dans chaque chapitre. Les sections marquées d'un astérisque peuvent être omises lors d'une première lecture.

Ce livre a été commencé à Saint-Pétersbourg, où nous avons bénéficié de discussions avec L.D. Faddeev, A.R. Its, N.A. Slavnov, E.K. Sklyanin, N.Yu. Reshetikhin, L.A. Takhtajan, V.E. Zakharov, P.P. Kulish, V.N. Popov, F.A. Smirnov et A.N. Kirillov. Le livre a été terminé à Stony Brook. Nous apprécions grandement l'atmosphère créative de l'Institut de Physique Théorique à Stony Brook. Nous avons bénéficié de discussions avec C.N. Yang, B. McCoy, B. Sutherland, M. Fowler, H. Thacker, H. Flaschka, A. Fokas, A. Newell, M. Ablowitz, J. Palmer, C. Tracy, L.L. Chau, R. Shrock, W. Weisberger, E. Melzer, F. Essler, D. Uglov, H. Frahm, K. Schoutens, F. Figueirido, E. Williams et S. Ray. Les auteurs tiennent à remercier David A. Coker pour la relecture et ses suggestions pédagogiques.

Nous remercions la NSF pour la subvention PHY-9107261.

