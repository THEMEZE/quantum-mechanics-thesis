La méthode de diffusion inverse quantique est un moyen de trouver des solutions exactes pour des modèles bidimensionnels en théorie quantique des champs et en physique statistique, tels que l'équation de sine-Gordon ou l'équation de Schrödinger non linéaire quantique. Ces modèles suscitent beaucoup d'attention parmi les physiciens et les mathématiciens.

Le présent texte est une introduction à ce domaine important et passionnant. Il se compose de quatre parties. La première traite de l'Ansatz de Bethe et du calcul de quantités physiques. Les auteurs donnent ensuite un exposé détaillé mais pédagogique de la méthode de diffusion inverse quantique avant de l'appliquer dans la deuxième moitié du livre au calcul des fonctions de corrélation. Il s'agit de l'une des applications les plus importantes de la méthode, et les auteurs ont apporté des contributions significatives à ce domaine. Ici, ils décrivent certaines des approches les plus récentes et générales et incluent quelques nouveaux résultats.

Ce livre sera une lecture essentielle pour tous les physiciens mathématiciens, au niveau de la recherche ou des études supérieures, travaillant en théorie des champs et en physique statistique.
