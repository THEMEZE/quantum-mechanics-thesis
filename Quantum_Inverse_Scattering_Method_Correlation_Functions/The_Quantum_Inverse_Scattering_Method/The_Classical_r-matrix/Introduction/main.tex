La méthode moderne de résolution des équations aux dérivées partielles est appelée la méthode inverse de diffusion classique (CISM). On peut la considérer comme une généralisation non linéaire de la transformation de Fourier. De nos jours, la méthode inverse de diffusion classique est une branche bien développée de la physique mathématique (voir la préface, références (1), (19), (10), (11), (18), (21)-(24), (29), (37), (45)). 

Dans ce chapitre, nous donnerons uniquement les informations nécessaires à la quantification qui sera effectuée dans le chapitre suivant. Les concepts de la représentation de Lax, de la matrice de transition et des identités de trace sont énoncés dans la section 1. Les équations aux dérivées partielles complètement intégrables apparaîtront à nouveau dans ce livre. Dans les chapitres XIV et XV, nous les dériverons pour les fonctions de corrélation quantiques. Dans ces chapitres, nous étudierons les équations différentielles complètement intégrables sous un angle différent. Nous appliquerons le problème de Riemann-Hilbert afin d'évaluer les asymptotiques. 

La matrice \( r \), qui permet de calculer les crochets de Poisson entre les éléments de matrice de la matrice de transition et également de construire les variables d'action-angle, est introduite dans la section 2. Comme expliqué là, l'existence de la matrice \( r \) garantit l'existence de la représentation de Lax. La matrice \( r \) satisfait une certaine relation bilinéaire (la relation de Yang-Baxter classique). L'existence de la matrice \( r \) garantit également l'existence d'un nombre infini de lois de conservation qui restreignent de manière essentielle la dynamique du système. Dans le chapitre suivant, la notion de la matrice \( r \) sera généralisée au cas quantique. Dans les deux premières sections de ce chapitre, des énoncés généraux sont démontrés par l'exemple de l'équation de Schrödinger non linéaire, qui est le modèle dynamique le plus simple (il convient de mentionner que dans le cas classique, ce nom est plus naturel que celui de gaz de Bose unidimensionnel). D'autres modèles (l'équation de sine-Gordon, le modèle de Mikhailov-Shabat-Zhiber) sont considérés dans la section 3. La notation tensorielle, dont l'application simplifie considérablement les calculs tant dans les cas classique que quantique, est discutée dans l'appendice de ce chapitre. 

Si le lecteur trouve ce chapitre trop bref, nous recommandons l'excellent livre de L.D. Faddeev et L.A. Takhtajan (voir [18] dans les références de la préface), qui décrit la CISM en détail explicite.
