

La méthode quantique de diffusion inverse relie l'Ansatz de Bethe à la théorie des équations différentielles classiques complètement intégrables. Ces équations sont parfois appelées équations de soliton. La méthode moderne pour les résoudre est appelée méthode classique de diffusion inverse. En un sens, cela constitue une généralisation non linéaire de la transformation de Fourier.

Dans cette partie, la méthode quantique de diffusion inverse est exposée. Les principales déclarations de la méthode classique de diffusion inverse nécessaires à la quantification sont fournies au Chapitre V, où la représentation de Lax est introduite. La structure hamiltonienne des modèles intégrables est également discutée, ainsi que le nombre infini d'intégrales de mouvement. La méthode la plus pratique pour analyser la structure hamiltonienne repose sur la \emph{r-matrice} classique. Certains modèles concrets seront considérés. Le Chapitre VI est consacré en particulier à la méthode quantique de diffusion inverse. La R-matrice, qui est l'objet principal de cette méthode, est introduite. L'équation de Yang-Baxter pour la R-matrice est discutée. Les principales déclarations de la méthode sont fournies et plusieurs exemples sont présentés. La formulation algébrique de l'Ansatz de Bethe, l'une des principales réalisations de la méthode quantique de diffusion inverse, est présentée au Chapitre VII. La notion de déterminant de la matrice de transition dans le cas quantique est introduite dans ce chapitre. (Ceci est étroitement lié au concept d'antipode dans les groupes quantiques.)

Les modèles intégrables de la théorie quantique des champs sur le réseau sont présentés au Chapitre VIII. La méthode quantique de diffusion inverse fournit un mécanisme pour transférer des modèles continus de la théorie quantique des champs au réseau tout en préservant la R-matrice. Pour les modèles classiques, cela signifie que la structure des variables d'action-angle reste la même. Dans le cas quantique, cela conduit à la conservation de la matrice de diffusion et des exposants critiques qui déterminent l'asymptotique à longue distance des fonctions de corrélation. Pour les modèles relativistes de la théorie des champs (comme le modèle de sine-Gordon), la variante sur le réseau fournit une solution rigoureuse au problème des divergences ultraviolettes. Nous pouvons étudier des modèles continus comme des modèles de réseau condensés (c'est-à-dire avec l'espacement du réseau \(A \rightarrow 0\)). Notre construction garantit qu'il n'y aura pas de transition de phase. La forme explicite de l'opérateur L permet la classification de tous les modèles intégrables ayant une R-matrice donnée.

La relation étroite de la méthode quantique de diffusion inverse avec d'autres méthodes de la physique mathématique contemporaine mérite d'être mentionnée. Tout d'abord, elle est liée aux groupes quantiques et à la théorie des nœuds. Elle est également associée à la méthode utilisée en physique statistique classique pour résoudre des modèles de réseau bidimensionnels. Nous abrégerons le nom « méthode quantique de diffusion inverse » par QISM. Nous devons noter que les caractéristiques physiques des particules - l'énergie habillée, le moment et la matrice S - sont toutes calculées dans le cadre de la QISM, exactement comme pour l'Ansatz de Bethe en coordonnées.

Il est intéressant de mentionner qu'ultérieurement dans le livre, nous obtiendrons des équations différentielles complètement intégrables pour les fonctions de corrélation quantiques. Nous les étudierons sous un angle différent. Le problème le plus important dans ce cas est de déterminer l'asymptotique à longue distance des fonctions de corrélation. Nous ferons cela par le biais du problème de Riemann-Hilbert.
