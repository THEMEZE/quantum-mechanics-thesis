Le gaz de Bose unidimensionnel est décrit par les champs de Bose quantiques canoniques \( \Psi(x, t) \) avec les relations de commutation canoniques à temps égal :


\begin{eqnarray}
	\left . \begin{array}{rcl}
		[ \operator{\Psi}(x, t),  \operator{\Psi}^\dagger(y, t) ]  &=&  \operator{\delta}(x - y) \\
		\left [ \operator{\Psi}(x, t),  \operator{\Psi}(y, t) \right ]   =  [ \operator{\Psi}^\dag(x, t),  \operator{\Psi}^\dag(y, t) ]  &=&  0 
	\end{array} \right . \label{eq.1.1}
\end{eqnarray}

Par la suite, l'argument \( t \) sera en général omis, car toutes les considérations de ce chapitre s'appliquent à un moment fixe dans le temps.

Le Hamiltonien du modèle est donné par

\begin{eqnarray}
	\operator{H} & = & \int dx \left [ \operator{\partial}_x \operator{\Psi}^\dag (x) \operator{\partial}_x \operator{\Psi}(x) + c \operator{\Psi}^\dag (x) \operator{\Psi}^\dag (x) \operator{\Psi} (x) \operator{\Psi} (x) \right ] 
\end{eqnarray}


où \( c \) est la constante de couplage. L'équation du mouvement correspondante
\begin{eqnarray}
	i \operator{\partial}_t \operator{\Psi}	 & = & - \operator{\partial}_x^2 \operator{\Psi} + 2 c \operator{\Psi}^\dag\operator{\Psi} \operator{\Psi}
\end{eqnarray}

est appelée l'équation de Schrödinger non linéaire (NS).

Pour $c > 0$, l'état fondamental à température nulle est une sphère de Fermi. Seul ce cas sera considéré par la suite. Le vide de Fock $\vert 0 \rangle$, défini par 


\begin{eqnarray}
	\forall x \in \mathbb{R} & \colon & \operator{\Psi} (x) \vert 0 \rangle = 	0 \label{eq.1.4}
\end{eqnarray}

est d'une grande importance.

It will be called the pseudovacum and it is to be distinguihed from the physical vacuum which is the ground state of the Hamiltonian (the Dirac sea). The dual pseudovacuum $\langle 0 \vert$ is defined as $\langle 0 \vert =  \vert 0 \rangle^\dag$ and satisfies the relations ... where the dagger denotes Hermitian conjugation. The number of particles aperator $\operator{Q}$ and momentum operator $\operator{P}$ are 

Il sera appelé le pseudovacuum et doit être distingué du vide physique, qui est l'état fondamental de l'Hamiltonien (la mer de Dirac). Le pseudovacuum dual $\langle 0 \vert$ est défini comme $\langle 0 \vert = \vert 0 \rangle^\dag$ et satisfait les relations 

\begin{eqnarray}
	\forall x \in \mathbb{R} & \colon & \langle 0 \vert \operator{\Psi}^\dag (x)  = 	0, ~\langle 0 \vert 0 \rangle = 1 
\end{eqnarray}

où la dague ($^\dag$) désigne la conjugaison hermitienne. 

L'opérateur du nombre de particules $\operator{Q}$ et l'opérateur de moment $\operator{P}$ sont définis comme 
\begin{eqnarray}
	\operator{Q} & = & \int \operator{\Psi}^\dag (x) \operator{\Psi} (x) \, d x \\
	\operator{P} & = & - \frac{i}2 \int \left \{  \operator{\Psi}^\dag(x) \operator{\partial}_x \operator{\Psi}(x) - \left [ \operator{\partial}_x \operator{\Psi}^\dag(x)\right ] \operator{\Psi}(x)\right \} dx \label{eq.1.7}
\end{eqnarray}

Ce sont des opérateurs hermitiens et ils constituent des intégrales du mouvement

\begin{eqnarray}
	[ \operator{H} , \operator{Q} ] = 	[ \operator{H} , \operator{P} ] = O. 
\end{eqnarray}

Il convient de noter que dans la Partie II, nous allons construire un nombre infini d'intégrales du mouvement. Nous pouvons maintenant chercher les fonctions propres communes $\vert \psi_N\rangle$ des opérateurs $\operator{H}$, $\operator{Q}$, et $\operator{P}$ :

\begin{eqnarray}
	\vert \psi ( \lambda_1 , \cdots , \lambda_N ) \rangle & = & \frac{1}{\sqrt{N!}} \int d^N z \, \chi_N ( z_1 , \cdots , z_N  ~\vert ~ \lambda_1 , \cdots , \lambda _N ) \operator{\Psi}^\dag (z_1 ) \cdots \operator{\Psi}^\dag (z_N )	 \vert 0 \rangle. \label{eq.1.9}
\end{eqnarray}

Ici, $\chi_N$ est une fonction symétrique de toutes les variables $z_j$. L'équation aux valeurs propres 

\begin{eqnarray}
	\operator{H} |\psi_N\rangle = E_N |\psi_N\rangle, \quad \operator{P} |\psi_N\rangle = p_N |\psi_N\rangle, \quad \operator{Q} |\psi_n\rangle = q_N |\psi_N\rangle,	
\end{eqnarray}

conduit au fait que $\chi_N$ est une fonction propre à la fois de l'Hamiltonien mécanique quantique $\operator{\mathcal{H}}_N$ et de l'opérateur du moment mécanique quantique $\operator{\mathcal{P}}_N$ :

\begin{eqnarray}
	&& \left \{ \begin{array}{rcl} \operator{ \mathcal{H}}_N & = & \displaystyle \sum_{j=1}^N   - \operator{\partial}_{z_j}^2 + 2c \sum_{1 \leq k < j \leq N } \operator{\delta} ( z_j - z_k) \\ \operator{\mathcal{P}}_N & = & \displaystyle \sum_{j=1}^N -i \operator{\partial}_{z_j}\end{array} \right .  \label{eq.1.11}\\
	&& ~~~\operator{ \mathcal{H}}_N \chi_N ~= ~	E_N \chi_N .\label{eq.1.12}
\end{eqnarray}

Cela peut être expliqué en utilisant l'opérateur du moment $\operator{P}$ (\ref{eq.1.7}) comme exemple. Tout d'abord, nous intégrons (\ref{eq.1.7}) par parties pour représenter $\operator{P}$ sous la forme :

\begin{eqnarray*}
	\operator{P} & = & i \int \left [ \operator{\partial}_x \operator{\Psi}^\dag(x)\right ] \operator{\Psi}(x) dx 
\end{eqnarray*}

Agir avec cet opérateur sur la fonction propre (\ref{eq.1.9}) donne 
\begin{eqnarray*}
	\operator{P}\vert \psi ( \lambda_1 , \cdots , \lambda_N ) \rangle & = & \frac{i}{ \sqrt{N!}} \int dx \int d^N z \, \chi_N ( z_1 , \cdots , z_N  ~\vert ~ \lambda_1 , \cdots , \lambda _N ) 	\left [\operator{\partial}_x \operator{\Psi}^\dag(x)\right ]  \\ & & \times \sum_{k = 1 }^N \operator{\Psi}^\dag(z_1) \cdots [\operator{\Psi}(x) , \operator{\Psi}^\dag(z_k)] \cdots \operator{\Psi}^\dag(z_N) \vert 0 \rangle
\end{eqnarray*}

où l'équation (\ref{eq.1.4}) a été utilisée. La formule (\ref{eq.1.1}) donne une fonction delta pour le commutateur, qui peut ensuite être intégrée pour donner

\begin{eqnarray*}
	\operator{P}\vert \psi ( \lambda_1 , \cdots , \lambda_N ) \rangle & = & \frac{i}{ \sqrt{N!}}  \int d^N z 	\, \chi_N ( z_1 , \cdots , z_N  ~\vert ~ \lambda_1 , \cdots , \lambda _N ) \\ & & \times  \sum_{k = 1 }^N  \operator{\Psi}^\dag(z_1) \cdots \left [\operator{\partial}_{z_k} \operator{\Psi}^\dag(z_k)\right ] \cdots \operator{\Psi}^\dag(z_N) \vert 0 \rangle. 
\end{eqnarray*}

Nous intégrons maintenant par parties par rapport à $z_k$ pour obtenir

\begin{eqnarray*}
	\operator{P}\vert \psi ( \lambda_1 , \cdots , \lambda_N ) \rangle & = & \frac{1}{ \sqrt{N!}}  \int d^N z 	 \left \{  -i \sum_{k = 1 }^N  \operator{\partial}_{z_k}\chi_N ( z_1 , \cdots , z_N  ~\vert ~ \lambda_1 , \cdots , \lambda _N ) \right \} \\ && \times \operator{\Psi}^\dag(z_1) \cdots \operator{\Psi}^\dag(z_N) \vert 0 \rangle. 
\end{eqnarray*}

Ainsi, nous avons prouvé que l'action de l'équation (\ref{eq.1.7}) sur l'équation (\ref{eq.1.9}) est équivalente à l'action de $\operator{\mathcal{P}}_N$ sur \(\chi_N\). La construction de l'Hamiltonien mécanique quantique est assez similaire.

Le problème de la théorie quantique des champs est donc réduit à un problème de mécanique quantique. L'Hamiltonien $\operator{\mathcal{H}}_N$ décrivant \(N\) particules bosoniques est répulsif pour \(c > 0\). En raison de la symétrie de \(\chi\) par rapport à toutes les \(z_i\), il est suffisant de considérer le domaine suivant \(T\) dans l'espace des coordonnées :

\begin{eqnarray}
	T & : & z_1 < z_2 < \cdots < z_N \label{eq.1.13}. 
\end{eqnarray}

Dans ce domaine, la fonction \(\chi_N\) est une fonction propre de l'Hamiltonien libre.

\begin{eqnarray}
	\left \{ \begin{array}{rcl} \operator{\mathcal{H}}_N^0 & = & \displaystyle - \sum_{j=1}^N \operator{\partial}_{z_j}^2 \\  \operator{\mathcal{H}}_N^0 \chi_N  & = & E_N \chi_N  \end{array} \right .	\label{eq.1.14}
\end{eqnarray}

Les conditions aux limites suivantes doivent être satisfaites :

\begin{eqnarray}
	( \operator{\partial}_{z_{j+1}}	- \operator{\partial}_{z_{j}} - c ) \chi_N &= & 0 , \quad z_{j+1} = z_j + 0 .\label{eq.1.15}
\end{eqnarray}

{\color{blue}

\begin{eqnarray*}
	\left ( \begin{array}{c} Y \\ Z \end{array} \right ) = \underbrace{\left ( \begin{array}{cc} -1 & 1  \\ 1/2  &  1/2 \end{array} \right )}_{P_{x \to X} }  \left ( \begin{array}{c} z_j \\ z_{j+1} \end{array} \right ), & & \left ( \begin{array}{c} z_j \\ z_{j+1} \end{array} \right ) = \underbrace{\left ( \begin{array}{cc} -1/2 & 1  \\ 1/2  &  1 \end{array} \right )}_{P_{X \to x} = P_{x \to X}^{-1} }  \left ( \begin{array}{c} Y \\ Z \end{array} \right ),\\
	\operator{\partial}_{\left ( \begin{array}{c} Y \\ Z \end{array} \right )} = \underbrace{\left ( \begin{array}{cc} -1/2 & 1/2  \\ 1  &  1 \end{array} \right )}_{P_{\operator{\partial}x \to \operator{\partial}X} =  {}^t(P_{x \to X}^{-1}) } \operator{\partial}_{  \left ( \begin{array}{c} z_j \\ z_{j+1} \end{array} \right )}, & & \operator{\partial}_{\left ( \begin{array}{c} z_j \\ z_{j+1} \end{array} \right )} = \underbrace{\left ( \begin{array}{cc} -1 & 1/2  \\ 1  &  1/2 \end{array} \right )}_{P_{\operator{\partial}X \to \operator{\partial}x} = P_{\operator{\partial}x \to \operator{\partial}X}^{-1} }  \operator{\partial}_{\left ( \begin{array}{c} Y \\ Z \end{array} \right )},
\end{eqnarray*}


}

L'équation (\ref{eq.1.14}) et la condition aux limites (\ref{eq.1.15}) sont équivalentes à l'équation (\ref{eq.1.12}). En effet, le potentiel dans (\ref{eq.1.11}) est égal à zéro dans le domaine \(T\). En intégrant l'équation (1.12) sur la variable \((z_{j+1} - z_j)\) dans la petite région \(|z_{j+1} - z_j| < \epsilon\), en considérant tous les autres \(z_k\) (\(k \neq j,j+1\)) comme fixes dans \(T\), on obtient exactement la condition (\ref{eq.1.15}).

Une fonction satisfaisant (\ref{eq.1.14}) et (\ref{eq.1.15}) peut être construite comme suit. Considérons la fonction propre de l'Hamiltonien (\ref{eq.1.14}) dans le domaine \(T\) donnée comme le déterminant de la matrice \(N \times N\) \(\exp\{i\lambda_j z_k \}\)

\begin{eqnarray}
	\det [ \exp\{i\lambda_j z_k \}] 
\end{eqnarray}

avec des nombres arbitraires \(\lambda_j\). Cette fonction est égale à zéro sur la frontière du domaine \(T\) en raison de son antisymétrie par rapport à \(z_k\). Il est alors facile de voir que la fonction \(\chi_N\) donnée par

\begin{eqnarray}
	\chi_N & \propto & \left [  \prod_{ 1 \leq k < j \leq N } \left ( \partial_{z_j} - \partial_{z_k} +c \right ) \right ] \det [ \exp\{i\lambda_j z_k \}] 	\label{eq.1.17}
\end{eqnarray}

satisfait les équations (\ref{eq.1.14}) et (\ref{eq.1.15}). Pour vérifier, par exemple, l'égalité

\begin{eqnarray}
	\left( \partial_{z_2} - \partial_{z_1} +c \right )	\chi_N & =& 	 0 , \quad z_{2} = z_1 + 0  \label{eq.1.18}.
\end{eqnarray}

réécrivons $\chi_N$ comme

\begin{eqnarray}
	\chi_N & =& \left( \partial_{z_2} - \partial_{z_1} + c \right ) \tilde{\chi}_N
\end{eqnarray}

où

\begin{eqnarray}
	\tilde{\chi}_N & \propto & 	\prod_{j=3 }^N \left ( \partial_{z_j} - \partial_{z_1} +c \right )	\left ( \partial_{z_j} - \partial_{z_2} +c \right ) \times \left [  \prod_{ 3 \leq k < j \leq N } \left ( \partial_{z_j} - \partial_{z_k} +c \right ) \right ] \det [ \exp\{i\lambda_j z_k \}] 	.
\end{eqnarray}

Cette fonction est antisymétrique par rapport à $z_1 \leftrightarrow z_2$,

\begin{eqnarray}
	\tilde{\chi}_N(z_1 , z_2) & =& - 	\tilde{\chi}_N(z_2 , z_1)	
\end{eqnarray}

et elle est égale à zéro lorsque $z_1 = z_2$. En revenant à l'équation (\ref{eq.1.18}),

\begin{eqnarray}
	\left [ \left ( \partial_{z_2} -  \partial_{z_1} \right )^2 - c^2   \right ] \tilde{\chi}_N & = & 0, \quad z_2 = z_1  \label{eq.1.22}	
\end{eqnarray}

nous voyons que le membre de gauche change de signe lorsque $z_1 \leftrightarrow z_2$, et donc l'égalité (\ref{eq.1.22}) est correcte. Nous pouvons de la même manière vérifier les autres conditions aux limites. Ainsi, $\chi_n$ dans (\ref{eq.1.17}) est la fonction propre souhaitée de l'Hamiltonien $\operator{\mathcal{H}}_N$ (\ref{eq.1.11}). Le déterminant dans (\ref{eq.1.17}) peut être écrit comme une somme sur toutes les permutations $\mathcal{P}$ des nombres $1, 2, \cdots, N$ :

\begin{eqnarray}
	\det [ \exp\{i\lambda_j z_k \}] & = & 	\sum_{\mathcal{P}} (-1)^{[\mathcal{P}]} \exp \left \{ i \sum_{n = 1}^N  z_n \lambda_{ \mathcal{P}(n) } \right \} 
\end{eqnarray}

où $[\mathcal{P}]$ désigne la parité de la permutation. On obtient, dans le domaine $T$ (\ref{eq.1.13}) :

\begin{eqnarray}
	\chi_N & = &	\left \{ N! \prod_{k<j} \left [ ( \lambda_j - \lambda_k )^2 + c^2 \right ] \right \}^{-1} \notag \\
	&&  \times \sum_{\mathcal{P}} (-1)^{[\mathcal{P}]} \exp \left \{ i \sum_{n = 1}^N  z_n \lambda_{ \mathcal{P}(n) } \right \} \prod_{k<j} \left [  \lambda_{\mathcal{P}(j)}- \lambda_{\mathcal{P}(k)}  -i c \right ]
\end{eqnarray}

avec la constante spécifiée. Continuons maintenant $\chi_N$ par symétrie à l'ensemble de $\mathbb{R}^N$ :

\begin{eqnarray}
	\chi_N & = &	\left \{ N! \prod_{k<j} \left [ ( \lambda_j - \lambda_k )^2 + c^2 \right ] \right \}^{-1} \notag \\
	&&  \times \sum_{\mathcal{P}} (-1)^{[\mathcal{P}]} \exp \left \{ i \sum_{n = 1}^N  z_n \lambda_{ \mathcal{P}(n) } \right \} \prod_{k<j} \left [  \lambda_{\mathcal{P}(j)}- \lambda_{\mathcal{P}(k)}  -i c \epsilon (z_j -z_k ) \right ]
\end{eqnarray}

où $\epsilon(x)$ est la fonction signe. La méthode décrite ci-dessus a apparemment été suggérée par M. Gaudin [9].
Une autre manière utile d'écrire $\chi_N$ est la suivante :

\begin{eqnarray}
	\chi_N & = &	\frac{(-i){\frac{N(N-1)}2}}{ \sqrt{N!}}\left \{  \prod_{1\leq k<j\leq N} \epsilon (z_j -z_k ) \right \}\notag \\
	&&  \times \sum_{\mathcal{P}} (-1)^{[\mathcal{P}]} \exp \left \{ i \sum_{k = 1}^N  z_k \lambda_{ \mathcal{P}(k) } \right \} \notag \\
	&& \times \exp \left \{  \frac{i}{2} \sum_{1\leq k<j\leq N} \epsilon (z_j -z_k ) \theta(\lambda_{\mathcal{P}(j)}- \lambda_{\mathcal{P}(k)}  )  \right \} \label{eq.1.26} 
\end{eqnarray}
où
\begin{eqnarray*}
	\theta ( x ) & = & i \ln \left ( \frac{ic + x }{ic - x } \right ) .
\end{eqnarray*}

La formule (\ref{eq.1.26}) détermine la fonction d'onde de Bethe ; cette fonction est réductible à deux particules. Il convient de mentionner que les fonctions d'onde de tous les modèles résolubles par l'Ansatz de Bethe ont une forme similaire à (\ref{eq.1.26}). Discutons maintenant des propriétés de la fonction d'onde $\chi_N$. La fonction $\chi_N$ est une fonction symétrique des variables $z_j \ (j = 1, \dots, N)$ et une fonction continue de chacune d'elles. Ces propriétés deviennent évidentes si l'on réécrit la représentation (\ref{eq.1.26}) sous la forme suivante :

\begin{eqnarray}
	\chi_N & = &	\frac{  \displaystyle \prod_{k<j} ( \lambda_j - \lambda_k ) }{ \sqrt{N! \displaystyle \prod_{k<j}  [ ( \lambda_j - \lambda_k )^2 +c^2 ] } }  \sum_{\mathcal{P}}\exp \left \{ i \sum_{n = 1}^N  z_n \lambda_{ \mathcal{P}(n) } \right \} \notag \\
	&& \times \prod_{k<j}  	\left [ 1 - \frac{ic \epsilon(z_j - z_k ) }{( \lambda_{ \mathcal{P}(j) }  - \lambda_{ \mathcal{P}(k) }  ) }.  \right ] 
\end{eqnarray}

On peut également voir à partir de cette formule que $\chi_N$ est une fonction antisymétrique des $\lambda_j$ :

\begin{eqnarray}
	\chi_N ( z_1 , \cdots , z_N \vert \lambda_1 , \cdots , \lambda_j , \cdots , \lambda_k , \cdots , \lambda_N ) & =& - 	\chi_N ( z_1 , \cdots , z_N \vert \lambda_1 , \cdots , \lambda_k , \cdots , \lambda_j , \cdots , \lambda_N ).
\end{eqnarray}

Ainsi, $\chi_N = 0 \ \text{si} \ \lambda_j = \lambda_k$, $j \neq k$. C’est la base du \textbf{principe d'exclusion de Pauli} pour les bosons en interaction en une dimension, qui joue un rôle fondamental dans la construction de l'état fondamental, appelé \textbf{mer de Dirac}. La démonstration complète du principe de Pauli est donnée dans la section VII.4.

Il est connu que le théorème reliant le spin et les statistiques ne s'applique pas dans des dimensions 1 + 1 de l'espace-temps. Par conséquent, certains modèles de bosons sont équivalents à des modèles de fermions. Par exemple, le \textbf{modèle de sine-Gordon} est équivalent au \textbf{modèle de Thirring massif}, tandis que le gaz de bosons unidimensionnel pour $c = \infty$ est équivalent au modèle de fermions libres.

Nous avons déjà construit les fonctions propres communes $\chi_N$ (équations (\ref{eq.1.26}), (\ref{eq.1.9})) des opérateurs $\operator{H}$, $\operator{P}$ et $\operator{Q}$, les valeurs propres correspondantes étant données par

\begin{eqnarray}
	E_N = \sum_{j = 1 }^N \lambda_j^2 ; ~ 	P_N = \sum_{j = 1 }^N \lambda_j^1 ; Q_N = \sum_{j = 1 }^N \lambda_j^0 = N 
\end{eqnarray}

Considérons maintenant les fonctions propres dans tout l'espace des coordonnées $\mathbb{R}^N : -\infty < z_j < \infty$ (avec $j = 1, \cdots, N$). La normalisation dans ce cas a été calculée dans \cite{g} :

\begin{eqnarray}
	\int_{-\infty}^{+\infty} d^N z \, \chi_N^\ast ( z_1 , \cdots , z_N \vert \lambda_1 , \cdots , \lambda_N ) \chi_N ( z_1 , \cdots , z_N \vert \mu_1 , \cdots , \mu_N ) & = & ( 2 \pi )^N 	\prod_{ j = 1 }^N \delta ( \lambda_j - \mu _1 ).
\end{eqnarray}

Les moments \(\{ \lambda \}\) et \(\{\mu\}\) sont supposés être ordonnés :

\begin{eqnarray}
	\lambda_1 < \lambda_2 < \cdots < \lambda_N , && 		\mu_1 < \mu_2 < \cdots < \mu_N 
\end{eqnarray}

Dans le même livre \cite{9}, la complétude du système \(\chi_N\) est également prouvée :

\begin{eqnarray}
	\int_{-\infty}^{+\infty} d^N \lambda \, \chi_N^\ast ( z_1 , \cdots , z_N \vert \lambda_1 , \cdots , \lambda_N ) \chi_N ( y_1 , \cdots , y_N \vert \lambda_1 , \cdots , \lambda_N ) & = & ( 2 \pi )^N 	\prod_{ j = 1 }^N \delta ( z_j - y_1 ).
\end{eqnarray}

\begin{eqnarray}
	z_1 < z_2 < \cdots < z_N , && 		y_1 < y_2 < \cdots < y_N .
\end{eqnarray}

