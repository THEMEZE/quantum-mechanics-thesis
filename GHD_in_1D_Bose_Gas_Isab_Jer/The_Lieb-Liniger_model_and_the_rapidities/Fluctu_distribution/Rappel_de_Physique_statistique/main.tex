\begin{figure}[H]
	\centering 
	\begin{tikzpicture}

		\begin{scope}[transform canvas={scale=0.6}]
			% Définition des couleurs avec les codes HTML
\definecolor{colorOne}{HTML}{443E46}
\definecolor{colorTwo}{HTML}{F6DEB8}
\definecolor{colorThree}{HTML}{908CA4}
\definecolor{colorFour}{HTML}{57659E}
\definecolor{colorFive}{HTML}{C57284}
\definecolor{colorSix}{HTML}{FF5B69}

% Raccourcis pour les couleurs
\def\colorOne{colorOne}
\def\colorTwo{colorTwo}
\def\colorThree{colorThree}
\def\colorFour{colorFour}
\def\colorFive{colorFive}
\def\colorSix{colorSix}

\def\colorslide{blue!50!black}

\def\Occupation{
	\def\traitx{0.3}
	\def\traity{0.5}
	\draw[shift={(0,0)}]
		(-13.5 , 0 ) edge [thick,line width=0.8ex ]( -3.2  , 0 )
		( -3.2 - \traitx  , 0 - \traity ) edge [thick,line width=0.8ex ]( -3.2 + \traitx  , 0 + \traity  )
		( -2.8 - \traitx  , 0 - \traity ) edge [thick,line width=0.8ex ]( -2.8 + \traitx  , 0 + \traity  )
		(-2.8 , 0 ) edge [thick,line width=0.8ex ](2.8  , 0 )
		( 2.8 - \traitx  , 0 - \traity ) edge [thick,line width=0.8ex ]( 2.8 + \traitx  , 0 + \traity  )
		( 3.2 - \traitx  , 0 - \traity ) edge [thick,line width=0.8ex ]( 3.2 + \traitx  , 0 + \traity  )
		(3.2, 0 ) edge [thick,line width=0.8ex,->,>=triangle 45 , color = black ]node [pos=1.01,below  ]{\huge$\theta$}	( 13  , 0 )
	;

	
	% Graduation abcsisse 
	% Définitions des listes
% Definitions of the lists
\def\listetuple{-9/\theta_{1}, -8/\theta_{2} , -5/\theta_{3} , -2/\theta_{a-1} , 0/\theta_{a} , 1/\theta_{a+1} , 2/\theta_{a+2} ,  5/\theta_{N-4} , 7/\theta_{N-3},8/\theta_{N-1},9/\theta_{N} }
\def\listetrais{-12 , -11, -10, -9 , -8 , -7 ,  -6 , -5, -4.5,-4, -2 , -1, 0 , 0.5, 1, 2, 4 , 5 ,  6 , 7 , 8 ,8.5, 9 ,  10 , 11, 12 }

% Loop over listetrais
\foreach \r in \listetrais {
    % Initialize found variable to zero
    % Initialize found variable to zero
    %\pgfmathsetmacro\found{0}
    \global\def\found{0}
    \xdef\nomtheta{}
    
    % Check if \r is in listetuple
    \foreach \x/\y in \listetuple { 
        \ifdim \r pt=\x pt % If \r matches any \x in listetuple
            \global\def\found{1} ;
            \xdef\nomtheta{\y} % Set \nomtheta to the corresponding \y
            %\pgfmathsetmacro\found{1} % Set found to 1            
            %\global\pgfmathsetmacro\found{1}
        \fi
    }
    
    %\node [circle, draw, red] (A) at (\r, 2) {\found , $\nomtheta$};
    
    % Draw the line and display \nomtheta if found
    \ifnum\found=1
        \draw[color=\colorOne, thick, line width=0.5ex] 
            (\r, -0.3) -- (\r, 0.3) node[red , pos=-0.5] {\large $\nomtheta$};
         \filldraw[line width=0.5ex, color=\colorSix, outer color=\colorSix, inner color=\colorSix] 
            (\r, 0) circle (4pt);
    \else 
        % Draw without \nomtheta and add a blue circle if not found
        \draw[color=\colorOne, thick, line width=0.5ex] 
            (\r, -0.3) -- (\r, 0.3);
        \filldraw[line width=0.5ex, color=\colorSix, outer color=\colorTwo, inner color=\colorTwo] 
            (\r, 0) circle (4pt); 
    \fi
}

\def\listetrais{-9.5/\theta_{i-1}/2/3, -6.5/\theta_{i}/1/4  ,   -1.5/\theta_{j}/2/4 , 1.5/\theta_{j+1}/-1/3 , 3.5/\theta_{\ell-1}/1/3 , 6.5/\theta_{\ell}/3/4 , 9.5/\theta(\theta_{\ell+1})/-1/3 };


			
}


\begin{scope}
	%\draw[help lines , width=1.5ex] (-8,-3) grid (8,3);\draw[help lines ,width=0.5ex , opacity = 0.5] (-3,-3) grid[step=0.1] (3,3));
	
	%\draw[help lines] 
	%	(-3,-3) edge[width=1.5ex] grid (3,3)	
	%	(-3,-3) edge[width=0.5ex , opacity = 0.5] grid (3,3)	
	%;
	\begin{scope}[shift={(0,1)},rotate=0,opacity=1,color=black]
		\Occupation	
		
		%\node[anchor=east, font=\bfseries] at (-11, 0) {\color{red}\large (T = 0 )} ;	
	\end{scope}
	
	
	
	
	\begin{scope}[shift={(-11.5,3)},rotate=0,opacity=1,color=black]
	
	\begin{scope}[shift={(-0,0)},rotate=0,opacity=1,color=black]
	
		\draw[shift={(0,0)} ,line width=1ex,rounded corners = 1ex,color=\colorOne , opacity =1 ,fill=\colorOne!00 , pattern={north east lines} , pattern color=\colorOne!00 ]
			(0 , -1 ) rectangle (5,1)
		;
		

		\begin{scope}[shift={(0.5,0.5)}]
			\draw[color=\colorOne, thick, line width=0.5ex] 
            (0, -0.3) -- (0, 0.3) ;
            \filldraw[line width=0.5ex, color=\colorSix, outer color=\colorSix, inner color=\colorSix] 
            (0, 0) circle (4pt);
            
            \node[anchor=west, font=\bfseries] at (0.2, 0) {\color{\colorSix}\large : quasi-particule};
		\end{scope}
		
		\begin{scope}[shift={(0.5,-0.5)}]
			\draw[color=\colorOne, thick, line width=0.5ex] 
            (0, -0.3) -- (0, 0.3) ;
            \filldraw[line width=0.5ex, color=\colorSix, outer color=\colorTwo, inner color=\colorTwo] 
            (0, 0) circle (4pt);
            
            \node[anchor=west, font=\bfseries] at (0.2, 0) {\color{\colorSix}\large : hole};
		\end{scope}

	\end{scope}

	
	
	\end{scope}


		
	
\end{scope}
				
		\end{scope}
			
		\draw[color = red , scale = 0.5 , draw = none  ] (-13.5 , -0.1) rectangle (13 , 5) ; 
							
	\end{tikzpicture}	
	\captionsetup{skip=10pt} % Ajoute de l’espace après la légende
\end{figure}



On écrit l'observable énergie et nombre :
	\begin{eqnarray}
		\operator{\mathcal{N}} & = & \sum_{ \{\theta_a\} }   \left ( \sum_{a = 1}^N  1 \right )  \vert \{ \theta_a\}\rangle	\langle \{ \theta_a \}\vert,\\
		\operator{\mathcal{E}} & = & \sum_{\{ \theta_a\}}  \left ( \sum_{a = 1}^N  \varepsilon ( \theta_a ) \right )   \vert \{ \theta_a\}\rangle	\langle \{ \theta_a \}\vert,		
	\end{eqnarray}
	
	avec $\sum_{a = 1}^N 1 \equiv \langle \operator{\mathcal{N}} \rangle_{ \{\theta_a \} }  \doteq  \langle \{ \theta_a \}\vert  \operator{\mathcal{N}} \vert\{ \theta_a\}\rangle  $ et $  \sum_{a = 1}^N  \varepsilon ( \theta_a ) \equiv \langle \operator{\mathcal{E}} \rangle_{\{\theta_a \}}  \doteq  \langle \{ \theta_a \}\vert  \operator{ \mathcal{E}}  \vert\{ \theta_a\}\rangle $.

	
	La probabilité que le système soit dans configuration $\{ \theta_a \}$  est 
	\begin{eqnarray}
		P_{\{\theta_a \}} & = & \frac{e^{- \beta \left ( \langle \operator{\mathcal{E}} \rangle_{\{\theta_a \}}   - \mu \langle \operator{\mathcal{N}} \rangle_{\{\theta_a \}} \right )}}{Z_{thermal}} = \frac{e^{- \beta \sum_{a=1}^N  ( \varepsilon( \theta_a )   - \mu  )}}{Z_{thermal}}	
	\end{eqnarray}
	
	avec la fonction de partition $Z_{thermal} = \sum_{\{ \theta_a\}}e^{- \beta \left ( \langle \operator{\mathcal{E}} \rangle_{\{\theta_a \}}   - \mu \langle \operator{\mathcal{N}} \rangle_{\{\theta_a \}} \right )} = \sum_{\{ \theta_a\}} e^{- \beta \sum_{a=1}^N  ( \varepsilon( \theta_a )   - \mu  )}$
	
	On peut commence à généraliser avec l'opérateur :
	
	
	\begin{eqnarray}
		\operator{\mathcal{O}}_i & = & \sum_{\{\theta_a \}} \langle \operator{\mathcal{O}}_i \rangle_{\{\theta_a \}}  \vert \{ \theta_a\}\rangle	\langle \{ \theta_a \}\vert
	\end{eqnarray}
	
	$\operator{\mathcal{O}}_i \in \{\operator{\mathcal{N}} , \operator{\mathcal{E}} - \mu \operator{\mathcal{N}} \} $  tel que $\sum_i \beta_i \langle \operator{\mathcal{O}}_i \rangle_{\{\theta_a \}} = \beta \left ( \langle \operator{\mathcal{E}} \rangle_{\{\theta_a \}}   - \mu \langle \operator{\mathcal{N}} \rangle_{\{\theta_a \}} \right ) $ et pour simplifier ici $Z \equiv Z_{thermal}$:
	
	\begin{aff}
	Sa  moyenne , variance et équartype de l'observable :
	\begin{eqnarray}
		\langle\operator{\mathcal{O}}_i \rangle & = & \sum_{\{\theta_a \}} 	\langle \operator{\mathcal{O}}_i \rangle_{\{\theta_a \}} \overbrace{\frac{e^{-\sum_i \beta_i \langle \operator{\mathcal{O}}_i \rangle_{\{\theta_a \}} }}{Z}}^{P_{\{\theta_a \}}} = - \left .\frac{1}{Z}\frac{ \partial Z }{ \partial \beta_i } \right )_{\beta_{j\neq i}} =  - \left .\frac{ \partial \ln Z }{ \partial \beta_i }  \right )_{\beta_{j\neq i}}\\
	%\end{eqnarray}	
	%et 	
	%\begin{eqnarray}
		\langle\operator{\mathcal{O}}_i^2 \rangle & = & \sum_{\{\theta_a \} } 	\langle \operator{\mathcal{O}}_i \rangle_{\{\theta_a \}}^2 \frac{e^{-\sum_i \beta_i \langle \operator{\mathcal{O}}_i \rangle_{\{\theta_a \}} }}{Z} = \left . \frac{1}{Z} \frac{ \partial^2 Z }{ {\partial \beta_i}^2 }  \right )_{\beta_{j\neq i}} \notag \\
	%\end{eqnarray}
	%\begin{eqnarray}
		\Delta_{\operator{\mathcal{O}}_i}^2  & = & 	\left \langle \left (\operator{\mathcal{O}}_i - \langle\operator{\mathcal{O}}_i \rangle \right )^2  \right \rangle  = 	\langle\operator{\mathcal{O}}_i^2 \rangle  -  \langle\operator{\mathcal{O}}_i \rangle^2 = \left . \frac{1}{Z} \frac{ \partial^2 Z }{ {\partial \beta_i}^2 }  \right )_{\beta_{j\neq i}} - \left ( \left . \frac{1}{Z}\frac{ \partial Z }{ \partial \beta_i }  \right )_{\beta_{j\neq i}}\right )^2  \\
		& = & \frac{\partial}{\partial \beta_i } \left ( \left . \frac{1}{Z} \frac{\partial Z}{\partial \beta_i }  \right )_{\beta_{j\neq i}}  \right )_{\beta_{j\neq i}} =  \left . \frac{\partial^2 \ln Z  }{{\partial \beta_i}^2 }  \right )_{\beta_{j\neq i}}  = - \left . 	\frac{\partial \langle\operator{\mathcal{O}}_i \rangle }{\partial \beta_i } \right )_{\beta_{j\neq i}}
	\end{eqnarray}
	\end{aff}
	
	si $\operator{\mathcal{O}}_i = \operator{\mathcal{N}}$ alors $\beta_i = - \beta \mu $ et si $\operator{\mathcal{O}}_i = \operator{\mathcal{E}} - \mu \operator{\mathcal{N}} $ alors $\beta_i = \beta$.\\
	
	\begin{eqnarray}
		\langle \operator{\mathcal{N}} \rangle  = \left .\frac{1}{\beta} \frac{ \partial \ln Z}{\partial \mu } \right )_T,  & & \Delta^2_{\operator{\mathcal{N}}} = \left . \frac{1}{\beta^2} \frac{ \partial^2 \ln Z}{{\partial \mu}^2 } \right )_T =  \left . \frac{1}{\beta} \frac{ \partial \langle \operator{\mathcal{N}} \rangle}{\partial \mu } \right )_T\\
		\langle \operator{\mathcal{E}} - \mu\operator{\mathcal{N}}  \rangle  = -\left . \frac{ \partial \ln Z}{\partial \beta } \right )_\mu,  & & \Delta^2_{\operator{\mathcal{E}} - \mu\operator{\mathcal{N}}} = \left .  \frac{ \partial^2 \ln Z}{{\partial \beta}^2 } \right )_\mu =  -\left .  \frac{ \partial \langle \operator{\mathcal{E}} - \mu\operator{\mathcal{N}} \rangle}{\partial \beta } \right )_\mu	\\
		\langle \operator{\mathcal{E}} \rangle  = \left [ \left .\frac{\mu}{\beta} \frac{ \partial}{\partial \mu } \right )_T -\left . \frac{ \partial }{\partial \beta } \right )_\mu  \right ]\ln Z,  & & \Delta^2_{\operator{\mathcal{E}} } = \left [ \left .\frac{\mu}{\beta} \frac{ \partial}{\partial \mu } \right )_T -\left . \frac{ \partial }{\partial \beta } \right )_\mu  \right ]^2\ln Z=  \left [ \left .\frac{\mu}{\beta} \frac{ \partial}{\partial \mu } \right )_T -\left . \frac{ \partial }{\partial \beta } \right )_\mu  \right ]\langle \operator{\mathcal{E}} \rangle	
	\end{eqnarray}

	
	\begin{eqnarray}
		\langle\operator{\mathcal{O}}_i \rangle & = & 	\sum_{\{\theta_a \} } 	\langle \{\theta_a \}  \vert \operator{\mathcal{O}}_i \vert \{\theta_a \}  \rangle \frac{e^{-\sum_i \beta_i \langle \operator{\mathcal{O}}_i \rangle_{\{\theta_a \}} }}{Z},\\
		& = & 	\sum_{\{\theta_b \}} \langle \{\theta_b \}  \vert  \operator{\mathcal{O}}_i \sum_{\{\theta_a \}}\frac{e^{-\sum_i \beta_i \langle \operator{\mathcal{O}}_i \rangle_{\{\theta_a \} } }}{Z} \vert \{\theta_a \}  \rangle  	\langle \{\theta_a \}  \vert  \{\theta_b \}  \rangle ,\\
		& = & Tr (  \operator{\mathcal{O}}_i \operator{\rho}) 
	\end{eqnarray}
	
	avec $\operator{\rho} = \sum_{\{\theta_a \}}\frac{e^{-\sum_i \beta_i \langle \operator{\mathcal{O}}_i \rangle_{\{\theta_a \}} }}{Z} \vert \{\theta_a \}  \rangle   	\langle \{\theta_a \} \vert $ et $Z = \sum_{\{\theta_a \}}e^{-\sum_i \beta_i \langle \operator{\mathcal{O}}_i \rangle_{ \{\theta_a \} } } $ tel que $Tr (  \operator{\rho}) = 1 $\\
	
	La matrice densité thermique est :
	\begin{eqnarray}
		\operator{\rho}_{thermal} & = & \frac{e^{- \beta \operator{H}}}{Z_{thermal}}, \\
		e^{-\beta \operator{H}} & = & 	\sum_{\{\theta_a \}} e^{- \beta \sum_{a=1}^N ( \varepsilon(\theta_a)- \mu ) } \vert \{ \theta_a\} \rangle \langle  \{ \theta_a\}  \vert 
	\end{eqnarray}


La matrice densité GGE avec $Z \equiv Z_{GGE}$ est :
	\begin{eqnarray}
		\operator{\rho}_{GGE}[f] & = & \sum_{\{\theta_a \}}\frac{e^{-\sum_{i = 1}^\infty  \beta_i \langle \operator{\mathcal{O}}_i \rangle_{\{\theta_a \}} }}{Z} \vert \{\theta_a \}  \rangle   	\langle \{\theta_a \} \vert. 
	\end{eqnarray}
	
	Dans le cas thermique, on peut remarquer que $\langle \operator{\mathcal{N}} \rangle_{\{ \theta_a\} } \propto \sum_{a = 1}^N \theta_a^0 $ et $\langle \operator{\mathcal{E}} \rangle_{\{ \theta_a\} } \propto \sum_{a = 1}^N \theta_a^2 $. On peut donc réécrire $\sum_{i = 1}^\infty  \beta_i \langle \operator{\mathcal{O}}_i \rangle_{ \{\theta_a \} }$
	
	\begin{eqnarray}
		\sum_{i = 1}^\infty  \beta_i \langle \operator{\mathcal{O}}_i \rangle_{ \{\theta_a \} } & = & \sum_{i = 0}^\infty \alpha_i \sum_{a = 1 }^N \theta_a^i		
	\end{eqnarray}
	
	et pour chaque $a \in \llbracket 1 , N  \rrbracket \colon \sum_i \alpha_i \theta_a^i$ converge donc on peut échanger les deux sommes soit 
	
	\begin{eqnarray}
		\sum_{i = 1}^\infty  \beta_i \langle \operator{\mathcal{O}}_i \rangle_{ \{\theta_a \} } & = & \sum_{a = 1 }^N  f(\theta_a) 
	\end{eqnarray}
	
	avec $f(\theta) =  \sum_{i = 0}^\infty \alpha_i  \theta^i$.	 Et on peut réecrire la matrice densité  :
	
	\begin{eqnarray}
		\operator{\rho}_{GGE}[f] & = & \frac{e^{-\operator{Q}[f]}}{Z_{GGE}}, \\
		e^{-\operator{Q}[f]} & = & 	\sum_{\{\theta_a \}} e^{- \sum_{a = 1}^N f(\theta_a) } \vert \{ \theta_a\} \rangle \langle  \{ \theta_a\}  \vert 
	\end{eqnarray}
 
	
	pour une certaine fonction $f$ relié à la charge $\operator{Q} [f]  = \sum_{\{\theta_a \}} \left ( \sum_{a = 1}^N f ( \theta_a )  \right ) \vert \{ \theta_a \} \rangle \langle \{ \theta_a \} \vert $.
	Et on peut réecrire la probabilité de la configuration $\{\theta_a\}$ : $ P_{\{ \theta_a \}} = \langle \{ \theta_a \}\vert \operator{\rho}_{GGE}[f] \vert  \{ \theta_a \} \rangle = e^{-\sum_{a = 1}^N f(\theta_a)} / Z $ avec $Z = \sum_{\{\theta_a \}} e^{-\sum_{a = 1}^N f(\theta_a)}$.\\
	
	 Nous aimerions calculer les valeurs d'attente par rapport à cette matrice de densité, par exemple
	La moyenne GGE d'un observable s'écrit ,
	\begin{aff}
	\begin{eqnarray}
		\langle \operator{\mathcal{O}} \rangle_{GGE} & \doteq & \displaystyle  \frac{\text{Tr} (\operator{\mathcal{O}}\operator{\rho}_{GGE}[f])}{\text{Tr} (\operator{\rho}_{GGE}[f])} = \frac{\text{Tr} (\operator{\mathcal{O}}e^{-\operator{Q}[f]})}{\text{Tr} (e^{-\operator{Q}[f]})}	 = \frac{\sum_{\{\theta_a \}} \langle  \{ \theta_a\}  \vert   \operator{\mathcal{O}} \vert \{ \theta_a\} \rangle e^{- \sum_{a = 1}^N f(\theta_a) }  }{\sum_{\{\theta_a  \}} e^{- \sum_{a = 1}^N  f(\theta_a) } }
		%& =  & \frac{ \sum_{\pi} \sum_{\vert \{\theta_a \}\rangle \vert \Pi } \langle  \{ \theta_a\}  \vert   \operator{\mathcal{O}} \vert \{ \theta_a\} \rangle e^{- \sum_{a = 1}^N f(\theta_a) }  }{\sum_{\pi} \sum_{\vert \{\theta_a \}\rangle \vert \Pi }  e^{- \sum_{a = 1}^N  f(\theta_a) } }
	\end{eqnarray}
	pour une certaine observable $\operator{\mathcal{O}}$.\\
	\end{aff}