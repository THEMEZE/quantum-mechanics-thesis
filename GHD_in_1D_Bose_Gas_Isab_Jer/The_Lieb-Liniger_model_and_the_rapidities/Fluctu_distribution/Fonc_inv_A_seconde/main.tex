On utilise la théorie de perturbation sur 

\begin{eqnarray*}
	\operator{A} & = & \operator{A}^{(0)} + \delta \theta \operator{V}
\end{eqnarray*}

avec $A^{(0)}_{\theta , \theta'} = \left ( \frac{ ( \Pi^c/\Pi^c_s)^{-1}}{\Pi^c_s - \Pi^c} \right )(\theta) \delta \theta   \delta_{\theta,\theta '}$ et $V_{\theta , \theta'} = \left \{ - \left [ \left ( \frac{1}{\Pi^c_s - \Pi^c } \right ) ( \theta)  + \left ( \frac{1}{\Pi^c_s - \Pi^c } \right ) ( \theta' )\right ] \frac{ \Delta( \theta'- \theta )}{ 2 \pi } + \left \langle  \frac{ \Pi^c/\Pi^c_s}{\Pi^c_s - \Pi^c} , \frac{\Delta_\theta}{2 \pi}\frac{\Delta_{\theta'}}{2 \pi} \right \rangle  \right \} \delta \theta $.\\


{\bf On peut aussi utiliser la formule de Neuman} si $\vert \delta \theta \vert \Vert  (\operator{A}^{(0)})^{-1} \operator{V} \Vert < 1 $,

\begin{eqnarray*}
	(\operator{A}^{(0)} + \delta \theta \operator{V})^{-1} = \sum_{k=0}^\infty (-\delta \theta)^k ((\operator{A}^{(0)})^{-1} \operator{V})^k(\operator{A}^{(0)})^{-1}
\end{eqnarray*}


\footnote{

{\color{magenta}
Hypothèse de départ : 

Soit $A$ une matrice inversible et $H$ une perturbation telle que \( \| A^{-1} H \| < 1 \).  
Cette condition garantit la convergence de la série géométrique matricielle. Nous cherchons à exprimer l’inverse de \( A + H \) en termes de \( A^{-1} \) et \( H \).

Étapes de la démonstration

Étape 1 : Formulation du problème

On commence avec l'identité matricielle fondamentale :

\begin{eqnarray*}
	(A + H)(A + H)^{-1}  &= & I	
\end{eqnarray*}
,


où \( I \) est la matrice identité. Multiplions par \( A^{-1} \) à gauche :

\begin{eqnarray*}
A^{-1}(A + H)(A + H)^{-1} & = & A^{-1}.
\end{eqnarray*}

En développant le produit matriciel :

\begin{eqnarray*}
	(I + A^{-1} H)(A + H)^{-1} &=& A^{-1}.
\end{eqnarray*}

Posons \( B = (A + H)^{-1} \). Cela donne :

\begin{eqnarray*}
	(I + A^{-1} H)B &= &A^{-1}.
\end{eqnarray*}

---

Étape 2 : Résolution de l’équation

Réorganisons pour isoler \( B \) :

\begin{eqnarray*}
	B &=& (I + A^{-1} H)^{-1} A^{-1}.
\end{eqnarray*}

Développons \( (I + A^{-1} H)^{-1} \). Si \( \| A^{-1} H \| < 1 \), nous pouvons appliquer la formule de la série géométrique matricielle :

\begin{eqnarray*}
	(I + A^{-1} H)^{-1}& = &\sum_{k=0}^\infty (-1)^k (A^{-1} H)^k.
\end{eqnarray*}

Substituons cette expression dans \( B \) :

\begin{eqnarray*}
	B &= &\left( \sum_{k=0}^\infty (-1)^k (A^{-1} H)^k \right) A^{-1}.
\end{eqnarray*}

En réorganisant :

\begin{eqnarray*}
	(A H)^{-1} &= &\sum_{k=0}^\infty (-1)^k (A^{-1} H)^k A^{-1}.
\end{eqnarray*}


Conditions de validité

La série converge si \( \| A^{-1} H \| < 1 \), c’est-à-dire que \( H \) est suffisamment "petite" par rapport à \( A \).  
Cette condition garantit que \( (I + A^{-1} H)^{-1} \) est bien définie et que la série géométrique converge.}
}

or 

\begin{eqnarray*}
	\left ( (\operator{A}^{(0)})^{-1} \operator{V}	(\operator{A}^{(0)})^{-1} \right)	_{\theta,\theta'} & = & (E_\theta^{(0)})^{-1} \delta_{\theta, \ell}  V^{\ell \kappa} (E_\kappa^{(0)})^{-1} \delta_{\kappa, \theta'} \\
	& = & (E_\theta^{(0)})^{-1} \delta_{\theta, \ell} {V^{\ell }}_{\theta'} (E_{\theta'}^{(0)})^{-1}\\
	& = & (E_\theta^{(0)})^{-1}  V_{\theta , \theta'} (E_{\theta'}^{(0)})^{-1}
\end{eqnarray*}

\begin{aff}
Donc une a l'ordre un en $\delta \theta (\operator{A}^{(0)})^{-1} \operator{V}$ 

\begin{eqnarray*}
	(\operator{A}^{-1})_{\theta,\theta'} & = &  ( (\Pi^c_s - \Pi^c)\Pi^c/\Pi^c_s ) ( \theta ) \delta_{\theta, \theta'}/\delta \theta + \mathscr{F}(\theta , \theta' ) ,	
\end{eqnarray*}

avec 

\begin{eqnarray*}
	\mathscr{F}(\theta , \theta' ) & = & \left [ (\Pi^c_s - \Pi^c )( \theta)  +  (\Pi^c_s - \Pi^c ) ( \theta' )\right ] \frac{\Pi^c}{\Pi^c_s}(\theta)\frac{\Pi^c}{\Pi^c_s}(\theta') \frac{ \Delta( \theta'- \theta )}{ 2 \pi }\\
	&&  - \left [ (\Pi^c_s - \Pi^c )( \theta)   (\Pi^c_s - \Pi^c ) ( \theta' )\right ] \frac{\Pi^c}{\Pi^c_s}(\theta)\frac{\Pi^c}{\Pi^c_s}(\theta')\left \langle  \frac{ \Pi^c/\Pi^c_s}{\Pi^c_s - \Pi^c} , \frac{\Delta_\theta}{2 \pi}\frac{\Delta_{\theta'}}{2 \pi} \right \rangle 	
\end{eqnarray*}
\end{aff}


{\bf Ou utiliser la théorie de pertubation},\\


$\operator{A}$  est symétrique donc diagonalisable. Elle est diagonalisable dans une base de vecteur propre orthonormé $\vert \varphi_\theta \rangle $ associé à la valeur propre $E_\theta$ soit 

\begin{eqnarray*}
	(\operator{A}^{(0)} + \delta \theta \operator{V} \vert \varphi_\theta \rangle & = & 	E_\theta \vert \varphi_\theta \rangle	
\end{eqnarray*}

avec 

\begin{eqnarray*}
	\vert \varphi_\theta \rangle  & = & \vert \varphi_\theta^{(0)} \rangle  + \delta \theta  \vert \varphi_\theta^{(1)}	\rangle  + 	(\delta \theta)^2 \vert \varphi_\theta^{(2)}	\rangle  + \cdots \\
	E_\theta & = & E_\theta^{(0)}   + \delta \theta E_\theta^{(1)}  + 	(\delta \theta)^2 E_\theta^{(2)} + \cdots 	
\end{eqnarray*}

tel que 

\begin{eqnarray*}
	\operator{A}^{(0)} \vert \varphi_\theta^{(0)} \rangle & = & E_\theta^{(0)}	 \vert \varphi_\theta^{(0)} \rangle 	
\end{eqnarray*}

Soit un Dl en $\delta \theta $ 

\begin{eqnarray*}
	\vert \varphi_\theta \rangle & = & \vert \varphi_\theta^{(0)} \rangle + \delta \theta \sum_{\theta' \neq \theta } \vert \varphi_{\theta'}^{(0)} \rangle   \frac{ V_{\theta', \theta}}{ E_\theta^{(0)} - E_{\theta'}^{(0)} } \\
	E_\theta & = & E_\theta^{(0)} + \delta \theta V_{\theta , \theta} + (\delta \theta)^2 \sum_{\theta' \neq \theta } \frac{ V_{\theta', \theta}^2}{ E_\theta^{(0)} - E_{\theta'}^{(0)} }
\end{eqnarray*}

On note $\operator{P}$ la matrace de passage de la base $\vert \varphi_\theta^{(0)} \rangle $ à $\vert \varphi_\theta \rangle $


\begin{eqnarray*}
	\operator{A} & = & {~}^t\operator{P}\operator{D}\operator{P}	\\
	\vert \varphi_\theta \rangle & = & 	\operator{P}\vert \varphi_\theta^{(0)} \rangle
\end{eqnarray*}

avec $\operator{D}$ diagonale avec $E_\theta$ dans la diagonale et 

\begin{eqnarray*}
	\operator{P} & = & 	\operator{1}	 + \delta \theta \operator{P}^{(1)}
\end{eqnarray*}

avec 

\begin{eqnarray*}
	 {(\operator{P}^{(1)})^{\theta}}_{\theta'} & = & \left \{ \begin{array}{rll} \frac{ V_{\theta,\theta'}}{ E_{\theta}^{(0)} - E_{\theta'}^{(0)} } & \mbox{si} & \theta \neq \theta' \\ 0 & \mbox{sinon}    \end{array}\right.	.	
\end{eqnarray*}

et dans la base $\vert \varphi_\theta^{(0)} \rangle $

\begin{eqnarray*}
	\operator{A}^{-1} & = & {~}^t\operator{P}\operator{D}^{-1} \operator{P}	
\end{eqnarray*}

avec 

\begin{eqnarray*}
	{(\operator{A}^{-1})^{\theta}}_{\theta'} & = & \left ({\operator{\delta}^{\theta}}_\ell +  \delta \theta {(\operator{P}^{(1)})_\ell}^\theta  \right )  \left ( 	(E^{-1})^\ell {\operator{\delta}^\ell}_\kappa \left ({\operator{\delta}^\kappa}_{\theta'} +  \delta \theta {(\operator{P}^{(1)})^\kappa}_{\theta'} \right ) \right )\\%\sum_\ell (\delta_{\theta,\ell} +  \delta \theta P^{(1)}_{\ell , \theta} ) \sum_\kappa 	E_\ell^{-1} \delta_{\ell,\kappa} (\delta_{\kappa,\theta'} +  \delta \theta P^{(1)}_{\kappa , \theta'} )
	& = &   \left ({\operator{\delta}^{\theta}}_\ell +  \delta \theta {(\operator{P}^{(1)})_\ell}^\theta  \right )  \left ( 	(E^{-1})^\ell  \left ({\operator{\delta}^\ell}_{\theta'} +  \delta \theta {(\operator{P}^{(1)})^\ell}_{\theta'} \right ) \right ) \\ %\sum_\ell (\delta_{\theta,\ell} +  \delta \theta P^{(1)}_{\ell , \theta} ) ( E_\ell^{-1} (\delta_{\ell,\theta'} + \delta \theta P^{(1)}_{\ell , \theta'}))
	& = & (E^{-1})^\theta  {\operator{\delta}^\theta}_{\theta'} - \delta \theta {(\operator{P}^{(1)})^\theta}_{\theta'} (  E_{\theta'}^{-1} - E_\theta^{-1}   ) - (\delta \theta)^2 {(\operator{P}^{(1)})^\theta}_{\ell}{(\operator{P}^{(1)})^\ell}_{\theta'}  E_{\ell}^{-1} %E_\theta^{-1} \delta_{\theta, \theta'} + \delta \theta P_{\theta,\theta'}^{(1)} ( E_\theta^{-1} - E_{\theta'}^{-1}  ) + (\delta \theta)^2 \sum_\ell P_{\theta,\ell}^{(1)}P_{\ell,\theta'}^{(1)}  E_{\ell}^{-1}  
\end{eqnarray*}

car $\operator{P}^{(1)}$ est antisymétrique ie $P_{\theta,\theta'}^{(1)} = -P_{\theta',\theta}^{(1)}$. Et avec 

\begin{eqnarray*}
	E_\theta^{-1} & = & \frac{1}{E_\theta^{(0)}} \frac{1}{ 1 + \delta \theta \frac{V_{\theta,\theta}}{ E_\theta^{(0)}}  + (\delta  \theta)^2 \sum_{\theta' \neq \theta} \frac{ V_{\theta',\theta}^2}{E_\theta^{(0)} ( E_\theta^{(0)} - E_{\theta'}^{(0)})	}	}\\
	& \underset{\delta \theta \to 0 }{=} & \frac{1}{E_\theta^{(0)}} - \delta \theta \frac{V_{\theta,\theta}}{ (E_\theta^{(0)})^2} + ( \delta \theta )^2  \left ( \frac{V_{\theta,\theta}^2}{ (E_\theta^{(0)})^3} - \sum_{\theta' \neq \theta} \frac{ V_{\theta',\theta}^2}{(E_\theta^{(0)})^2 ( E_\theta^{(0)} - E_{\theta'}^{(0)})	}\right) + o ( \Vert \delta \theta\Vert^3 ) 
\end{eqnarray*}

donc 

\begin{eqnarray*}
	E_\theta^{-1} - E_{\theta'}^{-1}	 & \underset{\delta \theta \to 0 }{=} &	 \underbrace{\frac{1}{E_\theta^{(0)}} - \frac{1}{E_{\theta'}^{(0)}}}_{\frac{E_{\theta'}^{(0)}-E_{\theta}^{(0)}}{E_{\theta}^{(0)}E_{\theta'}^{(0)}}} + \delta \theta \left ( \frac{V_{\theta',\theta'}}{ (E_{\theta'}^{(0)})^2} - \frac{V_{\theta,\theta}}{ (E_\theta^{(0)})^2}\right ) \\
	& & + \underbrace{(\delta \theta )^2 \left ( \frac{V_{\theta,\theta}^2}{ (E_\theta^{(0)})^3} - \frac{V_{\theta',\theta'}^2}{ (E_{\theta'}^{(0)})^3} + \sum_{\ell \neq \theta'} \frac{ V_{\ell,\theta'}^2}{(E_{\theta'}^{(0)})^2 ( E_{\theta'}^{(0)} - E_{\ell}^{(0)})	} - \sum_{\ell \neq \theta} \frac{ V_{\ell,\theta}^2}{(E_\theta^{(0)})^2 ( E_\theta^{(0)} - E_{\ell}^{(0)})	}\right ) + o ( \Vert \delta \theta \Vert^3 ) }_{ o ( \Vert \delta \theta \Vert^2 )}
\end{eqnarray*}

et 

\begin{eqnarray*}
	\sum_\ell P_{\theta,\ell}^{(1)}P_{\ell,\theta'}^{(1)}  E_{\ell}^{-1} & \underset{\delta \theta \to 0 }{=} &	 \sum_{\ell \notin \{ \theta , \theta' \} } 	\frac{ V_{\theta,\ell}^2}{E_\ell^{(0)} - 	E_{\theta}^{(0)}} 	\frac{ V_{\ell,\theta'}^2}{E_{\theta'}^{(0)} - 	E_\ell^{(0)}} \frac{1}{E_\ell^{(0)}}  + o (\Vert \delta \theta \Vert ) 
\end{eqnarray*}

Donc 

\begin{eqnarray*}
	(A^{-1})_{\theta, \theta'} & = & \frac{1}{E_\theta^{(0)}}\delta_{\theta,\theta'} - \delta \theta \left ( \frac{\overbrace{V_{\theta,\theta'}}^{0 ~si ~\theta = \theta'}}{E_\theta^{(0)}E_{\theta'}^{(0)}}\delta_{\theta \neq \theta'} + \frac{V_{\theta,\theta}}{(E_{\theta}^{(0)})^2} \delta_{\theta, \theta'}\right )\\
	&& \underbrace{(\delta \theta)^2 \left ( \frac{V_{\theta,\theta}^2}{ (E_\theta^{(0)})^3} \delta_{\theta,\theta'} - \sum_{\ell \neq \theta} \frac{ V_{\ell,\theta}^2}{(E_\theta^{(0)})^2 ( E_\theta^{(0)} - E_{\ell}^{(0)})	} +  \frac{V_{\theta,\theta'}}{E_{\theta'}^{(0)}-E_{\theta}^{(0)}} \left ( \frac{V_{\theta',\theta'}}{(E_{\theta'}^{(0)})^2}-\frac{V_{\theta,\theta}}{(E_{\theta}^{(0)})^2}\right) \delta_{\theta \neq \theta'}  - \sum_{\ell \notin \{ \theta , \theta' \} } 	\frac{ V_{\theta,\ell}^2}{E_\ell^{(0)} - 	E_{\theta}^{(0)}} 	\frac{ V_{\ell,\theta'}^2}{E_{\theta'}^{(0)} - 	E_\ell^{(0)}} \frac{1}{E_\ell^{(0)}}\right )  + o ( \Vert \delta \theta \Vert^3 ) }_{o ( \Vert \delta \theta \Vert^2}
\end{eqnarray*}\\


{\bf Ou revenire à la définition des fluctation } 

La lettre $E$ désigne un $\mathbb{K}$-e.v de dimension finie $n\in \mathbb{N}^\ast$. Soit $F$ un s.e.v de $E$ tel que

\begin{eqnarray*}
	E  & = & F \oplus F^\perp.	
\end{eqnarray*}

Soient $x \in E$ et $x_{\vert F }$ et   la restriction de $x$ à $F$.\\
Soient $u \in \mathcal{L}(E)$ et les restrictions  $u_{\vert F } \in \mathcal{L}(F)$ de $u$ à $F$ , $u_{\vert F^\perp } \in \mathcal{L}(F^\perp)$ de $u$ à $F^\perp$ , $u_{\vert F,F^\perp } \in \mathcal{L}(F,F^\perp)$ et $u_{\vert F^\perp, F } \in \mathcal{L}(F^\perp,F)$.\\
Soit $f$ une application de $E$ à valeur dans $\mathbb{K}$ et $f_{\vert F}$ sa restriction à $F$. \\

On veux calculer la moyenne de $f_{\vert F}$

\begin{eqnarray*}
	\langle f_{\vert F } \rangle  & = & \frac{\int d x \, f_{\vert F}(x) \exp \left ( - \frac{1}{2} ( x , u (x)) \right ) }{\int d x \, \exp \left ( - \frac{1}{2} ( x , u (x)) \right )}
\end{eqnarray*}

Je veux calculer les produit scalaire $\left ( x , u (x) \right )$ selon les différentes restriction. 

Pour visualiser les calcules plus facilement on note $\operator{U}$ la représentation matritielle de $u$ dans une basse orthogonale. Et bien que l'écriture soit faux mais pratique, on écrit 

\begin{eqnarray*}
	\left ( x , u (x) \right ) & = &  \left ( \begin{array}{cc} x_{\vert F } &  x_{\vert F^\perp }  \end{array} \right) \left ( \begin{array}{cc} \operator{U}_{\vert F} & \operator{U}_{\vert F , F^\perp}  \\ \operator{U}_{\vert F^\perp , F } & \operator{U}_{\vert F^\perp} \end{array} \right ) \left (  \begin{array}{c} x_{\vert F} \\  x_{\vert F^\perp} \end{array} \right ) ,
\end{eqnarray*}

Soit 

\begin{eqnarray*}
	\left ( x , u (x) \right ) & = & \left ( x_{\vert F } , u_{\vert F } \left (x_{\vert F } \right ) \right ) + \left ( x_{\vert F } , u_{\vert F , F^\perp  } \left (x_{\vert F^\perp } \right ) \right )  + \left ( x_{\vert F^\perp } , u_{\vert F^\perp , F  } \left (x_{\vert F } \right ) \right )  + \left ( x_{\vert F^\perp } , u_{\vert F^\perp } \left (x_{\vert F^\perp } \right ) \right ) 		
\end{eqnarray*}

De plus $u$ est auto-adjoint donc 

\begin{eqnarray*}
	\left ( x_{\vert F^\perp } , u_{\vert F^\perp , F  } \left (x_{\vert F } \right ) \right ) & = & \left ( x_{\vert F } , u_{\vert F , F^\perp  } \left (x_{\vert F^\perp } \right ) \right ) \\
	& = & \left ( u_{\vert F^\perp , F   }\left (x_{\vert F } \right )  ,  x_{\vert F^\perp }  \right )
\end{eqnarray*}

Soit 

\begin{eqnarray*}
	\left ( x , u (x) \right ) & = & \left ( x_{\vert F } , u_{\vert F } \left (x_{\vert F } \right ) \right ) +  2 \left ( u_{\vert F^\perp , F   }\left (x_{\vert F } \right )  ,  x_{\vert F^\perp }  \right )   + \left ( x_{\vert F^\perp } , u_{\vert F^\perp } \left (x_{\vert F^\perp } \right ) \right )		
\end{eqnarray*}

Donc 

\begin{eqnarray*}
	\int d x \, f_{\vert F}(x) \exp \left ( - \frac{1}{2} ( x , u (x)) \right )  & = & \int d x_{\vert F} \,  f(x_{\vert F}) \exp \left ( - \frac{1}{2} \left ( x_{\vert F } , u_{\vert F } \left (x_{\vert F } \right ) \right ) \right ) \\
	&& \times \int d x_{\vert F^\perp}\exp \left ( - \frac{1}{2} \left ( x_{\vert F^\perp } , u_{\vert F^\perp } \left (x_{\vert F^\perp } \right ) \right ) - \left ( u_{\vert F^\perp , F   }\left (x_{\vert F } \right )  ,  x_{\vert F^\perp }  \right ) \right )		
\end{eqnarray*}

or pour $u$ par forcement auto-adjoint

\begin{eqnarray*}
	\int dx \,  \exp \left ( -\frac{1}{2} (x , u(x)) + (b , x)  \right ) & = & \frac{\exp \left( \frac{1}{2}(b , u^{-1}(b)) \right) }{\sqrt{ \det \left( \frac{u}{2\pi} \right) }}	
\end{eqnarray*}

donc 

\begin{eqnarray*}
	\int d x_{\vert F^\perp}\exp \left ( - \frac{1}{2} \left ( x_{\vert F^\perp } , u_{\vert F^\perp } \left (x_{\vert F^\perp } \right ) \right ) - \left ( u_{\vert F^\perp , F   }\left (x_{\vert F } \right )  ,  x_{\vert F^\perp }  \right ) \right ) & = & 	\frac{\exp \left( \frac{1}{2}(u_{\vert F^\perp , F   }\left (x_{\vert F } \right ) , \left (u_{\vert F^\perp}\right )^{-1}(u_{\vert F^\perp , F   }\left (x_{\vert F } \right ))) \right) }{\sqrt{ \det \left( \frac{u_{\vert F^\perp}}{2\pi} \right) }}	\\
	& = & 	\frac{\exp \left( \frac{1}{2} \left( x_{\vert F} , \left( u_{\vert F, F^\perp} \circ \left (u_{\vert F^\perp}\right )^{-1} \circ u_{\vert F^\perp, F} \right) \left( x_{\vert F} \right) \right) \right)}{\sqrt{ \det \left( \frac{u_{\vert F^\perp}}{2\pi} \right) }}
\end{eqnarray*}

donc 

\begin{eqnarray*}
	\int d x \, f_{\vert F}(x) \exp \left ( - \frac{1}{2} ( x , u (x)) \right )  & = & \int d x_{\vert F} \,  f(x_{\vert F}) \exp \left ( - \frac{1}{2} \left ( x_{\vert F } , \left ( u_{\vert F }- u_{\vert F, F^\perp} \circ \left (u_{\vert F^\perp}\right )^{-1} \circ u_{\vert F^\perp, F} \right ) \left (x_{\vert F } \right ) \right ) \right )/\sqrt{ \det \left( \frac{u_{\vert F^\perp}}{2\pi} \right) } \\	
\end{eqnarray*}

Ce qui nous rassure car 

\begin{eqnarray*}
	\frac{1}{\sqrt{ \det \left( \frac{u}{2\pi} \right) }} & = & \int d x \ \exp \left ( - \frac{1}{2} ( x , u (x)) \right )\\	
	& = & \int d x_{\vert F} \ \exp \left ( - \frac{1}{2} \left ( x_{\vert F } , u_{\vert F } \left (x_{\vert F } \right ) \right ) \right ) \\
	&& \times \int d x_{\vert F^\perp}\exp \left ( - \frac{1}{2} \left ( x_{\vert F^\perp } , u_{\vert F^\perp } \left (x_{\vert F^\perp } \right ) \right ) - \left ( u_{\vert F^\perp , F   }\left (x_{\vert F } \right )  ,  x_{\vert F^\perp }  \right ) \right )	\\
	& = & 	\int d x_{\vert F} \, \exp \left ( - \frac{1}{2} \left ( x_{\vert F } , \left ( u_{\vert F }- u_{\vert F, F^\perp} \circ \left (u_{\vert F^\perp}\right )^{-1} \circ u_{\vert F^\perp, F} \right ) \left (x_{\vert F } \right ) \right ) \right )/\sqrt{ \det \left( \frac{u_{\vert F^\perp}}{2\pi} \right) }\\
	& = & \frac{1}{\sqrt{ \det \left( \frac{u_{\vert F }- u_{\vert F, F^\perp} \circ \left (u_{\vert F^\perp}\right )^{-1} \circ u_{\vert F^\perp, F}}{2\pi} \right)\det \left( \frac{u_{\vert F^\perp}}{2\pi} \right)}} 
\end{eqnarray*}

et d'aprés les complement et déterminent de Schur on a 

\begin{eqnarray*}
	\det (u ) & = & \det \left(u_{\vert F^\perp} \right) \det \left(u_{\vert F }- u_{\vert F, F^\perp} \circ\left (u_{\vert F^\perp}\right )^{-1} \circ u_{\vert F^\perp, F} \right)	
\end{eqnarray*}

Donc ça me rassure.\\

Donc

\begin{eqnarray*}
	\langle f_{\vert F } \rangle  & = & \frac{\int d x_{\vert F}  \, f(x_{\vert F}) \, \exp \left ( - \frac{1}{2} \left ( x_{\vert F } , \left ( u_{\vert F }- u_{\vert F, F^\perp} \circ \left (u_{\vert F^\perp}\right )^{-1} \circ u_{\vert F^\perp, F} \right ) \left (x_{\vert F } \right ) \right ) \right ) }{\int d x_{\vert F}   \, \exp \left ( - \frac{1}{2} \left ( x_{\vert F } , \left ( u_{\vert F }- u_{\vert F, F^\perp} \circ \left (u_{\vert F^\perp}\right )^{-1} \circ u_{\vert F^\perp, F} \right ) \left (x_{\vert F } \right ) \right ) \right )},\\
	& = & \sqrt{ \det \left( \frac{u_{\vert F }- u_{\vert F, F^\perp} \circ \left (u_{\vert F^\perp}\right )^{-1} \circ u_{\vert F^\perp, F}}{2\pi} \right)}\int d x_{\vert F}  \, f(x_{\vert F}) \, \exp \left ( - \frac{1}{2} \left ( x_{\vert F } , \left ( u_{\vert F }- u_{\vert F, F^\perp} \circ \left (u_{\vert F^\perp}\right )^{-1} \circ u_{\vert F^\perp, F} \right ) \left (x_{\vert F } \right ) \right ) \right )
\end{eqnarray*}

\begin{aff}
	On peut noter que 

\begin{eqnarray*}
	\left ( \left (u^{-1} \right )_{\vert F} \right )^{-1}  & = & 	u_{\vert F }- u_{\vert F, F^\perp} \circ (u_{\vert F^\perp})^{-1} \circ u_{\vert F^\perp, F}	
\end{eqnarray*}

\begin{eqnarray*}
	\det (u )\det \left( \left (u^{-1} \right )_{\vert F^\perp}  \right) & = & \det \left(u_{\vert F} \right) 	\\
	\det (u )\det \left( \left (u^{-1} \right )_{\vert F}  \right) & = & \det \left(u_{\vert F^\perp} \right)	
\end{eqnarray*}

\begin{eqnarray*}
	\langle f_{\vert F } \rangle  & = & \frac{\int d x \, f_{\vert F}(x) \exp \left ( - \frac{1}{2} ( x , u (x)) \right ) }{\int d x \, \exp \left ( - \frac{1}{2} ( x , u (x)) \right )}\\
	& = & \int d x_{\vert F}  \, f(x_{\vert F}) \, \exp \left ( - \frac{1}{2} \left ( x_{\vert F } , \left ( \left ( \left (u^{-1} \right )_{\vert F} \right )^{-1}  \right ) \left (x_{\vert F } \right ) \right ) \right )/\sqrt{ \det \left( 2\pi \left ( u^{-1} \right )_{\vert F} \right)}
\end{eqnarray*}

	
\end{aff}


or pour $(x_i , x_j)\in F^2$ et $u$ inversible 	

\begin{eqnarray*}
	\int d x  \, x_i x_j \exp \left ( - \frac{1}{2} ( x , u(x)) \right ) & = &  \frac{( e_i , u^{-1}(e_j))}{\sqrt{\det \left (\frac{u}{2\pi}\right ) }} 	
\end{eqnarray*}

Donc 

\begin{eqnarray*}
	\langle x_i x_j \rangle  & = & \Big ( e_i ,\overbrace{\left ( u_{\vert F }- u_{\vert F, F^\perp} \circ (u_{\vert F^\perp})^{-1} \circ u_{\vert F^\perp, F} \right )^{-1}}^{\left (u^{-1} \right )_{\vert F}}  \left (e_j\right ) \Big )
\end{eqnarray*}


\begin{aff}
	


Or $u_{\vert F }$ est inversible donc on peut écrire que 

\begin{eqnarray*}
	   \overbrace{\left ( u_{\vert F }- u_{\vert F, F^\perp} \circ (u_{\vert F^\perp})^{-1} \circ u_{\vert F^\perp, F} \right )^{-1}}^{\left (u^{-1} \right )_{\vert F}} & = & 	 \left (u_{\vert F }\right )^{-1}  + \\
	   &   + & \left (u_{\vert F }\right )^{-1} \circ u_{\vert F , F^\perp }\circ \overbrace{ \left ( u_{\vert F^\perp } - u_{\vert F^\perp , F } \circ  \left (u_{\vert F }\right )^{-1} \circ u_{\vert F , F^\perp  } \right )^{-1} }^{ \left (u^{-1} \right )_{\vert F^\perp } }	 \circ u_{\vert F^\perp , F } \circ  \left (u_{\vert F }\right )^{-1}
\end{eqnarray*}

Donc pour $(x_i , x_j)\in F^2$ et $u$ inversible 

\begin{eqnarray*}
	\langle x_i x_j \rangle  & = & \left ( e_i ,\left (u^{-1} \right )_{\vert F}  \left (e_j\right ) \right ),\\
	& = &  \left ( e_i ,\left ( \left (u_{\vert F }\right )^{-1} +  \left (u_{\vert F }\right )^{-1} \circ u_{\vert F , F^\perp }\circ  \left (u^{-1} \right )_{\vert F^\perp } 	 \circ u_{\vert F^\perp , F } \circ  \left (u_{\vert F }\right )^{-1}\right ) \left (e_j\right ) \right  ).	
\end{eqnarray*}






\end{aff}

\begin{aff}
\begin{eqnarray*}
	\left (u^{-1} \right )_{\vert F} & = & 	 \left (u_{\vert F }\right )^{-1}  +  \left (u_{\vert F }\right )^{-1} \circ u_{\vert F , F^\perp }\circ  \left ( u_{\vert F^\perp } \right)^{-1} \circ u_{\vert F^\perp , F } \circ  \left (u_{\vert F }\right )^{-1} + \\
	& + & \left (u_{\vert F }\right )^{-1} \circ u_{\vert F , F^\perp }\circ \left (u_{\vert F^\perp }\right )^{-1} \circ u_{\vert F ^\perp, F }\circ \left (u^{-1} \right )_{\vert F} \circ u_{\vert F, F ^\perp } \circ \left (u_{\vert F^\perp }\right )^{-1} \circ  u_{\vert F ^\perp, F } \circ \left (u_{\vert F }\right )^{-1}
\end{eqnarray*}

\begin{eqnarray*}
	p_1 & = & 	\left (u_{\vert F }\right )^{-1} \circ u_{\vert F , F^\perp } , \\
	p_2 & = & \left (u_{\vert F^\perp }\right )^{-1} \circ u_{\vert F ^\perp, F }
\end{eqnarray*}

\begin{eqnarray*}
	\left (u^{-1} \right )_{\vert F}  & = & \sum_{i = 0 }^\infty (p_	1 \circ p_2)^i \circ \left ( \left (u_{\vert F }\right )^{-1} + p_1 \circ \left ( u_{\vert F^\perp } \right)^{-1}  \circ p_i^\ast \right ) \circ	((p_	1 \circ p_2)^\ast )^i + \\
	&+ & \underset{i \to \infty}{\lim}  (p_	1 \circ p_2)^i \left (u^{-1} \right )_{\vert F} ((p_	1 \circ p_2)^\ast )^i
\end{eqnarray*}


	
\end{aff}



Pour $i = j$ on prend $F_i = \mathbf{Vect} ( e_i) $ soit 

\begin{eqnarray*}
	\langle x_i^2 \rangle  & = & \left ( e_i , \left ( u_{\vert F_i }- u_{\vert F_i, F_i^\perp} \circ u_{\vert F_i^\perp}^{-1} \circ u_{\vert F_i^\perp, F_i} \right )^{-1} \left (e_i\right ) \right ) 
\end{eqnarray*}

Pour $i \neq j$ on prend $F_{ij} = \mathbf{Vect} ( e_i , e_j) $ soit 

\begin{eqnarray*}
	\langle x_i x_j \rangle  & = & \left ( e_i , \left ( u_{\vert F_{ij} }- u_{\vert F_{ij}, F_{ij}^\perp} \circ u_{\vert F_{ij}^\perp}^{-1} \circ u_{\vert F_{ij}^\perp, F_{ij}} \right )^{-1} \left (e_j\right ) \right ) 
\end{eqnarray*}


 






%\begin{eqnarray*}
	%\langle \delta \Pi(\theta) \delta \Pi(\theta') \rangle & = & \frac{ \displaystyle \int \prod_{a \vert tranche} d \delta \Pi(\theta_a) \, \delta \Pi(\theta) \delta \Pi(\theta')\, e^{-\frac{1}{2}  \underset{a \vert tranche}{\sum}  \delta \Pi(\theta_a) \operator{A}_{\theta_a, \theta_a} \delta \Pi(\theta_a) - \underset{{\underset{a<b}{a ,b \vert tranche} }}{\sum}  \delta \Pi(\theta_a) \operator{A}_{\theta_a, \theta_b} \delta \Pi(\theta_b) } }{  \displaystyle  \int  \prod_{a \vert tranche} d \delta \Pi(\theta_a)  \, e^{ -\frac{1}{2}  \underset{a \vert tranche}{\sum} \delta \Pi(\theta_a) \operator{A}_{\theta_a, \theta_a} \delta \Pi(\theta_a) - \underset{{\underset{a<b}{a ,b \vert tranche} }}{\sum}   \delta \Pi(\theta_a) \operator{A}_{\theta_a, \theta_b} \delta \Pi(\theta_b) }}	
%\end{eqnarray*}

%et si on note $F_a$ la restriction de $\mathbb{R}^{\# tranche}$ à la $a$ eme composante et $F_a^\perp$ tel que $\mathbb{R}^{\# tranche} = F_a \oplus F_a^\perp$ donc 

%\begin{eqnarray*}
	%\left( \delta \Pi , \operator{A}(\delta \Pi ) \right)  & =& \left( \delta \Pi_{F_a} + \delta \Pi_{F_a^\perp }  , \operator{A}(\delta \Pi_{F_a} + \delta \Pi_{F_a^\perp } ) \right) \\
	%& = & 	\left(\delta \Pi_{F_a}, \operator{A}(\delta \Pi_{F_a}) \right) + \left(\delta \Pi_{F_a}, \operator{A}(\delta \Pi_{F_a^\perp}) \right) + \left(\delta \Pi_{F_a^\perp}, \operator{A}(\delta \Pi_{F_a}) \right) + \left(\delta \Pi_{F_a^\perp }, \operator{A}(\delta \Pi_{F_a^\perp}) \right)
%\end{eqnarray*}

%et $\operator{A}$ est symétrique donc $(\delta \Pi_{F_a^\perp}, \operator{A}(\delta \Pi_{F_a}) ) = ( \operator{A}(\delta \Pi_{F_a}) , \delta \Pi_{F_a^\perp}) = ( \delta \Pi_{F_a} , \operator{A}(\delta \Pi_{F_a^\perp}) )$

%donc 

%\begin{eqnarray*}
	%\left( \delta \Pi , \operator{A}(\delta \Pi ) \right )  & = & 	\left (\delta \Pi_{\vert F_a}, \operator{A}(\delta \Pi_{\vert F_a}) \right ) + 2\left (\delta \Pi_{\vert F_a}, \operator{A}(\delta \Pi_{\vert F_a^\perp}) \right) + \left (\delta \Pi_{\vert F_a^\perp }, \operator{A}(\delta \Pi_{\vert F_a^\perp})  \right )
%\end{eqnarray*}


%or 

%\begin{eqnarray*}
	%\int d \delta \Pi \, \exp \left ( - \frac{1}{2} (\delta \Pi ,  \operator{A}(\delta \Pi))  +  ( B ,  \delta \Pi )  \right )  & = & \frac{ \exp \left ( \frac{1}{2}  \left ( B , \operator{A}^{-1}(B) \right ) \right ) }{ \sqrt{ \det \left( \frac{\operator{A}}{2\pi} \right )} }	
%\end{eqnarray*}

%donc 

%\begin{eqnarray*}
	%\int d \delta \Pi_{ \vert F_a^\perp  }  \, \exp \left ( - \frac{1}{2}  \left ( \delta \Pi_{ \vert F_a^\perp  } , \operator{A}( \delta \Pi_{ \vert F_a^\perp  } ) \right ) + \left ( - \operator{A}\left ( \delta \Pi_{\vert F_a} \right ) ,  \delta \Pi_{\vert F_a^\perp }\right )   \right )  & = & \frac{ \exp \left ( \frac{1}{2}  \left (\operator{A}\left ( \delta \Pi_{\vert F_a} \right ) , \operator{A}^{-1}(\operator{A}\left ( \delta \Pi_{\vert F_a} \right )) \right ) \right ) }{ \sqrt{ \det \left( \frac{\operator{A}_{\vert F_a^\perp }}{2\pi} \right )} }	\\
	%& = & \frac{ \exp \left (\left ( \delta \Pi_{\vert F_a}  ,\operator{A}\left ( \delta \Pi_{\vert F_a} \right ) \right )  \right )}{ \sqrt{ \det \left( \frac{\operator{A}_{\vert F_a^\perp }}{2\pi} \right )} } 		
%\end{eqnarray*}




%donc 

%\begin{eqnarray*}
	%\langle (\delta \Pi(\theta))^2 \rangle & = & \frac{\displaystyle \int d \delta \Pi(\theta) \, (\delta \Pi(\theta) )^2 \exp \left ( -\frac{1}{2}\delta \Pi(\theta) \operator{A}_{\theta, \theta} \delta \Pi(\theta) \right ) }{\displaystyle \int d \delta \Pi(\theta) \,  \exp \left ( -\frac{1}{2}\delta \Pi(\theta) \operator{A}_{\theta, \theta} \delta \Pi(\theta) \right ) }\\
	%& = &  \frac{\displaystyle \int d \delta \Pi(\theta) \, (\delta \Pi(\theta) )^2 \exp \left ( -\frac{1}{2}\delta \Pi(\theta) \operator{A}_{\theta, \theta} \delta \Pi(\theta) \right ) }{\displaystyle \int d \delta \Pi(\theta) \,  \exp \left ( -\frac{1}{2}\delta \Pi(\theta) \operator{A}_{\theta, \theta} \delta \Pi(\theta) \right ) }	
%\end{eqnarray*}
















