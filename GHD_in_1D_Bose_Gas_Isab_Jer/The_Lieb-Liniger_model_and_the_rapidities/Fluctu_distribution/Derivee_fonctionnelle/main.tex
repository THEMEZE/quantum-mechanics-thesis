Pour insister que l'on a concsience de ce qu'est et de ce que n'est pas une dérivé fonctionnelle, On comment par donner une définition brèf admis par la comunoté physicien, mais suffisante pour la suite. \\

Soientt $\mathcal{F}$ une fonctionnelle et un fonctions $f$. Définir la dérivée fonctionnelle de $\mathcal{F}$ par rapport de sa variable $f$, $\frac{\delta \mathcal{F}}{\delta f}$ ,  est possible s'il y a la différentiabilité au sens de Fréchet de $\mathcal{F}$ en $f$. Pour tous fonction $h$ 

\begin{eqnarray*}
	\frac{\delta \mathcal{F}}{\delta f}[h] & \doteq & D_h \mathcal{F}(f)	
\end{eqnarray*}

où $D_h \mathcal{F}$ est la dérivée directionnelle de $\mathcal{F}$ dans la direction $h$.

\begin{eqnarray*}
	D_h \mathcal{F}(f) & = & \lim_{\varepsilon \to 0 } \frac{ \mathcal{F} ( f + \varepsilon h ) - \mathcal{F} ( f) }{\varepsilon}	
\end{eqnarray*}

cette dérivée étant bien définie au point $f$ car $\mathcal{F}$ est supposé différentiable en $f$.
