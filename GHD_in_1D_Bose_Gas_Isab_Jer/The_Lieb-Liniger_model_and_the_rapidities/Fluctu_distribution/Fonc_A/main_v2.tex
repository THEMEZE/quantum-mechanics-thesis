Maintenant $\{\theta_a \}$ désigne l'ensemble de rapidité où $a$ est l'indice des tranches. La probalilité de cette configuration $\{ \pi^d(\theta_a) \}$ est : 
	
	\begin{eqnarray}
		P_{\{ \pi^d(\theta_a)\} } & = & \frac{ e^{ A[\pi^d]  }}{Z} 	
	\end{eqnarray}
	
	ou $A[\pi^d] = S_{YY}[\pi^d] - \sum_{a\vert tranche} f(\theta_a) \pi^d ( \theta) \delta \theta $. On fait l'hypothèse que la configuration $\{ \overline{\pi}_a^d \}$ domine tous les configurations (ie $\forall \{ \theta_a\}  \colon P_{\{ \pi^d(\theta_a)\} } \leq P_{\{ \overline{\pi}^d(\theta_a)\} }$) ce qui implique que  
	\begin{eqnarray*}
		\forall \theta_a  \in \{\theta_a \}\colon  \left ( \frac{\partial A}{\partial \pi^d(\theta_a) } [\overline{\pi}^d]	 = 0 , ~ \& \,  \forall \theta_b \in \{\theta_a \} \colon  \forall \pi^d \colon \frac{\partial^2 A}{\partial \pi^d(\theta_a) \partial \pi^d(\theta_b) } [\overline{\pi^d}] \leq  0 	\right ) 
	\end{eqnarray*}
	
	\begin{figure}[H]
		\centering 
		\begin{tikzpicture}
			\begin{scope}[transform canvas={scale=0.5}]
			\input{The_Lieb-Liniger_model_and_the_rapidities/Fluctu_distribution/figures/fonc_A_discr_code}	
			
			\end{scope}
			
			\draw[color = red , scale = 0.5] (-2 , -1) rectangle (5, 6) ; 
				
			
		\end{tikzpicture}	
		\captionsetup{skip=10pt} % Ajoute de l’espace après la légende
	\end{figure}
	
	On note $\pi^d = \overline{\pi}^d + \delta \pi^d $, donc le debellopement limité de A autour de $\overline{\pi}^d$ est :
	\begin{eqnarray}
		A[\pi^d]  &  =  & 	A[\overline{\pi}^d]  + \frac{1}{1!} \sum_{a \vert tranche } \left \{ A^{(1)}\right \}_{a} [\overline{\pi}^d] \delta \pi^d(\theta_a)  + \frac{1}{2!} \sum_{a , b \vert tranche } \left \{ A^{(2)}\right \}_{a, b } [\overline{\pi}^d] \delta \pi^d(\theta_a)\delta  \pi^d(\theta_b) + R[\overline{\pi}^d], 
	\end{eqnarray}
	
	avec $ \left \{ A^{(k)}\right \}_{ i_1 , \cdots ,  i_k  }  = \frac{\partial^k A }{ \partial \pi^d ( \theta_{i_1} ) \cdots \partial \pi^d ( \theta_{i_k} )  } $ et $R[\overline{\pi}^d] = \sum_{ k = 3 }^\infty \frac{1}{k!} \sum_{i_1 , \cdots , i_k } \left \{ A^{(k)}\right \}_{ i_1 , \cdots ,  i_k  } [ \overline{\pi}^d ] \delta \pi^d(\theta_{i_1}) \cdots   \delta \pi^d(\theta_{i_k})$.
	
	On se restrins au second ordre soit $e^{R[\overline{\pi}^d]} \sim 1 $ soit 
	
	\begin{eqnarray}
		P_{\{\pi^d(\theta_a)\} } & 	\sim & B e^{\displaystyle \frac{1}{2} \sum_{a , b \vert tranche } \left \{ A^{(2)}\right \}_{a, b } [\overline{\pi}^d] \delta \pi^d(\theta_a)\delta  \pi^d(\theta_b) }, 
	\end{eqnarray}
	
	avec $ B = e^{A[\overline{\pi}^d]}/Z $.
	
	On reconnais une loi normale multidimensionelle
	\footnote{
	{\em \bf Loi normale multidimensionnelle} \\
	\begin{itemize}
		\item $x \in \mathbb{R}^N$ : fonction caractéristique : $\phi_{\mu , \Sigma} = \exp \left ( i x^T \mu - \frac{1}2 x^T \Sigma x  \right ) $.
		\item cas non-dégénéré où $Sigma$ est définie positive : $f_{ \mu , \Sigma } (x) = \frac{1}{(2 \pi)^{N/2} \det ( \Sigma ) ^{1/2}} \exp \left ( - \frac{1}{2} ( x - \mu )^T \Sigma^{-1} ( x - \mu ) \right ) $
		\item théorème centrale limite fait apparaitre un variable U de Gauss centré réduite 
			$$
			\begin{array}{cccc}
				& \mathbb{E} (U) = 0, & \mathbb{E} (U^2) = 1 & p_U(u) = \frac{1}{\sqrt{2 \pi}} e^{-\frac{1}{2} u^2 } \\
				\mbox{ si $X = \sigma U + \mu $}, & \mathbb{E} (X) = \mu, & \mathbb{E} ((X-\mu)^2) = \sigma^2 & p_X(x) = \frac{1}{\sigma \sqrt{2 \pi}} e^{-\frac{(x - \mu)^2}{2 \sigma^2}  }
			\end{array}
			$$
		\item loi unitaire à plusieurs variable 
			$$
			\begin{array}{ccc}
				 \mathbb{E} (U) = 0, & \mathbb{E} (UU^T) = id  & p_U(u) = \frac{1}{(2 \pi)^{N/2} } e^{-\frac{1}{2} u u^T } \\
			\end{array}
			$$
		\item loi générale à plusieurs variables
			$$
			\begin{array}{cccc}
				 X = a U + \mu & \mathbb{E} (X) = a \mathbb{E}(U) + \mu = \mu , & \mathbb{E} ((X-\mu)(X-\mu)^T) = \mathbb{E} (aUU^T a^T) = a a^T = \Sigma & p_X(x) = \frac{1}{(2 \pi)^{N/2}  \vert \Sigma\vert^2 } e^{-\frac{1}{2} (x-\mu)^T \Sigma^{-1} (x-\mu) } \\
			\end{array}
			$$
	\end{itemize}
	
	}
	:
	
	\begin{eqnarray}
		P_{\{\pi^d(\theta_a)\} } & \sim & 	\frac{1}{(2 \pi)^{N/2} \det ( \Sigma ) ^{1/2}} \exp \left ( - \frac{1}{2} ( x - \mu )^T \Sigma^{-1} ( x - \mu ) \right )	,
	\end{eqnarray}
	avec $( x - \mu ) = (\delta \pi^d ( \theta_1) , \underset{tranche}{ \cdots} )$ , $Z \sim (2 \pi)^{N/2}  \det ( \Sigma )^{1/2} e^{A[\overline{\pi}^d]}$ et la matrice écartype :
	\begin{eqnarray}
		\Sigma & =& - \left [ A^{(2)} \right ]^{-1} 	
	\end{eqnarray}