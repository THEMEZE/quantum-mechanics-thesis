Soit  $\{\theta_a \}$ l'ensemble de rapidité où $a$ est l'indice des tranches. La probalilité de cette configuration $\{ \Pi^c(\theta_a) \}$ est : 
	
	\begin{eqnarray}
		P_{\{ \Pi^d(\theta_a)\} } & = & \frac{ \displaystyle e^{ - \mathcal{A}[\Pi^c]  }}{\mathcal{Z}[\Pi^c]} 	
	\end{eqnarray}
	
	avec l'action 
	\begin{aff}
	\begin{eqnarray}
		\mathcal{A}[g] & = &  -\mathcal{S}_{YY}[g] + \langle f , g \rangle ,	
	\end{eqnarray}	
	\end{aff}

	
	qui est une fonctionnelle, et la fonction de partition $\mathcal{Z}[g] = \sum_g e^{-\mathcal{A}[g]}$ .
	Maintenant pour simplifier on note le produit scalaire sur l'espace des fonctions de $\mathbb{R}$ dans $\mathbb{R}$ , $\langle f , g \rangle = \sum_{a\vert tranche} f(\theta_a) g ( \theta_a) \delta \theta $.\\
	
On suppose que $\mathcal{A}$ est {\em différentiable} en tout point dans un ouvert $U$ dans l'espace des fonction d'argument réelle et à valeur réelle. Alors 

\begin{eqnarray*}
	\mathcal{A}[g + h]  & \underset{ h \to 0 }{=} \mathcal{A}[g] + d\mathcal{A}_g(h) + o(\Vert h \Vert ) 		
\end{eqnarray*}

avec $g \mapsto d\mathcal{A}_g$ l'{\em application différentielle} de $\mathcal{A}$.\\

Si de plus $\mathcal{A} \colon U \subset \mathbb{R}^{\# tranche} \to \mathbb{R}$, alors pour $g \in U $ , $\displaystyle \frac{ \partial \mathcal{A}}{ \partial x_a}(g)$ exite pour tous $a$ indice des tranches, et on a   
	
\begin{eqnarray*}
	d\mathcal{A}_g & = & \sum_{a\vert tranche} \frac{ \partial \mathcal{A}}{ \partial x_a}(g) d x_a ,		
\end{eqnarray*}

où $(d x_a )$ est la base duale dans $(\mathbb{R}^{\# tranche})^\ast$ de la base canonique de $\mathbb{R}^{\# tranche}$.\\

De plus si $\mathcal{A}$ est de classe $\mathcal{C}^k$ alors on applique la {\em formule de Taylor-Youg} ,

\begin{eqnarray*}
	\mathcal{A}[g + h]  & \underset{ h \to 0 }{=} & \mathcal{A}[g] +  \left[ \sum_{a\vert tranche} h_a \frac{ \partial \mathcal{A}}{ \partial x_a}(g) \right ] +  \frac{1}{2!} \left[ \sum_{a\vert tranche} h_a \frac{ \partial \mathcal{A}}{ \partial x_a}(g) \right ]^{[2]} +  \underbrace{\sum_{j = 3}^k \frac{1}{j!} \left[ \sum_{a\vert tranche} h_a \frac{ \partial \mathcal{A}}{ \partial x_a}(g) \right ]^{[j]}   + o(\Vert h \Vert^k )}_{o(\Vert h \Vert^2)} , 		
\end{eqnarray*}

avec pour $n \in \llbracket 1 , k \rrbracket$

\begin{eqnarray*}
	\left [ \sum_{a = 1}^{\# tranche} h_a \frac{ \partial \mathcal{A}}{ \partial x_a}(g) \right ]^{[n]} & = & \sum_{a_1 + \cdots + a_{\# tranche} = n } \frac{ n!}{a_1 ! \cdots a_{\# tranche} !} h_1^{a_1} \cdots h_{\# tranche}^{a_{\# tranche}} \frac{\partial^n  \mathcal{A}}{ \partial x_1^{a_1} \cdots \partial x_{\# tranche}^{a_{\# tranche}}} (g) 
\end{eqnarray*}

\begin{aff}
On se restrint à $\mathcal{A}$ est de classe $\mathcal{C}^2$ et on admet qu'il existe $g \in U $ tel que $d\mathcal{A}_g = 0$, de sorte que d'après la formule de Taylor-Youg 

\begin{eqnarray*}
	\mathcal{A}[g + h]  & \underset{ h \to 0 }{=} & \mathcal{A}[g]  +  \mathcal{Q}(h)/2  +  o(\Vert h \Vert^2) , 		
\end{eqnarray*}

où $\mathcal{Q}$ est la forme quadratique 

\begin{eqnarray*}
	 \mathcal{Q}(h) & = & \left[ \sum_{a\vert tranche} h_a \frac{ \partial \mathcal{A}}{ \partial x_a}(g) \right ]^{[2]} ~=~ \sum_{a,b\vert tranche} h_a  \frac{\partial^2 \mathcal{A}}{\partial x_a \partial x_b}(g) h_b~=~\sum_{a\vert tranche} h_a^2 \frac{\partial^2 \mathcal{A}}{{\partial x_a}^2} (g) + 2 \sum_{\underset{a<b}{a,b\vert tranche}} h_a h_b \frac{\partial^2 \mathcal{A}}{\partial x_a \partial x_b} (g)	
\end{eqnarray*}
\end{aff}

Alors 

\begin{enumerate}
	\item si $\mathcal{A}$ admet un minimum (resp. un maximum) relatif en $g$, $ \mathcal{Q}$ est une forme quadratique positive (resp. négative).
	\item si $ \mathcal{Q}$ est une forme quadratique définie positive (resp. définie négative), $\mathcal{A}$ admet une minimum (resp. un maximum) relatif en $g$.
\end{enumerate}

\begin{Rema}
\begin{itemize}
	\item Si la forme quadratique $ \mathcal{Q}$ est seulement supposée positive (et non pas définie positive), on n'est pas assuré d'avoir un minimum relatif en $g$. Par exemple, pour la fonction $\mathcal{A} \colon \mathbb{R} \to \mathbb{R} ~x \mapsto x^3 $ en $g=0$, $\mathcal{Q} = 0 $ est positive (elle est nulle) et pourtant, $\mathcal{A}$ n'admet pas d'extemum en $0$.
	\item Dire que la forme quadratique $ \mathcal{Q}$ est positive (resp. définie positive), c'est dire que la matrice symétrique $\operator{A} = \left \{ \mathcal{A}^{(2)} (g)\right \}_{a,b}$ avec  $\left \{ \mathcal{A}^{(2)} (g)\right \}_{a,b} = \left (\frac{\partial^2 \mathcal{A}}{\partial x_a \partial x_b}(g) \right )_{a,b}$ est positive (resp. définie positive), ou encore que les valeurs propres de $\operator{A}$ sont $\geq 0$ (resp. $>0$). La matrice $\mathbf{A}$ est appelée {\em matrice hessienne} de $\operator{A}$ au point critique $g$.\\
\end{itemize}
\end{Rema}

On fait l'hypothèse que la configuration $\{ \Pi^c(\theta_a) \}$ du col domine tous les configurations. On note %$\Pi = \Pi^c + \delta \Pi $

\begin{eqnarray*}
	\Pi  & =  &\Pi^c + \delta \Pi,
\end{eqnarray*}

tel que $\mathcal{A}$  admet un minimum en $\Pi^c$ et $\mathcal{Q}$ est définie positif.







%dans son voisinage (ie $\forall \{ \Pi^d(\theta_a)\} \in \mathcal{V}(\{ \overline{\Pi}^d(\theta_a)\})  \colon P_{\{ \Pi^d(\theta_a)\} } \leq P_{\{ \overline{\Pi}^d(\theta_a)\} }$) ce qui implique que  
	%\begin{eqnarray*}
		% \frac{\partial \mathcal{A}}{\partial \Pi^d(\theta_a) } [\overline{\Pi}^d]	 = 0 & \& & \forall (\{ \Pi^d(\theta_a)\},\{ \Pi^d(\theta_b)\})  \in \mathcal{V}^2(\{ \overline{\Pi}^d(\theta_a)\})  \colon  \frac{\partial^2 \mathcal{A}}{\partial \Pi^d(\theta_a) \partial \Pi^d(\theta_b) } [\overline{\Pi^d}] \leq  0  
	%\end{eqnarray*}
	
	\begin{figure}[H]
		\centering 
		\begin{tikzpicture}
			\begin{scope}[transform canvas={scale=0.6}]
			% Définition des couleurs avec les codes HTML
\definecolor{colorOne}{HTML}{443E46}
\definecolor{colorTwo}{HTML}{F6DEB8}
\definecolor{colorThree}{HTML}{908CA4}
\definecolor{colorFour}{HTML}{57659E}
\definecolor{colorFive}{HTML}{C57284}
\definecolor{colorSix}{HTML}{FF5B69}

% Raccourcis pour les couleurs
\def\colorOne{colorOne}
\def\colorTwo{colorTwo}
\def\colorThree{colorThree}
\def\colorFour{colorFour}
\def\colorFive{colorFive}
\def\colorSix{colorSix}

\def\colorslide{blue!50!black}



\begin{scope}
	% Tracer une courbe lisse entre des points
	\draw[shift={(0,0)} ,\colorOne]
		(-1 , 0 ) edge [thick,line width=0.8ex , ->,>=triangle 45  , \colorOne]( 5  , 0 )
	;
	\draw[shift={(0,0)}, color=\colorOne]
		(0, -1.0 ) edge [thick,line width=0.8ex , ->,>=triangle 45  ]node [pos=0.9,left ]{\huge$\mathcal{A}[\Pi]$}( 0  , 5 )
	;
	\draw[]
		(2.5, 0.12 ) edge [thick,line width=0.8ex ,\colorThree ]node [pos=1,below  ]{\huge$\{ \Pi^c(\theta_a) \}$} (2.5, -0.12 )	
	;
	
	\draw[]
		(2.5, -0.12 ) edge [thick,line width=0.4ex , dashed, \colorThree ] (2.5, 5.5 )
		(1.5, 1 ) edge [thick,line width=0.4ex , <->,>=triangle 45  , \colorThree ] (3.5, 1 )	
	;
    \draw[thick, line width=0.8ex , \colorFour] plot[smooth, tension=0.7] coordinates {
        (1, 5) (1.6 , 3 ) (2.5, 1) (3.5 , 3 )  (4, 5)
    };		
	
\end{scope}

	
			
			\end{scope}
			
			\draw[color = red , scale = 0.5 , draw = none ] (-2 , -1) rectangle (5, 6) ; 
				
			
		\end{tikzpicture}	
		\captionsetup{skip=10pt} % Ajoute de l’espace après la légende
	\end{figure}
	
	%On note $\Pi^d = \overline{\Pi}^d + \delta \Pi^d $, donc le debellopement limité de $\mathcal{A}$ autour de $\overline{\Pi}^d$ est :
	%\begin{eqnarray}
		%\mathcal{A}[\Pi^d]  &  =  &  \sum_{ k = 0 }^\infty \frac{1}{k!} \sum_{i_1 , \cdots , i_k } \left \{ \mathcal{A}^{(k)}\right \}_{ i_1 , \cdots ,  i_k  } [ \overline{\Pi}^d ] \delta \Pi^d(\theta_{i_1}) \cdots   \delta \Pi^d(\theta_{i_k}) 	\\
		%& = & \mathcal{A}[\overline{\Pi}^d]  + \frac{1}{1!} \sum_{a \vert tranche } \left \{ \mathcal{A}^{(1)}\right \}_{a} [\overline{\Pi}^d] \delta \Pi^d(\theta_a)  + \frac{1}{2!} \sum_{a , b \vert tranche } \left \{ \mathcal{A}^{(2)}\right \}_{a, b } [\overline{\Pi}^d] \delta \Pi^d(\theta_a)\delta  \Pi^d(\theta_b) + R[\overline{\Pi}^d], 
	%\end{eqnarray}
	
	%avec $ \left \{ \mathcal{A}^{(k)}\right \}_{ i_1 , \cdots ,  i_k  }  = \frac{\partial^k \mathcal{A} }{ \partial \Pi^d ( \theta_{i_1} ) \cdots \partial \Pi^d ( \theta_{i_k} )  } $ et $R[\overline{\Pi}^d] = \sum_{ k = 3 }^\infty \frac{1}{k!} \sum_{i_1 , \cdots , i_k } \left \{ \mathcal{A}^{(k)}\right \}_{ i_1 , \cdots ,  i_k  } [ \overline{\Pi}^d ] \delta \Pi^d(\theta_{i_1}) \cdots   \delta \Pi^d(\theta_{i_k})$.
	
	%On se restrins au second ordre soit $e^{R[\overline{\Pi}^d]} \sim 1 $ soit 
	
On simplifie les notation avec 

Pour simplifier les notations et généraliser, on note les fonction $g \equiv \Pi$. pour calculer la contribution dominante du col $g^c$, nous changeons de variable :


\begin{eqnarray*}
	g & = & g^c + h 	
\end{eqnarray*}

avec les fonction   $g^c \equiv \Pi^c$ et $h \equiv \delta \Pi$. Et les vecteurs dans $\mathbb{R}^{\# tranche}$

\begin{eqnarray*}
	\vect{g} & = & \vect{g}^c + \vect{h} 	
\end{eqnarray*}

avec les fonction $\vect{g} \equiv (\Pi(\theta_1), \cdots , \Pi(\theta_a) , \cdots \Pi(\theta_{\# tranche} )) $ ,  $\vect{g}^c \equiv (\Pi^c(\theta_1), \cdots , \Pi^c(\theta_a) , \cdots \Pi^c(\theta_{\# tranche} ))$ et $\vect{h} \equiv  (\delta\Pi(\theta_1), \cdots , \delta\Pi(\theta_a) , \cdots \delta\Pi(\theta_{\# tranche} ))$.\\
	
On definie la {\em matrice hessienne} de $\mathcal{A}$ au point critique

\begin{eqnarray*}
	\operator{A}  & = & \left \{ \mathcal{A}^{(2)} (g^c)\right \}_{a,b}
\end{eqnarray*} 

 avec  $\left \{ \mathcal{A}^{(2)} (g)\right \}_{a,b} = \left (\frac{\partial^2 \mathcal{A}}{\partial x_a \partial x_b}(g^c) \right )_{a,b}$.

	
\begin{eqnarray}
	P_{\{ \Pi(\theta_a) \}} & 	\sim & B e^{\displaystyle -\frac{1}{2} \vect{h}^t \operator{A} \vect{h}}, 
\end{eqnarray}
	
avec $ B = e^{-\mathcal{A}[ \Pi^c]}/Z $.
	
	On avoue ici loi normale multidimensionelle
	\footnote{
	{\em \bf Loi normale multidimensionnelle} \\
	\begin{itemize}
		\item $x \in \mathbb{R}^N$ : fonction caractéristique : $\phi_{\mu , \Sigma} = \exp \left ( i x^T \mu - \frac{1}2 x^T \Sigma x  \right ) $.
		\item cas non-dégénéré où $Sigma$ est définie positive : $f_{ \mu , \Sigma } (x) = \frac{1}{(2 \pi)^{N/2} \det ( \Sigma ) ^{1/2}} \exp \left ( - \frac{1}{2} ( x - \mu )^T \Sigma^{-1} ( x - \mu ) \right ) $
		\item théorème centrale limite fait apparaitre un variable U de Gauss centré réduite 
			$$
			\begin{array}{cccc}
				& \mathbb{E} (U) = 0, & \mathbb{E} (U^2) = 1 & p_U(u) = \frac{1}{\sqrt{2 \pi}} e^{-\frac{1}{2} u^2 } \\
				\mbox{ si $X = \sigma U + \mu $}, & \mathbb{E} (X) = \mu, & \mathbb{E} ((X-\mu)^2) = \sigma^2 & p_X(x) = \frac{1}{\sigma \sqrt{2 \pi}} e^{-\frac{(x - \mu)^2}{2 \sigma^2}  }
			\end{array}
			$$
		\item loi unitaire à plusieurs variable 
			$$
			\begin{array}{ccc}
				 \mathbb{E} (U) = 0, & \mathbb{E} (UU^T) = id  & p_U(u) = \frac{1}{(2 \pi)^{N/2} } e^{-\frac{1}{2} u u^T } \\
			\end{array}
			$$
		\item loi générale à plusieurs variables
			$$
			\begin{array}{cccc}
				 X = a U + \mu & \mathbb{E} (X) = a \mathbb{E}(U) + \mu = \mu , & \mathbb{E} ((X-\mu)(X-\mu)^T) = \mathbb{E} (aUU^T a^T) = a a^T = \Sigma & p_X(x) = \frac{1}{(2 \pi)^{N/2}  \vert \Sigma\vert^2 } e^{-\frac{1}{2} (x-\mu)^T \Sigma^{-1} (x-\mu) } \\
			\end{array}
			$$
	\end{itemize}
	
	}
	%:
	
	%\begin{eqnarray}
		%P_{\{\Pi^d(\theta_a)\} } & \sim & 	\frac{1}{(2 \pi)^{N/2} \det ( \Sigma ) ^{1/2}} \exp \left ( - \frac{1}{2} ( x - \mu )^T \Sigma^{-1} ( x - \mu ) \right )	,
	%\end{eqnarray}
	%avec $( x - \mu ) = (\delta \pi^d ( \theta_1) , \underset{tranche}{ \cdots} )$ , $Z \sim (2 \pi)^{N/2}  \det ( \Sigma )^{1/2} e^{-\mathcal{A}[\overline{\Pi}^d]}$ et la matrice écartype :
avec la fonction de partition 

\begin{eqnarray}
	Z &\sim  &(2 \pi)^{\# tranche/2}  \det ( \Sigma )^{1/2} e^{-\mathcal{A}[\Pi^c]}		
\end{eqnarray}

\begin{aff}
avec la matrice écartype 

\begin{eqnarray}
	\Sigma & =& \operator{A}^{-1} 	
\end{eqnarray}	
\end{aff}

	
	{\em Pourquoi une loi normal ? Notament pourquoi $Z \sim (2 \pi)^{\# tranche/2}  \det (\operator{A} )^{-1/2} e^{-\mathcal{A}[\Pi^c]}$ ?} On se pence plus sur la fonction de partition .