

	Dans la sous-section précédente, nous avons expliqué comment les observables physiques, telles que les valeurs d'attente des charges et des courants, deviennent des fonctionnelles de la distribution de rapidité $\rho(\theta)$ dans la limite thermodynamique. Cependant, nous n'avons pas expliqué comment construire des distributions de rapidité physiquement significatives (sauf pour l'état fondamental de l'hamiltonien de Lieb-Liniger, pour lequel $\nu(\theta)$) est une fonction rectangulaire, voir la sous-section (??)). {\em Par exemple, quelle est la distribution de rapidité correspondant à un état d'équilibre thermique à une température non nulle ?}\\
	
	La question a été répondue dans les travaux pionniers de Yang et Yang (1969), que nous allons maintenant examiner brièvement. Tout d'abord, nous observons qu'il existe de nombreuses choix différents de séquences d'états propres $(\{\theta_a\}_{ a \in \llbracket 1 , N \rrbracket} )_{ N \in \mathbb{Z}}$ qui conduisent à la même distribution de rapidité thermodynamique (??). La description du système en termes de distribution de rapidité $\rho ( \theta ) $ est seulement une description grossière : on devrait considérer la distribution de rapidité $\rho ( \theta ) $ comme caractérisant un macro-état du système, correspondant à un très grand nombre de micro-états possibles $\vert \{ \theta_a \} \rangle $. {\em Pour faire de la thermodynamique, il faut estimer le nombre de ces micro-états.}\\
	
	Pour estimer ce nombre, on se concentre sur une petite cellule de rapidité $[\theta, \theta+\delta\theta]$, qui contient $L\rho(\theta)\delta \theta$ rapidités. Les équations de Bethe (??) relient ces rapidités aux moments de fermions pa dans une cellule de moment $[p, p+\delta p]$, où $\delta p/\delta \theta$ est d'environ $2\pi \rho_s(\theta)$, voir l'équation (??). Il est important de noter que les moments de fermions pa sont soumis au principe d'exclusion de Pauli. Le nombre de micro-états est alors évalué en comptant le nombre de configurations de moments de fermions mutuellement distincts, équivalent à $L\rho (\theta)\delta \theta$, qui peuvent être placées dans la boîte $[p, p + \delta p]$. Comme l'espacement minimal entre deux moments est de $2\pi /L$, la réponse est

	\begin{eqnarray}
		\# \mbox{conf.} & \approx  & \frac{[ L \rho_s ( \theta ) \delta \theta ] ! }{ [ L \rho ( \theta ) \delta \theta ] ! [ L ( \rho_s ( \theta ) - \rho ( \theta ) )  \delta \theta ] ! } , 	
	\end{eqnarray}
	
	or 
	\begin{eqnarray}
		n! & \underset{n \to \infty}{\sim} n^n e^{-n} \sqrt{2\pi n}. 
	\end{eqnarray}

	\begin{eqnarray}
		\ln n! & \underset{n \to \infty}{\sim} & n \ln n \underbrace{- n + \ln \sqrt{2 \pi n }}_{o \left ( n \ln n \right ) } ,\\
		&  \underset{n \to \infty}{\sim} & n \ln n  
	\end{eqnarray}

	\begin{eqnarray}
		\ln \# \mbox{conf.} & \underset{n \to \infty}{\sim}   & L \left \{ \rho_s (\theta ) \ln [ L \rho_s ( \theta ) \delta \theta ] -  \rho (\theta ) \ln [ L \rho ( \theta ) \delta \theta ] - ( \rho_s (\theta ) - \rho ( \theta ) )  \ln [ L  ( \rho_s ( \theta ) - \rho ( \theta ) )  \delta \theta ] \right \} \delta \theta ,\\
		& \underset{n \to \infty}{\sim} &  L \left \{ \rho_s (\theta ) \ln \rho_s ( \theta )   -  \rho (\theta ) \ln \rho ( \theta ) - ( \rho_s (\theta ) - \rho ( \theta ) )  \ln   ( \rho_s ( \theta ) - \rho ( \theta ) )   \right \} \delta \theta ,	
	\end{eqnarray}
	
	Le nombre total de micro-états est le produit de toutes ces configurations pour toutes les cellules de rapidité $[\theta, \theta + \delta \theta]$. En prenant le logarithme et en remplaçant la somme par une intégrale sur $d \theta $, nous obtenons l'entropie de Yang-Yang.
	\begin{eqnarray}
		\ln \# \mbox{microstates.} & = & \int \ln \# \mbox{conf.},\\
	 	& \approx  &  L S_{YY} [ \rho ] , 	
	\end{eqnarray}
	
	\begin{eqnarray}
		S_{YY}[\rho] & \doteq & \int  \left ( \rho_s (\theta ) \ln \rho_s ( \theta )   -  \rho (\theta ) \ln \rho ( \theta ) - ( \rho_s (\theta ) - \rho ( \theta ) )  \ln   ( \rho_s ( \theta ) - \rho ( \theta ) )	\right ) d \theta 
	\end{eqnarray}
	
	
	
	
	La matrice densité thermique est :
	\begin{eqnarray}
		\hat{\rho}_{thermal} & = & \frac{e^{-H/T}}{Z_{thermal}}, \\
		e^{-H/T} & = & 	\sum_{\vert \theta_a \rangle} e^{- \sum_a ( \varepsilon(\theta_a)- \mu ) /T } \vert \{ \theta_a\} \rangle \langle  \{ \theta_a\}  \vert 
	\end{eqnarray}
	
	La matrice densité GGE est :
	\begin{eqnarray}
		\hat{\rho}_{GGE}[f] & = & \frac{e^{-Q[f]}}{Z_{GGE}}, \\
		e^{-Q[f]} & = & 	\sum_{\vert \theta_a \rangle} e^{- \sum_a f(\theta_a) } \vert \{ \theta_a\} \rangle \langle  \{ \theta_a\}  \vert 
	\end{eqnarray}
	
	pour une certaine fonction f. Nous aimerions calculer les valeurs d'attente par rapport à cette matrice de densité, par exemple
	La moyenne GGE d'un observable s'écrit ,
	
	\begin{eqnarray}
		\langle \mathcal{O} \rangle_{GGE} & \doteq & \frac{\text{Tr} (\mathcal{O}\hat{\rho}_{GGE}[f])}{\text{Tr} (\hat{\rho}_{GGE}[f])} = \frac{\text{Tr} (\mathcal{O}e^{-Q[f]})}{\text{Tr} (e^{-Q[f]})}	 = \frac{\sum_{\vert \theta_a \rangle} \langle  \{ \theta_a\}  \vert   \mathcal{O} \vert \{ \theta_a\} \rangle e^{- \sum_a f(\theta_a) }  }{\sum_{\vert \theta_a \rangle} e^{- \sum_a f(\theta_a) } }
	\end{eqnarray}
	
	pour une certaine observable $\mathcal{O}$. Lorsque l'observable O est suffisamment local, on croit que la valeur d'attente $\langle  \{ \theta_a\}  \vert   \mathcal{O} \vert \{ \theta_a\} \rangle$ ne dépend pas de l'état microscopique spécifique du système, de sorte qu'elle devient une fonctionnelle de $\rho$ dans la limite thermodynamique.
	\begin{eqnarray}
		\underset{\mbox{\tiny therm.}}{\lim} \langle  \{ \theta_a\}  \vert   \mathcal{O} \vert \{ \theta_a\} \rangle & = & \langle \mathcal{O}\rangle_{[\rho]}
	\end{eqnarray}
	
	Cette hypothèse est liée à une "Hypothèse de thermalisation généralisée des états propres", voir par exemple (Cassidy et al. 2011, Dymarsky and Pavlenko 2019, He et al. 2013, Pozsgay 2011, 2014, Vidmar and Rigol 2016). Sous cette hypothèse, on peut remplacer la somme précédente sur tous les états propres par une intégrale fonctionnelle sur la distribution des rapides à grains grossiers $\rho$,
	\begin{eqnarray}
		\underset{\mbox{\tiny therm.}}{\lim} \langle \mathcal{O}\rangle_{GGE} & = & \frac{\int \mathcal{D} \rho \langle \mathcal{O}\rangle_{[\rho]} \, \#\mbox{microstates.}\,  e^{-L\int f ( \theta ) \rho ( \theta ) d \theta } }{\int \mathcal{D} \rho  \, \#\mbox{microstates.}\,  e^{-L\int f ( \theta ) \rho ( \theta ) d \theta }},\\
		& = & \frac{\int \mathcal{D} \rho \langle \mathcal{O}\rangle_{[\rho]} \,   e^{L ( S_{YY}[\rho] -  \int f ( \theta ) \rho ( \theta ) d \theta) }}{\int \mathcal{D} \rho  \,  e^{L ( S_{YY}[\rho] -  \int f ( \theta ) \rho ( \theta ) d \theta)  }},
	\end{eqnarray}
	
	L'intégrale fonctionnelle est alors dominée par la distribution des racines qui minimise une fonctionnelle d'énergie libre (généralisée) :
	
	\begin{eqnarray}
		\frac{\delta}{\delta \rho}	\underset{\mbox{\tiny therm.}}{\lim} \langle \mathcal{O}\rangle_{GGE} = 0 & \Leftarrow & \frac{\delta}{\delta \rho} \left [ \int  f(\theta) \rho(\theta )  d\theta - S_{YY} [\rho] \right ] = 0  
	\end{eqnarray}
	
	En utilisant la définition de l'entropie de Yang-Yang et l'équation constitutive (??), on obtient la relation suivante entre la fonction $f(\theta)$ définissant la matrice de densité diagonale (??) et la distribution de rapidité $\rho (\theta)$ qui domine l'intégrale fonctionnelle (??).\\
	
	\begin{eqnarray}
		\frac{\delta}{\delta \rho}	\underset{\mbox{\tiny therm.}}{\lim} \langle \mathcal{O}\rangle_{GGE} = 0 & \Leftarrow &   	f(\theta) \rho(\theta )  -  \rho_s (\theta ) \ln \rho_s ( \theta ) +  \rho (\theta ) \ln \rho ( \theta ) + ( \rho_s (\theta ) - \rho ( \theta ) )  \ln   ( \rho_s ( \theta ) - \rho ( \theta ) ) = <0	
	\end{eqnarray}

	
	On rappel que le scattering shift $\Delta(\theta ) $ est donné par
	\begin{eqnarray*}
		\Delta( \theta ) & \doteq & \frac{d \phi }{d \theta } ( \theta ) = \frac{ 2c} { c^2 + \theta^2}	
	\end{eqnarray*}
	et que 
	\begin{eqnarray}
		\theta_a + \frac{1}{L} \sum_{b \neq a} 2 \arctan \left( \frac{\theta_a - \theta_b }{c} \right ) = p_a \quad \mbox{où}\quad \left \{\begin{array}{rclll} p_a & \in & \frac{2\pi}{L} \mathbb{Z} & \mbox{pour} & N \in 2\mathbb{N} + 1 ~\mbox{\em (odd)} \\ p_a & \in & \frac{2\pi}{L} \left ( \mathbb{Z}+ \frac{1}{2} \right ) & \mbox{pour} & N \in \mathbb{N} ~\mbox{\em(even)} \end{array} \right. 	
	\end{eqnarray}
	
	\begin{eqnarray}
		\frac{\delta \rho_s }{ \delta \rho } ( \theta ) & = & 	\int  \frac{d \theta ' }{ 2 \pi } \Delta ( \theta - \theta' ), \\
		\frac{ \delta S_{YY} }{ \delta \rho } [\rho ]  & = & \int  d\theta\left \{  \frac{\delta \rho_s }{ \delta \rho } \left ( \ln  \rho_s + {\color{blue} \cancel{ \color{black}1}}  \right ) - \left (  \ln \rho + {\color{blue} \cancel{ \color{black}1}} ) \right )  -  \left ( \frac{\delta \rho_s }{ \delta \rho } - 1 \right ) \left ( \ln ( \rho_s -\rho ) +  {\color{blue} \cancel{ \color{black} 1}}  \right ) \right  \} ( \theta )  ,\\
		& = & \int  d\theta \left \{ \ln \left ( \frac{ \rho_s}{ \rho} - 1 \right ) - \frac{\delta \rho_s }{ \delta \rho } \left ( \ln \left ( 1 - \frac{ \rho}{ \rho_s}\right ) \right ) \right \} ( \theta ) , \\
	\end{eqnarray}
	
	en injectant $\frac{\delta \rho_s }{ \delta \rho }$,
	
	\begin{eqnarray}
		\frac{ \delta S_{YY} }{ \delta \rho } [\rho ]  & = &\int  d\theta\left \{ \ln \left ( \frac{ \rho_s}{ \rho} - 1 \right ) - \int  \frac{d \theta ' }{ 2 \pi } \Delta ( \cdot  - \theta' ) \left ( \ln \left ( 1 - \frac{ \rho}{ \rho_s}\right ) \right ) \right \} ( \theta ) , \\  
		& = &\int  d\theta\left \{ \ln \left ( \frac{ \rho_s}{ \rho} - 1 \right )  \right \} ( \theta ) - \int  d\theta\left \{ \int  \frac{d \theta ' }{ 2 \pi } \Delta ( \cdot  - \theta' ) \left ( \ln \left ( 1 - \frac{ \rho}{ \rho_s}\right ) \right ) \right \} ( \theta ) , \\ 
		& = &\int  d\theta\left \{ \ln \left ( \frac{ \rho_s}{ \rho} - 1 \right )  \right \} ( \theta ) -  \int  d\theta' \int  \frac{d\theta}{2 \pi}  \left \{  \Delta ( \cdot  - \theta' ) \left ( \ln \left ( 1 - \frac{ \rho}{ \rho_s}\right ) \right ) \right \} ( \theta ) ,		
	\end{eqnarray}
	
	$\theta$ et $\theta'$ sont des variables muettes. On les echange dans la deuxième intégrale.
	
	\begin{eqnarray}
		\frac{ \delta S_{YY} }{ \delta \rho } [\rho ]  & = &\int  d\theta\left \{ \ln \left ( \frac{ \rho_s}{ \rho} - 1 \right )  \right \} ( \theta ) -  \int  d\theta    \int  \frac{d\theta'}{2 \pi}  \Delta (   \theta '  - \theta ) \left ( \ln \left ( 1 - \frac{ \rho  ( \theta ')}{ \rho_s  ( \theta ')}\right ) \right )  ,		
	\end{eqnarray}
	or $\delta$ est une fonction symétrique donc, 
	
	\begin{eqnarray}
		\frac{ \delta S_{YY} }{ \delta \rho } [\rho ]  & = & \int  d\theta\left \{ \ln \left ( \frac{ \rho_s}{ \rho} - 1 \right )  \right \} ( \theta ) -  \int  d\theta    \int  \frac{d\theta'}{2 \pi}  \Delta (   \theta   - \theta' ) \left ( \ln \left ( 1 - \frac{ \rho  ( \theta ')}{ \rho_s  ( \theta ')}\right ) \right )  ,	\\
		& = & \int  d\theta\left \{ \ln \left ( \frac{ \rho_s}{ \rho} - 1 \right ) - \frac{1}{2 \pi} \Delta \underset{ \tiny \theta }{\star} \ln \left ( 1 - \frac{ \rho}{ \rho_s}\right )  \right \} (\theta ) 		
	\end{eqnarray}
	
	où $\{ g \underset{ \tiny x }{\star} f \} ( x ) = \int d x g ( x - x') f ( x' ) $ pour simplifier les notation  $\underset{ \tiny x }{\star}\equiv  \star $ 
	
	Ainsi 
	
	\begin{eqnarray}
		\frac{\delta}{\delta \rho} \left [ \int  f(\theta) \rho(\theta )  d\theta - S_{YY} [\rho] \right ] & = & 	\int  d\theta\left \{ f - \ln \left ( \frac{ \rho_s}{ \rho} - 1 \right ) + \frac{1}{2 \pi} \Delta \star \ln \left ( 1 - \frac{ \rho}{ \rho_s}\right )  \right \} (\theta )
	\end{eqnarray}
	
	donc 
	
	\begin{eqnarray} 
		\frac{\delta}{\delta \rho}	\underset{\mbox{\tiny therm.}}{\lim} \langle \mathcal{O}\rangle_{GGE} = 0 & \Leftarrow & f = \ln \left ( \frac{ \rho_s}{ \rho} - 1 \right ) - \frac{1}{2 \pi} \Delta \star \ln \left ( 1 - \frac{ \rho}{ \rho_s}\right )		
	\end{eqnarray}
	
	Mais est-ce la seule solution ? 
 
	
	{\color{red} $\cdots$}\\
	
\subsubsection{Calcule de Bess}

On introduit $\rho_h$ tel que $\rho_s = \rho + \rho_h$. La thermodynamique de Yang-Yang nous donne 

\begin{eqnarray}
	\frac{\rho_h}{\rho} & = & e^{\beta \epsilon }, \\
	\nu \doteq \frac{\rho}{\rho_s} & = & \frac{1}{1+ e ^{\beta \epsilon} }	
\end{eqnarray}

Où la minimisation de l'énergie libre [Yang-Yang] donne :
\begin{eqnarray}
	\epsilon ( \theta ) & = & - \mu + \frac{\hbar^2 \theta^2 }{2 m} - \frac{1}{ \beta } \frac{1}{2 \pi} \left \{\Delta \star \ln \left ( 1 + e^{\beta \epsilon} \right )  \right \} ( \theta ), \\
	\rho &=&  \rho_s \nu   ~=~   \frac{\nu}{2 \pi } ( 1 + \{ \Delta \star \rho \} ) 	
\end{eqnarray}

\footnote{$$ 	2\pi \rho_s  =  1 + \{ \Delta \star \rho \} ~= ~ 1 + \{ \Delta \star \{\nu \rho_s\} \} $$}

Rappel des constantes :

\begin{eqnarray*}
	\beta & = & \frac{1}{ k_B T} ,\\
	\tilde{c} & =& \frac{1}{l_g}  = \frac{ m g_{1D}}{\hbar^2},\\
	t & = & \frac{1}{\beta E_g} = \frac{1}{ \beta} \frac{2 \hbar^2 }	{ m g_{1D}^2 } = \frac{1}{ \beta  \tilde{c} g_{1D} },\\
	\gamma & = & \frac{1}{n l_g } = \frac{1}{n} \frac{ m g_{1D}}{\hbar^2} = \frac{\tilde{c}}{n} 
\end{eqnarray*}

On fait les changement de variables :

\begin{eqnarray*}
	\tilde{\theta} & = & \frac{\theta}{\tilde {c}},\\
	\tilde{\mu} & =& \beta \mu ,\\
	\tilde{\epsilon} & = & \beta \epsilon , \\
	\tilde{\rho}(\tilde{\theta}) & = & \rho ( \theta ) 	
\end{eqnarray*}

soit 

\begin{eqnarray*}
	\frac{\tilde{\theta}^2}{t}	 & = & \beta \tilde{c} g_{1D} \frac{1}{\tilde{c}} \frac{\hbar^2}{m g_{1D}} \theta^2 = \beta \frac{ \hbar^2 \theta^2}{2 m}.
\end{eqnarray*}

et 

\begin{eqnarray*}
	\Delta ( \theta - \theta' ) d\theta' = \frac{2 \tilde{c}}{ \tilde{c}^2 + ( \theta - \theta')^2 } d \theta' = \frac{2}{ 1 + ( \tilde{\theta} - \tilde{\theta}')^2 } d\tilde{\theta}' = \tilde{\Delta} ( \tilde{\theta} - \tilde{\theta}') d \tilde{\theta}' 	
\end{eqnarray*}

\begin{eqnarray}
	\frac{\tilde{\rho_h}}{\tilde{\rho}} & = & e^{\tilde{\epsilon} }, \\
	\tilde{\nu} \doteq \frac{\tilde{\rho}}{\tilde{\rho_s}} & = & \frac{1}{1+ e ^{\tilde{\epsilon}} }	
\end{eqnarray}

Où la minimisation de l'énergie libre [Yang-Yang] donne :
\begin{eqnarray}
	\tilde{\epsilon} ( \tilde{\theta} ) & = & - \tilde{\mu} + \frac{\tilde{\theta}^2 }{t} - \frac{1}{2 \pi} \left \{\tilde{\Delta} \star \ln \left ( 1 + e^{\tilde{\epsilon}} \right )  \right \} ( \tilde{\theta} ), \label{eq:1Bess}\\
	\tilde{\rho} &=& \tilde{\rho_s} \tilde{\nu}   ~=~   \frac{\tilde{\nu}}{2 \pi}  ( 1 + \{ \tilde{\Delta} \star \tilde{\rho} \} ) 	 \label{eq:2Bess}
\end{eqnarray}

\footnote{\begin{eqnarray} 2\pi \tilde{\rho_s}  & =  &  1 + \{ \tilde{\Delta} \star \tilde{\rho} \} ~= ~ 1 + \{ \tilde{\Delta} \star \{\tilde{\nu} \tilde{\rho_s}\} \}  \label{eq:3Bess}\end{eqnarray}}

Après avoir résolue ces deux dernier equations, on peut calculer la densité linéaire $n$ , l'énergie totale $E$, l'entropy $S$, la pression $P$ : 

\begin{eqnarray}\label{eq:5Bess}
	(nl_g = ) \frac{1}{\gamma} & = & \int d \tilde{\theta} \,\tilde{\rho}(\tilde{\theta}), \\
	\beta \frac{E}{N} & = & \frac{\gamma}{t} \int d \tilde{\theta}\,  \tilde{\theta}^2 \tilde{\rho}(\tilde{\theta}), \\
	\frac{S}{N} & = & \gamma 	\int d \tilde{\theta} \,\tilde{\rho}(\tilde{\theta}) \left ( e^{\tilde{\epsilon}(\tilde{\theta})} \ln \left ( 1 + e^{-\tilde{\epsilon}(\tilde{\theta})}  \right ) + e^{\tilde{\epsilon}(\tilde{\theta})} \right )  ,\\
	\beta \frac{P}{n}  & = & \frac{\gamma}{2 \pi} \int d \tilde{\theta} \, \ln \left ( 1 + e^{-\tilde{\epsilon}(\tilde{\theta})}  \right ).
\end{eqnarray}


{ \bf \em Programmation :}

\begin{itemize}
	\item[$1$] Pour des couples ($\mu , T $) résoudre l'équation (\ref{eq:1Bess}). Avec $\tilde{e}$ on a $\tilde{\nu}$
	\item[$2$] Résoudre l'équation (\ref{eq:2Bess}) pour avoir $\tilde{\rho}$
	\item[$2 ~bis$] Résoudre l'équation (\ref{eq:3Bess}) pour avoir $\tilde{\rho}_s$. Et $\tilde{\rho} = \tilde{\nu} \tilde{\rho}_s$
	\item[$3$] Ou peut avoir accès à d'autre valeur thermodynamique avec les équation ci-dessus
\end{itemize}

{ \bf \em Résoudre :}



Les équations (\ref{eq:1Bess}) ,   (\ref{eq:2Bess}) et  (\ref{eq:3Bess}) sont de la forme .\\

Soit $x$ dans $K$ un e-vl  et une application $F \colon K \rightarrow K $

\begin{eqnarray}
	x & = & b + F ( x ) 	
\end{eqnarray}


$$
\begin{array}{c||ccc}
	& x & F( x )  & b \\
	\hline\hline
	(\ref{eq:1Bess}) & \tilde{\epsilon} & - \frac{1}{2 \pi} \left \{\tilde{\Delta} \star \ln \left ( 1 + e^{x} \right )  \right \} & - \tilde{\mu} + \frac{x \mapsto x^2 }{t}	\\
	(\ref{eq:2Bess}) & \tilde{\rho}  & \frac{\tilde{\nu}}{2 \pi} \cdot  \left \{\tilde{\Delta} \star x \right \} & \frac{\tilde{\nu}}{2 \pi} \\
	(\ref{eq:3Bess}) & \tilde{\rho}_s  & \frac{1}{2 \pi} \left \{\tilde{\Delta} \star \{\tilde{\nu} \cdot  x\} \right \} & \frac{1}{2 \pi} 
\end{array}
$$

{~}\\

La notation tensorielle 


\begin{eqnarray}
	x^i & = & b^i + \{F\}^i_j  x^j  	
\end{eqnarray}


$$
\begin{array}{c||ccc}
	& x^i & \{F\}^i_j  x^j & b^i\\
	\hline\hline
	(\ref{eq:1Bess}) & \tilde{\epsilon}^i & - \frac{1}{2 \pi} \left \{\tilde{\Delta} \star \ln \left ( 1 + e^{x } \right )  \right \}^i  & - \tilde{\mu} + \frac{x^i  \mapsto {x^i }^2 }{t}	\\
	(\ref{eq:2Bess}) & \tilde{\rho}^i  &  \left  \{ \frac{\tilde{\nu}}{2 \pi} \cdot \left \{\tilde{\Delta} \star x \right  \} \right \}^i & \frac{\tilde{\nu^i}}{2 \pi} \\
	(\ref{eq:3Bess}) & {\tilde{\rho}_s}^i  &  \frac{1}{2 \pi} \left \{\tilde{\Delta} \star \{\tilde{\nu} \cdot  x\} \right \}^i & \frac{1}{2 \pi} 
\end{array}
$$

Soit deux applications $f , g $. La notation Tensorielle de leurs multiplication et convolution est 
\begin{eqnarray}
	\{f \cdot   g \} ( x ) = f(x)  g(x) & \rightarrow & \{ f \cdot g \}^i x_i = \delta^i_j f^ix_i g^j x_j  \\
	\{f \star  g \} ( x ) = {\textstyle \int f ( x - x' ) g ( x' )  d x '}  & \rightarrow &	 \{ f \star g  \}^i x_i  = dx \, \delta^\sigma_j \{  f ( \cdot_i -\cdot_\sigma  )\}^{i\sigma } x_i x_\sigma   g^j x_j   
\end{eqnarray}


{~}\\

Numériquement on résout dans l'espace des vecteur $x_i$ et non dans l'espace dual $x^i$. Ce que je fait réellement c'est résoudre 
\begin{eqnarray}
	x_i  & = & b_i  + \{F(x_i) \}_i  	
\end{eqnarray}

$$
\begin{array}{c||ccc}
	& x_i & \{F(x_i) \}_i & b_i \\
	\hline\hline
	(\ref{eq:1Bess}) & \left \{\tilde{\epsilon}(\tilde{\theta}_i) \right \}_i  & -\frac{1}{2 \pi} \left \{\tilde{\Delta} \star \ln \left ( 1 + e^{x } \right )  \right \}_i  & - \tilde{\mu} + \frac{\tilde{\theta}_i^2 }{t}	\\
	(\ref{eq:2Bess}) & \left \{\tilde{\rho}(\tilde{\theta}_i) \right \}_i  &  \left  \{ \frac{\tilde{\nu}}{2 \pi} \cdot \left \{\tilde{\Delta} \star x \right  \} \right \}_i & \frac{\left \{\tilde{\nu}(\tilde{\theta}_i) \right \}_i}{2 \pi} \\
	(\ref{eq:3Bess}) &  \left \{\tilde{\rho}_s(\tilde{\theta}_i) \right \}_i & \frac{1}{2 \pi} \left \{\tilde{\Delta} \star \{\tilde{\nu} \cdot  x\} \right \}_i & \frac{1}{2 \pi} 
\end{array}\\
$$

Soit deux applications $f , g $. La notation Tensorielle de leurs multiplication et convolution est 
\begin{eqnarray}
	\{f \cdot   g \} ( x ) = f(x)  g(x) & \rightarrow & \{ \{ f \cdot g \}( x_i )\}_i   = \delta^j_i \delta^\sigma_j \{f ( x_\sigma ) \}_\sigma \{g(x_j ) \}_j    \\
	\{f \star  g \} ( x ) = {\textstyle \int f ( x - x' ) g ( x' )  d x '}  & \rightarrow &	 \{\{ f \star g  \}(x_i)\}_i  = dx \, \delta^\sigma_j \{  f ( x_i -x_\sigma  )\}_{i\sigma }    \{g ( x_j)\}_j    
\end{eqnarray}


\begin{enumerate}[label = Méthode \arabic*)]
	\item Résoudre :
		\begin{eqnarray}
			F_1 ( x  ) & = & b 	
		\end{eqnarray}
		avec $F_1 \colon x \mapsto x - F(x)$.\\

 

		
		
		Les équations (\ref{eq:2Bess}) et (\ref{eq:3Bess}) sont linéaires ; On peut écrire des équations matricielle :
		
		\begin{eqnarray}
			\{F_1\}  x   & = & b  	
		\end{eqnarray} 
		
		Pour $\tilde{\rho} ( \tilde{\theta} ) $ , $F_1 ( x )= x -  \frac{\tilde{\nu}}{2 \pi} \cdot \left \{\tilde{\Delta} \star x \right  \}  $ et $ b = \frac{\tilde{\nu}(\tilde{\theta}) }{2 \pi}$
		\begin{eqnarray}
			\left \{ Id - \frac{d \tilde{\theta}}{2 \pi} \mbox{diag} ( \tilde{\nu}   ( \tilde{\theta} )) \cdot \tilde{\Delta} ( \tilde{\theta} - \tilde{\theta }' ) \right \} x  &= & \frac{\tilde{\nu}(\tilde{\theta}) }{2 \pi}	
		\end{eqnarray}
		et $ \tilde{ \rho}_s = \frac{\tilde{\rho}}{\tilde{\nu}}$\\
		
		Pour $\tilde{\rho}_s ( \tilde{\theta} ) $ , $F_1 ( x )= x -  \frac{1}{2 \pi} \cdot \left \{\tilde{\Delta} \star (\tilde{\nu} \cdot x )  \right  \}  $ et $ b = \frac{1 }{2 \pi}$
		\begin{eqnarray}
			\left \{ Id - \frac{d \tilde{\theta}}{2 \pi}   \tilde{\Delta} ( \tilde{\theta} - \tilde{\theta }' )  \cdot \mbox{diag} ( \tilde{\nu}   ( \tilde{\theta} )) \right \} x  & = &  \frac{1 }{2 \pi}	
		\end{eqnarray}
		et $\tilde{ \rho} = \tilde{\rho}_s \cdot \tilde{\nu}$\\
		

		
		
		L'équations (\ref{eq:1Bess}) est non-linéaire. On peut avoir des solution cycles que l'on explique dans la deuxieme méthode.

		
	\item Pour trouver les points fixes de \(F_2: x \mapsto b + F(x)\), nous devons résoudre l'équation $$F_2(x) = x$$. Soit \(n\) un entier naturel, alors \(x_n = x^\ast + \delta x_n\) où \(x^\ast\) est un point fixe de \(F_2\) et \(\delta x_n\) est une petite perturbation.En développant \(F_2\) autour de \(x^\ast\), nous obtenons :
	\begin{eqnarray}
	x^\ast + \delta x_{n+1} = F_2(x^\ast + \delta x_n) = F_2(x^\ast) + F_2'(x^\ast)\delta x_n + \mathcal{O}(\delta x_n),
	\end{eqnarray}
	où \(F_2'(x^\ast)\) est la dérivée de \(F_2\) évaluée en \(x^\ast\). En négligeant les termes d'ordre supérieur \(\mathcal{O}(\delta x_n)\), nous avons :
	\begin{eqnarray}
		\delta x_{n+1} = F_2'(x^\ast)\delta x_n.
	\end{eqnarray}
	Pour que \(x^\ast\) soit stable, la norme du determinant du jacobien de \(F_2\) évaluée en \(x^\ast\) doit être inférieure à 1.
	
	Pour l'équation (\ref{eq:1Bess}) 
	\begin{eqnarray} 
		F_2' ( x ) =   - \frac{1}{2 \pi}\left \{ \tilde{\Delta} \underset{\tiny \tilde{\theta}}{\star} \frac{ e^x }{ 1 + e^{x} } \right \} 
	\end{eqnarray}
	
	\begin{eqnarray}
		\left ( \frac{ e^x }{ 1 + e^{x} } \right )' 	& = & \frac{ e^x }{ (1 + e^{x})^2 }     
	\end{eqnarray}
	
	donc 
	
	\begin{eqnarray}
		\underset{x \colon \mathbb{R} \to \mathbb{R} }{\mbox{Max}}\left ( \frac{ e^x }{ 1 + e^{x} } \right ) 	& < & \underset{ x \to \infty}{ \lim}  \left ( \frac{ e^x }{ 1 + e^{x} } \right ) = 1    
	\end{eqnarray}


	%et en utilisent l'inegalite de chochy-swhartz $\Vert g \star f \Vert_1 \leq  \Vert g\Vert_1 \Vert f\Vert_1 $ et 
	\begin{eqnarray}
		\Vert \tilde{\Delta} \Vert_1  & = & \int \frac{2}{1 + x^2 } dx = 2 \left [ \arctan (x) \right ]_{-\infty}^{+ \infty} = 2 \pi , \\
		\Vert F_2' \Vert_1 & < &  \frac{1}{2\pi} \underset{x \colon \mathbb{R} \to \mathbb{R} }{\mbox{Max}}\left ( \frac{ e^x }{ 1 + e^{x} } \right )  \Vert \tilde{\Delta} \Vert_1  = 1 
	\end{eqnarray}
	
	Si \(F_2\) est non linéaire, les solutions peuvent être des cycles. Par exemple, pour un cycle de longueur 2, nous devons résoudre l'équation :
	\begin{eqnarray}
		\{F_2 \circ F_2\}(x) = x.
	\end{eqnarray}
	
	\begin{eqnarray}
		x_{n+1} - x_{n} = \delta x_{n+1} - \delta x_{n} & = &  (F_2'(x^\ast) - 1 )\delta x_n + \mathcal{O}(\delta x_n) = \mathcal{O}(\delta x_n), \\
		\frac{\delta x_{n+1} - \delta x_{n}}{\delta x_{n}} = \mathcal{O}(1).
	\end{eqnarray}

	

\end{enumerate}



$$
\begin{array}{c||ccc}
	& x & F_1( x )  & b \\
	\hline\hline
	(\ref{eq:1Bess}) & \tilde{\epsilon} & x + \frac{1}{2 \pi} \left \{\tilde{\Delta} \star \ln \left ( 1 + e^{x} \right )  \right \} & - \tilde{\mu} + \frac{x \mapsto x^2 }{t}	\\
	(\ref{eq:2Bess}) & \tilde{\rho}  & x -\frac{\tilde{\nu}}{2 \pi} \cdot  \left \{\tilde{\Delta} \star x \right \} & \frac{\tilde{\nu}}{2 \pi} \\
	(\ref{eq:3Bess}) & \tilde{\rho}_s  & x - \frac{1}{2 \pi} \left \{\tilde{\Delta} \star \tilde{\nu} \cdot x \right \} & \frac{1}{2 \pi} 
\end{array}
$$

		{~}\\

		La notation tensorielle 		
		\begin{eqnarray}
			\{F_1\}^i_j  x^j   & = & b^i  	
		\end{eqnarray} 
		

		
$$
\begin{array}{c||ccc}
	& x^i & \{F_1\}^i_j  x^j & b^i\\
	\hline\hline
	(\ref{eq:1Bess}) & \tilde{\epsilon}^i & x^i + \frac{1}{2 \pi} \left \{\tilde{\Delta} \star \ln \left ( 1 + e^{x } \right )  \right \}^i  & - \tilde{\mu} + \frac{x^i  \mapsto {x^i }^2 }{t}	\\
	(\ref{eq:2Bess}) & \tilde{\rho}^i  & x^i - \left  \{ \frac{\tilde{\nu}}{2 \pi} \cdot \left \{\tilde{\Delta} \star x \right  \} \right \}^i & \frac{\tilde{\nu^i}}{2 \pi} \\
	(\ref{eq:3Bess}) & {\tilde{\rho}_s}^i  & x^i - \frac{1}{2 \pi} \left \{\tilde{\Delta} \star \tilde{\nu} \cdot  x \right \}^i & \frac{1}{2 \pi} 
\end{array}
$$
		

			
		
		Numériquement on résout dans l'espace des vecteur $x_i$ et non dans l'espace dual $x^i$. Ce que je fait réellement c'est résoudre 
		
		\begin{eqnarray}
			\{F_1(x_i) \}_i    & = & b_i   	
		\end{eqnarray}
		

		
$$
\begin{array}{c||ccc}
	& x_i & \{F_1(x_i) \}_i & b_i \\
	\hline\hline
	(\ref{eq:1Bess}) & \left \{\tilde{\epsilon}(\tilde{\theta}_i) \right \}_i  & x_i + \frac{1}{2 \pi} \left \{\tilde{\Delta} \star \ln \left ( 1 + e^{x } \right )  \right \}_i  & - \tilde{\mu} + \frac{\tilde{\theta}_i^2 }{t}	\\
	(\ref{eq:2Bess}) & \left \{\tilde{\rho}(\tilde{\theta}_i) \right \}_i  & x_i  - \left  \{ \frac{\tilde{\nu}}{2 \pi} \cdot \left \{\tilde{\Delta} \star x \right  \} \right \}_i & \frac{\left \{\tilde{\nu}(\tilde{\theta}_i) \right \}_i}{2 \pi} \\
	(\ref{eq:3Bess}) &  \left \{\tilde{\rho}_s(\tilde{\theta}_i) \right \}_i & x_i - \frac{1}{2 \pi} \left \{\tilde{\Delta} \star \tilde{\nu} \cdot  x \right \}_i & \frac{1}{2 \pi} 
\end{array}\\
$$
	
	
	
	





 


	

 






	



	
	

