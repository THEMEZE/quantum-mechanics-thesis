Pour un plus grand nombre de particules, les états propres de l'Hamiltonien (\ref{eq:I-0-6}) sur la ligne infinie sont des états de Bethe $\vert \{\theta_a\}\rangle $ étiquetés par un ensemble de $N$ nombres $\{\theta_a\}_{a \in \llbracket 1 , N \rrbracket } $, appelés les rapidités. Dans le domaine $x_1 < z_2 < \cdots < z_N$, la fonction d'onde est ({\color{blue} Gaudin 2014}, {\color{blue}Korepin et al. 1997}, {\color{blue}Lieb anz Liniger 1963}) :

\begin{eqnarray}
	\varphi_{\{\theta_a\}} ( z_1 , \cdots , z_N ) & = & \langle 0 \vert \Psi ( z_1 ) \cdots \Psi (z_N ) \vert \{ \theta_a \} \rangle \notag\\
	& \propto & \sum_\sigma ( - 1 ) ^{ \vert \sigma \vert } \left ( \prod_{ 1 \leq a < b \leq N } ( \theta_{\sigma(b)} - \theta_{\sigma(a)} - ic ) \right ) e^{i \sum_j z_i \theta_{ \sigma(j)}}\label{eq:I-2-17},
\end{eqnarray}

et elle est étendue à d'autres domaines par symétrie $z_i \leftrightarrow z_j$ . Ici, la somme s'étend à toutes les permutations $\sigma$ des $N$ éléments (donc il y a $N!$ termes) et $(-1)^{|\sigma|}$ est la signature de la permutation. L'impulsion et l'énergie de l'état propre (\ref{eq:I-2-17}) sont :