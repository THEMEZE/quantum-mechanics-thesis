\documentclass{standalone} % Utilise la classe standalone pour compiler uniquement le TikZ
\usepackage{tikz}          % Charge le package TikZ
\usepackage{xcolor}        % Gestion avancée des couleurs
\usepackage{amsmath}       % Gestion des formules mathématiques
\usetikzlibrary{arrows.meta} % Pour gérer les flèches avec des formes spécifiées comme "triangle"
\usetikzlibrary {decorations.pathmorphing}


% Définition de la fonction pour générer la grille
\newcommand{\drawgrid}[4]{


	\begin{scope}[opacity = 0.2]
	
	
    	% Paramètres : #1=xmin, #2=xmax, #3=ymin, #4=ymax

   	 	% Grille principale en cm
    	\draw[very thin, gray] (#1,#3) grid (#2,#4); % Grille de base (1 cm)
    
    	\draw[very thin, gray] (#1,#3) grid[step=0.1] (#2,#4);

    	\draw[thin, black] (#1,#3) edge[thick,line width=0.8ex,->,>=Stealth ] (#2,#3); 
    
    	\foreach \x in {#1,..., #2} {
        	\draw[shift={(\x, #3)}] (0,-0.1) edge[line width=0.8ex] node[pos = -1 , below] {\x} (0,0.1); % Lignes verticales en mm
    	}

    	\draw[thin, black] (#1,#3)edge[thick,line width=0.8ex,->,>=Stealth ] (#1,#4); 
    
    	\foreach \y in {#3,..., #4} {
        	\draw[shift={(#1, \y)}] (-0.1 , 0 ) edge[line width=0.8ex] node[pos = -1 , left ] {\y} (0.1 , 0); % Lignes verticales en mm
    	}
    
    \end{scope}


}


\newcommand{\Axex}[1][$x$]{ % "x" est la valeur par défaut de #1
    \draw
        (-10, 0) edge[thick, line width=0.8ex, ->, >=Triangle, color=colorOne] 
        node[pos=1.01, below]{ \huge #1 } (10, 0)
        ;
}

\newcommand{\Axetheta}[1][$\theta$]{
	\draw
		(-10, 0 ) edge [thick,line width=0.8ex,->,>=Triangle , color = \colorOne ]node [pos=1.01, above ]{\huge #1 } ( 10  , 0 )
	;
}

\newcommand{\Axedensity}[1][$n$]{
	\draw
		(0, -1 ) edge [thick,line width=0.8ex,->,>=Triangle , color = \colorOne ]node [pos=1.01,left  ]{\huge #1 } ( 0  , 6 )
	;
}

\newcommand{\Axedistribution}[1][$\nu$]{
	\draw
		(0, -1 ) edge [thick,line width=0.8ex,->,>=Triangle , color = \colorOne ]node [pos=1.01,left  ]{\huge #1 } ( 0  , 6 )
	;
}

\newcommand{\Axesphase}[2][$x$]{
	\Axex[#1]
	\begin{scope}[rotate = 90]
		\Axetheta[#2]	
	\end{scope}
}

\newcommand{\Axesdensity}[2][$x$]{
	\Axex[#1]
	\Axedensity[#2]
}

\newcommand{\Axesdistribution}[2][\theta]{
	\Axetheta[#1]
	\Axedistribution[#2]
}

\newcommand{\Axetemps}[1][Deformation]{
	\draw 
		(-5 , 0 ) edge [line width=2.8ex, color=\colorOne ]( 0, 0)
		(43 , 0 ) edge [line width=2.8ex, color=\colorOne , ->, >=Triangle] node[pos = 1.0 , right , scale = 2 ] { \fontsize{60pt}{720pt}\selectfont $\tilde{t}$} ( 50 , 0)
		( 0 , 1 ) edge [line width=2.0ex, color=\colorOne ] ( 0 , -1 )  
		( 43 , 1 ) edge [line width=2.0ex, color=\colorOne ] ( 43 , -1 ) 
		(0 , 0) edge[<->, > = stealth, line width=2.8ex, color=\colorOne ]  node [ rectangle , midway, fill = white , scale = 2 ] {\fontsize{600pt}{720pt}\selectfont \bf  #1}(43, 0)  
	;

}


\begin{document}



% Définition des couleurs avec les codes HTML
\definecolor{colorOne}{HTML}{443E46}
\definecolor{colorTwo}{HTML}{F6DEB8}
\definecolor{colorThree}{HTML}{908CA4}
\definecolor{colorFour}{HTML}{57659E}
\definecolor{colorFive}{HTML}{C57284}
\definecolor{colorSix}{HTML}{FF5B69}

% Raccourcis pour les couleurs
\def\colorOne{colorOne}
\def\colorTwo{colorTwo}
\def\colorThree{colorThree}
\def\colorFour{colorFour}
\def\colorFive{colorFive}
\def\colorSix{colorSix}




\def\Background{
	\draw[fill = \colorFour ]  
		(-10 , -10 ) rectangle ( 10 , 10 ) 
	;	

}






\pgfdeclareverticalshading{Insitut}{20cm}
 {color(0)=(\colorFour); color(2cm)=(\colorFive); color(4cm)=(\colorSix);color(10cm)=(\colorSix);  color(12cm)=(\colorFive);  color(14cm)=(\colorFour)}



\begin{tikzpicture}

% Placement de l'axe défini dans un cercle avec une échelle globale
%\Axes 

%% Pour voir plus precisement 
\clip (-21 , -85) rectangle (55 , 21) ;\draw[color = red ] (-20 , -84) rectangle (54 , 20) ;
%\clip (-20 , -15) rectangle (20 , 20) ; \draw[color = red ] (-20 , -14) rectangle (20 , 18) ; % voire Insitut
%\clip (10 , -15) rectangle (55 , 20) ; \draw[color = red ] (10 , -15) rectangle (55 , 20) ; % voire coupure 1 0ms  
%\clip (53 , -15) rectangle (95 , 20) ; \draw[color = red ] (53 , -15) rectangle (95 , 20) ; % voire coupure 1 18ms 
%\clip (53 , -55) rectangle (95 , 20) ; \draw[color = red ] (53 , -55) rectangle (95 , 20) ; % voire coupure 1 18ms 


% palette de couleur 
\node at (40,6) [rectangle , rotate = -90] { 
	\begin{tikzpicture}
	
	\node at (0,0) [rectangle, ] {
		\begin{tikzpicture}
			\draw[fill=\colorOne , color = \colorOne ] (-1,-1) rectangle (1,1);
		\end{tikzpicture}} ;
	\node at (3,0) [rectangle, ] {
		\begin{tikzpicture}
			\draw[fill=\colorTwo , color = \colorTwo] (-1,-1) rectangle (1,1);
		\end{tikzpicture}} ;
	\node at (6,0) [rectangle, ] {
		\begin{tikzpicture}
			\draw[fill=\colorThree , color = \colorThree] (-1,-1) rectangle (1,1);
		\end{tikzpicture}} ;
	\node at (9,0) [rectangle, ] {
		\begin{tikzpicture}
			\draw[fill=\colorFour , color = \colorFour] (-1,-1) rectangle (1,1);
		\end{tikzpicture}} ;
	\node at (12,0) [rectangle, ] {
		\begin{tikzpicture}
			\draw[fill=\colorFive , color = \colorFive] (-1,-1) rectangle (1,1);
		\end{tikzpicture}} ;
	\node at (15,0) [rectangle, ] {
		\begin{tikzpicture}
			\draw[fill=\colorSix  , color = \colorSix] (-1,-1) rectangle (1,1);
		\end{tikzpicture}} ;
	
	\end{tikzpicture}
	}; 
	

% Insitut 	
\node at (21,0) [rectangle ]{
	\begin{tikzpicture}[transform canvas={scale=1}] 
		
		% diagrale de phase 
		\node at (0,0) [circle, ] {
			\begin{tikzpicture}[transform canvas={scale=1}] 
				\begin{scope}
	 				% Appliquer un rectangle décoré et clipper la zone
    				\clip[decorate, decoration={random steps, segment length=3pt, amplitude=1pt}]
        			(-8,-8) rectangle (8,8); 
     				\node at (0,0) [circle  ] {\begin{tikzpicture}[transform canvas={scale=1}] \Background \end{tikzpicture}};
					\node at (0,0) [circle, ] {\pgfuseshading{Insitut}};
		
				\end{scope}
				\node at (0,0) [circle] { \begin{tikzpicture}[transform canvas={scale=1}] \Axesphase[$x$]{$\theta$} \end{tikzpicture}};

			\end{tikzpicture}
			};
	
		% density 
		\node at (0,13) [circle, ] {
			\begin{tikzpicture}[transform canvas={scale=1}] 
				\begin{scope}
					\clip[decorate, decoration={random steps, segment length=3pt, amplitude=1pt}]
        			(-8,-8) rectangle (8,8); 
        			\draw[line width=1.8ex , color = \colorFour , fill = \colorThree ] 
					(-10 , 0 ) -- (-10 , 5 ) -- (10 , 5 ) -- ( 10, 0 )  
					;				
				\end{scope}
		
				\Axesdensity[$x$]{$n$}
		
			\end{tikzpicture}

			};
	
	
		% distribution
		\node at (-13,0) [circle, rotate = 90 ] {
			\begin{tikzpicture}[transform canvas={scale=1}] 
				\begin{scope}
					\clip[decorate, decoration={random steps, segment length=3pt, amplitude=1pt}]
        			(-8,-8) rectangle (8,8);

					%\filldraw[smooth , line width=1.8ex , color = \colorFour , fill = \colorThree] plot coordinates {(-10,0) (-5,0.5) (-4,4) (4,4) (5,0.5) (10 , 0 )};
					\filldraw[  smooth, line width=1.8ex, color=\colorFour, fill=\colorThree] 
    						(-10,0)  .. controls (0,0) and (-10,5) .. (0,5) 
    						.. controls (10,5) and (0,0) .. (10,0);				
				\end{scope}		
				\Axesdistribution[$\theta$]{$\nu$}	
			\end{tikzpicture}

			};

	\end{tikzpicture}
};

% coupuur 1 O ms	
\node at (0,-35) [rectangle ]{
	\begin{tikzpicture}[transform canvas={scale=1}] 
		
		% diagrale de phase 
		\node at (0,0) [circle, ] {
			\begin{tikzpicture}[transform canvas={scale=1}] 
				\begin{scope}
	 				% Appliquer un rectangle décoré et clipper la zone
    				\clip[decorate, decoration={random steps, segment length=3pt, amplitude=1pt}]
        			(-8,-8) rectangle (8,8); 
     				\node at (0,0) [circle  ] {\begin{tikzpicture}[transform canvas={scale=1}] \Background \end{tikzpicture}};
     				\clip (0,-8) rectangle (10,8) ;
					\node at (0,0) [circle, ] {\pgfuseshading{Insitut}};
		
				\end{scope}
				\node at (0,0) [circle, ] {\begin{tikzpicture}[transform canvas={scale=1}] \Axesphase[$x$]{$\theta$} \end{tikzpicture}};

			\end{tikzpicture}
			};
	
		% density 
		\node at (0,13) [circle, ] {
			\begin{tikzpicture}[transform canvas={scale=1}] 
				\begin{scope}
					\clip[decorate, decoration={random steps, segment length=3pt, amplitude=1pt}]
        			(-8,-8) rectangle (8,8);
 
        			\filldraw[line width=1.8ex , color = \colorFour , fill = \colorThree ] 
						(-10 , 0 ) --  ( 0 , 0 ) -- ( 0 , 5 ) --(10 , 5 ) -- ( 10 , 0 )   
					;				
				\end{scope}
		
				\Axesdensity[x]{n}
		
			\end{tikzpicture}

			};
		
		% distribution x < 0 
		\node at (-13,0) [circle,   rotate = 90 , xscale = 1   ] {
			\begin{tikzpicture}[transform canvas={scale=1} ] 
				\begin{scope}
					\clip[decorate, decoration={random steps, segment length=3pt, amplitude=1pt}]
        			(-8,-8) rectangle (8,8);

					\filldraw[smooth , line width=1.8ex , color = \colorFour , fill = \colorThree] plot coordinates {(-10,0)  (10 , 0 )};				
				\end{scope}		
				\Axesdistribution[$\theta$]{$\nu ( x < 0 ) $}		
			\end{tikzpicture}

			};
	
		% distribution x > 0 
		\node at (13,0) [circle,   rotate = 90 , xscale = -1   ] {
			\begin{tikzpicture}[transform canvas={scale=1} ] 
				\begin{scope}
					\clip[decorate, decoration={random steps, segment length=3pt, amplitude=1pt}]
        			(-8,-8) rectangle (8,8);

					%\filldraw[smooth , line width=1.8ex , color = \colorFour , fill = \colorThree] plot coordinates {(-10,0) (-5,0.5) (-4,4) (4,4) (5,0.5) (10 , 0 )};
					\filldraw[  smooth, line width=1.8ex, color=\colorFour, fill=\colorThree] 
    						(-10,0)  .. controls (0,0) and (-10,5) .. (0,5) 
    						.. controls (10,5) and (0,0) .. (10,0);				
				\end{scope}		
				\Axesdistribution[$\theta$]{$\nu ( x > 0 ) $}		
			\end{tikzpicture}

			};

	\end{tikzpicture}
};


% coupuur 1 18 ms	
\node at (43,-35) [rectangle ]{
	\begin{tikzpicture}[transform canvas={scale=1}] 
		
		% diagrale de phase 
		\node at (0,0) [circle, ] {
			\begin{tikzpicture}[transform canvas={scale=1}] 
				\begin{scope}
	 				% Appliquer un rectangle décoré et clipper la zone
    				\clip[decorate, decoration={random steps, segment length=3pt, amplitude=1pt}]
        			(-8,-8) rectangle (8,8); 
     				\node at (0,0) [circle  ] {\begin{tikzpicture}[transform canvas={scale=1}] \Background \end{tikzpicture}};
     				\clip[smooth] plot coordinates {(-5,-10) (-2,-2.75) (2,2.75) (5,10) (10,10) 
     				 (10 , -10 )} ;
					\node at (0,0) [circle, ] {\pgfuseshading{Insitut}};
		
				\end{scope}
				\node at (0,0) [circle, ] {
					\begin{tikzpicture}[transform canvas={scale=1}] 
						\Axesphase[$x$]{$\theta$}
						\draw
    						(1, 0.2) edge[line width=1.8ex, color=\colorTwo] 
        						node[pos=1.1, below]{ \color{\colorTwo}\huge $x(s)$} (1, -0.2)
        					(0.2 , 1.3 ) edge[line width=1.8ex, color=\colorTwo] 
        						node[pos=1.1, left ]{ \color{\colorTwo}\huge $\theta(s)$} (-0.2 , 1.3 )
        					(1 , -0.2 ) edge[line width=0.8ex, color=\colorTwo , dashed ] (1 , 2 )
        					(-0.2 , 1.3 )  edge[line width=0.8ex, color=\colorTwo , dashed ] (1.5  , 1.3 )
        					; 
					\end{tikzpicture}};
				%\drawgrid{-10}{10}{-10}{10}
			\end{tikzpicture}
			};
	
		% density 
		\node at (0,13) [circle, ] {
			\begin{tikzpicture}[transform canvas={scale=1}] 
				\begin{scope}
					\clip[decorate, decoration={random steps, segment length=3pt, amplitude=1pt}]
        			(-8,-8) rectangle (8,8);
 
        			%\filldraw[smooth , line width=1.8ex , color = \colorFour , fill = \colorThree ] plot coordinates {(-10,0) (-3,0.4) (2.,4.5) (10,5) 
     				 %(10 , 0 )};
     				 \filldraw[smooth, line width=1.8ex, color=\colorFour, fill=\colorThree] 
    						(-10,0) -- (-2,0) 
    						.. controls (-1,0) and (2,5) .. (3,5) 
    						-- (10,5) -- (10,0);


				\end{scope}
		
				\Axesdensity[$x/t$]{$n$}
				

        				
				%\drawgrid{-10}{10}{-10}{10}
		
			\end{tikzpicture}

			};
		
		% distribution x  
		\node at (-13,0) [circle,   rotate = 90 , xscale = 1   ] {
			\begin{tikzpicture}[transform canvas={scale=1} ] 
				\begin{scope}
					\clip[decorate, decoration={random steps, segment length=3pt, amplitude=1pt}]
        			(-8,-8) rectangle (8,8);

					%\filldraw[smooth , line width=1.8ex , color = \colorFour , fill = \colorThree] plot coordinates {(-10,0) (-5,0.5) (-4,4) (4,4) (5,0.5) (10 , 0 )};
					
					\filldraw[ opacity = 0.5 , smooth, line width=1.8ex, color=\colorFour, fill=\colorThree] 
    						(-10,0)  .. controls (0,0) and (-10,5) .. (0,5) 
    						.. controls (10,5) and (0,0) .. (10,0);
					
					\filldraw[  opacity = 1 , smooth, line width=1.8ex, color=\colorFour, fill=\colorThree] 
    						(-10,0)  .. controls (0,0) and (-10,5) .. (0,5) 
    						.. controls (1.3,5) and (1.3,5) .. (1.3,4.9) -- (1.3 , 0 )-- ( 10 , 0 ) ;				
				\end{scope}
				\draw
    				(1.3, 0.2) edge[line width=1.8ex, color=\colorOne] 
        				node[pos=1.1, below]{ \color{\colorOne}\huge $\theta(s)$} (1.3, -0.2);		
				\Axesdistribution[$\theta$]{$\nu ( x(s) ) $}
				
				%\drawgrid{0}{10}{-10}{10}	
			\end{tikzpicture}

			};
	

	\end{tikzpicture}
};

%axe temporel 
\node at (0, -47) [circle ]{
	\begin{tikzpicture}[transform canvas={scale=1} ]
		\Axetemps[Deformation : $t = 18 ~ms$]
	\end{tikzpicture}
	};




% coupuur 2 0 ms	
\node at (0,-70) [rectangle ]{
	\begin{tikzpicture}[transform canvas={scale=1}] 
		
		% diagrale de phase 
		\node at (0,0) [circle, ] {
			\begin{tikzpicture}[transform canvas={scale=1}] 
				\begin{scope}
	 				% Appliquer un rectangle décoré et clipper la zone
    				\clip[decorate, decoration={random steps, segment length=3pt, amplitude=1pt}]
        			(-8,-8) rectangle (8,8); 
     				\node at (0,0) [circle  ] {\begin{tikzpicture}[transform canvas={scale=1}] \Background \end{tikzpicture}};
     				\clip[smooth] plot coordinates {(-5,-10) (-2,-2.75) (2,2.75) (5,10) (10,10) 
     				 (10 , -10 )} ;
     				 \clip (0.75,-10) rectangle (1.25 , 10 ) ;
					\node at (0,0) [circle, ] {\pgfuseshading{Insitut}};
		
				\end{scope}
				\node at (0,0) [circle, ] {
					\begin{tikzpicture}[transform canvas={scale=1}] 
						\Axesphase[$x$]{$\theta$}
						\draw
    						(1, 0.2) edge[line width=1.8ex, color=\colorTwo] 
        						node[pos=1.1, below]{ \color{\colorTwo}\huge $x(s)$} (1, -0.2)
        					(0.2 , 1.3 ) edge[line width=1.8ex, color=\colorTwo] 
        						node[pos=1.1, left ]{ \color{\colorTwo}\huge $\theta(s)$} (-0.2 , 1.3 )
        					(1 , -0.2 ) edge[line width=0.8ex, color=\colorTwo , dashed ] (1 , 2 )
        					(-0.2 , 1.3 )  edge[line width=0.8ex, color=\colorTwo , dashed ] (1.5  , 1.3 )
        					; 
					\end{tikzpicture}};
				%\drawgrid{-10}{10}{-10}{10}
			\end{tikzpicture}
			};
	
		% density 
		\node at (0,13) [circle, ] {
			\begin{tikzpicture}[transform canvas={scale=1}] 
				\begin{scope}
					\clip[decorate, decoration={random steps, segment length=3pt, amplitude=1pt}]
        			(-8,-8) rectangle (8,8);
 
        			%\filldraw[smooth , line width=1.8ex , color = \colorFour , fill = \colorThree ] plot coordinates {(-10,0) (-3,0.4) (2.,4.5) (10,5) 
     				 %(10 , 0 )};
     				 \filldraw[opacity = 0.5 , smooth, line width=1.8ex, color=\colorFour, fill=\colorThree] 
    						(-10,0) -- (-2,0) 
    						.. controls (-1,0) and (2,5) .. (3,5) 
    						-- (10,5) -- (10,0);
    				\filldraw[smooth, line width=1.8ex, color=\colorFour, fill=\colorThree] 
    						(-10,0) -- (0.75,0) -- (0.75 , 2.8)   
    						.. controls (0.75,2.9) and (1.24,3.5) .. (1.25,3.5) 
    						-- (1.25,0) -- (10,0);

					%\drawgrid{0}{10}{0}{10}
				\end{scope}
		
				\Axesdensity[$x$]{$n$}
				
				\draw[yshift = - 12]
					(0.5 , 0 )edge[line width=0.8ex, color=\colorOne , >-<, > = stealth,] node [pos = 0.5 , below]{\huge $\ell$}  ( 1.5 , 0) 	
				;
				

        				
				%\drawgrid{-10}{10}{-10}{10}
		
			\end{tikzpicture}

			};
		
		% distribution x  
		\node at (-13,0) [circle,   rotate = 90 , xscale = 1   ] {
			\begin{tikzpicture}[transform canvas={scale=1} ] 
				\begin{scope}
					\clip[decorate, decoration={random steps, segment length=3pt, amplitude=1pt}]
        			(-8,-8) rectangle (8,8);

					%\filldraw[smooth , line width=1.8ex , color = \colorFour , fill = \colorThree] plot coordinates {(-10,0) (-5,0.5) (-4,4) (4,4) (5,0.5) (10 , 0 )};
					
					\filldraw[ opacity = 0.5 , smooth, line width=1.8ex, color=\colorFour, fill=\colorThree] 
    						(-10,0)  .. controls (0,0) and (-10,5) .. (0,5) 
    						.. controls (10,5) and (0,0) .. (10,0);
					
					\filldraw[  opacity = 1 , smooth, line width=1.8ex, color=\colorFour, fill=\colorThree] 
    						(-10,0)  .. controls (0,0) and (-10,5) .. (0,5) 
    						.. controls (1.3,5) and (1.3,5) .. (1.3,4.9) -- (1.3 , 0 )-- ( 10 , 0 ) ;				
				\end{scope}
				\draw
    				(1.3, 0.2) edge[line width=1.8ex, color=\colorOne] 
        				node[pos=1.1, below]{ \color{\colorOne}\huge $\theta(s)$} (1.3, -0.2);			
				\Axesdistribution[$x$]{$\nu ( x(s) ) $}
				
				%\drawgrid{0}{10}{-10}{10}	
			\end{tikzpicture}

			};
	

	\end{tikzpicture}
};


% coupuur 2 30 ms	
\node at (43,-70) [rectangle ]{
	\begin{tikzpicture}[transform canvas={scale=1}] 
		
		% diagrale de phase 
		\node at (0,0) [circle, ] {
			\begin{tikzpicture}[transform canvas={scale=1}] 
				\begin{scope}
	 				% Appliquer un rectangle décoré et clipper la zone
    				\clip[decorate, decoration={random steps, segment length=3pt, amplitude=1pt}]
        			(-8,-8) rectangle (8,8); 
     				\node at (0,0) [circle  ] {\begin{tikzpicture}[transform canvas={scale=1}] \Background \end{tikzpicture}};
     				\clip (-30,-10) -- (-30 , 1.) -- (-8 , 1 ) .. controls (0,1) and (0,1) .. ( 8 , 1.5 ) -- (30 , 1.5 ) -- (30 , -10 ) ;
     				\node at (0,0) [circle, draw] {\pgfuseshading{Insitut}};

				\end{scope}
				\node at (0,0) [circle, ] {
					\begin{tikzpicture}[transform canvas={scale=1}] 
						\Axesphase[$x$]{$\theta$}
						\draw
    						(5, 0.2) edge[line width=1.8ex, color=\colorTwo] 
        						node[pos=1.1, below]{ \color{\colorTwo}\huge $x(s)$} (5, -0.2)
        					(0.2 , 1.3 ) edge[line width=1.8ex, color=\colorTwo] 
        						node[pos=1.1, left ]{ \color{\colorTwo}\huge $\theta(s)$} (-0.2 , 1.3 )
        					(5 , -0.2 ) edge[line width=0.8ex, color=\colorTwo , dashed ] (5 , 2 )
        					(-0.2 , 1.3 )  edge[line width=0.8ex, color=\colorTwo , dashed ] (6  , 1.3 )
        					; 
					\end{tikzpicture}};
				%\drawgrid{-10}{10}{-10}{10}
			\end{tikzpicture}
			};
	
		% density 
		\node at (0,13) [circle, ] {
			\begin{tikzpicture}[transform canvas={scale=1}] 
				\begin{scope}
					\clip[decorate, decoration={random steps, segment length=3pt, amplitude=1pt}]
        			(-8,-8) rectangle (8,8);
 
        			%\filldraw[smooth , line width=1.8ex , color = \colorFour , fill = \colorThree ] plot coordinates {(-10,0) (-3,0.4) (2.,4.5) (10,5) 
     				 %(10 , 0 )};
     				 \filldraw[  opacity = 1 , smooth, line width=1.8ex, color=\colorFour, fill=\colorThree] 
    						(-10,0)  .. controls (0,0) and (-5,5) .. (0,5) 
    						.. controls (1.3,5) and (1,0) .. (2,0) --  ( 10 , 0 ) ;

					%\drawgrid{0}{10}{0}{10}
				\end{scope}
		
				\Axesdensity[$x/\tau$]{$n*\tau/\ell$}
				
				

        				
				%\drawgrid{-10}{10}{-10}{10}
		
			\end{tikzpicture}

			};
		
		% distribution x  
		\node at (-13,0) [circle,   rotate = 90 , xscale = 1   ] {
			\begin{tikzpicture}[transform canvas={scale=1} ] 
				\begin{scope}
					\clip[decorate, decoration={random steps, segment length=3pt, amplitude=1pt}]
        			(-8,-8) rectangle (8,8);

					%\filldraw[smooth , line width=1.8ex , color = \colorFour , fill = \colorThree] plot coordinates {(-10,0) (-5,0.5) (-4,4) (4,4) (5,0.5) (10 , 0 )};
					
					\filldraw[ opacity = 0.5 , smooth, line width=1.8ex, color=\colorFour, fill=\colorThree] 
    						(-10,0)  .. controls (0,0) and (-10,5) .. (0,5) 
    						.. controls (10,5) and (0,0) .. (10,0);
					
					\filldraw[  opacity = 1 , smooth, line width=1.8ex, color=\colorFour, fill=\colorThree] 
    						(-10,0)  .. controls (0,0) and (-10,5) .. (0,5) 
    						.. controls (1.3,5) and (1.3,5) .. (1.3,4.9) -- (1.3 , 0 )-- ( 10 , 0 ) ;				
				\end{scope}
				\draw
    				(1.3, 0.2) edge[line width=1.8ex, color=\colorOne] 
        				node[pos=1.1, below]{ \color{\colorOne}\huge $\theta(s)$} (1.3, -0.2);			
				\Axesdistribution[$x$]{$\nu ( x(s) ) $}
				
				%\drawgrid{0}{10}{-10}{10}	
			\end{tikzpicture}

			};
	

	\end{tikzpicture}
};

%axe temporel 
\node at (0, -82) [circle ]{
	\begin{tikzpicture}[transform canvas={scale=1} ]
		\Axetemps[Expansion $\tau = 30 ~ms$]
	\end{tikzpicture}
	};


% gaduation tout 
\drawgrid{-20}{54}{-84}{20}


\end{tikzpicture}

\end{document}
