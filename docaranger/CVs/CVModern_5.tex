%%%%%%%%%%%%%%%%%
% This is an sample CV template created using altacv.cls
% (v1.7.4, 30 July 2025) written by LianTze Lim (liantze@gmail.com). Compiles with pdfLaTeX, XeLaTeX and LuaLaTeX.
%
%% It may be distributed and/or modified under the
%% conditions of the LaTeX Project Public License, either version 1.3
%% of this license or (at your option) any later version.
%% The latest version of this license is in
%%    http://www.latex-project.org/lppl.txt
%% and version 1.3 or later is part of all distributions of LaTeX
%% version 2003/12/01 or later.
%%%%%%%%%%%%%%%%

%% Use the "normalphoto" option if you want a normal photo instead of cropped to a circle
% \documentclass[10pt,a4paper,withhyper,normalphoto]{altacv}

\documentclass[10pt,a4paper,withhyper]{altacv}
%% AltaCV uses the fontawesome5 and simpleicons packages.
%% See http://texdoc.net/pkg/fontawesome5 and http://texdoc.net/pkg/simpleicons for full list of symbols.

% Change the page layout if you need to
\geometry{left=1.25cm,right=1.25cm,top=1.25cm,bottom=1.25cm,columnsep=1.2cm}

% The paracol package lets you typeset columns of text in parallel
\usepackage{paracol}

% Change the font if you want to, depending on whether
% you're using pdflatex or xelatex/lualatex
% WHEN COMPILING WITH XELATEX PLEASE USE
% xelatex -shell-escape -output-driver="xdvipdfmx -z 0" sample.tex
\iftutex
  % If using xelatex or lualatex:
  \setmainfont{Roboto Slab}
  \setsansfont{Lato}
  \renewcommand{\familydefault}{\sfdefault}
\else
  % If using pdflatex:
  \usepackage[rm]{roboto}
  \usepackage[defaultsans]{lato}
  % \usepackage{sourcesanspro}
  \renewcommand{\familydefault}{\sfdefault}
\fi

% Change the colours if you want to
\definecolor{SlateGrey}{HTML}{2E2E2E}
\definecolor{LightGrey}{HTML}{666666}
\definecolor{DarkPastelRed}{HTML}{450808}
\definecolor{PastelRed}{HTML}{8F0D0D}
\definecolor{GoldenEarth}{HTML}{E7D192}
\colorlet{name}{black}
#\colorlet{tagline}{PastelRed}
#\colorlet{heading}{DarkPastelRed}
#\colorlet{headingrule}{GoldenEarth}
#\colorlet{subheading}{PastelRed}
#\colorlet{accent}{PastelRed}
#\colorlet{emphasis}{SlateGrey}
#\colorlet{body}{LightGrey}

\colorlet{tagline}{black!90!white}
\colorlet{heading}{black!80!white}
\colorlet{headingrule}{black!70!white}
\colorlet{subheading}{black!60!white}
\colorlet{accent}{black!50!white}
\colorlet{emphasis}{black!40!white}
\colorlet{body}{LightGrey}


% Change some fonts, if necessary
\renewcommand{\namefont}{\Huge\rmfamily\bfseries}
\renewcommand{\personalinfofont}{\footnotesize}
\renewcommand{\cvsectionfont}{\LARGE\rmfamily\bfseries}
\renewcommand{\cvsubsectionfont}{\large\bfseries}


% Change the bullets for itemize and rating marker
% for \cvskill if you want to
\renewcommand{\cvItemMarker}{{\small\textbullet}}
\renewcommand{\cvRatingMarker}{\faCircle}
% ...and the markers for the date/location for \cvevent
% \renewcommand{\cvDateMarker}{\faCalendar*[regular]}
% \renewcommand{\cvLocationMarker}{\faMapMarker*}


% If your CV/résumé is in a language other than English,
% then you probably want to change these so that when you
% copy-paste from the PDF or run pdftotext, the location
% and date marker icons for \cvevent will paste as correct
% translations. For example Spanish:
% \renewcommand{\locationname}{Ubicación}
% \renewcommand{\datename}{Fecha}


%% Use (and optionally edit if necessary) this .tex if you
%% want to use an author-year reference style like APA(6)
%% for your publication list
% \input{pubs-authoryear.cfg}

%% Use (and optionally edit if necessary) this .tex if you
%% want an originally numerical reference style like IEEE
%% for your publication list
\input{pubs-num.cfg}

%% sample.bib contains your publications
\addbibresource{sample.bib}

\begin{document}
\name{Guillaume THÉMÈZE}
\tagline{Ph.D. in Quantum Physics}
%% You can add multiple photos on the left or right
%\photoR{2.8cm}{DSC_0549.JPG}
\photoR{4cm}{Portraits.pdf} % Replace with your picture
% \photoL{2.5cm}{Yacht_High,Suitcase_High}

\personalinfo{%
  % Not all of these are required!
  \email{guillaume.themeze@gmail.com}
  \phone{+33 6 25 69 08 44}
  %\mailaddress{Paris, France}
  \location{Paris, France}
  %\homepage{www.homepage.com}
  % \twitter{@twitterhandle}
  %\xtwitter{@x-handle}
  %\linkedin{your_id}
  \github{THEMEZE}
  %\orcid{0000-0000-0000-0000}
  %% You can add your own arbitrary detail with
  %% \printinfo{symbol}{detail}[optional hyperlink prefix]
  % \printinfo{\faPaw}{Hey ho!}[https://example.com/]

  %% Or you can declare your own field with
  %% \NewInfoFiled{fieldname}{symbol}[optional hyperlink prefix] and use it:
  %\NewInfoField{gitlab}{\faGitlab}[https://gitlab.com/]
  %\gitlab{your_id}
  %%
  %% For services and platforms like Mastodon where there isn't a
  %% straightforward relation between the user ID/nickname and the hyperlink,
  %% you can use \printinfo directly e.g.
  % \printinfo{\faMastodon}{@username@instace}[https://instance.url/@username]
  %% But if you absolutely want to create new dedicated info fields for
  %% such platforms, then use \NewInfoField* with a star:
  \NewInfoField*{mastodon}{\faMastodon}
  %% then you can use \mastodon, with TWO arguments where the 2nd argument is
  %% the full hyperlink.
  %\mastodon{@username@instance}{https://instance.url/@username}
}

\makecvheader
%% Depending on your tastes, you may want to make fonts of itemize environments slightly smaller
% \AtBeginEnvironment{itemize}{\small}

%% Set the left/right column width ratio to 5.5:4.5.
\columnratio{0.55}

% Start a 2-column paracol. Both the left and right columns will automatically
% break across pages if things get too long.
\begin{paracol}{2}

% ================= Research Experience ===================

\cvsection{Research Experience}

\cvevent{Ph.D. Research / Doctoral Researcher}{Institut d’Optique Graduate School (IOGS) / Charles Fabry Laboratory (LCF)}{2022--2025}{Palaiseau}
\begin{itemize}
\item Out-of-equilibrium quantum gases: combined theoretical, numerical and experimental work. Development of models, simulations, and local density measurements.
%\item See \cite{Dubois_2024,Dubois_2025,themeze2024seminar} for details.
\end{itemize}

\divider

\cvevent{Master 2 Internship / Research Intern}{(IOGS)/ (LCF)}{2022}{Palaiseau}
\begin{itemize}
\item Dynamics of a one-dimensional bosonic gas. Theoretical modeling, simulations, and experimental measurements.
\end{itemize}

\divider

\cvevent{Master 1 Internship / Research Intern}{Université Paris-Saclay}{2020}{Orsay}
\begin{itemize}
\item Bibliographic and numerical project: maximal expansion of Schwarzschild and Kerr black holes. Theoretical and numerical analysis during COVID lockdown.
\end{itemize}

\divider

\cvevent{Bachelor Internship / Research Intern}{l'École Polytechnique (l'X)  / Plasma Physics Laboratory (LPP) }{2019}{Palaiseau}
\begin{itemize}
\item Study of electric discharges in plasma. Theoretical and numerical modeling.
\end{itemize}

\divider

% ================= Teaching Experience ===================
\cvsection{Teaching Experience}
\cvevent{Teaching Assistant / TD/TP Instructor}{Institut Polytechnique de Paris (IP Paris) / (IOGS) /SupOptique \& Polytech }{2023--2025}{Palaiseau/Orsay}
\begin{itemize}
\item Tutorials in Quantum Mechanics and Signal Processing at SupOptique (BSc level).
\item Practical classes in lasers at Polytech (MSc level).
\item Over 150 hours of teaching experience.
\end{itemize}



% ================= Projects ===================
%\cvsection{Projects}

%\cvevent{Project 1}{Funding agency/institution}{}{}
%\begin{itemize}
%\item Details
%\end{itemize}

%\divider

%\cvevent{Project 2}{Funding agency/institution}{Project duration}{}
%A short abstract would also work.

%\medskip

% use ONLY \newpage if you want to force a page break for
% ONLY the current column
\newpage

% ================= A Day of My Life ===================
\cvsection{A Day of My Life}

% Adapted from @Jake's answer from http://tex.stackexchange.com/a/82729/226
% \wheelchart{outer radius}{inner radius}{
% comma-separated list of value/text width/color/detail}
\wheelchart{1.6cm}{0.5cm}{%
  6/8em/accent!30/{Sleep},
  7/10em/accent!60/{Research \& Thesis Work},
  3/8em/accent!50/{Teaching \& Mentoring},
  3/8em/accent!40/{Programming \& Development},
  2/8em/accent!20/{Sports \& Outdoor activities (Hiking, Paragliding)},
  3/8em/accent!35/{Creative hobbies}% (Drawing, Photography, Drone videos)
}

% ================= Most Proud of ===================
\cvsection{Most Proud of}

\cvachievement{\faTrophy}{Research Contribution}{Co-author of a Physical Review Letters article on 1D quantum gases, highlighting both theoretical and experimental expertise.}

\divider

\cvachievement{\faChalkboardTeacher}{Teaching Experience}{Over 150 hours of tutorials and practicals in quantum mechanics, signal processing, and laser physics at SupOptique and Polytech Paris.}

\divider

\cvachievement{\faProjectDiagram}{Interdisciplinary Skills}{Bridging physics, mathematics, and software development: numerical simulations, AI, and collaborative research projects.}

\divider

\cvachievement{\faGlobe}{International Collaboration}{Active participation in seminars, conferences, and teamwork in leading research groups on quantum gases.}


% ================= Strengths ===================

\cvsection{Strengths}

% Don't overuse these \cvtag boxes — they're just eye-candies and not essential. If something doesn't fit on a single line, it probably works better as part of an itemized list (probably inlined itemized list), or just as a comma-separated list of strengths.

\textbf{Physics \& Research:}
\cvtag{Mathematics} Pure mathematics (analysis, algebra, geometry, mathematical elegance); applied mathematics for physics (modeling, optimization, statistics).
\cvtag{Theoretical Physic} quantum mechanics and quantum gases; quantum field theory; statistical field theory; statistical physics; general relativity;personal interest in string theory, M-theory, supersymmetry, and quantum gravity.\\
\cvtag{Experimental Physics} laser physics, optics, and measurements on 1D bosonic gases.\\
\cvtag{Teaching} mathematics and physics tutorials, practicals, and lectures.

%\divider%\smallskip

\textbf{Programming:}
\cvtag{Python}
\cvtag{Julia}
\cvtag{C++}
\cvtag{\LaTeX /TikZ}
%High-Performance Computing (HPC)
numerical simulations, optimization\\

%\divider%\smallskip

\textbf{AI \& Data:}
\cvtag{Machine Learning \& Deep Learning}
\cvtag{algorithm optimization}
\cvtag{data analysis}\\

%\divider%\smallskip

\textbf{Software Development:}
\cvtag{Django (Python)}
\cvtag{Flutter (mobile apps)}
\cvtag{Web (HTML, CSS, JavaScript)}\\

%\divider%\smallskip

\textbf{Languages:}
\cvtag{French (native)}
\cvtag{English (scientific writing)}\\

%\divider%\smallskip

\textbf{Personal interests:}
\cvtag{Portrait drawing}
\cvtag{Photography}
\cvtag{Drone videography}
\cvtag{Hiking}
\cvtag{Paragliding}
\cvtag{DIY computing (Raspberry Pi, NAS)}\\


%% Switch to the right column. This will now automatically move to the second
%% page if the content is too long.
\switchcolumn

% ================= Education ===================

\cvsection{Education}

\cvevent{Ph.D.\ in Quantum Physics}{Institut d’Optique Graduate School (IOGS) / Charles Fabry Laboratory (LCF)}{2022--2025}{Palaiseau}
    Thesis on the out-of-equilibrium dynamics of one-dimensional bosonic gases.

\divider

\cvevent{M.Sc.\ Research-Oriented \ Master 2 Quantum, Light, Matter, Nanoscience (QLMN) }{(IOGS) / Université Paris-Saclay}{2021--2022}{Palaiseau}

\divider

\cvevent{M.Sc.\ Teacher-Oriented \ Master 2 Métiers de l'enseignement, de l'éducation et de la formation (MEEF) Physics-Chemistry }{École Normale Supérieure (ENS) Ulm \& Université Paris-Saclay}{2020--2021}{Paris/Montrouge}
    Teacher Training and Agrégation Preparation. Specialized program in pedagogy and preparation for the French national competitive exam (agrégation de physique-chimie). This year also validated the Magistère’s MSc requirement (M2 in Fundamental Physics).

\divider

\cvevent{B.Sc. \& M.Sc.\ Research-Oriented \ Magistère in Fundamental Physics (L3--M2)}{Université Paris-Sud}{2018--2021}{Orsay}
    Selective Excellence Program. Intensive training in theoretical and experimental physics, research internships, and complementary courses (languages, computing, management).

\divider

\cvevent{Preparatory Classes (MPSI/MP)}{Lycée Leconte de Lisle (Réunion) \& Lycée La Martinière (Lyon)}{2015--2018}{Lyon \& Réunion}
        Intensive training in Math and Physics.
        
%\divider

% use ONLY \newpage if you want to force a page break for
% ONLY the current column
\newpage


% ================= My Life Philosophy ===================

\cvsection{My Life Philosophy}

\begin{quote}
``Seeking elegance in both physics and mathematics, while turning complexity into clarity.''\\
%``Curiosity drives me to explore, rigor allows me to understand, and creativity helps me go further.''\\
%``I believe progress comes from the balance between deep focus, collaboration, and imagination.''\\
``Exploring the unknown with rigor, passion, and creativity.''\\
\end{quote}






% ================= Languages ===================

\cvsection{Languages}

\cvskill{French}{5}
\divider

\cvskill{English}{3}
\divider

\cvskill{Réunionnais}{5}
\divider



%\cvskill{German}{3.5} %% Supports X.5 values.

%% Yeah I didn't spend too much time making all the
%% spacing consistent... sorry. Use \smallskip, \medskip,
%% \bigskip, \vspace etc to make adjustments.
\medskip

% ================= Publications ===================
\cvsection{Publications}

%% Specify your last name(s) and first name(s) as given in the .bib to automatically bold your own name in the publications list.
%% One caveat: You need to write \bibnamedelima where there's a space in your name for this to work properly; or write \bibnamedelimi if you use initials in the .bib
%% You can specify multiple names, especially if you have changed your name or if you need to highlight multiple authors.
\mynames{Lim/Lian\bibnamedelima Tze,
  Wong/Lian\bibnamedelima Tze,
  Lim/Tracy,
  Lim/L.\bibnamedelimi T.}
%% MAKE SURE THERE IS NO SPACE AFTER THE FINAL NAME IN YOUR \mynames LIST

\nocite{*}

%\printbibliography[heading=pubtype,title={\printinfo{\faBook}{Books}},type=book]

%\divider

\printbibliography[heading=pubtype,title={\printinfo{\faFile*[regular]}{Journal Articles}},type=article]

\divider

\printbibliography[heading=pubtype,title={\printinfo{\faFile*[regular]}{Preprints}},type=misc]


\divider

\printbibliography[heading=pubtype,title={\printinfo{\faUsers}{Conference Proceedings}},type=inproceedings]



% ================= Referees ===================

\cvsection{Referees}

% \cvref{name}{email}{mailing address}
\cvref{Prof.\ Isabelle Bouchoule}{LCF}{isabelle.bouchoule@institutoptique.fr}
%{Address Line 1\\Address Line 2}

\divider

\cvref{MCF (HDR).\ Jean-Luc Raimbault}{l'X, LPP / Faculté des Sciences d’Orsay, Paris-Saclay}{jean-luc.raimbault@lpp.polytechnique.fr}
%{Address Line 1\\Address Line 2}


\end{paracol}


\end{document}
