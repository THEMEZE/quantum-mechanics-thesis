\chapter{Gyoto computations} \label{s:gyo}

\minitoc

\section{Introduction}


\textsf{Gyoto}\index{Gyoto} (\url{https://gyoto.obspm.fr}) is a free open-source
C++ code for computing orbits and ray-traced images in general relativity
\cite{VincePGP11}.
It has a Python interface and has the capability to integrate geodesics
not only in analytical spacetimes (such as Kerr) but also
in numerical ones, i.e. in spacetimes arising from numerical relativity.


\section{Image computations}

Here we provide the \textsf{Gyoto} input files, in XML format, that have been
used to produce the images shown in Chaps.~\ref{s:gis} and \ref{s:gik}.
To generate the images, it suffices to run \textsf{Gyoto} as
\begin{center}
\verb+gyoto input.xml output.fits+
\end{center}
By default, the computation is performed in parallel on 8 threads; you can
adapt to your CPU by changing the field \texttt{NThreads} in the file
\texttt{input.xml}. The output image is in FITS format and can be converted
to PNG or JPEG by most image processing programs, such as \textsf{GIMP}.

\subsection{Accretion disk around a Schwarzschild black hole} \label{s:gyo:Schwarz}

The input XML files for generating the images shown in Fig.~\ref{f:gis:disk_images}
are the files \verb+gis_disk*.xml+ in the directory\\
\url{https://github.com/egourgoulhon/BHLectures/tree/master/gyoto}


\subsection{Accretion disk around a Kerr black hole} \label{s:gyo:Kerr}

The input XML files for generating the images shown in Figs.~\ref{f:gik:img_disk_a50}
and \ref{f:gik:img_disk_a95}
are the files \verb+gik_a*.xml+ in the directory\\
\url{https://github.com/egourgoulhon/BHLectures/tree/master/gyoto}

