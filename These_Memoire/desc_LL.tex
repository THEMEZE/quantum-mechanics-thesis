Le gaz de Bose unidimensionnel est décrit par les operateurs champs quantiques de Bose canoniques  d'annihilation \( \Psi(x,t) \) et de création \( \Psi^\dag(x,t) \), satisfaisant les relations de commutation canoniques à temps égal $\left[ \operator{\Psi}(x,t), \operator{\Psi}^\dagger(y,t) \right] = \delta(x - y)$ et $ \left[ \operator{\Psi}(x,t), \operator{\Psi}(y,t) \right] = \left[ \operator{\Psi}^\dag(x,t), \operator{\Psi}^\dagger(y,t) \right] = 0$ . 


%\begin{eqnarray*}
%    \left[ \Psi(x,t), \Psi^\dagger(y,t) \right] &=& \delta(x - y), \\
%    \left[ \Psi(x,t), \Psi(y,t) \right] &=& \left[ \Psi^\dagger(x,t), \Psi^\dagger(y,t) \right] = 0
%\end{eqnarray*}




%Dans ce chapitre, l’argument temporel \( t \) sera omis, puisque toutes les considérations s’appliquent à un instant fixé dans le temps.

L’Hamiltonien du modèle s’écrit :

\begin{eqnarray*}
	\operator{H} & = & \int dx \left( \frac{\hbar^2}{2m} \partial_x \operator{\Psi}^\dagger \partial_x \operator{\Psi} + \frac{g}2  \operator{\Psi}^\dagger \operator{\Psi}^\dagger \operator{\Psi} \operator{\Psi} \right)	
\end{eqnarray*}


où \( m \) est la masse de la particule \( g \) est la constante de couplage 1D.  L’équation du mouvement correspondante 

\begin{eqnarray*}
	i\hbar \partial_t \operator{\Psi}  & = & -\frac{\hbar^2}{2m} \partial^2_x  \operator{\Psi}  + g 	\operator{\Psi}^\dag \operator{\Psi} \operator{\Psi}
\end{eqnarray*}

est l’équation de Schrödinger non linéaire (NS).

Les interraction entre particules sont répulsif donc $g > 0$. l’état fondamental à température nulle est une sphère de Fermi, et seul ce cas sera considéré par la suite.

Le vide de Fock \( |0\rangle \) est défini par $\Psi |0\rangle = 0$

%\[
%\Psi(x) |0\rangle = 0, \quad \forall x
%\]

%Il satisfait les relations suivantes :

%\[
%\Psi^\dagger(x) |0\rangle \neq 0, \quad \langle 0 | \Psi(x) = 0
%\]

%où le symbole dag (\(\dagger\)) désigne la conjugaison hermitienne.

Enfin, les opérateurs nombre de particules \( \operator{Q} \) et impulsion \( \operator{P} \) s’écrivent :

\begin{eqnarray*}
	\operator{Q}  &= & \int dx \, \operator{\Psi}^\dagger \operator{\Psi}\\
	\operator{P}  &=& \frac{1}{2} \int dx \, \operator{\Psi}^\dagger (-i \hbar \partial_x) \operator{\Psi	} + cc
\end{eqnarray*}

Ces opérateurs sont hermitiens et constituent des intégrales du mouvement $[\operator{H} , \operator{Q} ] = [\operator{H} , \operator{P}]  = 0 $ . Nous pouvons maintenant chercher les fonctions propres communes \( |\Psi_N\rangle \) des opérateurs \( \operator{H} \), \( \operator{P} \) et \( \operator{Q} \) :

\begin{eqnarray*}
	|\Psi_N ( \theta_1 , \cdots , \theta_N ) \rangle  & = &  \frac{1}{\sqrt{N!}} \int d^N x \, \varphi_N ( x_1 , \cdots , x_N \vert \theta_1 , \cdots , \theta_N ) \, \operator{\Psi}^\dag ( x_1 ) \cdots \operator{\Psi}^\dag ( x_N  ) \vert 0 \rangle 		
\end{eqnarray*}

Ici, \( \varphi_N \) est une fonction symétrique de toutes les variables \( x_j \). Les équations aux valeurs propres sont $\operator{H}|\Psi_N \rangle = E_N |\Psi_N \rangle $, $\operator{P}|\Psi_N \rangle = p_N |\Psi_N \rangle $ et $\operator{Q}|\Psi_N \rangle = N |\Psi_N \rangle $.

Il en résulte que \( \varphi_N \) est une fonction propre à la fois de l'Hamiltonien quantique  de Lieb et Liniger (LL)

\begin{eqnarray*}
	H_{LL} & =& \sum_{i = 1}^N \left \{  - \frac{ \hbar^2}{2m} \partial_{x_i}^2  + g  \sum_{j>i} \delta ( x_i - x_j ) \right \} \\
	H_{LL} \varphi_N & = & E_N \varphi_N 	
\end{eqnarray*}


et de l'opérateur de moment quantique \( P_N  = - i \hbar \sum_{i = 1}^N  \partial_{x_i} \) :
 





















