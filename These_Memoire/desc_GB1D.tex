{\color{red}Le gaz de Bose unidimensionnel avec interaction ponctuelle des particules (la variante quantique de l'équation de Schrödinger non linéaire)} est l'un des modèles principaux et les plus importants qui peut être résolu par la méthode de l'Ansatz de Bethe [14], [15]. Ce modèle a été minutieusement étudié ([1], [5], [17], [21] et [22]). {\color{red} Nous commencerons par la construction des fonctions propres de l'Hamiltonien dans un volume fini. Les quantités intéressantes d'un point de vue physique (dans la limite thermodynamique à température nulle) sont considérées ; la thermodynamique à température finie est également étudiée en détail. Un certain nombre d'idées essentielles qui seront appliquées à d'autres modèles sont introduites.}

{\color{red} La construction des fonctions propres de l'Hamiltonien est expliquée dans la section 1. Leur forme explicite et, en particulier, la réductibilité à deux particules, sont des caractéristiques communes des modèles résolubles par la méthode de l'Ansatz de Bethe. Des conditions aux limites périodiques sont imposées à la fonction d'onde dans la section 2 ; les équations de Bethe pour les moments des particules sont introduites et analysées. Pris sous forme logarithmique, ces équations réalisent la condition d'extrémum d'un certain fonctionnel, l'action correspondante étant appelée l'action de Yang-Yang. La transition vers la limite thermodynamique est envisagée dans la section 3. Dans cette même section, l'état fondamental du gaz est construit. La densité de distribution des particules dans l'espace des moments et l'énergie de l'état fondamental sont calculées. La méthode de transition vers la limite thermodynamique décrite dans cette section est assez générale et peut être appliquée à tout modèle résoluble par l'Ansatz de Bethe. Dans la section 4, les excitations au-dessus de l'état fondamental sont construites et leurs principales caractéristiques (énergie, moment et matrice de diffusion) sont déterminées à l'aide des équations de "dressing". L'état fondamental du modèle est la mer de Dirac (également appelée sphère de Fermi).}

{\color{red} La thermodynamique du modèle est présentée dans la section 5. L'approche par intégrale fonctionnelle est présentée. Elle permet de résoudre divers problèmes à température finie. Les équations de base décrivant l'état d'équilibre thermodynamique, l'équation de Yang-Yang en étant une d'entre elles, sont dérivées dans cette section. L'équation de Yang-Yang, qui est une équation intégrale non linéaire, est analysée dans la section 6. Le théorème montrant l'existence de solutions est prouvé.} L'état d'équilibre thermodynamique avec température tendant vers zéro est étudié dans la section 7. En examinant cette limite, nous pouvons obtenir des informations plus détaillées sur l'état fondamental de l'Hamiltonien à température nulle. La limite de couplage fort (dans laquelle le modèle est équivalent au modèle de fermions libres) est discutée. Les équations intégrales sont résolues exactement dans cette limite. Les excitations au-dessus de l'état d'équilibre thermodynamique sont étudiées et leur interprétation en termes de particules est donnée. Il est important de noter que la formule à température finie et à température nulle diffèrent uniquement par la mesure d'intégration. La thermodynamique des modèles exactement résolvables est tellement particulière qu'il est possible de construire des excitations stables à température finie, voir section 8. Les corrélations thermiques sont également discutées dans la section 8. Pour les modèles exactement résolvables, elles sont également très particulières : elles peuvent être représentées sous une forme similaire à celle à température nulle. Plus tard, dans la Partie IV (Chapitres XIII-XVI), cela sera utilisé pour l'évaluation explicite des fonctions de corrélation à température (même si elles dépendent du temps). La section 9 est consacrée à l'évaluation des corrections de taille finie à température nulle. Plus tard, elles seront utilisées pour le calcul des asymptotiques de longue distance des fonctions de corrélation à l'aide de la théorie des champs conformes.