\documentclass{standalone} % Utilise la classe standalone pour compiler uniquement le TikZ
%\usepackage{tkz-euclide}
\usetikzlibrary{intersections}
%\usetkzobj{all}
\usepackage[utf8]{inputenc}
\usepackage[frenchb]{babel}
\usepackage[babel=true,kerning=true]{microtype}
\usepackage{lmodern,bm}
\usepackage[T1]{fontenc}

%http://pgfplots.sourceforge.net/gallery.html
%\usepackage[usenames, dvipsnames,svgnames, x11names]{xcolor}

\usepackage{pgfplots}
\usepackage{graphics}

\usepackage{amsmath}
\usepackage{amsfonts}
\usepackage{esint} % package de symboles mathématiques
\usepackage{amssymb}
\usepackage{cancel}
 
\usepackage{tikz}
\usepackage[europeanresistor]{circuitikz}

%prévisualisation dessin par dessin
\usepackage[active,tightpage]{preview}
\PreviewEnvironment{tikzpicture}
\setlength\PreviewBorder{5pt}
%fin

\usetikzlibrary{decorations.markings,decorations.pathmorphing,decorations.pathreplacing}
\usetikzlibrary{calc,patterns,shapes.geometric}
%\tikzstyle arrowstyle=[scale=2] %taille des flèches

\usetikzlibrary{arrows,shapes,positioning}
\tikzstyle arrowstyle=[scale=1] %taille des flèches
\usetikzlibrary{fadings}
\tikzset{verre/.style={draw=SkyBlue,fill=SkyBlue!30}}
\tikzstyle simple=[postaction={decorate,decoration={markings,
    mark=at position .5 with {\arrow[scale=1,draw=red,>=stealth]{>}}}}]
\tikzstyle simplerev=[postaction={decorate,decoration={markings,
    mark=at position .5 with {\arrow[scale=1,draw=red,>=stealth]{<}}}}]

\tikzset{verre/.style={draw=SkyBlue,fill=SkyBlue!30}}
\usepackage{pgfplots}
\usetikzlibrary{decorations.pathreplacing,calligraphy,backgrounds}

\usetikzlibrary{arrows,snakes,backgrounds}

\usepackage{tikz}          % Charge le package TikZ
\usepackage{xcolor}        % Gestion avancée des couleurs
\usepackage{amsmath}       % Gestion des formules mathématiques
\usetikzlibrary{arrows.meta} % Pour gérer les flèches avec des formes spécifiées comme "triangle"
\usetikzlibrary {decorations.pathmorphing}
\usepackage{pgfplots}
\pgfplotsset{compat=1.18}
\usepackage[active,tightpage]{preview}
\PreviewEnvironment{tikzpicture}
\setlength\PreviewBorder{5pt}
%\usepgfplotslibrary{external}
%\tikzexternalize

%% Automatiser cela : une solution plus élégante  nouvelle page, tu peux utiliser la commande \newpage juste avant chaque \begin{tikzpicture}
%\let\oldtikzpicture\tikzpicture
%\let\endoldtikzpicture\endtikzpicture
%\renewenvironment{tikzpicture}{\newpage\oldtikzpicture}{\endoldtikzpicture}
%\usepackage{etoolbox} % dans le préambule, si pas encore utilisé
%\pretocmd{\tikzpicture}{\clearpage}{}{}


% Définition de la fonction pour générer la grille
\newcommand{\drawgrid}[4]{


	\begin{scope}[opacity = 0.2]
	
	
    	% Paramètres : #1=xmin, #2=xmax, #3=ymin, #4=ymax

   	 	% Grille principale en cm
    	\draw[very thin, gray] (#1,#3) grid (#2,#4); % Grille de base (1 cm)
    
    	\draw[very thin, gray] (#1,#3) grid[step=0.1] (#2,#4);

    	\draw[thin, black] (#1,#3) edge[thick,line width=0.8ex,->,>=Stealth ] (#2,#3); 
    
    	\foreach \x in {#1,..., #2} {
        	\draw[shift={(\x, #3)}] (0,-0.1) edge[line width=0.8ex] node[pos = -1 , below] {\x} (0,0.1); % Lignes verticales en mm
    	}

    	\draw[thin, black] (#1,#3)edge[thick,line width=0.8ex,->,>=Stealth ] (#1,#4); 
    
    	\foreach \y in {#3,..., #4} {
        	\draw[shift={(#1, \y)}] (-0.1 , 0 ) edge[line width=0.8ex] node[pos = -1 , left ] {\y} (0.1 , 0); % Lignes verticales en mm
    	}
    
    \end{scope}


}


\newcommand{\Axex}[1][$x$]{ % "x" est la valeur par défaut de #1
    \draw
        (-10, 0) edge[thick, line width=0.8ex, ->, >=Triangle, color=\colorThree] 
        node[shift = {(0,1)} , pos=1.01, below]{ \huge #1 } (10, 0)
        ;
}

\newcommand{\Axetheta}[1][$\theta$]{
	\draw
		(-10, 0 ) edge [thick,line width=0.8ex,->,>=Triangle , color = \colorThree ]node [pos=1.01,right]{\huge #1 } ( 10  , 0 )
	;
}

\newcommand{\Axedensity}[1][$n$]{
	\draw
		(0, -1 ) edge [thick,line width=0.8ex,->,>=Triangle , color = \colorThree ]node [pos=1.01,left  ]{\huge #1 } ( 0  , 6 )
	;
}

\newcommand{\Axedistribution}[1][$\nu$]{
	\draw
		(0, -1 ) edge [thick,line width=0.8ex,->,>=Triangle , color = \colorThree ]node [pos=1.01,left  ]{\huge #1 } ( 0  , 6 )
	;
}

\newcommand{\Axesphase}[2][$x$]{
	\Axex[#1]
	\begin{scope}[rotate = 90]
		\Axetheta[#2]	
	\end{scope}
}

\newcommand{\Axesdensity}[2][$x$]{
	\Axex[#1]
	\Axedensity[#2]
}

\newcommand{\Axesdistribution}[2][\theta]{
	\Axetheta[#1]
	\Axedistribution[#2]
}

\newcommand{\Axetemps}[1][Deformation]{
	\draw 
		(-5 , 0 ) edge [line width=2.8ex, color=\colorOne ]( 0, 0)
		(43 , 0 ) edge [line width=2.8ex, color=\colorOne , ->, >=Triangle] node[pos = 1.0 , right , scale = 2 ] { \fontsize{60pt}{720pt}\selectfont $\tilde{t}$} ( 50 , 0)
		( 0 , 1 ) edge [line width=2.0ex, color=\colorOne ] ( 0 , -1 )  
		( 43 , 1 ) edge [line width=2.0ex, color=\colorOne ] ( 43 , -1 ) 
		(0 , 0) edge[<->, > = stealth, line width=2.8ex, color=\colorOne ]  node [ rectangle , midway, fill = white , scale = 2 ] {\fontsize{600pt}{720pt}\selectfont \bf  #1}(43, 0)  
	;

}

\newcommand{\Palette}[1][\colorOne]{
	\clip[decorate, decoration={random steps, segment length=3pt, amplitude=3pt}]
        	(-1,-1) rectangle (1,1);
	\draw[fill=#1 , color = #1 ] (-2,-2) rectangle (2,2);
}






% Définition des couleurs avec les codes HTML
\definecolor{colorOne}{HTML}{443E46}
%\definecolor{colorTwo}{HTML}{F6DEB8}
\definecolor{colorTwo}{HTML}{FFFFFF}

\definecolor{colorThree}{HTML}{908CA4}
\definecolor{colorFour}{HTML}{57659E}
\definecolor{colorFive}{HTML}{C57284}
\definecolor{colorSix}{HTML}{FF5B69}

% Raccourcis pour les couleurs
\def\colorOne{colorOne}
\def\colorTwo{colorTwo}
\def\colorThree{colorThree}
\def\colorFour{colorFour}
\def\colorFive{colorFive}
\def\colorSix{colorSix}



% Graduation ordonné
	
\newcommand{\GraduationYMm}[2]{
	\foreach \r in {1,...,9} {
		\draw 
			({-0.25 * 0.5 * (cos(3.14*0.2*\r r)^2 + 1)}, #1*#2 - #2 + 0.1*#2*\r) 
			edge [thick, line width=0.3ex, color=\colorOne] 
			({0.25 * 0.5 * (cos(3.14*0.2*\r r)^2 + 1)}, #1*#2 - #2 + 0.1*#2*\r);
	}
}

\newcommand{\GraduationYCm}[4]{% #1: min, #2: max, #3: liste
  \foreach \R in {#1,...,#2} {
    \pgfmathparse{#3[\R-(#1)]} % index dans la liste
    \let\labelText\pgfmathresult

    \draw[color=\colorThree]
      (-0.3, \R*#4)
      edge [thick, line width=0.5ex, color=\colorOne]
      node [pos=-0.0, left]{\Large \labelText}
      (0.3, \R*#4);

    %\GraduationYMm{\R}{#4}
  }

  % Graduation supplémentaire à #2 + 3
  \pgfmathparse{#2 + 1}
  \edef\extendedPosition{\pgfmathresult}
  %\GraduationYMm{\extendedPosition}{#4}
}

\newcommand{\GraduationXMm}[2]{
	\foreach \r in {1,...,9} {
		\draw 
			( #1*#2 - #2 + 0.1*#2*\r,{-0.25 * 0.5 * (cos(3.14*0.2*\r r)^2 + 1)}) 
			edge [thick, line width=0.3ex, color=\colorOne] 
			( #1*#2 - #2 + 0.1*#2*\r,{0.25 * 0.5 * (cos(3.14*0.2*\r r)^2 + 1)});
	}
}

\newcommand{\GraduationXCm}[4]{% #1: min, #2: max, #3: liste
  \foreach \R in {#1,...,#2} {
    \pgfmathparse{#3[\R-(#1)]} % index dans la liste
    \let\labelText\pgfmathresult

    \draw[color=\colorThree]
      (\R*#4,-0.3)
      edge [thick, line width=0.5ex, color=\colorOne]
      node [pos=-0.0, below]{\Large \labelText}
      (\R*#4,0.3);

    %\GraduationXMm{\R}{#4}
  }

  % Graduation supplémentaire à #2 + 3
  \pgfmathparse{#2 + 1}
  \edef\extendedPosition{\pgfmathresult}
  %\GraduationXMm{\extendedPosition}{#4}
}

% Définition de la fonction pour générer la grille
\newcommand{\backgrowngrid}[6]{


	\begin{scope}[xscale = #5 , yscale = #6 ]
	
	
    	% Paramètres : #1=xmin, #3=xmax, #2=ymin, #4=ymax
    	%\clip[scale = 0.95 , decorate, decoration={random steps, segment length=0.1 cm, amplitude=0.2cm}](#1 , #2 ) rectangle ( #3 , #4 ) ;
    	\pgfmathparse{#1-1}  
    	\let\valXmin\pgfmathresult
    	\pgfmathparse{#2-1}  
    	\let\valYmin\pgfmathresult
    	\pgfmathparse{#3+1}  
    	\let\valXmax\pgfmathresult
    	\pgfmathparse{#4+1}  
    	\let\valYmax\pgfmathresult
    	
    	\draw[fill = \colorThree , opacity = 0.1]  (\valXmin , \valYmin ) rectangle ( \valXmax , \valYmax ) ;

   	 	% Grille principale en cm
    	\draw[very thin, line width=0.5ex ,  color=\colorTwo] (\valXmin , \valYmin ) grid ( \valXmax , \valYmax ); % Grille de base (#5 cm , #6 cm )
    
    	%\draw[very thin, line width=0.3ex ,  color=\colorTwo] (#1,#2) grid[step=0.1] (#3,#4);

    	%\draw[thin, black] (#1,#2) edge[thick,line width=0.8ex,->,>=Stealth ] (#3,#2); 
    
    	%\foreach \x in {#1,..., #3} {
    	%	\pgfmathparse{int(\x * #5)} % calcule et arrondit \pgfmathparse{round((\x * #5)/10)*10} Par exemple, au multiple de 10 le plus proche 
    	%	\let\val\pgfmathresult
        %	\draw[shift={(\x, #2)}] (0,-0.1) edge[line width=0.8ex] node[pos = -1 , below] {\val} (0,0.1); % Lignes verticales en mm
    	%}

    	%\draw[thin, black] (#1,#2)edge[thick,line width=0.8ex,->,>=Stealth ] (#1,#4); 
    
    	%\foreach \y in {#2,..., #4} {
    	%	\pgfmathparse{int(\y * #6)} % calcule et arrondit  
    	%	\let\val\pgfmathresult
        %	\draw[shift={(#1, \y)}] (-0.1 , 0 ) edge[line width=0.8ex] node[pos = -1 , left ] {\val} (0.1 , 0); % Lignes verticales en mm
    	%}
    
    \end{scope}


}



\begin{document}


%article_distribution_24-04-2024
\begin{tikzpicture}

	%\clip[scale = 0.95 , decorate, decoration={random steps, segment length=0.1 cm, amplitude=0.2cm}](#1 , #2 ) rectangle ( #3 , #4 ) ;
	\clip[](-14 , -3) rectangle (9.5 , 12) ;
	
	% Définition de l’échelle
	\def\echelleY{10}
	\pgfmathsetmacro{\echelleYCm}{11/6}
	\def\echelleX{50}
	\pgfmathsetmacro{\echelleXCm}{20/7}

	% Définition de la liste
	\def\listeLabelsY{{0,10,20,30,40,50,60}}
	\def\listeLabelsX{{-200,-150,-100,-50,0,50,100,150}}
	
	
	
	\pgfmathparse{-4*\echelleXCm}  
    \let\valXmin\pgfmathresult

    
    \begin{scope}
    	\pgfmathparse{-4*\echelleXCm}  
    	\let\valXmin\pgfmathresult
    	\clip[](\valXmin , -1) rectangle (9.2 , 11.5) ;
    	\backgrowngrid{-4}{0}{3}{6}{\echelleXCm}{\echelleYCm}
    	
    	\draw[
  			shift={(-11cm,6cm)},
  			line width=0.5ex,
  			draw=\colorThree,
  			fill=\colorThree!50!\colorTwo,
  			rounded corners=6pt,
  			opacity = 0.2
			] 
  			(0,0) rectangle (10.7,5);
  			
    	\draw[line width=0.8ex  ,color=\colorFive ,  fill = none] 	
			plot[smooth] file {data/donnee_bord_article_distribution_24-04-2024.table}
			(-10.5,10) edge[] node[shift = {(0,0)} , pos=1, right ]{\fontsize{19pt}{19pt}\selectfont \bf Profil de bord } (-9,10)
		;
		\draw[line width=1.0ex  ,color=\colorSix ,  fill = none] 	
			plot[] file {data/GHD_bord_article_distribution_24-04-2024.table}
			(-10.5,9) edge[] node[shift = {(0,0)} , pos=1, right ]{\fontsize{19pt}{19pt}\selectfont \bf Ajustement : T = 560~nK } (-9,9)
		;
		\draw[line width=1.0ex  ,color=\colorThree ,  fill = none] 	
			plot[] file {data/GHD_bord_2_article_distribution_24-04-2024.table}
			(-10.5,7) edge[] node[shift = {(0,0)} , pos=1, right ]{\fontsize{19pt}{19pt}\selectfont \bf   T = 1550~nK } (-9,7)
		;
		\draw[line width=0.8ex  ,color=\colorFour ,  fill = none] 
			plot[smooth] file {data/donnee_selection_1_article_distribution_24-04-2024.table}
			(-10.5,8) edge[] node[shift = {(0,0)} , pos=1, right ]{\fontsize{19pt}{19pt}\selectfont \bf Sélection} (-9,8)
		;
    \end{scope}
    
    
    % Graduations     
	\begin{scope}[shift = {(\valXmin,0)}]
		\GraduationYCm{0}{6}{\listeLabelsY}{\echelleYCm}	
	\end{scope}			
	\begin{scope}[shift = {(0,-1)}]	
		\GraduationXCm{-4}{3}{\listeLabelsX}{\echelleXCm}
	\end{scope}
	
	
	% Axxes 
	\draw[shift = {(0,-1)}](\valXmin, 0) edge[thick, line width=0.8ex, ->, >=Triangle, color=\colorOne] node[shift = {(0,-1)} , pos=0.5, below]{ \huge $x~(\mu m)$ } (9.5, 0);
    \draw   (\valXmin, -1) edge[thick, line width=0.8ex, ->, >=Triangle, color=\colorOne] node[shift = {(-1,0)} , pos=0.5, above , rotate = 90 ]{ \huge $n~({\mu m}^{-1})$ } (\valXmin, 12)
		
		
	;
	
	%\draw[line width=2.5ex , color = red ] (-14 , -1) rectangle (9.5 , 12) ;
	%\drawgrid{-13}{9.5}{-1}{12}
	
\end{tikzpicture}


%Assymetrie
\begin{tikzpicture}

	%\clip[scale = 0.95 , decorate, decoration={random steps, segment length=0.1 cm, amplitude=0.2cm}](#1 , #2 ) rectangle ( #3 , #4 ) ;
	\clip[](-12 , -4) rectangle (11 , 12) ;
	
	% Définition de l’échelle
	%\def\echelleY{10}
	\pgfmathsetmacro{\echelleYCm}{9/3}
	%\def\echelleX{50}
	\pgfmathsetmacro{\echelleXCm}{16/4}

	% Définition de la liste
	\def\listeLabelsY{{0,1,2,3}}
	\def\listeLabelsX{{-400,-200,0,200,400}}
	
	

    
    \begin{scope}

    	\clip[](-10 , -2) rectangle (10, 11.5) ;
    	\backgrowngrid{-2}{0}{2}{3}{\echelleXCm}{\echelleYCm}
    	
    	\draw[
  			shift={(2cm,8cm)},
  			line width=0.5ex,
  			draw=\colorThree,
  			fill=\colorThree!50!\colorTwo,
  			rounded corners=6pt,
  			opacity = 0.2
			] 
  			(0,0) rectangle (7,3);
  			
    	\draw[line width=0.8ex  ,color=\colorFour ,  fill = none] 	
			plot[smooth] file {data/donnee_article_asymetrie_24-04-202.table}
			(2.5,10) edge[] node[shift = {(0,0)} , pos=1, right ]{\fontsize{25pt}{19pt}\selectfont \bf $\tilde{n}(x)$ } (4,10)
		;
		\draw[line width=0.8ex  ,color=\colorFive,  fill = none] 	
			plot[smooth] file {data/donnee_2_article_asymetrie_24-04-202.table}
			(2.5,9) edge[] node[shift = {(0,0)} , pos=1, right ]{\fontsize{25pt}{19pt}\selectfont \bf $\tilde{n}(x-2x_s)$  } (4,9)
		;
    \end{scope}
    
    
    % Graduations     
	\begin{scope}[shift = {(-10,0)}]
		\GraduationYCm{0}{3}{\listeLabelsY}{\echelleYCm}	
	\end{scope}			
	\begin{scope}[shift = {(0,-2)}]	
		\GraduationXCm{-2}{2}{\listeLabelsX}{\echelleXCm}
	\end{scope}
	
	
	% Axxes 
	\draw[shift = {(0,-2)}](-10, 0) edge[thick, line width=0.8ex, ->, >=Triangle, color=\colorOne] node[shift = {(0,-1)} , pos=0.5, below]{ \huge $x~(\mu m)$ } (11, 0);
    \draw [shift = {(-10,0)}]  (0, -2) edge[thick, line width=0.8ex, ->, >=Triangle, color=\colorOne] node[shift = {(-1,0)} , pos=0.5, above , rotate = 90 ]{ \huge $n~({\mu m}^{-1})$ } (0, 12)
		
		
	;
	
	%\draw[line width=2.5ex , color = red ] (-14 , -1) rectangle (9.5 , 12) ;
	%\drawgrid{-11}{13}{-4}{12}
		
\end{tikzpicture}


%theorie 
\begin{tikzpicture}
	%\clip[scale = 0.95 , decorate, decoration={random steps, segment length=0.1 cm, amplitude=0.2cm}](#1 , #2 ) rectangle ( #3 , #4 ) ;
	%\clip[](-12 , -4) rectangle (11 , 12) ;
	
	% Définition de l’échelle
	%\def\echelleY{10}
	\pgfmathsetmacro{\echelleYCm}{10/6}
	%\def\echelleX{50}
	\pgfmathsetmacro{\echelleXCm}{10/4}

	% Définition de la liste
	\def\listeLabelsY{{0,20,40,60,80,100,120}}
	\def\listeLabelsX{{-12.5,-10,-7.5,-5,-2.5 , 0 , 2.5, 5}}
	
	

    
    \begin{scope}

    	%\clip[](-10 , -2) rectangle (10, 11.5) ;
    	\backgrowngrid{-4}{0}{2}{6}{\echelleXCm}{\echelleYCm}
    	
    	\draw[
  			shift={(2cm,8cm)},
  			line width=0.5ex,
  			draw=\colorThree,
  			fill=\colorThree!50!\colorTwo,
  			rounded corners=6pt,
  			opacity = 0.2
			] 
  			(0,0) rectangle (7,3);
  			
    	\draw[line width=0.8ex  ,color=\colorFour ,  fill = none] 	
			plot[smooth] file {data/GHD_article_distribution_24-04-2024.table}
			(2.5,10) edge[] node[shift = {(0,0)} , pos=1, right ]{\fontsize{25pt}{19pt}\selectfont \bf $\tau \times n_{\mbox{\tiny GHD}}(\tau \times \cdot)$ } (4,10)
		;
		\draw[line width=0.8ex  ,color=\colorFive,  fill = none] 	
			plot[smooth] file {data/Pi_article_distribution_24-04-2024.table}
			(2.5,9) edge[] node[shift = {(0,0)} , pos=1, right ]{\fontsize{25pt}{19pt}\selectfont \bf $\Pi(\cdot + x_0 \div \tau ) $  } (4,9)
		;
		\draw[line width=0.8ex  ,color=\colorOne, dashed ,  fill = none] 	
			plot[smooth] file {data/rho_article_distribution_24-04-2024.table}
			(2.5,8) edge[] node[shift = {(0,0)} , pos=1, right ]{\fontsize{25pt}{19pt}\selectfont \bf $\ell \times \rho ( \cdot + x_0 \div \tau ) $  } (4,8)
		;
    \end{scope}
    
    
    % Graduations     
	\begin{scope}[shift = {(-13,0)}]
		\GraduationYCm{0}{6}{\listeLabelsY}{\echelleYCm}	
	\end{scope}			
	\begin{scope}[shift = {(0,-1)}]	
		\GraduationXCm{-5}{2}{\listeLabelsX}{\echelleXCm}
	\end{scope}
	
	
	% Axxes 
	\draw[shift = {(-3,-1)}](-10, 0) edge[thick, line width=0.8ex, ->, >=Triangle, color=\colorOne] node[shift = {(0,-1)} , pos=0.5, below]{ \huge $\mu m / m s $ } (11, 0);
    \draw [shift = {(-13,0)}]  (0, -1) edge[thick, line width=0.8ex, ->, >=Triangle, color=\colorOne] node[shift = {(-1,0)} , pos=0.5, above , rotate = 90 ]{ \huge $ms/\mu m$ } (0, 12)
		
		
	;
	
	%\draw[line width=2.5ex , color = red ] (-14 , -1) rectangle (9.5 , 12) ;
	\drawgrid{-13}{13}{-4}{12}
	
\end{tikzpicture}






%\tikz \InsitutDiagram ; 


\end{document}
