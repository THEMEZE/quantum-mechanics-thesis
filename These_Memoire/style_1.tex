% !TEX encoding = IsoLatin                     
%cette premiÃère ligne permet au compilateur (souvent, dÃépend du compilateur) \LaTeX de reconnaitre l'encodage de votre fichier. 
\documentclass[10pt,titlepage,twoside]{book}       %pour un livre
%\documentclass{report}
%\documentclass[a3, 10pt,twoside]{article}          %pour un article
%\documentclass[twocolumn,amsfonts,showpacs,superscriptaddress,nofootinbib,aps,nolongbibliography]{revtex4-2}

\usepackage{array,multirow,graphicx}
\usepackage[T1]{fontenc}  % For correct {}s in \texttt
\usepackage{accents}
\usepackage[french]{babel}
\let\dddot\relax
\let\ddddot\relax
\usepackage{amsmath, amssymb, amsthm}
\usepackage[utf8]{inputenc}             
\usepackage{amstext, amsfonts, a4}
\usepackage{pgf,pgfarrows,pgfautomata,pgfheaps,pgfnodes,pgfshade}
\usepackage[left=0.05cm,right=0.05cm,top=0.05cm,bottom=0.05cm,nohead]{geometry}
\usepackage[pdfborder={0 0 0}]{hyperref}
\usepackage{xxcolor}
\usepackage{amscd}
\usepackage{dsfont}
\usepackage{float} 
\usepackage[Glenn]{fncychap} % Rejne Glenn\chapter
%\usetikzlibrary{circuits.ee.IEC}
\usepackage{minitoc}  % Charge le package
\dominitoc  % Active l'affichage des mini-sommaires
\usepackage{lipsum} % pour générer du texte aléatoire
%\usepackage{makeidx}


%\usepackage[tikz]{bclogo}
\usepackage{mathrsfs}%lagrangien
\usepackage{stmaryrd} %\llbracket
\usepackage{bbold}%Matrice 11
\usepackage{shapepar}
\usepackage{cancel}
\usepackage{bm}
\usepackage{pdfpages}
\usepackage[utf8]{inputenc}
\usepackage{marvosym}
\usepackage{enumitem}
\usepackage{import} % Charger le package import
\usepackage{listings} % Pour afficher du code MATLAB
\usepackage{pgfplots}
\usetikzlibrary {datavisualization}
\usetikzlibrary {arrows.meta,bending,positioning}
\usetikzlibrary {datavisualization.formats.functions}

%%%%%%%%%%%%%%%%%%%%%%%%%%%%%%%%% Pour le tag
%\usepackage[utf8]{inputenc}
%\usepackage{fourier}
\usepackage{mathtools}
%\usepackage{cleveref}
%\usepackage{xcolor}
%%%%%%%%%%%%%%%%%%%%%%%%%%%%%%%%%%


%%%%%%%%%%%%%%%%%%%%%%%%%%%%%%%%%%
\newcommand{\operatorvec}[1]{\hat{{\bm{#1}}}} % pour les operateur
\newcommand{\operator}[1]{\hat{\bm{#1}}} % pour les operaeur vecteur 
\newcommand{\operatormat}[1]{\operatorname{#1}} % pour les operaeur vecteur 
\newcommand{\operatortilde}[1]{\tilde{\bm{#1}}} % pour les opetateur avec un tilde
\newcommand{\operatortildevec}[1]{\tilde{\bm{#1}}}% pour les opetateur avec un tilde et vecteur 
%%%%%%%%%%%%%%%%%%%%%%%%%%%%%%%%%%

%\usepackage{tzplot}

%\usepakage{pstriks}
\usepackage[utf8]{inputenc}

%PREAMBULE pour schÃéma
\usepackage{pgfplots}
\usepackage{tikz}
\usepackage[european resistor, european voltage, european current]{circuitikz}
\usetikzlibrary{arrows,shapes,positioning}
\usetikzlibrary{decorations.markings,decorations.pathmorphing,
decorations.pathreplacing}
\usetikzlibrary{calc,patterns,shapes.geometric}
%FIN PREAMBULE

% PREAMBULE pour vercteur derivÃé
\usepackage[b]{esvect}    % For \vv

%\usepackage{caption}
%\usepackage{subcaption}

%\usepackage{graphicx}

\usepackage{fullpage}
\usepackage{eso-pic}
\usepackage{enumitem}

%\usepackage{pgfplots}
%\pgfplotsset{compat=1.17}







% --- Macro \xvec
\makeatletter
\newlength\xvec@height%
\newlength\xvec@depth%
\newlength\xvec@width%
\newcommand{\xvec}[2][]{%
  \ifmmode%
    \settoheight{\xvec@height}{$#2$}%
    \settodepth{\xvec@depth}{$#2$}%
    \settowidth{\xvec@width}{$#2$}%
  \else%
    \settoheight{\xvec@height}{#2}%
    \settodepth{\xvec@depth}{#2}%
    \settowidth{\xvec@width}{#2}%
  \fi%
  \def\xvec@arg{#1}%
  \def\xvec@dd{:}%
  \def\xvec@d{.}%
  \raisebox{.2ex}{\raisebox{\xvec@height}{\rlap{%
    \kern.05em%  (Because left edge of drawing is at .05em)
    \begin{tikzpicture}[scale=1]
    \pgfsetroundcap
    \draw (.05em,0)--(\xvec@width-.05em,0);
    \draw (\xvec@width-.05em,0)--(\xvec@width-.15em, .075em);
    \draw (\xvec@width-.05em,0)--(\xvec@width-.15em,-.075em);
    \ifx\xvec@arg\xvec@d%
      \fill(\xvec@width*.45,.5ex) circle (.5pt);%
    \else\ifx\xvec@arg\xvec@dd%
      \fill(\xvec@width*.30,.5ex) circle (.5pt);%
      \fill(\xvec@width*.65,.5ex) circle (.5pt);%
    \fi\fi%
    \end{tikzpicture}%
  }}}%
  #2%
}
\makeatother

% --- Override \vec with an invocation of \xvec.
\let\stdvec\vec
\renewcommand{\vec}[1]{\xvec[]{#1}}
% --- Define \dvec and \ddvec for dotted and double-dotted vectors.
\newcommand{\dvec}[1]{\xvec[.]{#1}}
\newcommand{\ddvec}[1]{\xvec[:]{#1}}
% FIN PREAMBULE


\def\pgf{\textsc{pgf}}
\def\pstricks{\textsc{pstricks}}
\def\Class#1{\hbox{\small#1}}
\def\bs{$\backslash$}

\def\Environment#1{\par\bigskip\noindent\textbf{Environment \texttt{#1}}\par}
\def\Command#1{\par\bigskip\noindent\textbf{Command \texttt{#1}}\par}
\long\def\Parameters#1{\medskip\noindent Parameters:
  \begin{enumerate}\itemsep=0pt\parskip=0pt
    #1
  \end{enumerate}}
\long\def\Description#1{\unskip\medskip\noindent Description: #1}
\def\Example{\par\medskip\noindent Example: }


\renewcommand*\descriptionlabel[1]{\hspace\labelsep\normalfont #1}


\def\declare#1{{\color{red!75!black}#1}}

\def\command#1{\list{}{\leftmargin=2em\itemindent-\leftmargin\def\makelabel##1{\hss##1}}%
\item\extractcommand#1@\par\topsep=0pt}
\def\endcommand{\endlist}
\def\extractcommand#1#2@{\strut\declare{\texttt{\string#1}}#2}


\def\environment#1{\list{}{\leftmargin=2em\itemindent-\leftmargin\def\makelabel##1{\hss##1}}%
\extractenvironement#1@\par\topsep=0pt}
\def\endenvironment{\endlist}
\def\extractenvironement#1#2@{%
\item{{\ttfamily\char`\\begin\char`\{\declare{#1}\char`\}}#2}%
  {\itemsep=0pt\parskip=0pt\item{\meta{environment contents}}%
  \item{\ttfamily\char`\\end\char`\{\declare{#1}\char`\}}}}


\def\smallpackage{\vbox\bgroup\package}
\def\endsmallpackage{\egroup\endpackage}
\def\package#1{\list{}{\leftmargin=2em\itemindent-\leftmargin\def\makelabel##1{\hss##1}}%
\extracttheme#1@\par\topsep=0pt}
\def\endpackage{\endlist}
\def\extracttheme#1#2@{%
\item{{{\ttfamily\char`\\usepackage}#2{\ttfamily\char`\{\declare{#1}\char`\}}}}}



\def\Environment#1{\par\bigskip\noindent\textbf{Environment \texttt{#1}}\par}
\def\Command#1{\par\bigskip\noindent\textbf{Command \texttt{#1}}\par}
\long\def\Parameters#1{\medskip\noindent Parameters:
  \begin{enumerate}\itemsep=0pt\parskip=0pt
    #1
  \end{enumerate}}
\long\def\Description#1{\unskip\medskip\noindent Description: #1}
\def\Example{\par\medskip\noindent Example: }

\newcommand{\w}[1]{\bf{#1}}\newcommand{\ww}[1]{\bf{#1}}
%\newcommand{\w}[1]{\vec{#1}}\newcommand{\ww}[1]{\overrightarrow{#1}}



\newcommand{\vertiii}[1]{{\left\vert\kern-0.25ex\left\vert\kern-0.25ex\left\vert#1\right\vert\kern-0.25ex\right\vert\kern-0.25ex\right\vert}}

%%%%%%%%%%%%%%%%%%%%%%%%%%%%%%%%%%%%%%%%%%%%%%%%%%%%%%%%%%%
    %%%%%%%% Accent Math
%%%%%%%%%%%%%%%%%%%%%%%%%%%%%%%%%%%%%%%%%%%%%%%%%%%%%%%%%%%
\makeatletter
\newcommand*{\math@auxii}[2][3]{{}\mkern#1mu\overline{\mkern-#1mu#2}}
\newcommand*{\math@auxi}[3][3]{\overset{\mkern#1mu\text{\scalebox{0.7}{#3}}\mkern-#1mu}{\smash{\math@auxii[#1]{#2}}\vphantom{#2}}}
\newcommand*{\mathco}[2][3]{\math@auxi[#1]{#2}{$\circ$}}
\newcommand*{\mathabc}[2][3]{\math@auxi[#1]{#2}{abc}}
\makeatother

%%%%%%%%%%%%%%%%%%%%%%%%%%%%%%%%%%%%%%%%%%%%%%%%%%%%%%%%%
%%%%algorithmes / pseudo code%%%
%%%%%%%%%%%%%%%%%%%%%%%%%%%%%%%%%%%%%%%%%%%%%%%%%%%%%%%%%%
\usepackage[french,ruled]{algorithm2e}

%%%%%
\usepackage{color,transparent}
%%%%%
%\usepackage{graphics,graphicx,subfigure,caption}
%\usepackage{graphics,graphicx,caption}
\usepackage{caption}
\usepackage{subcaption}
\usepackage{comment}
\usepackage{float} 
\setcounter{MaxMatrixCols}{30}%

%%%%%%%%%%%%%%%%%%%%%%%%%%%%%%%%%%%%%%%%
%%%%        Dimensions du texte (à adapter selon votre goût/le goût de l'Ãéditeur)
%%%%%%%%%%%%%%%%%%%%%%%%%%%%%%%%%%%%%%%%

\setlength\textwidth{20.5cm}          % largeur du texte
\setlength\topmargin{-1cm}          % marge en haut
\setlength\evensidemargin{-2cm}     % marge de gauche
\setlength\textheight{25cm}         % hauteur du texte
\setlength\oddsidemargin{\evensidemargin}


% Figures flottantes:
% fraction maximale d'une page pouvant etre occupe par une figure:
\renewcommand{\topfraction}{0.8}
% fraction minimale d'une page reservee pour le texte:
\renewcommand{\textfraction}{0.2}
% fraction minimale d'occupation de la page par une figure pleine page:
\renewcommand{\floatpagefraction}{0.7}

%%%%%%%%%%%%%%%%%%%%%%%%%%%%%%%%%%%%%%%%
%         D\'ecoupage des mots           %
%%%%%%%%%%%%%%%%%%%%%%%%%%%%%%%%%%%%%%%%
\hyphenation{}

%%%%%%%%%%%%%%%%%%%%%%%%%%%%%%%%%%%%%%%%
%%%%  Th\'eor\`emes, d\'efinitions, etc.
%%%%%%%%%%%%%%%%%%%%%%%%%%%%%%%%%%%%%%%%


% Il y a diffÃérents types d'ÃénoncÃés qui mÃéritent un environnement spÃécifique, voici une liste assez exhaustive. 
\theoremstyle{plain}
	\newtheorem{Theo}{Th\'eor\`eme}[section] %compteur commençant par le numÃéro de la section (on pourrait aussi faire commencer par le numÃéro de la sous-section - remplacer "section" par "subsection")
	\newtheorem{Prop}[Theo]{Proposition}        %mÃême compteur que pour les thÃéorÃèmes      
	\newtheorem{Prob}[Theo]{Probl\`eme}	    %idem
	\newtheorem{Lemm}[Theo]{Lemme}            %etc...
	\newtheorem{Coro}[Theo]{Corollaire}
	\newtheorem{Propr}[Theo]{Propri\'et\'e}
	\newtheorem{Conj}[Theo]{ Conjecture}
	\newtheorem{Aff}[Theo]{Affirmation}

    \newtheorem{TheoPrinc}{Th\'eor\`eme}     %compteur spÃécifique pour les thÃéorÃèmes les plus importants du papier
        
\theoremstyle{definition}
	\newtheorem{Defi}[Theo]{D\'efinition}
    \newtheorem{Exem}[Theo]{Exemple}
	\newtheorem{Nota}[Theo]{\Large Notation}

\theoremstyle{remark}
	\newtheorem{Rema}[Theo]{Remarque}
	\newtheorem{NB}[Theo]{N.B.}
	\newtheorem{Comm}[Theo]{Commentaire}
	\newtheorem{question}[Theo]{$\ast$ Question}
	\newtheorem{exer}[Theo]{Exercice}
	\newtheorem{Consequence}[Theo]{Conséquence}
	\newtheorem{Rap}[Theo]{Rappel}
	\newtheorem*{Merci}{Remerciements}
	
\usepackage{mdframed}

\mdfdefinestyle{propstyle}{%
linecolor=black,linewidth=2pt,%
hidealllines=true,
frametitlerule=true,%
frametitlebackgroundcolor=gray!20,
backgroundcolor=gray!10!white,
roundcorner=5pt,
innertopmargin=\topskip,
}

%\mdtheorem[style=propstyle]{prop}{Property}[chapter]
\mdtheorem[style=propstyle]{lemma}[prop]{Lemma}
\mdtheorem[style=propstyle]{TheoPrinc}{Th\'eor\`eme}[chapter]

% Définition d'un style personnalisé pour les Affirmations
\mdfdefinestyle{affirmestyle}{%
    linecolor=gray, % Couleur de la bordure
    linewidth=1pt, % Épaisseur de la bordure
    backgroundcolor=gray!10, % Couleur de fond (gris clair)
    roundcorner=5pt, % Coins arrondis
    innertopmargin=0pt, % Marge intérieure au-dessus du cadre
    innerbottommargin=10pt, % Marge intérieure en-dessous du cadre
    innerleftmargin=10pt, % Marge intérieure à gauche
    innerrightmargin=10pt, % Marge intérieure à droite
    skipabove=10pt, % Espace au-dessus du cadre
    skipbelow=10pt % Espace en-dessous du cadre
}

% Définition de l'environnement Affirmation
\theoremstyle{definition} % Style de théorème pour les affirmations
\newmdtheoremenv[style=affirmestyle]{aff}{Point clé n$^{\circ}$} % Environnement Affirmation avec le style personnalisé
	

%%%%%%%%%%%%%%%%%%%%%%%%%%%%%%%%%%%%%%%%
%%%%  Accolades, guillemets, etc.
%%%%%%%%%%%%%%%%%%%%%%%%%%%%%%%%%%%%%%%%

\def\ogg~{{\rm \og}}   % guillemets ouvrants
\def\fgg{{\rm \fg}}  % guillemets fermants



\def\nl{\newline}
\def\nn{\noindent}

\def\q{\nn}
\def\qq{\nn\quad}
\def\qqq{\nn\quad\quad}
\def\qqqq{\nn\quad\quad\quad}
\def\ce{\centerline}

%%%%%%%%%%%%%%%%%%%%%%%%%%%%%%%%%%%%%%%%
%%%%  Lettres et symboles math\'ematiques
%%%%%%%%%%%%%%%%%%%%%%%%%%%%%%%%%%%%%%%%

\def\emptyset{\varnothing}

%%%% Raccourcis pour les caractÃères gras mathÃématiques (ensembles R, N, Z, C etc)

\def\NN{{\mathbb N}}    %naturels
\def\ZZ{{\mathbb Z}}     %entiers relatifs
\def\RR{{\mathbb R}}    %rÃéels
\def\QQ{{\mathbb Q}}    %rÃéels
\def\CC{{\mathbb C}}    %complexes
\def\HH{{\mathbb H}}    %quaternions / espace hyperbolique
\def\AA{{\mathbb P}}     %espace projectif
\def\KK{{\mathbb K}}     %corps quelconques

\def\EE{{\mathbb E}}     %Experance
\def\VV{{\mathbb V}}     %Variance

%%%%%%raccourcis lettres calligraphiÃées
\def\cA{{\mathcal A}}  \def\cG{{\mathcal G}} \def\cM{{\mathcal M}} \def\cS{{\mathcal S}} \def\cB{{\mathcal B}}  \def\cH{{\mathcal H}} \def\cN{{\mathcal N}} \def\cT{{\mathcal T}} \def\cC{{\mathcal C}}  \def\cI{{\mathcal I}} \def\cO{{\mathcal O}} \def\cU{{\mathcal U}} \def\cD{{\mathcal D}}  \def\cJ{{\mathcal J}} \def\cP{{\mathcal P}} \def\cV{{\mathcal V}} \def\cE{{\mathcal E}}  \def\cK{{\mathcal K}} \def\cQ{{\mathcal Q}} \def\cW{{\mathcal W}} \def\cF{{\mathcal F}}  \def\cL{{\mathcal L}} \def\cR{{\mathcal R}} \def\cX{{\mathcal X}} \def\cY{{\mathcal Y}}  \def\cZ{{\mathcal Z}}

%%%%%%raccourcis lettres gothiques

\def\mfA{{\mathfrak A}} \def\mfA{{\mathfrak P}} \def\mfS{{\mathfrak S}}\def\mfZ{{\mathfrak Z}} \def\mfM{{\mathfrak M}} \def\mfQ{{\mathfrak Q}} \def\mfE{{\mathfrak E}} \def\mfL{{\mathfrak L}} \def\mfW{{\mathfrak W}} \def\mfR{{\mathfrak R}} \def\mfK{{\mathfrak K}} \def\mfX{{\mathfrak X}} \def\mfT{{\mathfrak T}} \def\mfJ{{\mathfrak J}} \def\mfC{{\mathfrak C}} \def\mfY{{\mathfrak Y}} \def\mfH{{\mathfrak H}} \def\mfV{{\mathfrak V}}\def\mfU{{\mathfrak U}}\def\mfG{{\mathfrak G}} \def\mfB{{\mathfrak B}} \def\mfI{{\mathfrak I}} \def\mfF{{\mathfrak F}} \def\mfN{{\mathfrak N}} \def\mfO{{\mathfrak O}} \def\mfD{{\mathfrak D}} 

\def\mfa{{\mathfrak a}} \def\mfp{{\mathfrak p}} \def\mfs{{\mathfrak s}}  \def\mfz{{\mathfrak z}} \def\mfm{{\mathfrak m}} \def\mfq{{\mathfrak q}}  \def\mfe{{\mathfrak e}} \def\mfl{{\mathfrak l}} \def\mfw{{\mathfrak w}} \def\mfr{{\mathfrak r}} \def\mfk{{\mathfrak k}} \def\mfx{{\mathfrak x}} \def\mft{{\mathfrak t}} \def\mfj{{\mathfrak j}} \def\mfc{{\mathfrak c}} \def\mfy{{\mathfrak y}} \def\mfh{{\mathfrak h}} \def\mfv{{\mathfrak v}} \def\mfu{{\mathfrak u}} \def\mfg{{\mathfrak g}} \def\mfb{{\mathfrak b}} \def\mfi{{\mathfrak i}} \def\mff{{\mathfrak f}} \def\mfn{{\mathfrak n}} \def\mfo{{\mathfrak o}} \def\mfd{{\mathfrak d}} 



%%raccourcis texte
\newcommand{\nbh}{\nobreakdash-\hspace*{0pt}}	% trait d'union insÃécable
\newcommand{\cad}{c'est\nbh \`a\nbh dire}		% c.-??-d.
\newcommand{\Cad}{C'est\nbh \`a\nbh dire}		% C.-??-d.

%%opÃérateurs   (ajoutez les raccourcis que vous voulez pour les opÃérateurs dont avez besoin)
\newcommand{\pgcd}{\operatorname{pgcd}}
\newcommand{\ppcm}{\operatorname{ppcm}}
\newcommand{\GL}{\operatorname{GL}}             %groupe linÃéaire
\newcommand{\Sp}{\mathsf{Sp}}			    %spectre
\newcommand{\End}{\operatorname{End}}         %Endomorphismes
\newcommand\Aut{\operatorname{Aut}}		  %etc...
\newcommand\ct{\operatorname{cotan}}         %cotangente
\newcommand{\dx}{\partial_x}                    
\newcommand{\dy}{\partial_y}
\newcommand{\Ker}{\operatorname{Ker}}
\newcommand{\Id}{\operatorname{Id}}
%\def\Im{\operatorname{Im}}                            %dans ce cas, \newcommand ne va pas marcher car la commande \Im existe dÃéjà. On utilise alors  \def de tex.

%%%%% a ajouter 

\usepackage{xcolor}
\newcommand{\varitem}[3][black]{%
  \item[%
   \colorbox{#2}{\textcolor{#1}{\makebox(5.5,7){#3}}}%
  ]
}


%%%%

\usepackage{scalerel}
\usepackage{xcolor}
\usepackage{stackengine}
\usepgflibrary {shadings}


\usetikzlibrary {decorations.pathmorphing}

\newcommand\dangersign[1]{%
    \renewcommand\stacktype{L}%
    \scaleto{\stackon[1.3pt]{\color{red}$\triangle$}{\tiny !}}{#1}%
}

\usepackage{tikz}
\tikzset{every picture/.style={execute at begin picture={\shorthandoff{:;!?};}}}
\tikzstyle{every picture}+=[remember picture]
\tikzstyle{na} = [shape=rectangle,inner sep=0pt]

% Commandes pour les flèches textuelles
\newcommand{\ptFleche}[2]{		% Déclaration d'une extrémité de flèche
    \tikz[baseline=(#1.base)]\node[na](#1){#2};
  }
%\newcommand{\Fleche}[5][thick]{	% Dessin de la flèche
%    \begin{tikzpicture}[overlay]
%        \path[->,#1](#2) edge [out=#4, in=#5] (#3);
%    \end{tikzpicture}
%  }
  
% \newcommand{\Flecheprim}[5][thick]{	% Dessin de la flèche
%    \begin{tikzpicture}[overlay]
%        \path[->,#1](#2) edge [out=#4, in=#5] (#3);
%    \end{tikzpicture}
%  }
%
\usepackage{marvosym}
\usepackage{changepage}

\usepackage{minitoc}
\usepackage{tocloft}
%\renewcommand{\cfttoctitle}{\hspace{-2em}} 
% Nastaveni obsahu
% Nastaveni obsahu

\usepackage{imakeidx}
\usepackage{fancyhdr}

%\usepackage{makeidx}
\makeindex[intoc=true]
\makeindex[name=pers, title=Index of person names, intoc=true]

\usepackage{xcolor}
\definecolor{linkcolor}{RGB}{0,0,180}
\PassOptionsToPackage{
    colorlinks=true,
    linkcolor=linkcolor,
    citecolor=linkcolor,
    urlcolor=linkcolor
}{hyperref}
\usepackage{hyperref}

%%%%%%%%%%%%%%%%%%%%%
%\definecolor{linkcolor}{RGB}{0,0,180}
\usepackage{titlesec}

% Appliquer la couleur à tous les niveaux de titre
\titleformat{\section}{\normalfont\color{linkcolor!90!black}\Large\bfseries}{\thesection}{1em}{}
\titleformat{\subsection}{\normalfont\color{linkcolor!70!black}\large\bfseries}{\thesubsection}{1em}{}
\titleformat{\subsubsection}{\normalfont\color{linkcolor!50!black}\normalsize\bfseries}{\thesubsubsection}{1em}{}
\titleformat{\paragraph}[runin]{\normalfont\color{linkcolor!30!black}\bfseries}{\theparagraph}{1em}{}
\titleformat{\subparagraph}[runin]{\normalfont\color{linkcolor!10!black}\itshape}{\thesubparagraph}{1em}{}
%%%%%%%%%%%%%%%%%%%%%%%

%%Couleurs dans la table des matières

\usepackage{tocloft}

% Modifier la couleur des entrées de la TOC
\renewcommand{\cftsecfont}{\color{linkcolor!90!black}}
\renewcommand{\cftsubsecfont}{\color{linkcolor!70!black}}
\renewcommand{\cftsubsubsecfont}{\color{linkcolor!50!black}}
\renewcommand{\cftparafont}{\color{linkcolor!30!black}}
\renewcommand{\cftsubparafont}{\color{linkcolor!10!black}}
%%%%%%%%%%%%%%%%%%%%%%%%%%%%%
% Reglages:
%
\hypersetup{pdftitle={Étude de la dynamique hors équilibre de boson unidimentionnel },
  pdfsubject={Quantum},
  pdfauthor={Guillaume THEMEZE <guillaume.themeze@gmail.fr>},
  pdfkeywords={LaTeX},
  colorlinks=true
}
\pagestyle{fancyplain}
\addtolength{\headwidth}{\marginparsep}
\addtolength{\headwidth}{\marginparwidth}
\renewcommand{\chaptermark}[1]{\markboth{#1}{}}
\renewcommand{\sectionmark}[1]{\markright{\thesection\ #1}}
\lhead[\fancyplain{}{\bfseries\thepage}]{}
\rhead[]{\fancyplain{}{\bfseries\thepage}}
\chead[\fancyplain{}{\bfseries\leftmark}]{\fancyplain{}{\bfseries\rightmark}}
\cfoot{}
%

%usepackage{titlesec}
% Changer la couleur des paragraphes en rouge par exemple :
%\titleformat{\paragraph}[runin] % ou [block] selon ce que tu veux
%  {\normalfont\color{red}\bfseries}
%  {\theparagraph}{1em}{}