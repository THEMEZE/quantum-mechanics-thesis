\documentclass[8pt, landscape]{report}
\usepackage[utf8]{inputenc}
\usepackage[T1]{fontenc}
\usepackage{amsmath, amssymb, graphicx, hyperref}
\usepackage[french]{babel}
%\usepackage[left=0.5cm,right=0.5cm,top=0.5cm,bottom=0.5cm,nohead]{geometry}
\usepackage[landscape,a3paper, left=0.5cm,right=0.5cm,top=0.5cm,bottom=0.5cm,nohead]{geometry}
\usepackage{multicol}
\usepackage{xcolor}
\usepackage{enumitem}
\usepackage{etoolbox}

% Couleurs personnalisées
\definecolor{level1}{RGB}{0,0,120}
\definecolor{level2}{RGB}{0,100,0}
\definecolor{level3}{RGB}{120,0,0}
\definecolor{level4}{RGB}{80,80,0}
\definecolor{level5}{RGB}{60,60,60}

% Style de texte par niveau
\newcommand{\textlevelone}{\color{level1}\sffamily\large}
\newcommand{\textleveltwo}{\color{level2}\rmfamily\normalsize}
\newcommand{\textlevelthree}{\color{level3}\ttfamily\small}
\newcommand{\textlevelfour}{\color{level4}\itshape\footnotesize}
\newcommand{\textlevelfive}{\color{level5}\scriptsize\upshape}

% Style de label par niveau
\setlist[enumerate,1]{label=\textcolor{level1}{\textbf{\arabic*.}}, leftmargin=1em, before=\textlevelone}
\setlist[enumerate,2]{label=\textcolor{level2}{\textbf{\alph*)}}, leftmargin=2em, before=\textleveltwo}
\setlist[enumerate,3]{label=\textcolor{level3}{\roman*)}, leftmargin=3em, before=\textlevelthree}
\setlist[enumerate,4]{label=\textcolor{level4}{\Alph*)}, leftmargin=4em, before=\textlevelfour}

%\setlist[itemize,1]{label=\textcolor{level1}{$\bullet$}, leftmargin=1em, before=\textlevelone}
\setlist[itemize,1]{label=\textcolor{level5}{--}, leftmargin=5em, before=\textlevelfive}

\setlist[itemize,2]{label=\textcolor{level2}{--}, leftmargin=2em, before=\textleveltwo}
\setlist[itemize,3]{label=\textcolor{level3}{$\ast$}, leftmargin=3em, before=\textlevelthree}
\setlist[itemize,4]{label=\textcolor{level4}{$\cdot$}, leftmargin=4em, before=\textlevelfour}

%\geometry{a4paper, inner=3.5cm, outer=2.5cm, top=2.5cm, bottom=2.5cm, twoside} % proposition
%\title{Titre de ta thèse}
%\author{Ton Nom}
%\date{\today}

\begin{document}

%\begin{multicols}{3} % 3 colonnes
%
%\begin{enumerate}
%	\item Introduction générale
%
%	\item Modèle de Lieb-Liniger et approche Bethe Ansatz
%	
%		\begin{enumerate}
%			\item Rappel
%				\begin{enumerate}
%					\item Dynamique particule libre dans espace à une dimension , périodique
%					\item Première à seconde quantification
%						\begin{enumerate}
%							\item Outil pour le Bethe Antsatz
%						\end{enumerate}					
%				\end{enumerate}
%			\item Description du modèle de Lieb-Liniger 
%				\begin{enumerate}
%					\item Introduction au modèle de gaz de Bose unidimensionnel et Hamiltonien du modèle 
%						\begin{enumerate}
%							\item Champs de Bose
%							\item Expression de l’Hamiltonien
%							\item Commutation canonique
%							\item Équation du mouvement associée
%						\end{enumerate}	
%					\item Fonction d’onde et Hamiltonien et moment à N corps
%						\begin{enumerate}
%							\item Cas d’une particule libre dans une boîte périodique : base des états propres à une particule
%								\begin{itemize}
%									\item Contexte physique du système.
%									\item Détermination des états propres. 
%									\item État à une particule dans la base de Fock. 
%									\item Orthonormalité de la base. 
%									\item Fonction d’onde à une particule.
%									\item Hamiltonien dans le cas à une particule. 
%									\item Hamiltonien dans le cas à une particule et Action de l’Hamiltonien. 
%									\item Équation de Schrödinger différentielle.
%									\item Résolution avec conditions périodiques. 
%									\item Énergies quantifiées. 
%									\item Notation adoptée et interprétation. 
%								\end{itemize}
%							\item Deux particules
%								\begin{itemize}
%									\item Introduction au système à deux bosons avec interaction de contact. 
%									\item Forme générale de l’état à deux particules dans la base de Fock 
%									\item Définition et orthonormalisation de la base positionnelle bosonique. 
%									\item Symétrie et normalisation de la fonction d’onde à deux particules. 
%									\item Réécriture de l’Hamiltonien du champ. 
%									\item Action de l’Hamiltonien sur l’état à deux particules et Forme explicite de l’Hamiltonien effectif à deux corps
%									\item Changement de variables : coordonnées du centre de masse et relatives. 
%									\item Résolution du problème de centre de masse et de coordonnée relative.
%									\item Forme symétrique de la fonction d’onde pour bosons.
%									\item Condition de discontinuité à cause du potentiel delta. 
%									\item Détermination de la phase $\Phi$. 
%									\item Phase de diffusion à un corps.
%									\item Lien entre phase de diffusion et décalage temporel : interprétation semi-classique.
%									\item Retour aux coordonnées du laboratoire. 								
%								\end{itemize}
%							\item Cas général à N particules : l’Ansatz de Bethe	
%						\end{enumerate}
%					\item Opérateurs conservés (intégrales du mouvement) 
%						\begin{enumerate}
%							\item Opérateurs nombre de particules Q et moment P
%							\item Propriétés
%							\item L’état propre
%						\end{enumerate}
%				\end{enumerate}	
%			\item Équation de Bethe et distribution de rapidité
%				\begin{enumerate}
%					\item Condition aux bords périodiques et équation de Bethe Ansatz 
%					\item Thermodynamique du gaz de Lieb-Liniger à température nulle 
%					\item Excitations élémentaires à température nulle
%				\end{enumerate}
%		\end{enumerate}
%		
%	\item Théorie thermodynamique et équilibre généralisé
%		\begin{enumerate}
%			\item Dynamique hors équilibre et notions d’équilibre 
%				\begin{enumerate}
%					\item Limit Thermodynamique 
%						\begin{enumerate}
%							\item Limite
%							\item The dressing
%						\end{enumerate}
%					\item Notion d’état d’équilibre généralisé (GGE)
%						\begin{enumerate}
%							\item Configuration des états à N particules. 
%							\item Observables diagonales dans la base des états propres. 
%							\item Définition générale d’observables conservées.
%							\item Principe de maximisation de l’entropie.
%							\item Définition de la matrice densité et de la fonction de partition
%							\item Interprétation physique des multiplicateurs de Lagrange.
%							\item Probabilité d’un état à rapidités fixées.
%							\item Moyenne d’un observable et dérivées de Z.
%							\item Moments d’ordre supérieur et fluctuations.
%							\item Cas particulier de l’équilibre thermique.
%						\end{enumerate}
%					\item Rôle des charges conservées extensives et quasi-locales
%						\begin{enumerate}
%							\item Écriture des observables thermodynamiques comme sommes sur les rapidités.
%							\item Interprétation fonctionnelle et échange des sommes.
%							\item Expression de la matrice densité généralisée. 
%							\item Probabilité associée à une configuration de rapidités
%							\item Moyennes d’observables dans le GGE.
%							\item Interpretation. 
%							\item Rôle dans le formalisme GGE.
%							\item D’un point de vue mathématique.
%						\end{enumerate}
%				\end{enumerate}
%			\item Thermodynamique de Bethe et relaxation
%				\begin{enumerate}
%					\item Statistique des macro-états : entropie de Yang-Yang et moyennes dans le GGE
%						\begin{enumerate}
%							\item Macro-états et entropie dans la TBA.
%							\item Distribution de rapidité comme macro-état.
%							\item Dénombrement local des configurations microcanoniques.
%							\item Estimation asymptotique à l’aide de Stirling. 
%							\item Entropie de Yang-Yang : définition . 
%							\item Énergie généralisée. 
%							\item Observables locales dans la limite thermodynamique.
%							\item Passage à la limite continue.
%							\item Formule fonctionnelle pour les moyennes.
%						\end{enumerate}
%					\item Équations intégrales de la TBA
%						\begin{enumerate}
%							\item Moyenne des observables dans l’ensemble généralisé de Gibbs.
%								\begin{itemize}
%									\item Approximation au point selle.
%									\item Développement fonctionnel au premier ordre.
%									\item Équation intégrale de la TBA.
%								\end{itemize}
%						\end{enumerate}
%						
%				\end{enumerate}
%		\end{enumerate}
%		
%	\item Hydrodynamique généralisée
%		\begin{enumerate}
%			\item Hydrodynamique et régimes asymptotiques 
%				\begin{enumerate}
%					\item Hydrodynamique classique des systèmes chaotiques 
%					\item Hydrodynamique des systèmes intégrables et distribution de rapidité 
%					\item Équation d’hydrodynamique généralisée (GHD) 
%				\end{enumerate}
%		\end{enumerate}
%		
%	\item Fluctuations autour des états d’équilibre
%		\begin{enumerate}
%			\item Introduction
%			\item Développement autour du point selle
%			\item Définition de la fonction de corrélation 
%			\item Fluctuations autour de la distribution moyenne et rôle de la Hessienne
%			\item Fluctuations autour de la distribution moyenne
%			\item Fonction correlation du nombre d’atomes et de l’énergie 
%		\end{enumerate}
%		
%	\item Expériences sur les gaz 1D hors-équilibre
%		\begin{enumerate}
%			\item Présentation de l’expérience
%				\begin{enumerate}
%					\item Piégeage transverses et longitudinale 
%				\end{enumerate}
%			\item Outil de sélection spatial 
%		\end{enumerate}
%		
%	\item Protocoles expérimentaux avancés 
%		\begin{enumerate}
%			\item Dispositif expérimental
%				\begin{enumerate}
%					\item Préparation et Confinement du Gaz Ultra-Froid de 87Rb
%					\item Confinement Longitudinal et Stabilisation du Piège Quartique
%					\item Sélection Spatiale et Réalisation de la Coupure Bipartite 
%					\item Dynamique Après Coupure	
%				\end{enumerate}
%			\item Prédictions de la GHD 
%			\item Données expérimentales
%			\item Sonder la distribution locale des rapidités
%			\item Détails sur les calculs 
%				\begin{enumerate}
%					\item Facteur d’occupation et distribution de rapidité à l’équilibre thermique
%					\item Dynamique du contour dans l’espace des phases $(x,\theta)$
%					\item Simulation de la déformation du bord
%					\item Simulation de l’expansion
%				\end{enumerate}
%		\end{enumerate}
%		
%	\item Mise en place d’un confinement longitudinale dipolaire 
%		\begin{enumerate}
%			\item Calculs analytiques pour le confinement dipolaire
%				\begin{enumerate}
%					\item Transformation de jauge et simplification du Hamiltonien 
%						\begin{enumerate}
%							\item Cadre sans potentiel vecteur. 
%							\item Hamiltonien simplifié.
%							\item Conclusion – Simplification par transformation de jauge.
%						\end{enumerate}
%					\item Effet Stark dynamique et interaction dipolaire atomique 
%						\begin{enumerate}
%							\item Polarisabilité scalaire, vectorielle et tensorielle dans les états fins
%								\begin{itemize}
%									\item Interprétation physique
%									\item Cas des atomes alcalins (ex. Rubidium)
%								\end{itemize}
%						\end{enumerate}
%					\item Cas du Rubidium 87 dans une polarisation rectiligne 
%						\begin{enumerate}
%							\item Champ électrique appliqué 
%							\item Cas de désaccords très importants. 
%								\begin{itemize}
%									\item Décalage d’énergie au second ordre. 
%									\item Structure orbitale et opérateur dipolaire.
%									\item Application du théorème de Wigner-Eckart. 
%									\item Application au cas 5S $\to $ 5P et q= 0. 	
%								\end{itemize} 
%							\item Piégeage dipolaire d’un atome à deux niveaux — généralités.
%								\begin{itemize}
%									\item Introduction. 
%									\item Système à deux niveaux et interaction avec le champ. 
%									\item Expression du potentiel.
%									\item Conditions de validité.
%									\item Interprétation physique. 
%									\item Confinement optique. 
%									\item Taux de diffusion spontanée. 
%									\item Bilan — compromis intensité / désaccord  	
%								\end{itemize}
%							\item Structure fine et base des états $|L,S; J,m_J\rangle$.
%								\begin{itemize}
%									\item Décalage d’énergie au second ordre. 
%									\item Projection dans la base découplée. 
%									\item Application au cas 5S1/2 $\to$ 5P1/2,3/2 avec q = 0. 
%									\item Potentiel dipolaire 	
%								\end{itemize}
%
%						\end{enumerate}
%					\item Quelle longueur d’onde du laser?
%					\item Quelle Puissance du laser?
%				\end{enumerate}
%			\item laser , MOPA, etc .. Données
%		\end{enumerate}
%	\item Conclusion et perspectives
%
%\end{enumerate}
%
%\end{multicols}

\begin{multicols}{3} % 3 colonnes

\begin{enumerate}
    \item Présentation de l’expérience
    \begin{enumerate}
        \item Dispositif expérimental
        \begin{enumerate}
            \item La puce atomique
            \item Contrôleur et séquenceur
            \item Présentation des différentes étapes pour la production d’un gaz de Bose 1D
            \item Système lasers
        \end{enumerate}
        \item Piégeage dans le guide modulé
        \begin{enumerate}
            \item Principe de piégeage magnétique par un fil
            \item Piégeage transverse
            \begin{enumerate}
                \item Problème de rugosité
                \item Mesure de la fréquence transverse
            \end{enumerate}
            \item Piégeage longitudinal
            \begin{enumerate}
                \item Découplage des confinements transverses et longitudinaux
                \item Piégeage harmonique
                \item Champ magnétique résiduel
                \item Mesure de la fréquence longitudinale
                \item Piégeage quartique
                \item Réalisation expérimentale d’un piège quartique
                \item Instabilités du piège quartique
            \end{enumerate}
        \end{enumerate}
        \item Stabilité de l’expérience
        \begin{enumerate}
            \item Sensibilité aux bruits magnétiques extérieurs
            \item Régime permanent
            \item Détérioration de la puce
        \end{enumerate}
    \end{enumerate}

    \item Techniques d’analyse
    \begin{enumerate}
        \item Système d’imagerie
        \begin{enumerate}
            \item Imagerie par absorption après temps de vol
            \item Imagerie par absorption in situ
            \item Défauts d’imagerie
        \end{enumerate}
        \item Thermométrie
        \begin{enumerate}
            \item Température Yang-Yang
            \begin{enumerate}
                \item Principe de mesure
                \item Prise en compte de la population des états transverses
            \end{enumerate}
            \item Thermométrie par étude des ondulations de densité
            \begin{enumerate}
                \item Spectre de puissance des ondulations de densité
                \item Quasi-condensat homogène
                \item Cas des petits vecteurs d’ondes
                \item Cas des grands vecteurs d’ondes
                \item Quasi-condensat non-homogène
                \item Mesures expérimentales
                \item Commentaire sur les mesures de températures
            \end{enumerate}
        \end{enumerate}
    \end{enumerate}

    \item Mise en place d’un outil de sélection spatiale
    \begin{enumerate}
        \item Principe de sélection
        \item Mise en place expérimentale
        \begin{enumerate}
            \item Contrôle du DMD
            \item Montage optique
            \item Mise au point
            \item Imagerie sur les atomes
        \end{enumerate}
        \item Caractérisation de la sélection
        \begin{enumerate}
            \item Estimation de la puissance nécessaire
            \item Mesure de la puissance nécessaire
            \item Imagerie par fluorescence
            \item Limitations
        \end{enumerate}
    \end{enumerate}

    \item Expansion longitudinale d’un gaz de Bose 1D
    \begin{enumerate}
        \item Expansion longitudinale
        \begin{enumerate}
            \item Protocole expérimental
            \item Compensation de la gravité
        \end{enumerate}
        \item Profil de densité d’un gaz dans un piège harmonique
        \begin{enumerate}
            \item Équations de Gross-Pitaevskii dépendantes du temps
            \item Équations d’état
            \item Solutions analytiques homothétiques
            \begin{enumerate}
                \item Facteur d’échelle
                \item Solutions analytiques homothétiques
                \item Évolution temporelle du facteur d’échelle
            \end{enumerate}
            \item Régimes particuliers
            \begin{enumerate}
                \item Régime asymptotique à temps longs
                \item Régime à temps courts
            \end{enumerate}
            \item Régime de crossover
            \begin{enumerate}
                \item Méthodes numériques
                \item Caractérisation de l’évolution du profil
            \end{enumerate}
            \item Analyse des données expérimentales
            \item Limites expérimentales
        \end{enumerate}
        \item Fluctuations de phases dans un piège harmonique
        \begin{enumerate}
            \item Hypothèse de suivi adiabatique
            \item Cas des petits vecteurs d’ondes
            \item Validation de l’hypothèse
            \item Mesures expérimentales et perspectives
        \end{enumerate}
        \item Évolution des fluctuations d’une tranche homogène
        \begin{enumerate}
            \item Protocole expérimental
            \item Résultats expérimentaux et perspectives
        \end{enumerate}
    \end{enumerate}

    \item Sonde locale de la distribution de rapidités
    \begin{enumerate}
        \item Protocole expérimental
        \item Mesures à l’équilibre
        \item Dynamique d’expansion : comportements attendus
        \begin{enumerate}
            \item Comportement hydrodynamique
            \item Comparaison aux équations hydrodynamiques GP
        \end{enumerate}
        \item Comparaison aux équations GHD
        \begin{enumerate}
            \item Régime asymptotique
            \item Hypothèse thermique sur une tranche
            \item Hypothèse thermique sur différentes tranches
            \item Au-delà de l’hypothèse thermique
        \end{enumerate}
        \item Effet du processus de sélection
        \begin{enumerate}
            \item Effet de l’intensité du faisceau
            \item Autres précisions
        \end{enumerate}
        \item Systèmes hors équilibre
        \begin{enumerate}
            \item Protocole de cisaillement
            \item Mesures expérimentales
        \end{enumerate}
    \end{enumerate}
\end{enumerate}

\end{multicols}


%\maketitle
%\tableofcontents

%\chapter*{Introduction générale}
%\addcontentsline{toc}{chapter}{Introduction générale}

%\chapter{Contexte scientifique et état de l’art}
%\section{Introduction au domaine}
%\section{Les principaux résultats existants}
%\section{Problématique de la thèse}

%\chapter{Méthodes théoriques et outils}
%\section{Formalismes mathématiques utilisés}
%\section{Outils numériques / expérimentaux}
%\section{Validation des méthodes}

%\chapter{Résultats principaux}
%\section{Première étude / Modèle}
%\section{Deuxième étude / Simulation}
%\section{Analyse et discussion}

%\chapter{Vers de nouvelles perspectives}
%\section{Limites actuelles}
%\section{Ouvertures théoriques / expérimentales}
%\section{Applications potentielles}

%\chapter*{Conclusion générale}
%\addcontentsline{toc}{chapter}{Conclusion générale}

%\appendix
%\chapter{Annexes}
%\section{Démonstrations complémentaires}
%\section{Données supplémentaires}

%\bibliographystyle{plain}
%\bibliography{bibliographie}

\end{document}
