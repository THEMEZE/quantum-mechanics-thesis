Les modèles mécaniques dans les dimensions espace-temps sont présentés dans cet article. Cette méthode a été suggérée pour la première fois par H. Bethe en 1931 [1] et est traditionnellement appelée l'Ansatz de Bethe. Par la suite, la méthode a été développée par Hulthen, Yang et Yang, Lieb, Sutherland, Baxter, Gaudin et d'autres (voir [2], [3], et [4]).

Nous commençons la présentation par l'Ansatz de Bethe en coordonnées, non seulement pour des raisons historiques, mais aussi en raison de sa simplicité et de sa clarté. {\color{red} La matrice de diffusion à plusieurs particules apparaît comme étant égale au produit des matrices à deux particules pour les modèles intégrables. Cette propriété de réductibilité à deux particules est d'une importance primordiale lors de la construction de la fonction d'onde de Bethe. L'une des caractéristiques importantes des modèles intégrables est qu'il n'y a pas de production multiple de particules hors des coquilles de masse. Cette propriété est étroitement liée à l'existence d'un nombre infini de lois de conservation dans de tels modèles ; cela sera expliqué dans la Partie II.}

Quatre modèles principaux, à savoir le gaz de Bose unidimensionnel, le magnétisme de Heisenberg, le modèle de Thirring massif et le modèle de Hubbard, sont considérés dans la Partie I. Les fonctions propres des hamiltoniens de ces modèles sont construites. {\color{red} L'application des conditions aux limites périodiques mène à un système d'équations pour les valeurs permises des moments. Celles-ci sont connues sous le nom d'équations de Bethe. Ce système peut également être dérivé d'un certain principe variationnel, l'action correspondante étant appelée l'action de Yang-Yang. Elle joue un rôle important dans l'étude des modèles. Les équations de Bethe sont également utiles dans la limite thermodynamique. L'énergie de l'état fondamental, la vitesse du son, etc., peuvent être calculées dans cette limite. Les excitations au-dessus de l'état fondamental, c'est-à-dire les particules physiques, sont également étudiées. Pour définir leurs caractéristiques physiques, la technique des équations de "dressing" est introduite et étudiée. La thermodynamique du modèle est expliquée en détail.}

Le matériel de cette Partie est organisé comme suit. La théorie du gaz de Bose unidimensionnel avec une interaction répulsive ponctuelle entre les particules est présentée dans le premier chapitre. La solution du magnétisme de Heisenberg X X Z dans un champ magnétique externe est donnée dans le deuxième chapitre. Le modèle quantique du champ spinor avec une auto-interaction à quatre points dans deux dimensions espace-temps est résolu dans le troisième chapitre. Cela est généralement appelé le modèle de Thirring massif, et est équivalent au modèle de sine-Gordon (dans le secteur de charge nulle). Dans le dernier chapitre de la Partie I, le modèle de Hubbard des fermions interactifs sur un réseau est brièvement abordé.


Le gaz de Bose unidimensionnel est décrit par les champs quantiques de Bose canoniques \( \Psi(x,t) \) avec les relations de commutation canoniques à temps égal :

Plus tard, l'argument \( t \) sera en règle générale omis, puisque toutes les considérations de ce chapitre s'appliquent à un instant fixé dans le temps.  
L'Hamiltonien du modèle est

où \( c \) est la constante de couplage. L'équation du mouvement correspondante est

est appelée l'équation de Schrödinger non linéaire (NS).  

Pour \( c > 0 \), l'état fondamental à température nulle est une sphère de Fermi. Seul ce cas sera considéré par la suite.  

Le vide de Fock \( |0\rangle \), défini par

est important. Il sera appelé le pseudovacuum et doit être distingué du vide physique, qui est l'état fondamental de l'Hamiltonien (la mer de Dirac). Le pseudovacuum dual \(\langle 0|\) est défini comme \(\langle 0| = |0\rangle\) et satisfait les relations

où le symbole dag (\(\dagger\)) désigne la conjugaison hermitienne. L'opérateur nombre de particules \(Q\) et l'opérateur impulsion \(P\) sont

