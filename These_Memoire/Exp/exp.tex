%\section{Présentation de l’expérience}
%\section*{Introduction}
%
%\section{Refroidissement}
%
%\section{Imagerie}
%\subsection{Prubleme d'iamgerie et idée numerique}
%
%\section{Confinement transverse}
%
%\section{Confinement longitudinale}
%
%\subsection{Evolution logitudinale}
%
%\section{Outil de sélection spatial}
%
%\subsection{Mesure de distribution de rapidités locales $\rho(x , \theta ) $  pour des systèmes en équilibre}
%
%%\subsection{Piégeage transverses et longitudinale}
%%\section{Outil de sélection spatial}
%%%\section{Mesure de $\rho(x , \theta ) $ }
%
%%\section{Mesure de distribution de rapidités locales $\rho(x , \theta ) $  pour des systèmes en équilibre}

\section*{Introduction}

\begin{itemize}
	\item Objectif du chapitre : présentation synthétique de l’expérience
	\item Distinction claire des contributions : mise en place initiale (précédents doctorants), développement (travail de Léa Dubois), contribution personnelle (prise de données, analyses spécifiques, participation à certaines manipulations)
	\item Rôle de l’expérience dans l’étude de la dynamique des gaz de Bose 1D
\end{itemize}

Ce chapitre présente l’expérience utilisée pour étudier les gaz unidimensionnels de rubidium ultra-froids. Nous décrivons l’architecture du dispositif, les méthodes d’imagerie et d’analyse, ainsi que les protocoles expérimentaux auxquels j’ai participé. Le développement initial du refroidissement et du piégeage avant la puce a été réalisé par d’anciens doctorants. La mise en place du piégeage sur la puce et du système de sélection spatiale à l’aide d’un DMD a été initiée par Léa Dubois, alors en première année de doctorat à mon arrivée. Mon travail s’est concentré principalement sur la prise de données, l’analyse et la participation à certaines expériences spécifiques telles que l’expansion longitudinale et la mesure locale de la distribution de rapidité.


\section{Présentation générale de l’expérience}
\subsection{Vue d’ensemble du dispositif}
\begin{itemize}
    \item Architecture générale : production, piégeage, manipulation et imagerie.
    \item Systèmes étudiés : gaz de rubidium 87 dans des pièges 1D.
    \item Objectifs : exploration de dynamiques hors équilibre.
\end{itemize}

\subsection{Historique et contributions successives}
\begin{itemize}
    \item Étapes de refroidissement et piégeage initial : travaux antérieurs (voir thèses citées).
    \item Développement du piégeage 1D sur puce et du DMD : thèse de Léa Dubois.
    \item Contributions personnelles : prise de données, protocoles dynamiques, analyse.
\end{itemize}

\section{Le dispositif expérimental}
\subsection{Production et refroidissement des atomes (non détaillé ici, renvoi à d'autres travaux)}
\begin{itemize}
    \item Source chaude de rubidium, MOT, molasses optique.
    \item Refroidissement à des températures sub-$\mu~K$ Refroidissement sub-Doppler (détails renvoyés aux travaux précédents).
\end{itemize}

\subsection{Piégeage magnétique sur puce}
\begin{itemize}
    \item Présentation de la puce atomique.
    \item Confinement transverse et longitudinal.
    \item Régime 1D : conditions d’accès (\(\hbar \omega_\perp \gg k_B T\)).
    \item Problèmes de rugosité, stabilité magnétique.
\end{itemize}

\subsection{Génération de potentiels modulés}
\begin{itemize}
    \item Courants modulés pour créer des pièges harmoniques ou quartiques.
    \item Découplage transverse/longitudinal.
\end{itemize}

\section{Sélection spatiale avec DMD}
\subsection{Motivation et principe}
\begin{itemize}
    \item Besoin de préparer des tranches homogènes.
    \item Intérêt dans les protocoles hors équilibre.
\end{itemize}

\subsection{Mise en place technique (initiée par Léa Dubois)}
\begin{itemize}
    \item Dispositif optique de projection.
    \item Contrôle numérique des motifs.
    \item Calibration et stabilité.
\end{itemize}

\subsection{Utilisation dans les protocoles}
\begin{itemize}
    \item Formes utilisées : boîtes, barrières, coupures.
    \item Préparation initiale contrôlée du gaz.
    \item Exemples de protocoles expérimentaux utilisant le DMD
\end{itemize}

\section{Techniques d’imagerie et d’analyse}
\subsection{Imagerie par absorption}
\begin{itemize}
    \item Imagerie \textit{in situ} et après temps de vol.
    \item Résolution, limites instrumentales.
\end{itemize}

\subsection{Analyse des profils}
\begin{itemize}
    \item Extraction des densités, tailles, températures.
    \item Distribution longitudinale.
    \item Estimation de la température par ajustement Yang-Yang (optionnel si pertinent).
\end{itemize}

\section{Expériences et protocoles étudiés}
Cette section peut être la plus personnelle, en précisant ton rôle à chaque fois.
\subsection{Expansion longitudinale}
\begin{itemize}
    \item Protocole d’expansion (libération longitudinale, maintien du confinement transverse).
    \item Suivi de l’évolution du profil.
    \item Analyse à différents temps d’expansion
    \item Comparaison aux modèles analytiques : solutions homothétiques, GP, asymptotiques.
\end{itemize}

\subsection{Sonde locale de distribution de rapidité}
\begin{itemize}
    \item Principe de la mesure : coupure d’une tranche puis expansion.
    \item Rôle du DMD dans la sélection.
    \item Accès à la distribution de vitesse locale.
    \item Comparaison avec les prédictions GHD.
    \item Limites et incertitudes
\end{itemize}

\section{Discussion sur les limites et les perspectives}
\begin{itemize}
    \item Contraintes techniques (bruit, alignement, stabilité de la puce…).
    \item Améliorations potentielles (résolution, contrôle du potentiel, automatisation).
    \item Perspectives pour d’autres types d’expériences (étude de chocs, turbulence quantique, etc.)
\end{itemize}

\section*{Conclusion}
\begin{itemize}
    \item Résumé de l’architecture du dispositif).
    \item Méthodes d’analyse utilisées et robustesse.
    \item Importance de l’expérience dans le contexte de l’étude des gaz quantiques unidimensionnels
\end{itemize}
Ce chapitre a présenté les éléments essentiels du dispositif expérimental, les méthodes d’imagerie, ainsi que les expériences auxquelles j’ai participé. L’ensemble constitue une plateforme performante pour l’étude de la dynamique de gaz 1D hors équilibre.

\appendix
\section*{Annexes}
\begin{itemize}
    \item Schémas techniques (puce, DMD, optique).
    \item Tableaux de paramètres expérimentaux.
    \item Exemples de motifs DMD utilisés.
\end{itemize}