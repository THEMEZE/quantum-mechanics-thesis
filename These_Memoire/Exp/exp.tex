%\section{Présentation de l’expérience}
%\section*{Introduction}
%
%\section{Refroidissement}
%
%\section{Imagerie}
%\subsection{Prubleme d'iamgerie et idée numerique}
%
%\section{Confinement transverse}
%
%\section{Confinement longitudinale}
%
%\subsection{Evolution logitudinale}
%
%\section{Outil de sélection spatial}
%
%\subsection{Mesure de distribution de rapidités locales $\rho(x , \theta ) $  pour des systèmes en équilibre}
%
%%\subsection{Piégeage transverses et longitudinale}
%%\section{Outil de sélection spatial}
%%%\section{Mesure de $\rho(x , \theta ) $ }
%
%%\section{Mesure de distribution de rapidités locales $\rho(x , \theta ) $  pour des systèmes en équilibre}

\section*{Introduction}

\begin{itemize}
	\item Objectif du chapitre : présentation synthétique de l’expérience
	\item Distinction claire des contributions : mise en place initiale (précédents doctorants), développement (travail de Léa Dubois), contribution personnelle (prise de données, analyses spécifiques, participation à certaines manipulations)
	\item Rôle de l’expérience dans l’étude de la dynamique des gaz de Bose 1D
\end{itemize}

Ce chapitre présente l’expérience utilisée pour étudier les gaz unidimensionnels de rubidium ultra-froids. Nous décrivons l’architecture du dispositif, les méthodes d’imagerie et d’analyse, ainsi que les protocoles expérimentaux auxquels j’ai participé. Le développement initial du refroidissement et du piégeage avant la puce a été réalisé par d’anciens doctorants. La mise en place du piégeage sur la puce et du système de sélection spatiale à l’aide d’un DMD a été initiée par Léa Dubois, alors en première année de doctorat à mon arrivée. Mon travail s’est concentré principalement sur la prise de données, l’analyse et la participation à certaines expériences spécifiques telles que l’expansion longitudinale et la mesure locale de la distribution de rapidité.


\paragraph{Objectif du chapitre}  
Ce chapitre a pour objectif de fournir une présentation synthétique et structurée du dispositif expérimental utilisé pour étudier la dynamique de gaz de Bose unidimensionnels ultra-froids. Il constitue un socle indispensable pour comprendre les protocoles expérimentaux développés au cours de ma thèse et les analyses présentées dans les chapitres suivants.

\paragraph{Architecture générale}  
Nous présentons d'abord l’architecture complète de l’expérience, depuis la production des atomes jusqu’à leur imagerie, en passant par les étapes de refroidissement, de piégeage magnétique sur puce, de manipulation optique, et de génération de potentiels. Cette description s’accompagne d’une mise en contexte des contributions historiques au dispositif.

\paragraph{Contributions successives et personnelles}  
Une attention particulière est portée à la répartition chronologique des contributions. Les étapes initiales (source atomique, MOT, piège DC) ont été développées par d’anciens doctorants. La mise en place du piégeage 1D sur puce ainsi que l’utilisation du DMD pour la sélection spatiale ont été réalisées au cours de la thèse de Léa Dubois. Mon travail s’inscrit dans cette continuité et concerne principalement la prise de données, l’analyse de protocoles dynamiques, ainsi que la participation à certaines opérations de maintenance et d’optimisation du système.

\paragraph{Rôle du dispositif dans la thèse}  
Ce dispositif permet d’explorer des phénomènes hors équilibre dans des gaz quantiques 1D. Il constitue une plateforme particulièrement adaptée à l’étude de protocoles d’expansion, de sondes locales, ou de dynamiques guidées par la théorie hydrodynamique généralisée (GHD), qui sont au cœur de cette thèse.




\section{Présentation générale de l’expérience}
\subsection{Vue d’ensemble du dispositif}
\begin{itemize}
    \item Architecture générale : production, piégeage, manipulation et imagerie.
    \item Systèmes étudiés : gaz de rubidium 87 dans des pièges 1D.
    \item Objectifs : exploration de dynamiques hors équilibre.
\end{itemize}

\subsection{Historique et contributions successives}
\begin{itemize}
    \item Étapes de refroidissement et piégeage initial : travaux antérieurs (voir thèses citées).
    \item Développement du piégeage 1D sur puce et du DMD : thèse de Léa Dubois.
    \item Contributions personnelles : prise de données, protocoles dynamiques, analyse.
\end{itemize}

\section{Le dispositif expérimental}
\subsection{Système laser et contrôle de fréquence}
\label{sec:systeme_laser}

\paragraph{Laser maître 1 : référence de fréquence}
La référence principale de fréquence pour l'ensemble des faisceaux utilisés dans l'expérience est fournie par un laser à cavité étendue, développé au SYRTE. Ce laser est asservi par spectroscopie d’absorption saturée sur la transition D2 du $^{87}$Rb, au croisement des transitions $|F=2\rangle \rightarrow |F'=2,3\rangle$. Ce signal de référence est utilisé pour verrouiller les autres sources laser par battement optique.

\paragraph{Laser repompeur}
Un laser DFB (Distributed Feedback Diode) est utilisé pour produire le faisceau repompeur, permettant de transférer les atomes retombés dans l’état $|F=1\rangle$ vers l’état $|F=2\rangle$. Ce laser est asservi à une fréquence distante de 6\,GHz de celle du maître 1, en utilisant un montage de battement optique et mélange avec un oscillateur à 6.6\,GHz. Une diode Fabry-Perot injectée par la DFB permet d’amplifier la puissance au-delà de 100\,mW.

\paragraph{Laser maître 2 : laser principal de manipulation}
Un second laser à cavité étendue, identique au maître 1, est asservi par battement optique à la fréquence du maître 1. Il est amplifié par un amplificateur à semi-conducteur évasé (Tapered Amplifier), permettant d’atteindre une puissance de sortie supérieure à 1\,W. Ce faisceau est ensuite divisé en plusieurs branches pour alimenter :
\begin{itemize}
    \item le Piège Magnéto-Optique (PMO),
    \item la mélasse optique,
    \item le pompage optique,
    \item l’imagerie par absorption,
    \item le faisceau de sélection.
\end{itemize}

\paragraph{Contrôle de fréquence et polarisation}
Les fréquences des différents faisceaux sont ajustées via des Modulateurs Acousto-Optiques (AOM), tandis que leur polarisation et leur intensité sont contrôlées à l’aide de cubes PBS en combinaison avec des lames demi-onde motorisées ou fixes. Cette configuration assure une grande flexibilité dans la mise en œuvre des différentes phases expérimentales.

\paragraph{Remarque}
Une description plus détaillée du montage laser et de son verrouillage peut être trouvée dans la thèse de A.~Johnson~\cite{Johnson2016}. L’ensemble a été maintenu et utilisé sans modifications majeures au cours de ma thèse.


\subsection{Production et refroidissement des atomes (non détaillé ici, renvoi à d'autres travaux)}
{\color{blue}
\begin{itemize}
    \item Source chaude de rubidium, MOT, molasses optique.
    \item Refroidissement à des températures sub-$\mu~K$ Refroidissement sub-Doppler (détails renvoyés aux travaux précédents).
\end{itemize}
}
Le dispositif expérimental permet de produire des gaz de rubidium ultra-froids, avec pour objectif final l’obtention de gaz unidimensionnels dans le régime quantique dégénéré. La production suit une séquence expérimentale déjà bien établie, initialement développée par d’anciens doctorants (voir par exemple la thèse d’A. Johnson~\cite{Johnson2016}), puis réoptimisée au début de la thèse de Léa-Dubois ~\cite{L.Dubois2024} sous la supervision d’I. Bouchoule.

\paragraph{Libération des atomes de rubidium}
Les atomes de rubidium 87 sont libérés à partir d’un \emph{dispenser}, placé directement dans l’enceinte à vide, sur le côté de la monture de la puce atomique. Ce composant, parcouru par un courant de \( 4.5\,\mathrm{A} \) pendant environ \( 4.8\,\mathrm{s} \), émet un flux d’atomes thermiques dans la chambre à vide.

\paragraph{Capture par piège magnéto-optique (PMO)}
Les atomes thermiques sont ralentis et piégés à l’aide d’un piège magnéto-optique. Celui-ci utilise quatre faisceaux laser (dont deux sont réfléchis par la puce) et un champ quadrupolaire magnétique généré par des bobines. Le nuage ainsi formé se situe à quelques millimètres de la surface de la puce.

\paragraph{Rapprochement vers la puce}
Pour rapprocher les atomes de la puce, on transfère le champ quadrupolaire depuis les bobines vers un champ généré par le fil en forme de U de la puce (fil bleu dans la Fig.~\ref{fig:puce}). Ce fil est parcouru par un courant variant de \( 3.6\,\mathrm{A} \) à \( 1.5\,\mathrm{A} \), ce qui rapproche le nuage à quelques centaines de micromètres de la surface.

\paragraph{Mélasse optique}
Une phase de mélasse optique permet un refroidissement sub-Doppler des atomes capturés. Un système d’imagerie provisoire est utilisé à cette étape pour visualiser le nuage atomique, dont la taille dépasse le champ d’observation du système d’imagerie final.

\paragraph{Pompage optique}
Afin de polariser les atomes dans l’état magnétique \( |F=2,\,m_F=2\rangle \), un pompage optique est effectué avec un faisceau circulairement polarisé \( \sigma^+ \), résonant sur la transition \( |F=2\rangle \rightarrow |F'=2\rangle \).


\subsection{Piégeage magnétique sur puce}
{\color{blue}
\begin{itemize}
    \item Présentation de la puce atomique.
    \item Confinement transverse et longitudinal.
    \item Régime 1D : conditions d’accès (\(\hbar \omega_\perp \gg k_B T\)).
    \item Problèmes de rugosité, stabilité magnétique.
\end{itemize}
}

\subsubsection{Piégeage magnétique sur puce}
\label{sec:piegeage_puce}

\paragraph{Principe général}
Les atomes de rubidium sont piégés grâce à une puce atomique intégrée dans l’enceinte à vide. Une puce atomique est un circuit microfabriqué contenant des micro-fils dans lesquels circulent des courants permettant de générer des champs magnétiques à géométrie contrôlée. Ce dispositif, développé dans les années 1990, permet une miniaturisation du système de piégeage et un accès à des confinements forts, particulièrement adaptés à l'étude de gaz de Bose unidimensionnels.

\paragraph{Structure de la puce utilisée}
La puce utilisée au cours de cette expérience a été conçue en collaboration avec S.~Bouchoule, A.~Durnez et A.~Harouri (C2N). Elle repose sur un substrat de carbure de silicium sur lequel est déposé le circuit électrique. Ce dernier est recouvert d’une couche de résine BCB, aplanie par des cycles d’enduction et d’attaque plasma. Une fine couche d’or (\(\sim200\,\mathrm{nm}\)) est finalement évaporée afin de permettre l’utilisation de la puce comme miroir pour l’imagerie à \(780\,\mathrm{nm}\). La puce est soudée à l’indium sur une monture en cuivre inclinée à \(45^\circ\) par rapport à l’axe optique.

\paragraph{Fils de piégeage et géométrie des champs}
Plusieurs fils sont intégrés à la puce pour assurer les différentes étapes du piégeage et du transport des atomes : un fil en forme de Z est utilisé pour le piégeage initial (DC), tandis que trois micro-fils (symétriques et parallèles) sont utilisés pour former un guide unidimensionnel par courants alternatifs (AC). La géométrie des fils a été optimisée pour minimiser la dissipation de chaleur, limiter les couplages parasites et améliorer la symétrie du piège. Dans la zone d’intérêt, les atomes sont piégés à environ \(15\,\mu\mathrm{m}\) au-dessus des fils, soit à \(8\,\mu\mathrm{m}\) au-dessus de la surface de la puce.

\paragraph{Confinement transverse et longitudinal}
Le confinement transverse est assuré principalement par la géométrie des fils et la présence de champs magnétiques externes. Sa fréquence élevée permet d’atteindre des énergies de confinement \(\hbar \omega_\perp\) bien supérieures aux énergies thermiques et chimiques du système, condition nécessaire à l’accès au régime 1D :
\[
k_B T, \mu \ll \hbar \omega_\perp.
\]
Le confinement longitudinal, plus faible, est modulable par combinaison de champs magnétiques externes et courants dans les fils additionnels.

\paragraph{Avantages du piégeage sur puce}
Comparé aux systèmes utilisant des réseaux optiques 2D, le piégeage sur puce ne fournit qu’un seul tube, ce qui permet un meilleur accès aux fluctuations locales de densité et aux observables résolues spatialement. Ce type de dispositif est ainsi particulièrement adapté à l'étude de la thermodynamique et de la dynamique de gaz 1D isolés.

\paragraph{Limitations et effets parasites}
Parmi les limitations spécifiques au piégeage sur puce figurent la rugosité des potentiels magnétiques due aux imperfections des fils, qui peut induire des modulations parasites du confinement longitudinal. De plus, la stabilité du dispositif est sensible aux champs parasites magnétiques externes ainsi qu’aux échauffements dus aux courants continus.



\paragraph{Chargement dans le piège DC}
Après le pompage optique, les atomes sont transférés dans un piège magnétique combinant un courant continu circulant dans le fil en forme de Z de la puce (fil orange dans la Fig.~\ref{fig:puce}) et un champ magnétique externe. Ce piège, noté \emph{piège DC}, permet un confinement transverse important. Un refroidissement par évaporation radiofréquence est alors réalisé pendant environ \( 2.3\,\mathrm{s} \), ce qui abaisse la température du nuage à environ \( 1\,\mu\mathrm{K} \), pour un nombre d’atomes typiquement autour de \( 2.5 \times 10^5 \).

\paragraph{Imagerie finale}
À l’issue de ce refroidissement, les atomes sont observés avec le système d’imagerie final (voir Fig.~\ref{fig:imagerieFinale}), adapté aux tailles caractéristiques du gaz dans le piège. Une image typique de ce nuage est présentée en Fig.~\ref{fig:nuageDC}.

\paragraph{Transfert vers le guide unidimensionnel}
À la suite du premier refroidissement, les atomes sont transférés du piège DC vers un guide unidimensionnel produit par des courants alternatifs circulant dans trois micro-fils intégrés à la puce (fils jaunes dans la Fig.~\ref{fig:puce}). Ce transfert permet de réaliser un piège de type "guide 1D", où la dynamique longitudinale est découplée du confinement transverse.

Ce dispositif a été mis en place au cours de la thèse de Léa Dubois~\cite{TheseLea}. J’ai utilisé ce guide dans le cadre des protocoles expérimentaux sur l’expansion longitudinale et les sondes locales de distribution de rapidité.

\paragraph{Rampes de courant et compensation du mouvement}
Pour effectuer un transfert adiabatique, cinq rampes de courant linéaires d'une durée de \( 50 \text{ à } 60\,\mathrm{ms} \) chacune sont appliquées successivement. Pendant cette opération :
\begin{itemize}
    \item les courants dans le fil Z (piège DC) diminuent progressivement ;
    \item les courants dans les micro-fils du guide augmentent jusqu’à atteindre environ \( 50\,\mathrm{mA} \) ;
    \item un courant initial de \( 0.5\,\mathrm{A} \) est appliqué dans deux autres fils notés \( D \) et \( D' \) (fils vert et bleu sur la Fig.~\ref{fig:puce}), puis ajusté pendant les rampes afin de maintenir fixe la position du centre de masse du nuage.
\end{itemize}

Ce protocole précis a permis d’optimiser la stabilité du transfert, en minimisant les oscillations résiduelles dans le guide 1D.

\paragraph{Refroidissement final et accès au régime unidimensionnel}
Une dernière phase de refroidissement par évaporation radiofréquence est ensuite réalisée dans le guide AC. Ce refroidissement, mené dans le piège à forte anisotropie, permet d’atteindre le régime unidimensionnel, caractérisé par la hiérarchie d’énergies :
\[
k_B T, \mu \ll \hbar \omega_\perp
\]
où \( \omega_\perp \) est la fréquence de confinement transverse, \( \mu \) le potentiel chimique et \( T \) la température du gaz.

Les gaz obtenus contiennent typiquement entre \( 3 \times 10^3 \) et \( 1.5 \times 10^4 \) atomes, pour des températures de l’ordre de \( 50 \text{ à } 200\,\mathrm{nK} \). La Fig.~\ref{fig:gaz1D} montre un exemple de tel gaz observé avec le système d’imagerie final.


\paragraph{Remarques expérimentales}
Lorsque j’ai rejoint l’équipe, la première année thèse de Léa Dubois touchait à sa fin et le dispositif expérimental était en fonctionnement stable. Les différentes étapes du cycle (dispenser, PMO, mélasse, pompage optique, piège DC, transfert vers le guide, évaporation finale) avaient été mises en place et optimisées pendant les premières années de sa thèse, sous la supervision d’I. Bouchoule.Le cycle expérimental complet dure environ 15 secondes. Une description plus détaillée peut être trouvée dans la thèse d’A. Johnson~\cite{Johnson2016}.


Pendant ma première année, j’ai principalement participé à la prise de données en collaboration avec Léa. Grâce à la qualité de son travail, le dispositif était globalement très fiable, ce qui a permis de mener des campagnes expérimentales riches sans intervention lourde. Néanmoins, cette stabilité avait pour contrepartie que je n’ai pas été directement impliqué dans la résolution des pannes complexes ou dans le reconditionnement complet de la manipulation, ce qui a limité ma formation sur les aspects de maintenance approfondie du dispositif.

En revanche, peu avant la fin de la thèse de Léa et au début de ma troisième année, nous avons observé une chute significative du nombre d’atomes capturés. Sous la supervision d’I. Bouchoule, une intervention lourde a alors été décidée : nous avons cassé le vide pour diagnostiquer le problème. Il s’est avéré que les connecteurs du dispenser étaient endommagés. L’opération a été mise à profit pour installer un nouveau dispenser et remplacer la puce atomique.

Cette opération a mobilisé plusieurs personnes du laboratoire et de ses partenaires : S. Bouchoule (C2N) et Anne [Nom complet à préciser] ont participé à la manipulation et à la pose de la puce, tandis que j’ai pu assister à l’étuvage de l’enceinte à vide avec F. Nogrette. Après cette intervention, j’ai suivi avec I. Bouchoule le réajustement progressif de la séquence de refroidissement : alignement des faisceaux, réglages de la mélasse, optimisation du chargement dans le piège DC, puis dans le guide.

Cet épisode m’a permis de me confronter plus directement aux paramètres critiques du cycle d’évaporation et à la reprise d’une séquence complète. Toutefois, le départ de Léa, qui maîtrisait tous les aspects de la manipulation, a marqué une rupture importante dans la continuité des savoir-faire pratiques liés à cette expérience.


\begin{center}
	({fig:puce} — Schéma de la puce atomique avec fils U, Z, AC, D et D'.)
\end{center}
\begin{center}
	({fig:imagerieFinale} — Schéma optique du système d’imagerie final)
\end{center}
\begin{center}
	[{fig:nuageDC} — Image du gaz dans le piège DC après évaporation]
\end{center}
\begin{center}
	[{fig:gaz1D} — Image typique d’un gaz dans le régime 1D]
\end{center}



\subsection{Génération de potentiels modulés}
\begin{itemize}
    \item Courants modulés pour créer des pièges harmoniques ou quartiques.
    \item Découplage transverse/longitudinal.
\end{itemize}

\paragraph{Caractérisation des potentiels longitudinal et transverse}

Pour atteindre le régime unidimensionnel, les potentiels de piégeage doivent être très asymétriques : un confinement transverse fort et un confinement longitudinal faible. La fréquence transverse \(\omega_\perp\) doit être suffisamment élevée pour geler les degrés de liberté dans cette direction, avec la condition \(\mu, k_B T \ll \hbar \omega_\perp\).

\paragraph{Potentiel longitudinal}

Le confinement longitudinal est produit par des courants continus ou modulés dans certains fils. Dans certains protocoles spécifiques, on utilise un potentiel quartique \( V_\parallel(x) = c_4 x^4 \). Le système reste dans le régime 1D tant que la longueur caractéristique longitudinale reste beaucoup plus grande que la transverse.

\paragraph{Potentiel transverse}

Le confinement transverse est réalisé à l’aide de trois micro-fils parallèles situés sur la puce : un fil central parcouru par un courant \( I \), et deux fils latéraux par des courants opposés \(-I\). Cette configuration crée un piège transverse harmonique avec une fréquence \(\omega_\perp\) contrôlable par la valeur du champ \( B_0 \) et le courant. Les atomes sont piégés à environ \( d = 15~\mu\text{m} \) au-dessus de la puce. La fréquence maximale accessible expérimentalement est de l’ordre de \( \sim 100~\text{kHz} \).

\paragraph{Effet de rugosité et suppression par modulation}

La rugosité des micro-fils induit des fluctuations parasites du champ magnétique le long du guide. Pour supprimer cet effet, les courants sont modulés à haute fréquence (environ 400~kHz). Grâce à cette modulation rapide, les atomes ne ressentent que le potentiel moyen, dans lequel la composante parasite longitudinale du champ s’annule. Ce procédé permet d’obtenir un potentiel transverse régulier et stable, avec une fréquence efficace \[ f_\perp = \frac{f_\perp^{(0)}}{\sqrt{2}}. \]

\paragraph{Découplage des confinements transverse et longitudinal.}
Dans notre dispositif, le confinement transverse est assuré par les micro-fils modulés, tandis que le confinement longitudinal est généré par quatre fils extérieurs (D, D', d, d'). L’analyse du potentiel magnétique moyen montre que, sous l’hypothèse d’un champ de bobine homogène et dominant, les contributions transverse et longitudinale du potentiel sont découplées. Cette propriété est cruciale pour nos expériences : elle permet de modifier la géométrie du potentiel longitudinal sans perturber le confinement transverse, facilitant ainsi l’exploration de différentes configurations dynamiques.

\paragraph{Piégeage longitudinal harmonique.}
Un piège longitudinal harmonique est réalisé en appliquant des courants égaux dans les fils D et D', disposés de manière symétrique. Le champ magnétique longitudinal produit conduit à un potentiel quadratique local :
\[
V_\parallel(x) = V_0 + \frac{1}{2} m \omega_\parallel^2 x^2,
\]
avec une fréquence $\omega_\parallel$ contrôlée par le courant et la géométrie de la puce. En pratique, des fréquences jusqu’à 150 Hz sont atteintes pour des courants de 4 A. Une correction peut être nécessaire pour prendre en compte un champ magnétique résiduel $B_{0v}$, responsable d’un déplacement du centre du nuage atomique.

\paragraph{Piégeage longitudinal quartique.}
L’ajout de deux fils supplémentaires (d et d') permet de modifier la forme du potentiel longitudinal jusqu’à l’ordre 4. En ajustant les courants dans les quatre fils, on peut annuler le terme quadratique et obtenir un potentiel quartique :
\[
V_\parallel(x) = a_0 + a_4 x^4.
\]
Cette configuration est particulièrement adaptée pour générer des profils de densité homogènes, comme requis dans certaines expériences de transport. Le transfert des atomes du piège harmonique vers le piège quartique est réalisé de manière diabatique (changement rapide du potentiel), car un transfert adiabatique entraînait des pertes importantes.



\section{Sélection spatiale avec DMD}
\subsection{Motivation et principe}
{\color{blue}
\begin{itemize}
    \item Besoin de préparer des tranches homogènes.
    \item Intérêt dans les protocoles hors équilibre.
\end{itemize}
}

\paragraph{Objectif du dispositif de sélection}

L’outil de sélection spatiale a été conçu pour permettre une action locale sur le gaz atomique. Il présente deux objectifs principaux. D’une part, il permet de mesurer la distribution de rapidité localement résolue, en sélectionnant une tranche du gaz avant de la libérer et de suivre son expansion. D’autre part, il offre la possibilité de créer des situations hors équilibre en retirant une partie du gaz à l’équilibre, ce qui perturbe la configuration initiale et initie une dynamique.

\paragraph{Intérêt pour les protocoles hors équilibre}

Ce dispositif permet ainsi de générer des protocoles analogues à des configurations classiques comme le pendule de Newton, ou de sonder directement la dynamique d’un gaz de Lieb-Liniger dans des conditions contrôlées. Il constitue une brique essentielle pour les expériences de dynamique et de transport quantique.


\subsection{Mise en place technique (initiée par Léa Dubois)}

{\color{blue}
\begin{itemize}
    \item Dispositif optique de projection.
    \item Contrôle numérique des motifs.
    \item Calibration et stabilité.
\end{itemize}
}

\paragraph{Principe de sélection par pression de radiation}

La sélection repose sur l’illumination d’une zone définie du gaz avec un faisceau quasi-résonant avec la transition cyclique \( F=2 \rightarrow F'=3 \) de la ligne D2 du rubidium. Les atomes subissent une pression de radiation due aux cycles absorption/émission spontanée, ce qui les pousse hors du piège ou les amène dans un état non piégé.

\paragraph{Façonnage spatial du faisceau}

La sélection doit être spatialement résolue. Le profil d’intensité dans le plan des atomes est de type binaire :
\[
I(x) = 
\begin{cases}
0 & \text{si } x \in [x_1, x_2] \\
I_0 & \text{sinon}
\end{cases}
\]
ce qui permet de préserver ou d’éjecter les atomes selon leur position longitudinale.

\paragraph{Utilisation du DMD}

Pour générer ce profil, un DMD (Digital Micromirror Device) est utilisé. Il s’agit d’une matrice de \(1024 \times 768\) micro-miroirs orientables individuellement (±12°). En inclinant ces miroirs, on contrôle localement la réflexion de la lumière. L’image du DMD est projetée directement sur le plan des atomes, en imagerie directe.

\paragraph{Avantages du DMD}

Le DMD permet une reconfiguration rapide et programmable du motif de lumière. Cette technologie est largement utilisée dans les expériences d’atomes froids pour produire des potentiels structurés, homogénéiser un faisceau ou adresser localement les atomes.

\paragraph{Alternatives possibles}

Il est possible, en théorie, d’atteindre un effet similaire par un transfert cohérent des atomes vers un état anti-piégé via un pulse micro-onde ou une transition Raman. Cependant, la méthode par pression de radiation est plus simple à mettre en œuvre et adaptée à nos objectifs expérimentaux.

\paragraph{Principe de l’expulsion par pression de radiation}

Un atome illuminé par un faisceau proche de la résonance peut être expulsé du piège soit par transition vers un état anti-piégé, soit par effet de pression de radiation. Cette dernière génère une accélération suffisante pour fournir une énergie cinétique supérieure à la profondeur du puits magnétique. Le nombre de photons diffusés nécessaire peut être estimé à partir de la conservation de l’impulsion : une vingtaine de photons suffisent typiquement à extraire un atome du piège dans nos conditions.

\paragraph{Modèle de diffusion et estimation du seuil}

Le taux de diffusion de photons est modélisé à l’aide d’un taux \(\Gamma_{\mathrm{sc}}\), dépendant de l’intensité \(I\), de l’intensité de saturation \(I_{\mathrm{sat}}\), d’un paramètre \(\alpha\) (lié à la polarisation et au champ magnétique) et du désaccord \(\delta\). À résonance, et pour un temps d’illumination \(\tau_p\), on peut estimer le nombre total de photons diffusés par atome par \(N_{\mathrm{sc}} = \tau_p \Gamma_{\mathrm{sc}}\).

\paragraph{Mesures expérimentales de la puissance nécessaire}

La puissance minimale nécessaire pour éjecter tous les atomes d’une zone illuminée est déterminée en fixant un temps d’illumination donné, puis en variant l’intensité du faisceau. L’analyse est réalisée après un délai d’attente de \(\sim 10\) ms, pour s’assurer que seuls les atomes encore piégés soient détectés. Il est observé que 99$\%$ des atomes sont retirés à partir d’un rapport \(I/I_{\mathrm{sat}} \simeq 0.12\).

\paragraph{Mesures de photons diffusés par fluorescence}

La quantité de photons diffusés est également mesurée par l’analyse du signal de fluorescence capté par la caméra. En calibrant le rapport entre photons détectés et photons diffusés (en tenant compte de l’efficacité optique du système), le nombre moyen de photons nécessaires pour éjecter un atome est confirmé expérimentalement autour de 20. Un ajustement du modèle de diffusion permet d’estimer le paramètre \(\alpha \simeq 0.4\).

\paragraph{Saturation et effets Doppler}

À fort temps d’illumination (\(\tau_p > 150\,\mu\)s), une saturation du nombre de photons diffusés est observée, interprétée comme un effet géométrique : les atomes accélérés atteignent physiquement la puce atomique et cessent de contribuer au signal. Une correction Doppler peut être introduite dans le modèle, mais reste négligeable (\(< 5\%\)) dans les régimes expérimentaux utilisés.

\paragraph{Limitations expérimentales de la sélection}

Plusieurs effets peuvent limiter l'efficacité ou la propreté de la sélection :
\begin{itemize}
    \item La diffraction liée à la taille finie de l’objectif entraîne un flou de l’ordre de \(1{-}2\,\mu\)m au bord des zones éclairées.
    \item Une diffusion parasite par la puce peut se produire à forte intensité si tout le DMD est illuminé ; cela est évité en réduisant la taille transverse du faisceau à quelques micro-miroirs seulement.
    \item Des inhomogénéités d’éclairement dues à la gaussienne du faisceau et au speckle peuvent conduire à une sur-illumination de certaines zones. Un effort a été fait pour homogénéiser l’intensité en sortie de fibre.
    \item La réabsorption des photons diffusés pourrait entraîner un échauffement du gaz restant. Un désaccord en fréquence de 15 MHz a été testé pour éviter ce phénomène, sans effet visible sur la température du gaz.
\end{itemize}

\paragraph{Mesures de l’impact sur le gaz restant}

La température du gaz sélectionné est comparée avant et après sélection via l’analyse des fluctuations de densité après temps de vol. Aucun changement significatif de température ni d’élargissement n’a été observé. Ces résultats suggèrent que, dans les conditions expérimentales utilisées, la sélection ne perturbe pas significativement les atomes restants.




\subsection{Utilisation dans les protocoles}

{\color{blue}
\begin{itemize}
    \item Formes utilisées : boîtes, barrières, coupures.
    \item Préparation initiale contrôlée du gaz.
    \item Exemples de protocoles expérimentaux utilisant le DMD
\end{itemize}
}

\paragraph{Sélection locale et mesure de rapidité}

En sélectionnant une tranche du gaz, on peut ensuite couper le confinement longitudinal et laisser cette tranche s’étendre. Le profil de densité asymptotique obtenu après un long temps d’expansion est proportionnel à la distribution de rapidité locale du gaz initial. Ce protocole permet ainsi une mesure résolue de \(\rho(x,t \to \infty) \sim \rho(v)\).

\paragraph{Génération d’états hors équilibre}

La sélection permet également de créer des discontinuités dans le profil de densité, et donc d’initier une dynamique hors équilibre. Par exemple, on peut ne conserver que deux paquets séparés de gaz, qui vont alors osciller l’un vers l’autre. Cette configuration est analogue à un pendule de Newton quantique.

\paragraph{Formes utilisées}

Les motifs projetés par le DMD peuvent prendre différentes formes : boîtes, barrières, coupures, etc. Cette flexibilité rend l’outil extrêmement précieux pour explorer diverses configurations initiales et protocoles dynamiques.

\paragraph{Contrôle logiciel du DMD}

Le pilotage du DMD repose sur l’utilisation d’un module intégré fourni par Vialux (V7001-SuperSpeed), qui comprend les bibliothèques logicielles ALP-4. Plusieurs configurations du DMD peuvent être chargées en mémoire au début de chaque cycle expérimental, puis sélectionnées en cours de séquence à l’aide d’un signal digital. Le temps de commutation des miroirs est inférieur à \(30\,\mu\mathrm{s}\), ce qui est compatible avec les protocoles étudiés.

\paragraph{Partage du faisceau avec la voie d’imagerie}

Le faisceau utilisé pour la sélection spatiale est prélevé à partir du faisceau sonde déjà accordé sur la transition \(F=2 \rightarrow F'=3\) de la raie D2. Le partage est réalisé à l’aide d’un cube séparateur de polarisation placé en aval d’une lame demi-onde, permettant de contrôler la puissance injectée dans la fibre optique. Ce choix simplifie la mise en œuvre en évitant d’ajouter une source laser supplémentaire.

\paragraph{Blocage du faisceau de sélection}

Deux systèmes permettent de couper le faisceau de sélection pendant le cycle expérimental :
\begin{itemize}
    \item un cache mécanique (type électro-aimant), utilisé pour un blocage longue durée ;
    \item un modulateur acousto-optique (AOM), permettant de produire des impulsions brèves de quelques dizaines de \(\mu\mathrm{s}\), en amont du séparateur.
\end{itemize}
Pour garantir que le faisceau ne perturbe pas l’imagerie, le cache mécanique reste fermé pendant l’utilisation du faisceau sonde.

\paragraph{Montage optique de projection}

Le faisceau façonné par le DMD est projeté dans le plan des atomes à l’aide d’un système optique permettant de sélectionner l’ordre 0 de diffraction. L’ensemble des optiques est dimensionné (diamètre \(50\,\mathrm{mm}\)) pour limiter la diffraction. L’alignement est effectué en superposant le faisceau de sélection à la voie d’imagerie.

\paragraph{Grandissement et champ couvert}

Le montage permet de couvrir une zone de l’ordre de \(600\,\mu\mathrm{m}\) dans le plan des atomes, soit plus que la longueur typique d’un nuage (\(\sim 400\,\mu\mathrm{m}\) pour \(f_{\parallel}=5\,\mathrm{Hz}\)). Le grandissement est déterminé par les focales utilisées : une focale \(f_1 = 750\,\mathrm{mm}\) du côté du DMD, et \(f = 32\,\mathrm{mm}\) pour l’objectif côté atomes, donnant \(G = f/f_1 \approx 0.043\).

\paragraph{Visualisation et interface}

Le contrôle du DMD s’effectue via une interface graphique permettant de prévisualiser les configurations de miroirs. Une capture d’écran de cette interface est présentée dans la Fig.~\ref{fig:dmd_interface}, où la zone active réfléchie est visualisée en rouge. Cette interface est pilotée de manière automatisée pendant le déroulement de la séquence expérimentale.


\section{Techniques d’imagerie et d’analyse}
\subsection{Imagerie par absorption}
{\color{blue}
\begin{itemize}
    \item Imagerie \textit{in situ} et après temps de vol.
    \item Résolution, limites instrumentales.
\end{itemize}
}

\paragraph{Système d’imagerie par absorption}

L’imagerie est réalisée à l’aide d’une caméra CCD à déplétion profonde, optimisée pour une grande efficacité quantique à la longueur d’onde de 780 nm. On utilise des techniques d’imagerie par absorption permettant d’extraire la densité optique \( D(x, z) \), elle-même reliée à la densité atomique 3D via la loi de Beer-Lambert. Le profil de densité linéaire \( n(x) \) est obtenu par intégration sur les directions transverses.

\paragraph{Imagerie après temps de vol}

En appliquant un champ magnétique vertical (\( B = 8\,\mathrm{G} \)), la polarisation du faisceau peut être rendue circulaire (\( \sigma^+ \)) pour adresser la transition fermée \( |F=2, m_F=2\rangle \rightarrow |F'=3, m_F'=3\rangle \). Cette configuration assure une meilleure définition de la section efficace d’absorption. Un temps de vol de quelques ms est utilisé avant l’imagerie, permettant également de décomprimer le nuage.

\paragraph{Imagerie in situ}

Sans champ magnétique, les atomes sont imagés à $7~\mu m$ de la puce, ce qui implique une double absorption du faisceau incident et réfléchi. Dans ce cas, la transition n’est pas fermée, ce qui nécessite une calibration du facteur de conversion entre la densité mesurée et la densité réelle. Un ajustement linéaire permet de relier les profils in situ aux profils obtenus après temps de vol.

\paragraph{Choix des paramètres d’imagerie}

L’intensité du faisceau sonde est choisie typiquement à \( I_0/I_{\mathrm{sat}} \approx 0.3 \) pour optimiser le rapport signal sur bruit tout en restant dans une zone de linéarité acceptable. Dans ces conditions, le nombre de photons diffusés est de l’ordre de \( N_{\mathrm{sc}} \approx 230 \) et le rayon de diffusion reste comparable à la résolution du système d’imagerie (\( \sim 2.6\,\mu \mathrm{m} \)).

\paragraph{Limites du modèle de Beer-Lambert}

La validité de la loi de Beer-Lambert repose sur une approximation à une particule. Dans le cas des gaz fortement denses ou quasi 1D, les effets collectifs, les réabsorptions et les couplages dipolaires peuvent invalider ce modèle. Pour cette raison, même pour l’imagerie in situ, un temps de vol court (\( \sim 1\,\mathrm{ms} \)) est souvent appliqué afin de diluer le gaz transversalement.

\paragraph{Défauts et instabilités expérimentales}

Plusieurs limitations instrumentales ont été identifiées :
\begin{itemize}
    \item La caméra initialement utilisée montrait des motifs parasites aléatoires ainsi qu’un offset variant au cours du temps. Le remplacement de la caméra a permis de résoudre ces problèmes.
    \item Des franges d’interférences apparaissaient lors de la division des images d’absorption, probablement dues à des effets Fabry-Pérot dans les optiques. Le désaxage du faisceau d’imagerie a permis d’en limiter l’impact.
    \item Des photons résiduels, même en l’absence de faisceau sonde, ont été détectés. Ces derniers proviennent vraisemblablement de diffusions multiples dans le système optique.
\end{itemize}

\paragraph{Conclusion}

La combinaison de l’imagerie in situ et après temps de vol, ainsi qu’une calibration soigneuse des paramètres optiques et expérimentaux, permettent d’accéder à des profils de densité fiables malgré les limites intrinsèques du système d’imagerie. Une attention particulière a été portée à la réduction des artefacts expérimentaux afin de garantir la précision des mesures.


\subsection{Analyse des profils}

{\color{blue}
\begin{itemize}
    \item Extraction des densités, tailles, températures.
    \item Distribution longitudinale.
    \item Estimation de la température par ajustement Yang-Yang (optionnel si pertinent).
\end{itemize}
}


\section{Expériences et protocoles étudiés}
Cette section peut être la plus personnelle, en précisant ton rôle à chaque fois.
\subsection{Expansion longitudinale}
\begin{itemize}
    \item Protocole d’expansion (libération longitudinale, maintien du confinement transverse).
    \item Suivi de l’évolution du profil.
    \item Analyse à différents temps d’expansion
    \item Comparaison aux modèles analytiques : solutions homothétiques, GP, asymptotiques.
\end{itemize}

\subsection{Sonde locale de distribution de rapidité}
\begin{itemize}
    \item Principe de la mesure : coupure d’une tranche puis expansion.
    \item Rôle du DMD dans la sélection.
    \item Accès à la distribution de vitesse locale.
    \item Comparaison avec les prédictions GHD.
    \item Limites et incertitudes
\end{itemize}

\section{Discussion sur les limites et les perspectives}
\begin{itemize}
    \item Contraintes techniques (bruit, alignement, stabilité de la puce…).
    \item Améliorations potentielles (résolution, contrôle du potentiel, automatisation).
    \item Perspectives pour d’autres types d’expériences (étude de chocs, turbulence quantique, etc.)
\end{itemize}

\section*{Conclusion}
\begin{itemize}
    \item Résumé de l’architecture du dispositif).
    \item Méthodes d’analyse utilisées et robustesse.
    \item Importance de l’expérience dans le contexte de l’étude des gaz quantiques unidimensionnels
\end{itemize}
Ce chapitre a présenté les éléments essentiels du dispositif expérimental, les méthodes d’imagerie, ainsi que les expériences auxquelles j’ai participé. L’ensemble constitue une plateforme performante pour l’étude de la dynamique de gaz 1D hors équilibre.

\paragraph{Résumé de l’architecture expérimentale}  
Nous avons décrit les éléments clés du dispositif utilisé : un système de refroidissement laser basé sur trois sources couplées, un piégeage magnétique sur puce optimisé pour réaliser des géométries unidimensionnelles, une plateforme de modulation de potentiel via un DMD, et un système d’imagerie haute résolution. L’ensemble permet une manipulation fine des nuages atomiques dans un cadre reproductible et stable.

\paragraph{Méthodes d’analyse et robustesse}  
L’imagerie par absorption, couplée à une analyse rigoureuse des profils atomiques, fournit des outils fiables pour extraire les grandeurs pertinentes : densités, tailles, températures, distributions de vitesses. Ces méthodes ont permis de confronter les résultats expérimentaux à des prédictions théoriques de type GHD ou Yang-Yang.

\paragraph{Importance du dispositif pour la thèse}  
Ce dispositif a été essentiel pour mener à bien les expériences présentées dans cette thèse. Il offre à la fois un contrôle local (grâce au DMD), un bon confinement transverse (grâce à la puce) et une imagerie précise. La plateforme est ainsi bien adaptée pour étudier des systèmes 1D fortement corrélés hors équilibre, et pour tester les prédictions de la physique statistique intégrable.

\paragraph{Perspectives}  
Malgré ses atouts, le dispositif présente des limitations techniques (rugosité magnétique, sensibilité à l’alignement, etc.) qui laissent entrevoir des pistes d’amélioration. Des développements futurs pourraient notamment viser à augmenter la résolution spatiale, automatiser davantage les séquences, ou explorer d'autres régimes dynamiques comme la turbulence ou les collisions de chocs quantiques.



%\appendix
\section*{Annexes}
\begin{itemize}
    \item Schémas techniques (puce, DMD, optique).
    \item Tableaux de paramètres expérimentaux.
    \item Exemples de motifs DMD utilisés.
\end{itemize}