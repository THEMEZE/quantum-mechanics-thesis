\minitoc

\section*{Introduction}

%Considérons un système homogène de taille $L$. Faisons l'hypothèse qu'après relaxation, le système est décrite par un ensemble généralisé de Gibbs (GGE). On a vus dans le chapire d'avant que la valeur moyenne des observables $\langle\operator{\mathcal{O}}\rangle$ dans un ensemble statistique s'écrit comme une intégrale formelle

Considérons un système intégrable unidimensionnel, homogène, de longueur \(L\). Comme établi dans le chapitre~(\ref{chap:relaxation}), sa relaxation vers un état stationnaire est décrite par un \textit{Generalized Gibbs Ensemble} (GGE). Dans la limiter thermodynamiques la moyenne d’une observable locale $\underset{\mbox{\tiny therm.}}{\lim}\langle \operator{\mathcal{O}} \rangle_{GGE}$ que pour plus de clairetais , dans les calcule future je noterais \(\langle\operator{\mathcal{O}}\rangle\) se formule alors comme une intégrale fonctionnelle sur les distributions de rapidité~\cite{YangYang1969} :

%Considérons un système intégrable unidimensionnel, homogène, de longueur  \(L\). Comme discuté au chapitre~\ref{chap:relaxation}, sa relaxation vers un état stationnaire est décrite par un \textit{Generalized Gibbs Ensemble} (GGE). Dans la limite thermodynamique, la moyenne d’une observable locale, notée désormais \(\langle\operator{\mathcal{O}}\rangle\) pour alléger les notations, s’écrit comme une intégrale fonctionnelle sur les distributions de rapidité~\cite{YangYang1969} :
\begin{eqnarray}
	\underset{\mbox{\tiny therm.}}{\lim}\langle \operator{\mathcal{O}} \rangle_{GGE} & = & \frac{\int \mathcal{D} \rho \; e^{L (\mathcal{S}_{YY}[\rho] - \mathcal{W}[\rho])} \, \langle\operator{\mathcal{O}} \rangle_{[\rho]}}{\int \mathcal{D} \rho \; e^{L (\mathcal{S}_{YY}[\rho] - \mathcal{W}[\rho])}}, \label{chap:fluctu:eq:ensemble_average}
\end{eqnarray}
avec $\langle\operator{\mathcal{O}} \rangle_{[\rho]}$ la valeur de l’observable dans un état propre caractérisé par la distribution de rapidité $\rho$, $\mathcal{S}_{YY}$ l'entropie de Yang-Yang \eqref{chap.2.entropi.int} et $\mathcal{W}$ l'énergie généralisé par unité de longueur \eqref{chap.2.W.int}.

\medskip

Dans la limite thermodynamique \(L\to\infty\), cette moyenne fonctionelle \eqref{chap:fluctu:eq:ensemble_average} sur tous les profils de rapidité $\rho$, avec un poids $\propto e^{(\mathcal{S}_{YY}[\rho] - \mathcal{W}[\rho])}$. Dans la limite thermodynamique $L \to \infty$ , cette intégrale est dominée par la configuration $ \rho_{eq}$ qui maximise le poids exponentiel, c’est-à-dire la distribution de rapidité la plus probable. Ainsi, la moyenne de l’observable s’écrit au premier ordre de la méthode du point selle :
\begin{eqnarray}
	\underset{\mbox{\tiny therm.}}{\lim}\langle \operator{\mathcal{O}} \rangle_{GGE} & \approx & \langle\operator{\mathcal{O}} \rangle_{[\rho_{eq}]}	
	\label{chap:fluctu:eq:ensemble_average:approx}
\end{eqnarray}

	

%\paragraph{Pourquoi étudier les fluctuations ?}

%L’hypothèse selon laquelle, après relaxation, le système est décrit par un \textit{Generalized Gibbs Ensemble} (GGE) constitue un fondement majeur de notre compréhension des dynamiques hors équilibre dans les systèmes intégrables. Cette hypothèse a des implications théoriques profondes et mérite d’être testée expérimentalement. Toutefois, la seule connaissance de la distribution de rapidité moyenne \(  \rho_{eq} \) ne permet pas de confirmer l'adéquation du GGE. En effet, plusieurs ensembles statistiques peuvent mener à une même valeur moyenne de \( \rho \). Pour lever cette ambiguïté, il est nécessaire d’étudier les fluctuations autour de la distribution typique, notées \( \delta \rho \), définies par 
%\(
%	 \rho  =   \rho_{eq} + \delta \rho .	
%\)
%Leur étude nécessite de pousser le développement fonctionnel
%de la {\em fonction thermodynamique effective} : 
%\(
%(\mathcal{S}_{YY}-\mathcal{W})[\rho]
%\)
%au second ordre en \(\delta\rho\).\\

%Si la GGE décrit correctement la valeur moyenne de $\rho(\theta)$ après relaxation, il est naturel de se demander si elle capture aussi les {\bf fluctuations}  autour de cette moyenne. Autrement dit, notre objectif est de vérifier expérimentalement si la GGE est l’" {\em ensemble statistique correct} " pour l’état d’équilibre en étudiant non seulement la distribution moyenne des rapidités, mais aussi ses fluctuations. \\

%Plusieurs travaux récents ont souligné l’intérêt de sonder ces fluctuations. De Nardis et al. montrent que la mesure de la {\em structure dynamique} de la densité après quench permet de reconstruire entièrement l’état stationnaire, c’est-à-dire la distribution de rapidités du GGE \cite{??} . Concrètement, ils mettent en évidence que l’analyse du facteur de structure dynamique fournit accès à chacune des " {\em températures effectives} " $\beta_i$ et donc à la distribution macroscopique $\rho(\theta)$ \cite{??} \cite{??} . Ainsi, en mesurant les corrélations dynamiques du gaz (quantité accessible via spectroscopie ou fluctuations de densité), on peut tester si les fluctuations observées concordent avec celles prédites par la GGE. \\

%En pratique, l’étude des fluctuations de $\rho(\theta)$ consiste à analyser la dispersion des vitesses (ou rapidités) sur des répétitions expérimentales du même quench. Si la GGE est correcte, la variance et les corrélations de ces fluctuations devraient correspondre aux prédictions de $\operator{\rho}^{(\mathcal{S})}_{\mathrm{GGE}}$ \eqref{chap.TBA.op.rho.S}. À terme, une telle analyse permettrait de confirmer que l’état final du gaz suit bien la statistique GGE plutôt que la distribution thermique classique. En résumé, les fluctuations de la distribution de rapidités constituent un test clé de la validité de la GGE pour modéliser les résultats expérimentaux dans le modèle de Lieb-Liniger \cite{??}\cite{??}. \\

%-----------------------------------
\paragraph{Pourquoi étudier les fluctuations ?}

L’hypothèse selon laquelle, après relaxation, le système est décrit par un \textit{Generalized Gibbs Ensemble} (GGE) constitue un fondement majeur de notre compréhension des dynamiques hors équilibre dans les systèmes intégrables. Cette hypothèse, bien que robuste théoriquement, appelle à être testée expérimentalement.

Toutefois, la seule connaissance de la distribution de rapidité moyenne \( \rho_{\mathrm{eq}} \) ne permet pas, à elle seule, de confirmer la validité du GGE. En effet, plusieurs ensembles statistiques peuvent mener à une même valeur moyenne de \( \rho(\theta) \). Pour lever cette ambiguïté, il est nécessaire d’étudier les \textbf{fluctuations} autour de la distribution typique, notées \( \delta \rho \), définies par :
\(
\rho = \rho_{\mathrm{eq}} + \delta \rho.
\)
Cela nécessite de pousser le développement fonctionnel de la \emph{fonction thermodynamique effective} \( (\mathcal{S}_{YY} - \mathcal{W})[\rho] \) à l’ordre quadratique en \( \delta \rho \).

\medskip
Si la GGE décrit correctement la valeur moyenne de \( \rho(\theta) \) après relaxation, il est naturel de se demander si elle capture également les \emph{fluctuations} autour de cette moyenne. Autrement dit, notre objectif est de tester si la GGE constitue le \textit{bon ensemble statistique} pour l’état stationnaire, en analysant non seulement la distribution moyenne des quasi-particules, mais aussi ses fluctuations.

%\medskip
%Plusieurs travaux récents ont mis en lumière l’intérêt expérimental de sonder ces fluctuations. De Nardis et al.\ ont notamment montré que la mesure de la \textit{structure dynamique} de la densité, après un quench, permet de reconstruire entièrement l’état stationnaire, c’est-à-dire la distribution \( \rho(\theta) \) du GGE~\cite{??}. En particulier, l’analyse du facteur de structure dynamique permet d’extraire les différentes \textit{températures effectives} \( \beta_i \) du GGE, et donc d’accéder à la distribution macroscopique des quasi-particules~\cite{??, ??}.

%\medskip
%Ainsi, en mesurant les corrélations dynamiques du gaz — accessibles expérimentalement via la spectroscopie ou les fluctuations de densité — on peut tester si les fluctuations observées concordent avec celles prédites par la GGE.

%\medskip
%Concrètement, cela consiste à analyser la dispersion des vitesses (ou rapidités) sur plusieurs répétitions expérimentales d’un même quench. Si la GGE décrit correctement l’état stationnaire, la variance et les corrélations des fluctuations de \( \rho(\theta) \) devraient être en accord avec les prédictions du formalisme fluctuationnel issu de l’entropie \( \operator{\rho}^{(\mathcal{S})}_{\mathrm{GGE}} \), cf.~\eqref{chap.TBA.op.rho.S}.
{\color{blue} { \color{red} (en traveaux)} 
\medskip
Plusieurs travaux récents ont mis en lumière l’intérêt expérimental de sonder ces fluctuations. De Nardis et al.\ ont notamment montré que la mesure de la \textit{structure dynamique} de la densité, après un quench, permet de reconstruire entièrement l’état stationnaire, c’est-à-dire la distribution \( \rho(\theta) \) du GGE~\cite{DeNardis2017}. En particulier, l’analyse du facteur de structure dynamique permet d’extraire les différentes \textit{températures effectives} \( \beta_i \) du GGE, et donc d’accéder à la distribution macroscopique des quasi-particules~\cite{Goldstein2013,CauxKonik2012}.

\medskip
Ainsi, en mesurant les corrélations dynamiques du gaz — accessibles expérimentalement via la spectroscopie ou les fluctuations de densité — on peut tester si les fluctuations observées concordent avec celles prédites par la GGE.

\medskip
Concrètement, cela consiste à analyser la dispersion des vitesses (ou rapidités) sur plusieurs répétitions expérimentales d’un même quench. Si la GGE décrit correctement l’état stationnaire, la variance et les corrélations des fluctuations de \( \rho(\theta) \) devraient être en accord avec les prédictions du formalisme fluctuationnel issu de l’entropie \( \operator{\rho}^{(\mathcal{S})}_{\mathrm{GGE}} \), cf.~\eqref{chap.TBA.op.rho.S}.

\medskip
Le lien entre fluctuations, fonctions de réponse, et ensembles de Gibbs généralisés (GGE) a suscité un intérêt croissant dans les systèmes quantiques intégrables. Le formalisme des charges quasi-locales et des potentiels conjugués dans le GGE a été précisé dans le modèle de Lieb–Liniger par Pálmai et Konik~\cite{Palmai2018}, qui montrent comment structurer la matrice densité en termes de fonctionnelles de rapidité. L'identité fondamentale liant la dérivée fonctionnelle de l'entropie de Yang–Yang au noyau de fluctuations $\chi(\theta,\theta')$ est également dérivée dans ce cadre.

La relation fluctuation–réponse dans les gaz bosoniques unidimensionnels a été étudiée en profondeur par De Nardis et al.~\cite{DeNardis2017}, qui proposent une méthode pour reconstruire les fluctuations thermiques à partir de fonctions de réponse dynamiques, en comparant mesures expérimentales et théories thermodynamiques. D'autres travaux, comme ceux de Goldstein et Andrei~\cite{Goldstein2013}, ou de Caux et Konik~\cite{CauxKonik2012}, examinent en détail la relaxation vers un GGE à la suite d’un quench quantique, et en particulier le rôle de la distribution de rapidité dans la description des états stationnaires.

Enfin, les travaux de Essler, Mussardo et Panfil~\cite{Essler2015} donnent une perspective unificatrice dans le cadre des théories de champs intégrables, où la fluctuation–réponse s’exprime à partir de noyaux intégrables et de corrélateurs généralisés.
}

\medskip
En résumé, l’étude des fluctuations de la distribution de rapidités fournit un test clé de la validité du GGE pour modéliser les résultats expérimentaux dans le modèle de Lieb–Liniger~\cite{??, ??}.

\medskip
Ce chapitre est consacré à cette extension, qui permettra :
\begin{itemize}[label = $\bullet$]
  \item d’obtenir les matrices de susceptibilité \(\chi_{ij}\)  et les corrélations gaussiennes du GGE ;
  %\item de relier ces fluctuations aux observables expérimentales  (temps de vol, bruit de densité, etc.) ;
  \item de fournir la base théorique des équations d’hydrodynamique généralisée au second ordre.
\end{itemize}
Nous commencerons par rappeler le formalisme variationnel, puis nous dériverons l’action quadratique régissant \(\delta\rho\).  Les contraintes d’intégrabilité et la structure du noyau \(\Delta\) joueront un rôle clé dans la diagonalisation de cette action, ouvrant la voie aux prédictions quantitatives sur les fluctuations GGE.


\section{Fluctuation-réponse et susceptibilités dans les états d’équilibre généralisés}

%Soit un système intégrable dans un état d’équilibre généralisé (GGE), On a vus dans le chapitre \ref{} on à  défini par une matrice densité de la forme :

Considérons un système intégrable dans un état d’équilibre généralisé (GGE). Comme nous l'avons vu au chapitre~(\ref{chap:relaxation}), un tel état est décrit par une matrice densité de la forme :

\begin{eqnarray*}
	\operator{\rho}_{\mathrm{GGE}}^{(\mathcal{S})}[w] \doteq  \frac{1}{Z^{(\mathcal{S})}[w]}\, e^{-\operator{\mathcal{Q}}^{(\mathcal{S})}[w]},\quad \text{où} \quad Z^{(\mathcal{S})}[w] \doteq \mathrm{Tr}\left(e^{-\operator{\mathcal{Q}}^{(\mathcal{S})}[w]}\right),
\end{eqnarray*}
où \( \operator{\mathcal{Q}}[w] \) est une charge généralisée déterminée par le poids/potentiel spectral \( w(\theta) \).\\

Nous avons vu que l'espérance d'une charge généralisée \( \operator{\mathcal{Q}}[f_1] \) dans cet état, notée au chapitre~(\ref{chap:relaxation}) \( \langle \operator{\mathcal{Q}}^{(\mathcal{S})}[f_1] \rangle_{\operator{\rho}_{\mathrm{GGE}}^{(\mathcal{S})}[w]} \), peut être simplifiée, dans ce chapitre, en :
\begin{eqnarray*}
	\langle \operator{\mathcal{Q}}[f_1] \rangle_w =  -  \left . \frac{ \delta \ln Z [w] }{ \delta f_1  } \right)_{w}  = \mathrm{Tr} \left[ \operator{\rho}_{\mathrm{GGE}}[w]\, \operator{\mathcal{Q}}[f_1] \right]
= \frac{1}{Z[w]}\, \mathrm{Tr} \left[ e^{-\operator{\mathcal{Q}}[w]}\, \operator{\mathcal{Q}}[f_1] \right] .	
\end{eqnarray*}

Par souci de lisibilité, nous omettrons les indices \((\mathcal{S})\) : le caractère local des observables étant désormais implicite.


%On a vus  que l’espérance d’une observable $\operator{\mathcal{Q}}^{(\mathcal{S})}[f_1]$ dans le GGE , noté dans le chapitre \ref{} par $\langle \operator{\mathcal{Q}}^{(\mathcal{S})}[f_1] \rangle_{\operator{\rho}_{\mathrm{GGE}}^{(\mathcal{S})}[w]}$ que pour soucis d'aléger dans lé résonement on notéra $\langle \operator{\mathcal{Q}}[f_1] \rangle$, et j'enlève les $(\mathcal{S})$, le caractère locale étant maintenat inplicite , est donnée par :

\medskip

%La dérivée fonctionnelle de l'espérance de la charge $\operator{\mathcal{Q}}[f_1]$ par rapport au potentiel conjugué $f_2$ est donnée par :
La dérivée fonctionnelle de l'espérance de \( \operator{\mathcal{Q}}[f_1] \) par rapport à une autre fonction test \( f_2 \), représentant un potentiel conjugué, donne la réponse linéaire croisée :

\begin{eqnarray*}
	\chi_w[f_1, f_2] \doteq   \left . \frac{ \delta^2 \ln Z [w] }{ \delta f_1 \delta f_2 } \right)_{w} =  - \left.\frac{\delta \langle \operator{\mathcal{Q}}[f_1] \rangle}{\delta f_2} \right )_w = C_w(f_1, f_2) , 	
\end{eqnarray*}

avec le fonction corrélation à deux points : 

\begin{eqnarray*}
	C_w(f_1, f_2) \doteq  	\langle ( \operator{\mathcal{Q}}[f_1] - \langle \operator{\mathcal{Q}}[f_1] \rangle_w ) ( \operator{\mathcal{Q}}[f_2] - \langle \operator{\mathcal{Q}}[f_f] \rangle_w  ) \rangle_w = \langle \operator{\mathcal{Q}}[f_1] \operator{\mathcal{Q}}[f_2] \rangle_w - \langle \operator{\mathcal{Q}}[f_1] \rangle_w \langle \operator{\mathcal{Q}}[f_2] \rangle_w.
\end{eqnarray*}


Autrement dit, la dérivée fonctionnelle seconde du logarithme de la fonction de partition fournit la covariance des charges \( \operator{\mathcal{Q}}[f_1] \) et \( \operator{\mathcal{Q}}[f_2] \), illustrant le principe de fluctuation-réponse dans ce contexte diagonal.

\paragraph{Démonstration.}

Pour établir le lien entre réponse linéaire et fluctuations, nous allons partir de l'expression intégrale des charges généralisées, puis montrer que la susceptibilité associée correspond bien à la fonction de corrélation des fluctuations de densité de rapidité.


\subparagraph{Charge généralisé sous forme intégrale.}
Pour le démontrer on écrit la charge généralisé sous forme intégrale : 
\begin{eqnarray*}
\operator{\mathcal{Q}}[f] = L\int d\theta f(\theta) \operator{\rho}(\theta)
\end{eqnarray*}
où $\operator{\rho}(\theta)$ est l'opérateur densité spectal , agissant comme :
\(
\operator{\rho}(\theta) \ket{ \{ \theta_a \} } = \frac{1}{L} \sum \delta ( \theta - \theta_a ) \ket{ \{ \theta_a \} }.
\)

\subparagraph{Dérivée fonctionnelle.}
On dérive cette expression par rapport à $w_2$. En utilisant la dérivée d’un rapport :
\(
	\left.\frac{\delta}{\delta f_2} \left( \frac{A[w]}{Z[w]} \right) \right |_{w} = \frac{1}{Z[w]}\, \left.\frac{\delta A[w]}{\delta f_2} \right|_{w} - \frac{A[w]}{Z[w]^2}\, \left.\frac{\delta Z[w]}{\delta f_2}\right|_{w} ,
\)
avec $A[w] \equiv \mathrm{Tr}[e^{-\operator{\mathcal{Q}}[w]} \operator{\mathcal{Q}}[f_1]]$, on obtient :
\begin{eqnarray*}
	\left . \frac{\delta \langle \operator{\mathcal{Q}}[f_1] \rangle_w}{\delta f_2} \right )_w
	= \frac{1}{Z[w]}\, \frac{\delta A[w]}{\delta f_2}
	- \langle \operator{\mathcal{Q}}[f_1] \rangle_w \cdot \frac{1}{Z[w]}\, \frac{\delta Z[w]}{\delta f_2}.
\end{eqnarray*}

\subparagraph{Dérivées explicites.}
Utilisons la propriété démontrée précédemment :
\(
	\frac{\delta \operator{\mathcal{Q}}[w]}{\delta f_2} = \operator{\mathcal{Q}}[f_2]	.
\)
En notant que la dérivée d’une exponentielle diagonale est simple :
\(
\frac{\delta}{\delta f_2} e^{-\operator{\mathcal{Q}}[w]} = - \operator{\mathcal{Q}}[f_2]\, e^{-\operator{\mathcal{Q}}[w]},
\)
on obtient :
\begin{eqnarray*}
\frac{\delta A[w]}{\delta f_2} = - \mathrm{Tr}\left[ \operator{\mathcal{Q}}[f_2]\, e^{-\operator{\mathcal{Q}}[w]} \operator{\mathcal{Q}}[f_1] \right],
\quad
\frac{\delta Z[w]}{\delta f_2} = - \mathrm{Tr}\left[ \operator{\mathcal{Q}}[f_2]\, e^{-\operator{\mathcal{Q}}[w]} \right].
\end{eqnarray*}

\subparagraph{Regroupement.}
En regroupant les termes, on trouve :
\begin{eqnarray*}
	\chi_w[f_1, f_2]&  \doteq & - \left.\frac{\delta \langle \operator{\mathcal{Q}}[f_1] \rangle_w}{\delta f_2} \right )_w  \\
&=& -\frac{1}{Z[w]}\, \mathrm{Tr}\left[ \operator{\mathcal{Q}}[w_2]\, e^{-\operator{\mathcal{Q}}[w]}\, \operator{\mathcal{Q}}[f_1] \right]
+ \langle \operator{\mathcal{Q}}[f_1] \rangle_w \cdot \frac{1}{Z[w]}\, \mathrm{Tr}\left[ \operator{\mathcal{Q}}[f_2]\, e^{-\operator{\mathcal{Q}}[w]} \right] \\
&=& \langle \operator{\mathcal{Q}}[f_2]\, \operator{\mathcal{Q}}[f_1] \rangle_w - \langle \operator{\mathcal{Q}}[f_1] \rangle_w\, \langle \operator{\mathcal{Q}}[f_2] \rangle_w.
\end{eqnarray*}


\subparagraph{Conclusion.} Dans un état diagonal tel que le GGE, la susceptibilité est égale à la covariance entre charges généralisées, illustrant le principe de \emph{fluctuation-réponse}.


\paragraph{Corrélations spectrales et susceptibilité.}

La susceptibilité fonctionnelle peut s’exprimer, dans la base des fonctions test, comme une forme bilinéaire projetée sur la fonction de corrélation spectrale :
\begin{eqnarray*}
\chi_w[f_1, f_2] = C_w[f_1, f_2] = L^2 \iint d\theta\, d\theta'\, f_1(\theta)\, f_2(\theta')\, C_w(\theta, \theta'),
\end{eqnarray*}
où la fonction de corrélation spectrale \( C_w(\theta, \theta') \) est définie comme la covariance des fluctuations de densité spectrale :
\begin{eqnarray*}
C_w(\theta, \theta') \doteq \langle \delta \operator{\rho}(\theta)\, \delta \operator{\rho}(\theta') \rangle_w = \langle \operator{\rho}(\theta)\, \operator{\rho}(\theta') \rangle_w - \langle \operator{\rho}(\theta) \rangle_w\, \langle \operator{\rho}(\theta') \rangle_w.
\end{eqnarray*}

Par ailleurs, l’espérance de la densité spectrale s’obtient comme une dérivée fonctionnelle du logarithme de la fonction de partition :
\begin{eqnarray*}
\langle \operator{\rho}(\theta) \rangle_w = - \frac{1}{L} \left. \frac{\delta \ln Z[w]}{\delta w(\theta)} \right)_w,
\end{eqnarray*}
et la susceptibilité spectrale, définie comme la réponse de \( \langle \rho(\theta) \rangle_w \) à une variation de \( w(\theta') \), s’écrit :
\begin{eqnarray*}
\chi(\theta, \theta') = \left. \frac{1}{L^2} \frac{\delta^2 \ln Z[w]}{\delta w(\theta)\, \delta w(\theta')} \right)_w = - \left. \frac{1}{L} \frac{\delta \langle \operator{\rho}(\theta) \rangle_w}{\delta w(\theta')} \right)_w.
\end{eqnarray*}

Ainsi, dans l’état diagonal \( \operator{\rho}_{\mathrm{GGE}}[w] \), la susceptibilité spectrale coïncide avec la fonction de corrélation \( C_w(\theta, \theta') \), illustrant le principe de \emph{fluctuation-réponse}.



\medskip


\paragraph{Remarques.}
La matrice $\chi_w[f_1, f_2]$ s’interprète donc comme la covariance de la densité spectrale projetée sur les fonctions test $f_1$ et $f_2$.

\medskip
Un cas particulier d’intérêt est la susceptibilité d’une seule charge :
\begin{eqnarray*}
\chi_w[f] \equiv \chi_w[f, f] = \mathrm{Var}(\operator{\mathcal{Q}}[f]) = \langle \operator{\mathcal{Q}}[f]^2 \rangle_w - \langle \operator{\mathcal{Q}}[f] \rangle_w^2
= L^2 \iint d\theta\, d\theta'\, f(\theta)\, f(\theta')\, C_w(\theta, \theta').
\end{eqnarray*}

\section{Limite thermodynamique, structure variationnelle et susceptibilités}

\subsection{Susceptibilités spectrales et structure variationnelle de l’entropie}

Dans l’approximation thermodynamique, l’opérateur de densité spectrale $\operator{\rho}(\theta)$ est remplacé par sa valeur moyenne macroscopique $\rho_{\mathrm{eq}}(\theta)$, représentant la densité continue de quasi-particules dans l’état d’équilibre.

Dans le cadre du GGE continu, l’état macroscopique est entièrement déterminé par une densité spectrale $\rho(\theta)$ qui maximise l’entropie de Yang-Yang $\mathcal{S}_{\mathrm{YY}}[\rho]$, sous la contrainte d’une charge généralisée fixée. Le poids spectral $w$ est fixé. On a vus dans les chapitre (\ref{chap:relaxation}) que la condition d’équilibre s’écrit alors :
\(
\left. \frac{\delta \mathcal{S}_{\mathrm{YY}}[\rho]}{\delta \rho(\theta)} \right)_{\rho = \rho_{\mathrm{eq}}} = \left. \frac{\delta \mathcal{W}[\rho]}{\delta \rho(\theta)} \right)_{\rho = \rho_{\mathrm{eq}}} =  w(\theta).
\)

\medskip
\paragraph{Dérivée fonctionnelle.}
On peut considérer cette équation comme une relation implicite définissant $w$ comme une fonctionnelle de $\rho$. La dérivée fonctionnelle de cette relation donne :
\[
\left.\frac{\delta w(\theta)}{\delta \rho(\theta')}\right)_{\rho = \rho_{\mathrm{eq}}} = \left. \frac{\delta^2 \mathcal{S}_{\mathrm{YY}}[\rho]}{\delta \rho(\theta)\, \delta \rho(\theta')} \right)_{\rho = \rho_{\mathrm{eq}}} \equiv \mathcal{H}^{(\mathcal{S}_{\mathrm{YY}})}(\theta, \theta'),
\]
où $\mathcal{H}^{(\mathcal{S}_{\mathrm{YY}})}$ est l'opérateur (généralement négatif) de courbure fonctionnelle de l'entropie.

\paragraph{Inversion.}
On en déduit que la réponse de $\rho$ à une variation du potentiel conjugué $w$ est donnée par l'inverse fonctionnel :
\[
\frac{\delta \rho_{\mathrm{eq}}(\theta)}{\delta w(\theta')} = \left( \mathcal{H}^{(\mathcal{S}_{\mathrm{YY}})} \right)^{-1} (\theta, \theta') \equiv - L \chi(\theta, \theta').
\]

%La fonction $\chi(\theta, \theta')$ définit la matrice de susceptibilité spectrale, qui décrit la réponse linéaire de la densité d’équilibre à une perturbation infinitésimale du potentiel conjugué. Par le principe de fluctuation-réponse, elle coïncide avec la matrice de corrélation des fluctuations thermodynamiques de la densité spectrale :
%\[
%\chi(\theta, \theta') = \langle \delta \rho(\theta) \, \delta \rho(\theta') \rangle.
%\]

\paragraph{Interprétation physique.}
La matrice $\chi(\theta, \theta')$ possède plusieurs interprétations équivalentes :
\begin{itemize}[label = $\bullet$]
  \item Elle mesure la réponse linéaire de la densité spectrale à une variation du potentiel conjugué :
  \[
  \chi(\theta, \theta') = - \frac{1}{L}\frac{\delta \rho_{\mathrm{eq}}(\theta)}{\delta w(\theta')}.
  \]
  \item Dans un état diagonal (comme le GGE), elle coïncide avec la matrice de corrélation, selon le principe fluctuation-réponse :
  \[
  \chi(\theta, \theta') = \langle \delta \operator{\rho}(\theta)\, \delta \operator{\rho}(\theta') \rangle_w.
  \]
  \item Dans la limite thermodynamique, où $\operator{\rho}(\theta) \to \rho(\theta)$, on peut omettre les chapeaux :
  \[
  \chi(\theta, \theta') = \langle \delta \rho(\theta)\, \delta \rho(\theta') \rangle_w.
  \]
\end{itemize}

\paragraph{Résumé.}
L'équation variationnelle d'équilibre :
\(
\left.\frac{\delta \mathcal{S}_{YY}[\rho_{eq}]}{\delta \rho(\theta)} \right|_{\rho = \rho_{\mathrm{eq}}} = w(\theta)
\)
implique, par différentiation fonctionnelle et inversion de l’opérateur $\mathcal{H}^{(\mathcal{S}_{\mathrm{YY}})}$, que :
\[
\chi(\theta, \theta') =   - \frac{1}{L}\frac{\delta \rho_{\mathrm{eq}}(\theta)}{\delta w(\theta')} = \langle \delta \rho(\theta)\, \delta \rho(\theta') \rangle  = - \left( L \mathcal{H}^{(\mathcal{S}_{\mathrm{YY}})} \right)^{-1} (\theta, \theta') .
\]

\noindent
En résumé, dans la limite thermodynamique, la matrice de susceptibilité spectrale $\chi(\theta, \theta')$ encode à la fois :

\begin{itemize}[label = $\bullet$] 
  \item la réponse linéaire de la densité d'équilibre à une variation du poids qpectral $w(\theta)$ ;
  \item la covariance (ou corrélation) entre densités spectrales ;
  \item la courbure du potentiel thermodynamique générateur du GGE.
\end{itemize}

%\paragraph{Remarque : Sur la linéarité de la contrainte.}
%Il est important de noter que la fonctionnelle de contrainte $\mathcal{W}[\rho] = \int w(\theta) \rho(\theta)\, d\theta$ est \emph{linéaire} en $\rho$ : le potentiel $w(\theta)$ y est une fonction test fixée et ne dépend pas de $\rho$. Par conséquent, sa dérivée fonctionnelle est simplement :
%\[
%\frac{\delta \mathcal{W}[\rho]}{\delta \rho(\theta)} = w(\theta), \qquad \text{et} \qquad \frac{\delta^2 \mathcal{W}[\rho]}{\delta \rho(\theta)\, \delta \rho(\theta')} = 0.
%\]
%Il ne faut donc \emph{pas} confondre cette structure linéaire avec l’équation d’équilibre
%\[
%\left. \frac{\delta \mathcal{S}_{\mathrm{YY}}[\rho]}{\delta \rho(\theta)} \right|_{\rho = \rho_{\mathrm{eq}}} = w(\theta),
%\]
%qui définit implicitement $w(\theta)$ comme une fonctionnelle de $\rho$. Dans ce second cadre, une variation de $\rho$ entraîne une variation de $w$, et on peut écrire :
%\[
%\frac{\delta w(\theta)}{\delta \rho(\theta')} = \left. \frac{\delta^2 \mathcal{S}_{\mathrm{YY}}[\rho]}{\delta \rho(\theta)\, \delta \rho(\theta')} \right|_{\rho = \rho_{\mathrm{eq}}}.
%\]
%Ainsi, seule l'entropie $\mathcal{S}_{\mathrm{YY}}$ porte la courbure fonctionnelle, et non la contrainte linéaire $\mathcal{W}[\rho]$.




%\section{Développement quadratique de l’action effective : formalisme variationnel}
%
%\subsection*{Expansion fonctionnelle autour de \(\rho_{\mathrm{eq}}\)}

%Comme rappelé dans l’introduction, la moyenne d’une observable dans le GGE
%est dominée, en \(L \to \infty\), par la distribution de rapidité
%\(\rho_{\mathrm{eq}}\) qui maximise l’action effective :
%\[
%\Phi[\rho] \coloneqq \mathcal{S}_{YY}[\rho] - \mathcal{W}[\rho].
%\]
%Nous nous intéressons à présent aux fluctuations autour de ce point
%selle, en développant fonctionnellement l’action à l’ordre 2 :
%\[
%\rho(\theta) = \rho_{\mathrm{eq}}(\theta) + \delta \rho(\theta).
%\]

%\paragraph{Développement de Taylor fonctionnel.}

%Sous une variation infinitésimale \(\delta \rho\), l’action
%s’écrit à l’ordre quadratique :
%\begin{equation}
%\Phi[\rho] = \Phi[\rho_{\mathrm{eq}}]
%+ \underbrace{\int d\theta\; \left.
%    \frac{\delta \Phi}{\delta \rho(\theta)}
%  \right|_{\rho=\rho_{\mathrm{eq}}}
%  \delta \rho(\theta)}_{\text{= 0 par stationnarité}}
%+ \frac{1}{2} \iint d\theta\,d\theta'\; \delta \rho(\theta)\,
%\mathcal{H}(\theta,\theta')\,\delta \rho(\theta')
%+ \mathcal{O}(\delta\rho^3),
%\label{eq:dev-fonctionnel-2}
%\end{equation}
%où \(\mathcal{H}\) est l’opérateur hessien de \(\Phi\) :
%\[
%\mathcal{H}(\theta,\theta') =
%\left. \frac{\delta^2 \Phi[\rho]}{\delta \rho(\theta)\, \delta \rho(\theta')}
%\right|_{\rho=\rho_{\mathrm{eq}}}.
%\]

%\paragraph{Structure de \(\mathcal{H}\).}
%L’opérateur hessien s’écrit naturellement comme la différence :
%\[
%\mathcal{H} = \mathcal{H}^{(S)} - \mathcal{H}^{(W)},
%\]
%avec
%\begin{align*}
%\mathcal{H}^{(S)}(\theta,\theta') &=
%\left. \frac{\delta^2 \mathcal{S}_{YY}}{\delta \rho(\theta)\, \delta \rho(\theta')} \right|_{\rho = \rho_{\mathrm{eq}}}, \\
%\mathcal{H}^{(W)}(\theta,\theta') &=
%\left. \frac{\delta^2 \mathcal{W}}{\delta \rho(\theta)\, \delta \rho(\theta')} \right|_{\rho = \rho_{\mathrm{eq}}}.
%\end{align*}
%
%\subsection*{Forme explicite du hessien : fluctuations gaussiennes}
%
%Il est classique (cf. appendix~\ref{appendix:YYentropy}) que
%l’entropie de Yang–Yang admet un hessien diagonal en base des \(\theta\) :
%\begin{equation}
%\mathcal{H}^{(S)}(\theta,\theta') =
%\frac{\delta(\theta-\theta')}{\nu_{\!eq}(\theta)\big(1-\nu_{\!eq}(\theta)\big)}.
%\label{eq:hessian-SYY}
%\end{equation}
%
%Quant à l’énergie généralisée,
%si l’observable est extensive (fonctionnelle additive),
%le terme \(\mathcal{H}^{(W)}\) est symétrique, et typiquement local ou à noyau fini.
%
%\medskip
%Au total, les fluctuations du GGE au voisinage de \(\rho_{\mathrm{eq}}\)
%sont décrites, à l’ordre quadratique, par une théorie de champs gaussienne
%dont l’action effective est donnée par :
%\begin{equation}
%\delta^2 \Phi[\delta \rho] = \frac{1}{2} \iint d\theta\,d\theta'\; \delta \rho(\theta)\,
%\left[ \frac{\delta(\theta-\theta')}{\nu_{\!eq}(\theta)(1-\nu_{\!eq}(\theta))} - \mathcal{H}^{(W)}(\theta,\theta') \right]
%\delta \rho(\theta').
%\label{eq:action-gaussienne}
%\end{equation}
%
%Cette structure permet l’évaluation explicite des variances et corrélations
%d’observables (voir chap.~\ref{chap:observables}) ainsi que
%la dérivation des matrices de susceptibilité, au cœur de la description
%des fluctuations thermodynamiques et du transport.
%
%\subsection*{Commentaires physiques}
%
%\begin{itemize}[label = $\bullet$]
%\item L’apparition de \(\nu_{\!eq}(1-\nu_{\!eq})\) dans le hessien
%  reflète une structure de type Fermi–Dirac,
%  même dans un système bosonique,
%  conséquence de l’exclusion statistique induite par l’intégrabilité.
%\item La matrice hessienne contrôle la réponse du système à une
%  variation infinitésimale de \(\rho\) et, par là, encode les
%  fluctuations statiques du GGE.
%\item Ce développement quadratique justifie le caractère
%  gaussien des fluctuations dans le régime thermodynamique,
%  et sera à la base des extensions hydrodynamiques de type MFT
%  (Macroscopic Fluctuation Theory).
%\end{itemize}



%\section{Développement autour du point selle}

%Cette configuration vérifie donc l’{\bf équation de point selle} :

%\begin{eqnarray*}
%	\left. \frac{\delta \mathcal{S}_{YY}[\rho]}{\delta \rho} \right|_{\rho = \langle \rho \rangle } 	 & = & \left. \frac{\delta \mathcal{W}[\rho]}{\delta \rho} \right|_{\rho = \langle \rho \rangle }.
%\end{eqnarray*}

%On développe de Taylor-Young à l’ordre quadratique :

%\begin{eqnarray*}
%	\mathcal{S}_{YY}[\rho] - \mathcal{W}[\rho] & \approx & \mathcal{S}_{YY}[\langle\rho\rangle] - \mathcal{W}[\langle\rho\rangle] + \frac{1}2 \left. \frac{\delta^2 (\mathcal{S}_{YY}[\rho] - \mathcal{W}[\rho]) }{(\delta \rho)^2} \right|_{\rho = \langle \rho \rangle }	(\delta \rho)^2,
%	\label{chap:fluctu:eq:action}	
%\end{eqnarray*}

%avec $-\left. \frac{\delta^2 (\mathcal{S}_{YY}[\rho] - \mathcal{W}[\rho]) }{(\delta \rho)^2} \right|_{\rho = \langle \rho \rangle }$ désigne la forme bilinéaire symétrique définie positive.

%avec
%\begin{eqnarray*}
%	\left. \frac{\delta^2 (\mathcal{S}_{YY}[\rho] - \mathcal{W}[\rho]) }{(\delta \rho)^2} \right|_{\rho = \langle \rho \rangle } & = & 
%\iint d\theta \, d\theta' \; 
%\left. \frac{\delta^2 (\mathcal{S}_{YY}[\rho] - \mathcal{W}[\rho]) }{\delta \rho(\theta) \delta \rho(\theta')} \right|_{\rho = \langle \rho \rangle } 
%\delta \rho(\theta) \delta \rho(\theta').
%\end{eqnarray*}

%Cette approximation fait de l’intégrale une {\bf intégrale gaussienne}.


%------------------------------------------------------------------
%\section{Fonction de corrélation à deux points}
%
%\subsection*{Définition}
%
%Soient deux observables macroscopiques \(\operator{\mathcal{A}}\) et \(\operator{\mathcal{B}}\) (e.g. énergie, nombre de particules, courant, etc.). On définit leur \textbf{corrélation connectée} (parfois appelée \emph{fonction de corrélation à deux points}) par
%\begin{equation}
%  C_{\operator{\mathcal{A}},\operator{\mathcal{B}}}
%  \;=\;
%  \bigl\langle
%    \bigl(\operator{\mathcal{A}}-\langle\operator{\mathcal{A}}\rangle\bigr)
%    \bigl(\operator{\mathcal{B}}-\langle\operator{\mathcal{B}}\rangle\bigr)
%  \bigr\rangle
%  \label{eq:corr-conn}
%\end{equation}
%
%\subsection*{Observables linéaires en \(\rho\)}
%
%Considérons à présent le développement à l’ordre linéaire de la valeur propre d’une observable fonctionnelle $\operator{\mathcal{A}}$ autour de la densité moyenne~$\rho_{eq} $ :
%\begin{eqnarray*}
%	\langle\operator{\mathcal{A}}\rangle_{[\rho]} \approx 	\langle\operator{\mathcal{A}}\rangle_{[\rho_{eq}]} + \langle\delta\operator{\mathcal{A}}\rangle_{[\rho_{eq}]}, \quad \mbox{soit} \quad \langle\delta\operator{\mathcal{A}}\rangle_{[\rho_{eq}]} \approx \langle \operator{\mathcal{A}} - \langle\operator{\mathcal{A}}\rangle_{[\rho_{eq}]}\rangle_{[\rho]}  
%\end{eqnarray*}
%avec
%\begin{eqnarray*}
%	\langle\operator{\mathcal{A}}\rangle_{[\rho_{eq}]} = \int d\theta\,\mathsf{a}(\theta)\,\rho_{eq}(\theta), \quad \mbox{et} \quad \langle\delta\operator{\mathcal{A}}\rangle_{[\rho_{eq}]} = \int d\theta\,\mathsf{a}(\theta)\,\delta \rho(\theta)
%\end{eqnarray*} 
%
%où \(\mathsf{a}(\theta)\)  le densités spectrales associées à la valeur moyenne de le l'operateur $\operator{\mathcal{A}}$ .
%
%En s’appuyant sur les équations~\eqref{chap:fluctu:eq:ensemble_average}, \eqref{chap:fluctu:eq:ensemble_average:approx} et (\ref{eq:corr-conn}), et en remarquant que l’intégrale fonctionnelle est dominée par la contribution gaussienne autour du point selle, on obtient que la corrélation~\eqref{eq:corr-conn} est donnée par la formule des \textbf{fluctuations gaussiennes} :
%\begin{eqnarray}
%	C_{\operator{\mathcal{A}}, \operator{\mathcal{B}}} &=& \iint d\theta \, d\theta' \; \mathsf{a}(\theta) \, \mathsf{b}(\theta') \, 
%	\left\langle \delta \rho(\theta) \, \delta \rho(\theta') \right\rangle.
%	\label{eq:2point}
%\end{eqnarray}

\subsection{Fluctuations gaussiennes autour de l’équilibre thermodynamique}

Une autre approche pour accéder aux fluctuations de la densité spectrale $\rho(\theta)$ dans un état de GGE consiste à exploiter le développement quadratique de l’action effective autour de l’équilibre thermodynamique. Cette méthode, dite \emph{gaussienne}, repose sur le fait que, dans la limite thermodynamique, l’intégrale fonctionnelle définissant le GGE est dominée par les configurations proches du point-selle $\rho_{\mathrm{eq}}$.

On peut alors développer l’action $\mathcal{S}_{YY} - \mathcal{W}$ à second ordre autour de ce point d'équilibre :
\begin{eqnarray*}
	(\mathcal{S}_{YY} - \mathcal{W})[\rho] \approx (\mathcal{S}_{YY} - \mathcal{W})[\rho_{eq}]  + \underbrace{\int d\theta \left . \frac{ \delta (\mathcal{S}_{YY} - \mathcal{W})}{\delta \rho (\theta) }  \right\vert_{\rho = \rho_{eq} } \delta \rho(\theta)}_{\text{= 0 par stationnarité}} - \frac{1}{2} \int d\theta d \theta'\mathcal{H}(\theta, \theta' )\delta \rho(\theta)\delta \rho(\theta') + \mathcal{O}(\delta \rho^3) 
\end{eqnarray*}
où
\(
\displaystyle \mathcal{H}(\theta, \theta' ) = -\left.\frac{\delta^{2}(\mathcal{S}_{YY}-\mathcal{W})}{\delta\rho(\theta)\,\delta\rho(\theta')}\right|_{\rho = \rho_{\mathrm{eq}}}
\)
est le \textit{hessien} de l’action effective.

Sous l’approximation gaussienne autour de l'équilibre, la covariance des fluctuations est donnée par :
\begin{eqnarray}
\langle \delta \rho(\theta) \, \delta \rho(\theta') \rangle &=& \frac{
 \displaystyle \int \mathcal{D} \delta \rho \; \delta \rho(\theta) \, \delta \rho(\theta') 
    \, \exp \left[ 
        -\frac{L}{2} 
        \iint  d \theta_1 d \theta_2 \; 
        \delta \rho(\theta_1) \, \mathcal{H}(\theta_1, \theta_2 )  \, \delta \rho(\theta_2) 
    \right]
}{
    \displaystyle \int \mathcal{D} \delta \rho \; 
    \exp \left[ 
        -\frac{L}{2} 
        \iint  d \theta_1 d \theta_2 \; 
        \delta \rho(\theta_1) \, \mathcal{H}(\theta_1, \theta_2 )  \, \delta \rho(\theta_2) 
    \right]
}, \nonumber \\
&=& \frac{1}{L} \mathcal{H}^{-1}(\theta, \theta' ), \label{eq:fluctuations}
\label{chap:fluctu:eq:fluctuations}
\end{eqnarray}
confirmant ainsi que la matrice de susceptibilité spectrale $\chi(\theta, \theta')$ coïncide avec l’inverse du hessien.

\medskip

Ces relations posent les bases d'une description quantifiée des fluctuations de densité de rapidité, essentielles pour tester expérimentalement la validité du GGE, comprendre les corrélations à longue distance, et accéder aux propriétés dynamiques fines des systèmes intégrables en une dimension.\\

Ce développement quadratique justifie le caractère gaussien des fluctuations dans le régime thermodynamique, et sera à la base des extensions hydrodynamiques de type MFT (Macroscopic Fluctuation Theory).

\paragraph{Structure de \(\mathcal{H}\).}

L’opérateur hessien de l’action effective se décompose naturellement comme la différence entre deux contributions fonctionnelles :
\[
\mathcal{H} = \mathcal{H}^{(\mathcal{W})} - \mathcal{H}^{(\mathcal{S}_{YY})},
\]
où
\[
\mathcal{H}^{(\mathcal{W})}(\theta,\theta') := \left. \frac{\delta^2 \mathcal{W}}{\delta \rho(\theta)\, \delta \rho(\theta')} \right|_{\rho = \rho_{\mathrm{eq}}}, 
\quad \text{et} \quad 
\mathcal{H}^{(\mathcal{S}_{YY})}(\theta,\theta') := \left. \frac{\delta^2 \mathcal{S}_{YY}}{\delta \rho(\theta)\, \delta \rho(\theta')} \right|_{\rho = \rho_{\mathrm{eq}}}.
\]

L’opérateur inverse \(\mathcal{H}^{-1}\) est défini par la relation fonctionnelle :
\begin{eqnarray}
	(\mathcal{H}^{-1} \cdot \mathcal{H})(\theta, \theta') = (\mathcal{H} \cdot \mathcal{H}^{-1})(\theta, \theta') = \int d\theta'' \; \mathcal{H}(\theta, \theta'')\, \mathcal{H}^{-1}(\theta'', \theta') = \delta(\theta - \theta'),
	\label{chap:fluctu:eq:hessienner.prod.inv} 	
\end{eqnarray}
où \(\delta(\theta - \theta')\) désigne la distribution de Dirac, et non une variation.

\medskip

On remarque tout d’abord que \(\mathcal{H}^{(\mathcal{W})} = 0\), car l’énergie généralisée par unité de longueur s’écrit simplement comme un couplage linéaire en \(\rho\) :
\[
\mathcal{W}[\rho] = \int d\theta \, w(\theta)\, \rho(\theta),
\]
avec un poids spectral \(w(\theta)\) fixé \eqref{chap.2.W.int}. La seconde dérivée fonctionnelle de \(\mathcal{W}\) s’annule donc identiquement.

En revanche, la courbure fonctionnelle provient entièrement de l'entropie de Yang–Yang, dont l’expression \eqref{chap.2.entropi.int} est donnée par :
\[
\mathcal{S}_{YY}[\rho] = \int d\theta \left[
\rho_s(\theta) \ln \rho_s(\theta) - \rho(\theta) \ln \rho(\theta) - (\rho_s(\theta) - \rho(\theta)) \ln(\rho_s(\theta) - \rho(\theta))
\right],
\]
où \(\rho_s(\theta)\) désigne la densité d’états accessibles (ou densité totale) liée à \(\rho\) par les équations de Bethe.

Ainsi, l’opérateur de fluctuation \(\mathcal{H}\) coïncide avec la hessienne négative de l'entropie \(\mathcal{S}_{YY}\), et détermine complètement la covariance spectrale à l’équilibre.



%\subsection*{Lien avec les susceptibilités}
%
%Le hessien \(\mathcal{H}\) est l’analogue fonctionnel
%de la matrice de susceptibilité classique :
%pour deux charges conservées linéaires
%\(Q_i=\int d\theta\,q_i(\theta)\rho(\theta)\),
%on retrouve
%\[
%  \chi_{ij}
%  =
%  \frac{\partial\langle Q_i\rangle}{\partial\mu_j}
%  =
%  -\frac{1}{L}
%  \iint d\theta\,d\theta'\;
%  q_i(\theta)\,\mathcal{H}^{-1}(\theta,\theta')\,q_j(\theta').
%\]
%Ainsi, les corrélations \eqref{eq:2point} généralisent
%la relation fluctuation–réponse :
%elles encodent la réponse linéaire (susceptibilité)
%et les fluctuations thermiques d’une même matrice
%\(\mathcal{H}^{-1}\).
%
%\medskip
%Ces résultats posent les fondations d’une description quantitative des
%fluctuations de densité de rapidité — clé pour tester expérimentalement
%la validité du GGE, analyser les corrélations à longue portée et
%caractériser les propriétés dynamiques des modèles intégrables en
%une dimension.


%\section*{Introduction}

%L’hypothèse selon laquelle, après relaxation, un système quantique intégrable est décrit par un ensemble généralisé de Gibbs (GGE) constitue un fondement majeur de notre compréhension des dynamiques hors équilibre. Cette hypothèse repose sur la conservation d’un grand nombre de charges locales et quasi-locales, et permet d’expliquer pourquoi de tels systèmes ne thermalisaient pas au sens conventionnel.

%Toutefois, la seule connaissance de la distribution de rapidité moyenne $\langle \rho(\theta) \rangle$ ne suffit pas à établir de manière univoque la nature de l’ensemble statistique sous-jacent. En effet, plusieurs ensembles distincts peuvent mener à une même valeur moyenne de $\rho$. Pour lever cette ambiguïté et sonder la structure statistique fine de l’état stationnaire, il est nécessaire d’étudier les \emph{fluctuations de densité de rapidité}, définies comme
%\begin{equation}
%\rho(\theta) = \langle \rho(\theta) \rangle + \delta \rho(\theta).
%\end{equation}

%Dans le modèle de Lieb-Liniger moyenné à grande échelle (coarse-grained), les valeurs moyennes des observables $\langle \hat{A} \rangle$ dans un ensemble statistique peuvent être formulées comme une intégrale fonctionnelle :
%\begin{eqnarray}
%\langle \hat{A} \rangle_W &=& \frac{\int \mathcal{D} \rho \; e^{L (S[\rho] - W[\rho])} \, A[\rho]}{\int \mathcal{D} \rho \; e^{L (S[\rho] - W[\rho])}}, \label{eq:ensemble_average}
%\end{eqnarray}
%où $A[\rho]$ est la valeur de l’observable dans un état propre caractérisé par la distribution de rapidité $\rho(\theta)$, $\mathcal{D}\rho$ désigne l'intégrale fonctionnelle sur toutes les distributions possibles, $L$ est la taille du système, $S[\rho]$ est l'entropie de Yang-Yang, donnée par :
%\begin{eqnarray}
%S[\rho] &=& \int d\theta \; s(\nu(\theta)) \, \rho_s(\theta), \label{eq:entropy}
%\end{eqnarray}
%et $W[\rho]$ est le poids fonctionnel définissant l'ensemble :
%\begin{eqnarray}
%W[\rho] &=& \int d\theta \; w(\theta) \, \rho(\theta). \label{eq:weight}
%\end{eqnarray}

%Dans la limite thermodynamique ($L \to \infty$), l’intégrale fonctionnelle peut être évaluée par la méthode du point col (saddle point), ce qui conduit à l’équation de Yang-Yang :
%\begin{eqnarray}
%w(\theta) &=& \frac{\delta S[\rho]}{\delta \rho(\theta)}. \label{eq:yangyang}
%\end{eqnarray}

%Cette équation fixe complètement les valeurs moyennes des observables locales (fonctions à un point). Cependant, pour accéder aux fonctions de corrélation à deux points, on s’intéresse aux corrélations des fluctuations $\delta \rho$, telles que :


%\section{Fluctuations autour de l’état typique}

%La fonction de partition des états s’exprime comme une fonctionnelle de la distribution de rapidité \( \rho \) :
%\begin{eqnarray}
%	Z & = & \sum_\rho \exp\left( -\mathcal{A}(\rho) \right).
%\end{eqnarray}

%Dans la section \textit{\textbf{Entropie de Yang-Yang}}~(\ref{??}), nous avons montré que l’action \( \mathcal{A}(\rho) \) s’écrit sous la forme :
%\begin{eqnarray}
%	\mathcal{A}(\rho) & \doteq & - L \mathcal{S}_{YY}(\rho) + L \int f(\theta)\, \rho(\theta) \, d\theta, \label{eq:action}
%\end{eqnarray}
%où \( L \) désigne la taille du système ,  \( \mathcal{S}_{YY} \) désigne la fonctionnelle d'entropie de Yang-Yang, définie en~(\ref{??}), et \( f \) la fonction génératrice associée aux charges conservées, introduite en~(\ref{??}).

%Dans cette même section, nous avons également montré que la distribution \( \rho^c \) maximisant la contribution à la fonction de partition correspond à un point stationnaire de l'action, et est entièrement déterminée par la fonction \( f \). Cette distribution \( \rho^c \) définit ainsi l’état macroscopique typique du système au sein de l’ensemble statistique considéré.



%Afin de caractériser ces fluctuations, nous développons l’action \( \mathcal{A}(\rho) \) au voisinage de \( \rho^c \). Par construction, \( \rho^c \) étant un point stationnaire, la différentielle première de \( \mathcal{A} \) en ce point est nulle : \( d\mathcal{A}_{\rho^c} = 0 \) (cf. équation~(\ref{??})). En appliquant un développement de Taylor-Young à l’ordre deux en \( \delta \rho \), on obtient :
%\begin{eqnarray}
%	\mathcal{A}(\rho^c + \delta \rho) & \underset{ \delta \rho \to 0 }{=} & \mathcal{A}(\rho^c) + \frac{1}{2} \left. \frac{\delta^2 \mathcal{A}}{{\delta \rho}^2} \right|_{\rho^c} (\delta \rho) + \mathcal{O}((\delta \rho)^3),
%\end{eqnarray}
%où \( \left. \frac{\delta^2 \mathcal{A}}{\delta \rho^2} \right|_{\rho^c} \) désigne la forme bilinéaire symétrique (a priori définie positive) associée à la hessienne de l’action. Celle-ci s’écrit également à partir de la dérivée fonctionnelle seconde de l’entropie de Yang-Yang au point \( \rho^c \) (via l’équation~\eqref{eq:action}) :
%\begin{eqnarray*}
%	\left. \frac{\delta^2 \mathcal{A}}{\delta \rho^2} \right|_{\rho^c} &=& -L \left. \frac{\delta^2 \mathcal{S}_{YY}}{{\delta \rho}^2} \right|_{\rho^c}.
%\end{eqnarray*}

%La dérivée seconde \( \left. \frac{\delta^2 \mathcal{S}_{YY}}{\delta \rho^2} \right|_{\rho^c} \) étant strictement négative, on conclut que \( \left. \frac{\delta^2 \mathcal{A}}{\delta \rho^2} \right|_{\rho^c} \) est bien définie positive. La taille \( L \) étant grande devant les autres échelles du problème , ainsi,l’action croît fortement dès que la configuration \( \rho \) s’éloigne de \( \rho^c \) (voir Fig.~\ref{fig:grap.A}).

%Cette approximation quadratique constitue la base du traitement gaussien des fluctuations autour de l’état typique, permettant d’accéder aux corrélations et à la structure fine de l’ensemble statistique effectif.



\begin{figure}[H]
	\centering
	\begin{subfigure}[b]{0.3\textwidth}
		\begin{tikzpicture}
			\begin{scope}[shift={(2,0)}]
				\begin{scope}[transform canvas={scale=0.6}]
					\input{Figures/Fonc_A_code}	
			
				\end{scope}
			
				\draw[color = red , scale = 0.5 , draw = none  ] (-2 , -1) rectangle (5, 6) ; 	
			\end{scope}
					
		\end{tikzpicture}
		\caption{}
		\label{fig:grap.A}
	\end{subfigure}
	\hfill
	\begin{subfigure}[b]{0.65\textwidth}
		\begin{tikzpicture}

			\begin{scope}[shift={(-11,2.7)}]
				\begin{scope}[transform canvas={scale=0.6}]
					% Définition des couleurs avec les codes HTML
\definecolor{colorOne}{HTML}{443E46}
\definecolor{colorTwo}{HTML}{F6DEB8}
\definecolor{colorThree}{HTML}{908CA4}
\definecolor{colorFour}{HTML}{57659E}
\definecolor{colorFive}{HTML}{C57284}
\definecolor{colorSix}{HTML}{FF5B69}

% Raccourcis pour les couleurs
\def\colorOne{colorOne}
\def\colorTwo{colorTwo}
\def\colorThree{colorThree}
\def\colorFour{colorFour}
\def\colorFive{colorFive}
\def\colorSix{colorSix}

\def\colorslide{blue!50!black}

\def\Occupation{
	\def\traitx{0.3}
	\def\traity{0.5}
	\draw[shift={(0,0)}]
		(-13.5 , 0 ) edge [thick,line width=0.8ex ]( -3.2  , 0 )
		( -3.2 - \traitx  , 0 - \traity ) edge [thick,line width=0.8ex ]( -3.2 + \traitx  , 0 + \traity  )
		( -2.8 - \traitx  , 0 - \traity ) edge [thick,line width=0.8ex ]( -2.8 + \traitx  , 0 + \traity  )
		(-2.8 , 0 ) edge [thick,line width=0.8ex ](2.8  , 0 )
		( 2.8 - \traitx  , 0 - \traity ) edge [thick,line width=0.8ex ]( 2.8 + \traitx  , 0 + \traity  )
		( 3.2 - \traitx  , 0 - \traity ) edge [thick,line width=0.8ex ]( 3.2 + \traitx  , 0 + \traity  )
		(3.2, 0 ) edge [thick,line width=0.8ex,->,>=triangle 45 , color = black ]node [pos=1.01,below  ]{\huge$\theta$}	( 13  , 0 )
	;
	\draw[shift={(0,0)}, color=\colorOne]
		(-10.5 , -1.5 ) edge [thick,line width=0.8ex , ->,>=triangle 45  ]( -10.5  , 4.5 )
	;
		
	\foreach \r in {1 , ... , 3 } {
%		\draw[
%		decoration={
%		markings,
%    	mark connection node=my node,
%    	mark=at position 0 with{\node [blue,transform shape] (my node) {\large \r};}},
%		color=gray, thick, 
%		line width=0.5ex] decorate { 
%            (-11.0, \r) -- (-10.1, \r )}
%        ;
        \draw[
			color=\colorOne,
			] 
            (-11.0, \r) edge[color=\colorThree , thick,line width=0.5ex] node [pos=-0.5 ]{\large\color{\colorFour} $\frac{\r}{\delta \theta}$ } (-10.3, \r )
        	;
	
	}
	

	
	% Graduation abcsisse 
	% Définitions des listes
% Definitions of the lists
\def\listetuple{-9/\theta_{1}, -8/\theta_{2} , -5/\theta_{3} , -2/\theta_{a-1} , 0/\theta_{a} , 1/\theta_{a+1} , 2/\theta_{a+2} ,  5/\theta_{N-4} , 7/\theta_{N-3},8/\theta_{N-1},9/\theta_{N} }
\def\listetrais{-12 , -11, -10, -9 , -8 , -7 ,  -6 , -5, -4.5,-4, -2 , -1, 0 , 0.5, 1, 2, 4 , 5 ,  6 , 7 , 8 ,8.5, 9 ,  10 , 11, 12 }

% Loop over listetrais
\foreach \r in \listetrais {
    % Initialize found variable to zero
    % Initialize found variable to zero
    %\pgfmathsetmacro\found{0}
    \global\def\found{0}
    \xdef\nomtheta{}
    
    % Check if \r is in listetuple
    \foreach \x/\y in \listetuple { 
        \ifdim \r pt=\x pt % If \r matches any \x in listetuple
            \global\def\found{1} ;
            \xdef\nomtheta{\y} % Set \nomtheta to the corresponding \y
            %\pgfmathsetmacro\found{1} % Set found to 1            
            %\global\pgfmathsetmacro\found{1}
        \fi
    }
    
    %\node [circle, draw, red] (A) at (\r, 2) {\found , $\nomtheta$};
    
    % Draw the line and display \nomtheta if found
    \ifnum\found=1
        \draw[color=\colorOne, thick, line width=0.5ex] 
            (\r, -0.3) -- (\r, 0.3) node[red , pos=-0.5] {\large $\nomtheta$};
         \filldraw[line width=0.5ex, color=\colorSix, outer color=\colorSix, inner color=\colorSix] 
            (\r, 0) circle (4pt);
    \else 
        % Draw without \nomtheta and add a blue circle if not found
        \draw[color=\colorOne, thick, line width=0.5ex] 
            (\r, -0.3) -- (\r, 0.3);
        \filldraw[line width=0.5ex, color=\colorSix, outer color=\colorTwo, inner color=\colorTwo] 
            (\r, 0) circle (4pt); 
    \fi
}

\def\listetrais{-9.5/\theta_{i-1}/2/3, -6.5/\theta_{i}/1/4  ,   -1.5/\theta_{j}/2/4 , 1.5/\theta_{j+1}/-1/3 , 3.5/\theta_{\ell-1}/1/3 , 6.5/\theta_{\ell}/3/4 , 9.5/\theta(\theta_{\ell+1})/-1/3 };



\foreach \r/\nomx/\y/\ys in \listetrais {
	\draw[
		decoration={
		markings,
    	mark connection node=my node,
    	mark=at position .5 with{\node [blue,transform shape] (my node) {\large \color{\colorFour} $\nomx$};}},
		color=\colorThree , thick, 
		line width=0.5ex] decorate { 
            (\r, 0.12) -- (\r, -1.2)}
        ;
     
     \ifdim \y pt > -1 pt 
     	\draw[
			decoration={
			markings,
    		mark connection node=my node,
    		mark=at position .5 with{\node [blue,transform shape] (my node) {\large \color{\colorFour} $\pi^d(\nomx) $};}},
			color=\colorThree, thick, 
			line width=0.5ex] decorate { 
            (\r, \y) -- (\r +3, \y)}
        ;
        \draw[
			decoration={
			markings,
    		mark connection node=my node,
    		mark=at position .5 with{\node [blue,transform shape] (my node) {\large \color{\colorFive} $\pi_s^d(\nomx) $};}},
			color=\colorFive, thick, 
			line width=0.5ex] decorate { 
            (\r, \ys) -- (\r +3, \ys)}
        ;
     \fi 
     \ifdim \r pt= -1.5 pt
     	\draw[
     		decoration={
			markings,
    		mark connection node=my node,
    		mark=at position .5 with{\node [blue,transform shape] (my node) {\large \color{\colorFour}  $\delta \theta $};},
    		%mark=at position 0.1  with {\arrow[blue, line width=0.5ex]{<}},
    		%mark=at position 1  with {\arrow[blue, line width=0.5ex]{>}}
    		},
        	color=\colorThree,
        	thick,
        	line width=0.5ex,
        	%arrows={Computer Modern Rightarrow[line cap=round]-Computer Modern Rightarrow[line cap=round]}
   			](\r, -1.2) edge[arrows={Computer Modern Rightarrow[line cap=round]-}] (\r + 0.4, -1.2)decorate {
    		(\r, -1.2) -- (\r + 3, -1.2)}(\r + 2, -1.2) edge[arrows={-Computer Modern Rightarrow[line cap=round]}] (\r + 3, -1.2)
    		;
    \fi
			
	
}


			
}


\begin{scope}
	%\draw[help lines , width=1.5ex] (-8,-3) grid (8,3);\draw[help lines ,width=0.5ex , opacity = 0.5] (-3,-3) grid[step=0.1] (3,3));
	
	%\draw[help lines] 
	%	(-3,-3) edge[width=1.5ex] grid (3,3)	
	%	(-3,-3) edge[width=0.5ex , opacity = 0.5] grid (3,3)	
	%;
	\begin{scope}[shift={(0,1)},rotate=0,opacity=1,color=black]
		\Occupation	
		
		%\node[anchor=east, font=\bfseries] at (-11, 0) {\color{red}\large (T = 0 )} ;	
	\end{scope}
	
	
	
	
	\begin{scope}[shift={(-10.5,7)},rotate=0,opacity=1,color=black]
	
	\begin{scope}[shift={(-0,0)},rotate=0,opacity=1,color=black]
	
		\draw[shift={(0,0)} ,line width=1ex,rounded corners = 1ex,color=\colorOne , opacity =1 ,fill=\colorOne!00 , pattern={north east lines} , pattern color=\colorOne!00 ]
			(0 , -1 ) rectangle (5,1)
		;
		

		\begin{scope}[shift={(0.5,0.5)}]
			\draw[color=\colorOne, thick, line width=0.5ex] 
            (0, -0.3) -- (0, 0.3) ;
            \filldraw[line width=0.5ex, color=\colorSix, outer color=\colorSix, inner color=\colorSix] 
            (0, 0) circle (4pt);
            
            \node[anchor=west, font=\bfseries] at (0.2, 0) {\color{\colorSix}\large : quasi-particule};
		\end{scope}
		
		\begin{scope}[shift={(0.5,-0.5)}]
			\draw[color=\colorOne, thick, line width=0.5ex] 
            (0, -0.3) -- (0, 0.3) ;
            \filldraw[line width=0.5ex, color=\colorSix, outer color=\colorTwo, inner color=\colorTwo] 
            (0, 0) circle (4pt);
            
            \node[anchor=west, font=\bfseries] at (0.2, 0) {\color{\colorSix}\large : hole};
		\end{scope}

	\end{scope}
	
	\begin{scope}[shift={(6,0)},rotate=0,opacity=1,color=black]	
		
		\draw[shift={(0,0)} ,line width=1ex,rounded corners = 1ex,color=\colorOne , opacity =1 ,fill=\colorOne!00 , pattern={north east lines} , pattern color=\colorOne!00 ]
			(0 , -1 ) rectangle (7.5,1)
		;
		
		\node[anchor=west] at (0.5, 0.5) {\color{\colorFour}\large $\pi^d$ };\node[anchor=west, font=\bfseries] at (0.9, 0.5) {\color{\colorFour}\large : quasi-particule distribution};
		
		\node[anchor=west] at (0.5, -0.5) {\color{\colorFour}\large $\pi_h^d$ };\node[anchor=west, font=\bfseries] at (0.9, -0.5) {\color{\colorFour}\large  : hole distribution};
		
	\end{scope}
	
	\begin{scope}[shift={(14.5,0)},rotate=0,opacity=1,color=black]	
		
		\draw[shift={(0,0)} ,line width=1ex,rounded corners = 1ex,color=\colorOne , opacity =1 ,fill=\colorOne!00 , pattern={north east lines} , pattern color=\colorOne!00 ]
			(0 , -0.5 ) rectangle (7.0,0.5)
		;
		
		\node[anchor=west] at (0.5, 0) {\color{\colorFour}\large ${\color{\colorFive}\pi_s^d} = \pi^d + \pi_h^d $ };\node[anchor=west, font=\bfseries] at (2.9, 0) {\color{\colorFour}\large {\color{\colorFive} : density of states}};
		
	\end{scope}
	
	
	\end{scope}


		
	
\end{scope}

	
			
				\end{scope}
				\begin{scope}[scale=1]
					\draw[color = red , scale = 1 , draw = none  ] (-1 , -1) rectangle (5, 5) ; 
				\end{scope}	
			\end{scope}
			
		\end{tikzpicture}
		\caption{}
		\label{fig.fluctu.A}
	\end{subfigure}
	\caption{}
	\label{fig:diag_fig}
\end{figure}


%\begin{figure}[H]
%	\centering 
%	\begin{tikzpicture}
%		\begin{scope}[shift={(0,0)}]
%			\begin{scope}[transform canvas={scale=0.6}]
%				\input{Figures/Fonc_A_code}	
			
%			\end{scope}
			
%			\draw[color = red , scale = 0.5 , draw = none  ] (-2 , -1) rectangle (5, 6) ; 	
%		\end{scope}
		
%		\begin{scope}[shift={(19,-1)}]
%			\begin{scope}[transform canvas={scale=0.6}]
%				% Définition des couleurs avec les codes HTML
\definecolor{colorOne}{HTML}{443E46}
\definecolor{colorTwo}{HTML}{F6DEB8}
\definecolor{colorThree}{HTML}{908CA4}
\definecolor{colorFour}{HTML}{57659E}
\definecolor{colorFive}{HTML}{C57284}
\definecolor{colorSix}{HTML}{FF5B69}

% Raccourcis pour les couleurs
\def\colorOne{colorOne}
\def\colorTwo{colorTwo}
\def\colorThree{colorThree}
\def\colorFour{colorFour}
\def\colorFive{colorFive}
\def\colorSix{colorSix}

\def\colorslide{blue!50!black}

\def\Occupation{
	\def\traitx{0.3}
	\def\traity{0.5}
	\draw[shift={(0,0)}]
		(-13.5 , 0 ) edge [thick,line width=0.8ex ]( -3.2  , 0 )
		( -3.2 - \traitx  , 0 - \traity ) edge [thick,line width=0.8ex ]( -3.2 + \traitx  , 0 + \traity  )
		( -2.8 - \traitx  , 0 - \traity ) edge [thick,line width=0.8ex ]( -2.8 + \traitx  , 0 + \traity  )
		(-2.8 , 0 ) edge [thick,line width=0.8ex ](2.8  , 0 )
		( 2.8 - \traitx  , 0 - \traity ) edge [thick,line width=0.8ex ]( 2.8 + \traitx  , 0 + \traity  )
		( 3.2 - \traitx  , 0 - \traity ) edge [thick,line width=0.8ex ]( 3.2 + \traitx  , 0 + \traity  )
		(3.2, 0 ) edge [thick,line width=0.8ex,->,>=triangle 45 , color = black ]node [pos=1.01,below  ]{\huge$\theta$}	( 13  , 0 )
	;
	\draw[shift={(0,0)}, color=\colorOne]
		(-10.5 , -1.5 ) edge [thick,line width=0.8ex , ->,>=triangle 45  ]( -10.5  , 4.5 )
	;
		
	\foreach \r in {1 , ... , 3 } {
%		\draw[
%		decoration={
%		markings,
%    	mark connection node=my node,
%    	mark=at position 0 with{\node [blue,transform shape] (my node) {\large \r};}},
%		color=gray, thick, 
%		line width=0.5ex] decorate { 
%            (-11.0, \r) -- (-10.1, \r )}
%        ;
        \draw[
			color=\colorOne,
			] 
            (-11.0, \r) edge[color=\colorThree , thick,line width=0.5ex] node [pos=-0.5 ]{\large\color{\colorFour} $\frac{\r}{\delta \theta}$ } (-10.3, \r )
        	;
	
	}
	

	
	% Graduation abcsisse 
	% Définitions des listes
% Definitions of the lists
\def\listetuple{-9/\theta_{1}, -8/\theta_{2} , -5/\theta_{3} , -2/\theta_{a-1} , 0/\theta_{a} , 1/\theta_{a+1} , 2/\theta_{a+2} ,  5/\theta_{N-4} , 7/\theta_{N-3},8/\theta_{N-1},9/\theta_{N} }
\def\listetrais{-12 , -11, -10, -9 , -8 , -7 ,  -6 , -5, -4.5,-4, -2 , -1, 0 , 0.5, 1, 2, 4 , 5 ,  6 , 7 , 8 ,8.5, 9 ,  10 , 11, 12 }

% Loop over listetrais
\foreach \r in \listetrais {
    % Initialize found variable to zero
    % Initialize found variable to zero
    %\pgfmathsetmacro\found{0}
    \global\def\found{0}
    \xdef\nomtheta{}
    
    % Check if \r is in listetuple
    \foreach \x/\y in \listetuple { 
        \ifdim \r pt=\x pt % If \r matches any \x in listetuple
            \global\def\found{1} ;
            \xdef\nomtheta{\y} % Set \nomtheta to the corresponding \y
            %\pgfmathsetmacro\found{1} % Set found to 1            
            %\global\pgfmathsetmacro\found{1}
        \fi
    }
    
    %\node [circle, draw, red] (A) at (\r, 2) {\found , $\nomtheta$};
    
    % Draw the line and display \nomtheta if found
    \ifnum\found=1
        \draw[color=\colorOne, thick, line width=0.5ex] 
            (\r, -0.3) -- (\r, 0.3) node[red , pos=-0.5] {\large $\nomtheta$};
         \filldraw[line width=0.5ex, color=\colorSix, outer color=\colorSix, inner color=\colorSix] 
            (\r, 0) circle (4pt);
    \else 
        % Draw without \nomtheta and add a blue circle if not found
        \draw[color=\colorOne, thick, line width=0.5ex] 
            (\r, -0.3) -- (\r, 0.3);
        \filldraw[line width=0.5ex, color=\colorSix, outer color=\colorTwo, inner color=\colorTwo] 
            (\r, 0) circle (4pt); 
    \fi
}

\def\listetrais{-9.5/\theta_{i-1}/2/3, -6.5/\theta_{i}/1/4  ,   -1.5/\theta_{j}/2/4 , 1.5/\theta_{j+1}/-1/3 , 3.5/\theta_{\ell-1}/1/3 , 6.5/\theta_{\ell}/3/4 , 9.5/\theta(\theta_{\ell+1})/-1/3 };



\foreach \r/\nomx/\y/\ys in \listetrais {
	\draw[
		decoration={
		markings,
    	mark connection node=my node,
    	mark=at position .5 with{\node [blue,transform shape] (my node) {\large \color{\colorFour} $\nomx$};}},
		color=\colorThree , thick, 
		line width=0.5ex] decorate { 
            (\r, 0.12) -- (\r, -1.2)}
        ;
     
     \ifdim \y pt > -1 pt 
     	\draw[
			decoration={
			markings,
    		mark connection node=my node,
    		mark=at position .5 with{\node [blue,transform shape] (my node) {\large \color{\colorFour} $\pi^d(\nomx) $};}},
			color=\colorThree, thick, 
			line width=0.5ex] decorate { 
            (\r, \y) -- (\r +3, \y)}
        ;
        \draw[
			decoration={
			markings,
    		mark connection node=my node,
    		mark=at position .5 with{\node [blue,transform shape] (my node) {\large \color{\colorFive} $\pi_s^d(\nomx) $};}},
			color=\colorFive, thick, 
			line width=0.5ex] decorate { 
            (\r, \ys) -- (\r +3, \ys)}
        ;
     \fi 
     \ifdim \r pt= -1.5 pt
     	\draw[
     		decoration={
			markings,
    		mark connection node=my node,
    		mark=at position .5 with{\node [blue,transform shape] (my node) {\large \color{\colorFour}  $\delta \theta $};},
    		%mark=at position 0.1  with {\arrow[blue, line width=0.5ex]{<}},
    		%mark=at position 1  with {\arrow[blue, line width=0.5ex]{>}}
    		},
        	color=\colorThree,
        	thick,
        	line width=0.5ex,
        	%arrows={Computer Modern Rightarrow[line cap=round]-Computer Modern Rightarrow[line cap=round]}
   			](\r, -1.2) edge[arrows={Computer Modern Rightarrow[line cap=round]-}] (\r + 0.4, -1.2)decorate {
    		(\r, -1.2) -- (\r + 3, -1.2)}(\r + 2, -1.2) edge[arrows={-Computer Modern Rightarrow[line cap=round]}] (\r + 3, -1.2)
    		;
    \fi
			
	
}


			
}


\begin{scope}
	%\draw[help lines , width=1.5ex] (-8,-3) grid (8,3);\draw[help lines ,width=0.5ex , opacity = 0.5] (-3,-3) grid[step=0.1] (3,3));
	
	%\draw[help lines] 
	%	(-3,-3) edge[width=1.5ex] grid (3,3)	
	%	(-3,-3) edge[width=0.5ex , opacity = 0.5] grid (3,3)	
	%;
	\begin{scope}[shift={(0,1)},rotate=0,opacity=1,color=black]
		\Occupation	
		
		%\node[anchor=east, font=\bfseries] at (-11, 0) {\color{red}\large (T = 0 )} ;	
	\end{scope}
	
	
	
	
	\begin{scope}[shift={(-10.5,7)},rotate=0,opacity=1,color=black]
	
	\begin{scope}[shift={(-0,0)},rotate=0,opacity=1,color=black]
	
		\draw[shift={(0,0)} ,line width=1ex,rounded corners = 1ex,color=\colorOne , opacity =1 ,fill=\colorOne!00 , pattern={north east lines} , pattern color=\colorOne!00 ]
			(0 , -1 ) rectangle (5,1)
		;
		

		\begin{scope}[shift={(0.5,0.5)}]
			\draw[color=\colorOne, thick, line width=0.5ex] 
            (0, -0.3) -- (0, 0.3) ;
            \filldraw[line width=0.5ex, color=\colorSix, outer color=\colorSix, inner color=\colorSix] 
            (0, 0) circle (4pt);
            
            \node[anchor=west, font=\bfseries] at (0.2, 0) {\color{\colorSix}\large : quasi-particule};
		\end{scope}
		
		\begin{scope}[shift={(0.5,-0.5)}]
			\draw[color=\colorOne, thick, line width=0.5ex] 
            (0, -0.3) -- (0, 0.3) ;
            \filldraw[line width=0.5ex, color=\colorSix, outer color=\colorTwo, inner color=\colorTwo] 
            (0, 0) circle (4pt);
            
            \node[anchor=west, font=\bfseries] at (0.2, 0) {\color{\colorSix}\large : hole};
		\end{scope}

	\end{scope}
	
	\begin{scope}[shift={(6,0)},rotate=0,opacity=1,color=black]	
		
		\draw[shift={(0,0)} ,line width=1ex,rounded corners = 1ex,color=\colorOne , opacity =1 ,fill=\colorOne!00 , pattern={north east lines} , pattern color=\colorOne!00 ]
			(0 , -1 ) rectangle (7.5,1)
		;
		
		\node[anchor=west] at (0.5, 0.5) {\color{\colorFour}\large $\pi^d$ };\node[anchor=west, font=\bfseries] at (0.9, 0.5) {\color{\colorFour}\large : quasi-particule distribution};
		
		\node[anchor=west] at (0.5, -0.5) {\color{\colorFour}\large $\pi_h^d$ };\node[anchor=west, font=\bfseries] at (0.9, -0.5) {\color{\colorFour}\large  : hole distribution};
		
	\end{scope}
	
	\begin{scope}[shift={(14.5,0)},rotate=0,opacity=1,color=black]	
		
		\draw[shift={(0,0)} ,line width=1ex,rounded corners = 1ex,color=\colorOne , opacity =1 ,fill=\colorOne!00 , pattern={north east lines} , pattern color=\colorOne!00 ]
			(0 , -0.5 ) rectangle (7.0,0.5)
		;
		
		\node[anchor=west] at (0.5, 0) {\color{\colorFour}\large ${\color{\colorFive}\pi_s^d} = \pi^d + \pi_h^d $ };\node[anchor=west, font=\bfseries] at (2.9, 0) {\color{\colorFour}\large {\color{\colorFive} : density of states}};
		
	\end{scope}
	
	
	\end{scope}


		
	
\end{scope}

	
			
%			\end{scope}
%			\begin{scope}[scale=1]
%				\draw[color = red , scale = 1 , draw = none  ] (-1 , -1) rectangle (5, 5) ; 
%			\end{scope}	
%		\end{scope}

		
			
%	\end{tikzpicture}	
%	\captionsetup{skip=10pt} % Ajoute de l’espace après la légende
%	\label{fig.fluctu.A}
%\end{figure}

\subsection{Expression de la Hessienne}

%Dans le cadre de l'approximation gaussienne autour du point selle \( \rho = \langle \rho \rangle \), les fluctuations de la densité \( \delta \rho(\theta) = \rho(\theta) - \langle \rho(\theta) \rangle \) sont contrôlées par le développement quadratique de l'action effective \( \mathcal{S}_{YY}[\rho] - \mathcal{W}[\rho] \). %La forme quadratique associée s’écrit alors :

%\begin{equation}
%    \mathcal{S}_{\text{eff}}[\rho] \approx \mathcal{S}_{\text{eff}}[\langle \rho \rangle] + \frac{1}{2} \int d\theta \, d\theta' \; \delta \rho(\theta) \, \mathcal{H}(\theta, \theta') \, \delta \rho(\theta'),
%\end{equation}



La Hessienne $ \mathcal{H}^{(\mathcal{S}_{YY})}$ se décompose alors
\begin{eqnarray}
	\mathcal{H}^{(\mathcal{S}_{YY})}(\theta, \theta') & = & \mathcal{D}(\theta, \theta') + \mathcal{V}(\theta, \theta')
	\label{chap:fluctu:eq:hessienne3} 	
\end{eqnarray}
avec une partie diagonale irrégulière reflètant une structure de type Fermi–Dirac, même dans un système bosonique,conséquence de l’exclusion statistique induite par l’intégrabilité,
\begin{eqnarray}
	\mathcal{D}(\theta, \theta') & = & \left ( \frac{1}{\rho_{\! s , eq}(\theta) \nu_{\! eq}(\theta) (1 - \nu_{\! eq}(\theta)) }\right )  \delta(\theta, \theta') 	
\end{eqnarray}
avec une partie symétrique régulière
\begin{eqnarray}
	\mathcal{V}(\theta, \theta')	 & = & - \left ( \frac{1}{\rho_{\! s , eq}(\theta) (1 - \nu_{\! eq}(\theta) ) } + \frac{1}{\rho_{\! s, eq}(\theta')   (1 - \nu_{\! eq}(\theta')) }\right )\frac{\Delta(\theta - \theta')}{2\pi}\\
	&&  + \int d \theta''  \frac{\nu_{\! eq}(\theta'')}{\rho_{\, s, eq}(\theta'')(1 - \nu_{\! eq}(\theta'')) } \frac{\Delta(\theta - \theta'' )}{2\pi} \frac{\Delta(\theta''  - \theta')}{2\pi}
	\label{chap:fluctu:eq:reg}
\end{eqnarray}
 
 en notant $\rho_{\! eq}(\theta) = \nu_{\! eq}(\theta)\rho_{\! s, eq}(\theta)$.\\



%\subsection*{Forme explicite du hessien : fluctuations gaussiennes}
%
%Il est classique (cf. appendix~\ref{appendix:YYentropy}) que
%l’entropie de Yang–Yang admet un hessien diagonal en base des \(\theta\) :
%\begin{equation}
%\mathcal{H}^{(S)}(\theta,\theta') =
%\frac{\delta(\theta-\theta')}{\nu_{\!eq}(\theta)\big(1-\nu_{\!eq}(\theta)\big)}.
%\label{eq:hessian-SYY}
%\end{equation}
%
%Quant à l’énergie généralisée,
%si l’observable est extensive (fonctionnelle additive),
%le terme \(\mathcal{H}^{(W)}\) est symétrique, et typiquement local ou à noyau fini.
%
%\medskip
%Au total, les fluctuations du GGE au voisinage de \(\rho_{\mathrm{eq}}\)
%sont décrites, à l’ordre quadratique, par une théorie de champs gaussienne
%dont l’action effective est donnée par :
%\begin{equation}
%\delta^2 \Phi[\delta \rho] = \frac{1}{2} \iint d\theta\,d\theta'\; \delta \rho(\theta)\,
%\left[ \frac{\delta(\theta-\theta')}{\nu_{\!eq}(\theta)(1-\nu_{\!eq}(\theta))} - \mathcal{H}^{(W)}(\theta,\theta') \right]
%\delta \rho(\theta').
%\label{eq:action-gaussienne}
%\end{equation}
%
%Cette structure permet l’évaluation explicite des variances et corrélations
%d’observables (voir chap.~\ref{chap:observables}) ainsi que
%la dérivation des matrices de susceptibilité, au cœur de la description
%des fluctuations thermodynamiques et du transport.
%
%\subsection*{Commentaires physiques}
%
%\begin{itemize}[label = $\bullet$]
%\item L’apparition de \(\nu_{\!eq}(1-\nu_{\!eq})\) dans le hessien
%  reflète une structure de type Fermi–Dirac,
%  même dans un système bosonique,
%  conséquence de l’exclusion statistique induite par l’intégrabilité.
%\item La matrice hessienne contrôle la réponse du système à une
%  variation infinitésimale de \(\rho\) et, par là, encode les
%  fluctuations statiques du GGE.
%\item Ce développement quadratique justifie le caractère
%  gaussien des fluctuations dans le régime thermodynamique,
%  et sera à la base des extensions hydrodynamiques de type MFT
%  (Macroscopic Fluctuation Theory).
%\end{itemize}


%%%%%%%%%%%%%%%%%%%%%

%On note \( \mathcal{H}(\theta, \theta') \)  la Hessienne fonctionnelle de  \( \mathcal{S}_{YY}[\rho] - \mathcal{W}[\rho] \)
%
%%où \( \mathcal{H}(\theta, \theta') \) désigne la Hessienne fonctionnelle de l’action effective, définie par :
%
%\begin{eqnarray}
%    \mathcal{H}(\theta, \theta') = - \left. \frac{\delta^2}{\delta \rho(\theta) \, \delta \rho(\theta')} \left( \mathcal{S}_{YY}[\rho] - \mathcal{W}[\rho] \right) \right|_{\rho = \langle \rho \rangle}.
%\end{eqnarray}
%
%La fonction de corrélation à deux points des fluctuations \( \delta \rho \) est alors donnée par l’inverse de la Hessienne :
%
%\begin{eqnarray}
%    \langle \delta \rho(\theta) \, \delta \rho(\theta') \rangle = \frac{1}{L} \, \mathcal{H}^{-1}(\theta, \theta'),
%    \label{chap:fluctu:eq:fluctuations}
%\end{eqnarray}
%
%où \( \mathcal{H}^{-1} \) est défini comme le noyau de l’opérateur inverse au sens fonctionnel :
%
%\begin{eqnarray}
%     (\mathcal{H}^{-1} \cdot  \mathcal{H})(\theta, \theta')  = (\mathcal{H}\cdot  \mathcal{H}^{-1})(\theta, \theta') = \int d\theta'' \; \mathcal{H}(\theta, \theta'') \, \mathcal{H}^{-1}(\theta'', \theta') = \delta(\theta - \theta')
%     \label{chap:fluctu:eq:hessienner.prod.inv} 
%\end{eqnarray}
%
%où le dernier $\delta$ ,désigne la fonction delta de Dirac, et non une différentielle.\\
%
%Dans un premier temps calculons $\mathcal{H}(\theta, \theta')$. Puisque l'énergie généralisé s'écrit
%
%\begin{eqnarray}
%	\mathcal{W}[\rho] & = & \int w(\theta) \rho(\theta) \, d \theta , 	
%\end{eqnarray}
%
%alors sa différentielle seconde est nulle et la Hessienne se réécrit alors 
%
%\begin{eqnarray}
%	\mathcal{H}(\theta, \theta') = - \left. \frac{\delta^2\mathcal{S}_{YY}[\rho]}{\delta \rho(\theta) \, \delta \rho(\theta')} \right|_{\rho = \langle \rho \rangle},		
%\end{eqnarray}
%
%alors en injectant l'entropie de Yang-Yang que je rappelle 
%
%\begin{eqnarray}
%	\mathcal{S}_{YY}[\rho] & = & \int d \theta \, \left ( \rho_s \ln \rho_s - \rho \ln \rho - ( \rho_s - \rho ) \ln ( \rho_s - \rho ) \right ) ( \theta ) 		
%\end{eqnarray}
%
%la Hessienne se décompose alors
%
%\begin{eqnarray}
%	\mathcal{H}(\theta, \theta') & = & \mathcal{D}(\theta, \theta') + \mathcal{V}(\theta, \theta')
%	\label{chap:fluctu:eq:hessienne3} 	
%\end{eqnarray}
%
%avec une partie diagonale irrégulière
%
%\begin{eqnarray}
%	\mathcal{D}(\theta, \theta') & = & \left ( \frac{1}{\rho_{\! s , eq}(\theta) \langle \nu(\theta) \rangle (1 - \langle \nu(\theta) \rangle) }\right )  \delta(\theta, \theta') 	
%\end{eqnarray}
%
%avec une partie symétrique régulière
%
%\begin{eqnarray}
%	\mathcal{V}(\theta, \theta')	 & = & - \left ( \frac{1}{\rho_{\! s , eq}(\theta) (1 - \langle \nu(\theta) \rangle) } + \frac{1}{\langle \rho_s(\theta') \rangle  (1 - \langle \nu(\theta') \rangle) }\right )\frac{\Delta(\theta - \theta')}{2\pi}\\
%	&&  + \int d \theta''  \frac{\langle \nu(\theta'') \rangle}{\langle \rho_s(\theta'') \rangle (1 - \langle \nu(\theta'') \rangle) } \frac{\Delta(\theta - \theta'' )}{2\pi} \frac{\Delta(\theta''  - \theta')}{2\pi}
%	\label{chap:fluctu:eq:reg}
%\end{eqnarray}
% 
% en notant $\langle \rho \rangle = \langle \rho_s \rangle \langle \rho \rangle$.\\
 
 \subsection{Fluctuations autour de la distribution moyenne et inversion de la Hessienne}
 
 On cherche alors \( \mathcal{H}^{-1} \) aussi sous la forme 
\begin{eqnarray}
	\mathcal{H}^{-1}(\theta, \theta') & = & \mathcal{D}^{-1}(\theta, \theta') + \mathcal{B}(\theta, \theta') 
	\label{chap:fluctu:eq:hessienner.inv.1}	
\end{eqnarray}
avec une partie diagonale irrégulière
\begin{eqnarray}
	\mathcal{D}^{-1}(\theta, \theta') & = & (\rho_{\! s , eq}(\theta)  \nu_{\! eq}(\theta)(1 -\nu_{\! eq}(\theta)))  \delta(\theta, \theta') 
	 \label{chap:fluctu:eq:irreg.inv}	
\end{eqnarray}
tel que 
\begin{eqnarray}
    (\mathcal{D}^{-1} \cdot  \mathcal{D})(\theta, \theta')  = (\mathcal{D}\cdot  \mathcal{D}^{-1})(\theta, \theta') =  \int d\theta'' \; \mathcal{D}(\theta, \theta'') \, \mathcal{D}^{-1}(\theta'', \theta') = \delta(\theta - \theta'),
    \label{chap:fluctu:eq:irreg.prod.inv}
\end{eqnarray}
avec une partie symetrique régulière $\mathcal{B}$.\\
Les equations (\ref{chap:fluctu:eq:hessienne3}) , (\ref{chap:fluctu:eq:hessienner.inv.1}) , (\ref{chap:fluctu:eq:hessienner.prod.inv}) et (\ref{chap:fluctu:eq:irreg.prod.inv}) , il vient que cette série d'implication
\begin{eqnarray}
	\left \{\begin{array}{rcl} \mathcal{H}\cdot\mathcal{H}^{-1} & =& \delta \\  \mathcal{H}^{-1}\cdot\mathcal{H} & =& \delta  \end{array} \right.  \Rightarrow	 \left \{\begin{array}{rcl} \mathcal{H}\cdot\mathcal{B} & =& - \mathcal{V}\cdot\mathcal{D}^{-1} \\  \mathcal{B}\cdot\mathcal{H} & =& - \mathcal{D}^{-1}\cdot\mathcal{V}  \end{array} \right. \Rightarrow \left \{\begin{array}{rcl} \mathcal{B} & =& - \mathcal{H}^{-1}\cdot\mathcal{V}\cdot \mathcal{D}^{-1}  \\  \mathcal{B} & =& - \mathcal{D}^{-1}\cdot\mathcal{V}\cdot \mathcal{H}^{-1}  \end{array} \right.	
\end{eqnarray}
Du fait que tous ces fonctions ($\mathcal{H}$ , $\mathcal{D}$ , $\mathcal{V}$ et inverse ) soit symétriques alors l'equation ci-dessus ne forme qu'une et $\mathcal{B}$ étant donc symétrique .
Donc en utilisant (\ref{chap:fluctu:eq:reg}) et (\ref{chap:fluctu:eq:irreg.inv}) 
\begin{eqnarray*}
	\mathcal{B}(\theta , \theta ' )  & = & - (\mathcal{D}^{-1}\cdot\mathcal{V}\cdot \mathcal{H}^{-1})(\theta , \theta') 	,\\
	& = & 	(\rho_{\! s , eq}(\theta) \nu_{\! eq}(\theta) (1 -  \nu_{\! eq}(\theta) )) \times \\
	&& ~~~~~~~~~~~~~\left \{  \frac{\Delta}{2\pi} \star  \left [ \left (  \frac{1}{\rho_{\! s, eq}(\theta)  (1 - \nu_{\! eq}(\theta)  )} +  \frac{1}{\rho_{\! s, eq}(\cdot) (1 - \nu_{\! eq} (\cdot )  )}\right )  \, \mathcal{H}^{-1}( \cdot , \theta ' )   \right. \right .\\
	&&  ~~~~~~~~~~~~~~~~~~~~~~~~~\left . \left .   -  \frac{\nu_{\! eq}(\cdot)}{\rho_{\! s, eq}(\cdot) ( 1 - \nu_{\! eq}(\cdot) )}  \, \left ( \frac{\Delta}{2\pi} \star \mathcal{H}^{-1}( \cdot , \theta ' )   \right )  \right ] \right \} (\theta),
\end{eqnarray*}
où \( (f \star g)(x) \) désigne la convolution \( \int f(x - t)\, g(t)\, dt \).
En injectant cette dernier équation et \eqref{chap:fluctu:eq:hessienner.inv.1} , dans \eqref{chap:fluctu:eq:fluctuations} , il vient que une éqution implicite:
\begin{eqnarray*}
	\langle \delta \rho(\theta) \delta	\rho(\theta') \rangle & = & \frac{1}{L} \mathcal{D}^{-1}(\theta , \theta') + \\
	&& 	(\rho_{\! s , eq}(\theta) \nu_{\! eq}(\theta) (1 -  \nu_{\! eq}(\theta) )) \times \\
	&& ~~~~~~~~~~~~~\left \{  \frac{\Delta}{2\pi} \star  \left [ \left (  \frac{1}{\rho_{\! s, eq}(\theta)  (1 - \nu_{\! eq}(\theta)  )} +  \frac{1}{\rho_{\! s, eq}(\cdot) (1 - \nu_{\! eq} (\cdot )  )}\right )  \, \mathcal{H}^{-1}( \cdot , \theta ' )   \right. \right .\\
	&&  ~~~~~~~~~~~~~~~~~~~~~~~~~\left . \left .   -  \frac{\nu_{\! eq}(\cdot)}{\rho_{\! s, eq}(\cdot) ( 1 - \nu_{\! eq}(\cdot) )}  \, \left ( \frac{\Delta}{2\pi} \star \mathcal{H}^{-1}( \cdot , \theta ' )   \right )  \right ] \right \} (\theta),
\end{eqnarray*}





Cette expression explicite des corrélations permet d'évaluer les fluctuations des grandeurs macroscopiques comme le nombre total de particules ou l'énergie, en les exprimant comme des observables linéaires de la densité \( \rho(\theta) \).




%\section{Fluctuations autour du profil stationnaire}

%Dans les systèmes intégrables à grand nombre de particules, la description hydrodynamique est régie par la densité de rapidité \( \rho(\theta) \), qui encode la répartition des quasi-particules sur l’axe des rapidités \( \theta \). À l'équilibre, le profil stationnaire \( \rho^c(\theta) \) maximise la probabilité d’observer une configuration donnée. On s’intéresse ici aux fluctuations autour de ce profil stationnaire, et plus précisément aux corrélations à deux points des fluctuations \( \delta \rho(\theta) = \rho(\theta) - \rho^c(\theta) \).

%Pour ce faire, 
%On discrétise l’axe des rapidités en petites cellules \( [\theta, \theta + \delta\theta] \), chacune contenant un nombre moyen de quasi-particules donné par \( L \rho(\theta) \, \delta\theta \).%, où \( L \) désigne la taille du système. 

%Dans cette discrétisation, le développement quadratique de l’action \( \mathcal{A}[\rho] \) autour du profil stationnaire \( \rho^c \) s’écrit comme un produis scalaire:

%\begin{eqnarray*}
%    \left. \frac{\delta^2 \mathcal{A}}{{\delta \rho}^2} \right|_{\rho^c}(\delta \rho )  &= &  
%    \sum_{a,b} \delta \rho(\theta_a)  
%    \frac{\partial^2 \mathcal{A}}{\partial (\delta \rho(\theta_a)) \, \partial (\delta \rho(\theta_b))} \Big|_{\rho^c} 
%    \delta \rho(\theta_b).
%\end{eqnarray*}

%Cette structure quadratique mène naturellement à une intégrale fonctionnelle gaussienne pour la probabilité des fluctuations. Les corrélations à deux points des fluctuations s’expriment alors comme :

%\begin{eqnarray*}
%    \langle \delta \rho(\theta) \, \delta \rho(\theta') \rangle 
%    & = &   
%    \frac{  \displaystyle \int \mathcal{D}(\delta \rho) \, \delta \rho(\theta) \, \delta \rho(\theta') 
%    \exp\left( -\frac{1}{2} \sum_{a,b} \delta \rho(\theta_a) \, 
%    \operatormat{A}_{\theta_a, \theta_b} \, \delta \rho(\theta_b) \right) }
%    { \displaystyle \int \mathcal{D}(\delta \rho) \, 
%    \exp\left( -\frac{1}{2} \sum_{a,b} \delta \rho(\theta_a) \, 
%    \operatormat{A}_{\theta_a, \theta_b} \, \delta \rho(\theta_b) \right) } 
%    = \left( \operatormat{A}^{-1} \right)_{\theta, \theta'},
%\end{eqnarray*}

%où \( \operatormat{A} \) est la matrice hessienne de l’action évaluée au point stationnaire \( \rho^c \), c’est-à-dire :

%On note 

%\begin{eqnarray*}
%    \operatormat{A}_{\theta, \theta'} & = & \left. 
%    \frac{\partial^2 \mathcal{A}}{\partial(\delta \rho(\theta)) \, \partial (\delta \rho(\theta'))}\right|_{\rho^c} = - L  \left. 
%    \frac{\partial^2 \mathcal{S}_{YY}}{\partial(\delta \rho(\theta)) \, \partial (\delta \rho(\theta'))}\right|_{\rho^c}
%    .
%\end{eqnarray*}

%Dans le cas considéré, le profil stationnaire s’écrit sous la forme \( \rho^c = \rho^c_s \, \nu^c \), avec \( \nu^c \) la fonction d’occupation des quasi-particules et \( \rho^c_s \) la densité d’états disponibles. La matrice \( \operatormat{A} \) se décompose alors en deux contributions :

%\begin{eqnarray*}
%    \operatormat{A} & = & \operatormat{D} + \delta\theta \, \operatormat{V},
%\end{eqnarray*}

%avec une partie diagonale:

%\begin{eqnarray*}
%    D_{\theta, \theta'} & = &
%    L \, \delta\theta \left( \frac{1}{\rho^c_s(\theta) \nu^c(\theta)(1 - \nu^c(\theta)) } \right) 
%    \delta_{\theta,\theta'}, 
%\end{eqnarray*}

%et 

%\begin{eqnarray*}
%    V_{\theta, \theta'}  &=& L \, \delta\theta \left\{ 
%    - \left( 
%     \frac{1}{\rho^c_s(\theta)(1 - \nu^c(\theta))} 
%    +  \frac{1}{\rho^c_s(\theta')(1 - \nu^c(\theta'))}
%    \right) \frac{\Delta(\theta' - \theta)}{2\pi}  
%    +  \sum_{\theta''} \delta\theta 
%    \frac{\nu^c(\theta'') }{\rho^c_s(\theta'')  (1 - \nu^c(\theta'') )} 
%    \frac{\Delta(\theta'' - \theta)}{2\pi} 
%    \frac{\Delta(\theta'' - \theta')}{2\pi} \right\}.
%\end{eqnarray*}

%Pour obtenir une expression explicite de la matrice $\operatormat{A}^{-1}$, il est tentant d'utiliser une approche basée sur la théorie des perturbations, ce qui revient à appliquer la formule de Neumann. Toutefois, cette méthode n'est pas applicable ici car $\left\| \delta\theta \, \operatormat{D}^{-1} \operatormat{V} \right\| > 1$.
%De plus, même en faisant tendre $\delta\theta \to 0$, cette norme reste constante. Cela est dû à une dégénérescence des valeurs propres de la matrice $\operatormat{A}^{-1}$ lorsque $\delta\theta$ devient petit.

%Une écriture possible de l'inverse de la matrice $\operatormat{A}^{-1}$ est donnée par :

%\begin{eqnarray*}
%    \operatormat{A}^{-1} &= &\operatormat{D}^{-1} - \delta\theta \,  \operatormat{D}^{-1}\operatormat{V} \operatormat{A}^{-1},
%\end{eqnarray*}

%où le premier terme est singulier (irrégulier) et le second régulier.\\

%En faisant tendre la tailles des cellules vers 0 , \( \delta \theta \to 0 \), 
%\begin{eqnarray*}
%	\sum_{\theta_a} \delta \theta & \underset{\delta \theta \to 0 }{\rightarrow} & \int d\theta ,	
%\end{eqnarray*}
%les fluctuations s'écrivent alors sous la forme :


%\begin{eqnarray*}
%	\langle \delta \rho(\theta) \, \delta \rho(\theta') \rangle  &= & \frac{\rho^c_s(\theta) \nu^c(\theta)(1 - \nu^c(\theta))}{L \, \delta\theta} \delta ( \theta - \theta')		\\
%	& + & L \, \rho^c_s(\theta) \nu^c(\theta)(1 - \nu^c(\theta)) \left \{  \frac{\Delta}{2\pi} \star  \left [ \left (  \frac{1}{\rho^c_s(\theta) (1 - \nu^c(\theta))} +  \frac{1}{\rho^c_s (1 - \nu^c)}\right )  \, \langle \delta \rho \, \delta \rho(\theta') \rangle  \right. \right .\\
%	&&  ~~~~~~~~~~~~~~~~~~~~~~~~~~~~~~~~~~~~~~~~~~~~~\left . \left .   -  \frac{\nu^c}{\rho_s^c ( 1 - \nu )}  \, \left ( \frac{\Delta}{2\pi} \star \langle \delta \rho \, \delta \rho(\theta') \rangle   \right )  \right ] \right \} (\theta),
%\end{eqnarray*}

%où \( (f \star g)(x) \) désigne la convolution \( \int f(x - t)\, g(t)\, dt \).

%avec une permière partie irrégulière et une seconde partide régulière .



%Cette matrice encode la structure des fluctuations du système dans l’espace des configurations de rapidité.



\section{Lien entre dérivée fonctionnelle et réponse linéaire aux facteurs de Lagrange}

\paragraph{Hypothèses.}
On considère un potentiel spectral défini par une combinaison linéaire :
\[
w(\theta) = \sum_j \beta_j f_j(\theta),
\]
où les \( \beta_j \) sont les multiplicateurs de Lagrange associés aux charges généralisées :
\[
\operator{Q}_i = \mathcal{Q}[f_i] = L \int d\theta\, f_i(\theta)\, \operator{\rho}(\theta),
\]
et \( \operator{\rho}(\theta) \) est l’opérateur de densité spectrale.

\paragraph{Dérivée par rapport à \( \beta_j \).}
On utilise la règle de chaîne fonctionnelle :
\[
\left. \frac{\partial \langle \operator{Q}_i \rangle_w }{ \partial \beta_j } \right)_{\beta_{k \ne j}} = \int d\theta'\, \left. \frac{\delta \langle \operator{Q}_i \rangle_w }{ \delta w(\theta') } \right)_{w} \cdot \frac{\partial w(\theta')}{\partial \beta_j}.
\]
Comme \( \partial w(\theta') / \partial \beta_j = f_j(\theta') \), on obtient :
\[
\left. \frac{\partial \langle \operator{Q}_i \rangle_w }{ \partial \beta_j } \right)_{\beta_{k \ne j}} = \int d\theta'\, \left. \frac{\delta \langle \operator{Q}_i \rangle_w }{ \delta w(\theta') } \right)_{w} f_j(\theta').
\]

\paragraph{Expression de la dérivée fonctionnelle.}
On écrit :
\[
\left. \frac{\delta \langle \operator{Q}_i \rangle_w }{ \delta w(\theta') } \right)_{w} = L \int d\theta\, f_i(\theta)\, \left. \frac{\delta \langle \operator{\rho}(\theta) \rangle_w }{ \delta w(\theta') } \right)_{w} = - L^2 \int d\theta\, f_i(\theta)\, \chi_w(\theta, \theta'),
\]
où \( \chi(\theta, \theta') \doteq  - \frac{1}{L} \left. \frac{\delta \langle \operator{\rho}(\theta) \rangle_w }{ \delta w(\theta') } \right)_{w} \) est la {\em matrice de susceptibilité spectrale locale}.

\paragraph{Formule finale.}
En injectant dans l'expression précédente, on obtient :
\[
\left. \frac{\partial \langle \operator{Q}_i \rangle }{ \partial \beta_j } \right|_{\beta_{k \ne j}} = L \int d\theta\, f_i(\theta) \int d\theta'\, \chi(\theta, \theta') f_j(\theta') = L^2 \iint d\theta\, d\theta'\, f_i(\theta)\, \chi(\theta, \theta')\, f_j(\theta').
\]

\paragraph{Résultat.}
On définit alors la {\em matrice de susceptibilité croisée} \( \chi_{ij} \) par :
\[
\boxed{
\chi_{ij} \doteq - \left. \frac{\partial \langle \operator{Q}_i \rangle_w }{ \partial \beta_j } \right)_{\beta_{k \ne j}} = - \left. \frac{\delta \langle \operator{\mathcal{Q}}[f_i] \rangle_w }{ \delta f_j } \right)_{w} \doteq  \chi[f_i , f_j ] = L^2 \iint d\theta\, d\theta'\, f_i(\theta)\, \chi_w(\theta, \theta')\, f_j(\theta')
}
\]

Ce résultat relie :
\begin{itemize}[label =$\bullet$] 
  \item la réponse linéaire de \( \langle \operator{Q}_i \rangle_w \) à une perturbation \( \beta_j \),
  \item la dérivée fonctionnelle de l’observable \( \operator{\mathcal{Q}}[f_i] \) par rapport à sa fonction test,
  \item la projection de la matrice de susceptibilité spectrale locale \( \chi_w(\theta, \theta') \) sur \( f_i, f_j \).
\end{itemize}

\subsection{Cas particuliers : énergie et nombre de particules}

Nous testons à présent notre expression des fluctuations dans le cas particulier de l'équilibre thermique. Le système est supposé en contact avec un bain à température \( T \) et potentiel chimique \( \mu \). Le poids spectral prend alors la forme canonique :
\(
w(\theta) = \beta \varepsilon(\theta) - \beta \mu,
\)
avec \( \beta = 1 / (k_B T) \) et \( \varepsilon(\theta) \) l’énergie spectrale (par exemple \( \theta^2/2 \) pour des particules libres).
 

\paragraph{Cas du nombre de particules.}
On choisit la fonction test \( f_{\operator{Q}}(\theta) = 1 \), ce qui définit la charge associée :
\[
\operator{Q} = \operator{\mathcal{Q}}[1] = L \int d\theta\, \operator{\rho}(\theta).
\]
Son facteur de Lagrange / potentiel conjugué dans \( w(\theta) \) est simplement :
\[
w(\theta) \supset -\beta \mu \cdot 1.
\]
La susceptibilité associée s’écrit alors :
\[
\chi_{\operator{Q},\operator{Q}} = \left . \frac{\partial \langle \operator{Q} \rangle_w}{\partial (\beta \mu)} \right)_\beta  = L^2 \iint d\theta\, d\theta'\, \chi_w(\theta, \theta').
\]

\paragraph{Cas de l’énergie.}
On prend maintenant \( f_{\operator{H}}(\theta) = \varepsilon(\theta) \), l’énergie spectrale (ici par ex. \( \theta^2/2 \) pour des particules libres), ce qui donne :
\[
\operator{H} = \operator{\mathcal{Q}}[\varepsilon] = L \int d\theta\, \varepsilon(\theta)\, \operator{\rho}(\theta),
\]
et facteur de Lagrange / potentiel conjugué  est simplement \( \beta \varepsilon(\theta) \), avec :
\[
w(\theta) \supset \beta \varepsilon(\theta).
\]
On en déduit la susceptibilité thermique :
\[
\chi_{\operator{H},\operator{H}} := -\left.\frac{\partial \langle \operator{H} \rangle_w}{\partial \beta} \right)_{\beta \mu}  = L^2 \iint d\theta\, d\theta'\, \varepsilon(\theta)\, \chi_w(\theta, \theta')\, \varepsilon(\theta').
\]

\paragraph{Évaluation numérique.}

%On fais les calcule des correlation pour $\gamma = g/n $ fixé à ?? et $t = 1/(\beta g^2)$ entre ?? et ??.Les points correspondants sont représentés en bleu sur un diagramme de phase du modèle de Lieb-Liniger (voir Fig.~\ref{fig:diag}).

%Ce même diagramme contient également un point rouge correspondant à \( t = ?? \). Les fluctuations associées à ce point sont représentées en niveaux de couleur sur un graphique 2D (voir Fig.~\ref{fig.fluctu.A})

Les corrélations de \( \rho(\theta) \) sont calculées numériquement pour un couplage \(\gamma = g/n\) fixé, et une température réduite \( t = 1 / (\beta g^2) \) variant dans un intervalle donné. Les points correspondants sont indiqués en \textbf{bleu} dans le diagramme de phase du modèle de Lieb-Liniger (Fig.~\ref{fig:diag}).

Dans ce même diagramme, un \textbf{point rouge} correspond à des conditions fixes (\(T = 60~\mathrm{nK}\), \( \mu = 27~\mathrm{nK} \)), pour lesquelles la carte des fluctuations \( \delta \rho \) est représentée en niveaux de couleur (Fig.~\ref{fig.fluctu.A}).


\begin{figure}[H]
	\centering
	\begin{subfigure}[b]{0.45\textwidth}
		\includegraphics[width=\textwidth]{Figures/diagram.png}
		\caption{Diagramme de phase du modèle de Lieb-Liniger.}
		\label{fig:diag}
	\end{subfigure}
	\hfill
	\begin{subfigure}[b]{0.45\textwidth}
		\includegraphics[width=\textwidth]{Figures/fluctu.png}
		\caption{ \( \mathcal{B}(\theta, \theta') \).}
		\label{fig.fluctu.A}
	\end{subfigure}
	\caption{(a) Diagramme de phase du modèle de Lieb-Liniger à l’équilibre thermique. Différents régimes asymptotiques sont séparés par des transitions progressives. Les points bleus représentent les fluctuations calculées numériquement pour différentes températures. Les coordonnées sont données par \( \gamma = \frac{m g}{\hbar^2 n} \) et \( t = \frac{k_B T}{m g^2/\hbar^2} \). (b) Représentation en niveaux de couleur de la partie régulière $\mathcal{B}$ des fluctuations \( \delta \rho \) pour \( T = 60~\mathrm{nK} \) et \( \mu = 27~\mathrm{nK} \) (point rouge dans (a)).}
	\label{fig:diag_fig}
\end{figure}

\paragraph{Comparaison avec les dérivées thermodynamiques.}

Les résultats obtenus à partir de l’analyse quadratique de l’action (fluctuations de \( \rho \)) sont comparés aux fluctuations extraites directement par différentiation des observables thermodynamiques \( \langle \operator{Q} \rangle \) et \( \langle \operator{H} \rangle \). Ces comparaisons sont présentées dans la Fig.~\ref{fig.fluctu.A_com} et révèlent une excellente concordance.



%Les résultats obtenus à l’aide de cette méthode thermodynamique sont comparés à ceux issus du calcul direct des fluctuations de \( \rho \). Ces comparaisons sont représentées sur la Fig.~\ref{fig.fluctu.A_com}, et montrent une excellente concordance entre les deux approches.

\begin{figure}[H]
	\centering 
	\includegraphics[width=1\textwidth]{Figures/fluctuations_plot_log_gamma=-1.342.png}	
	%\includegraphics[width=1\textwidth]{Figures/fluctuations_relativ_plot_log_gamma=-1.342.png}	
	\caption{Comparaison numérique entre les fluctuations calculées à partir de l’analyse quadratique de l’action (fluctuations de \( \rho \)) et celles obtenues par dérivées thermodynamiques des observables moyennes.}
	\label{fig.fluctu.A_com}
\end{figure}

%\paragraph{🔸 Remarque sur les corrélations globales.}
%Dans les deux cas, les fluctuations totales \( \langle \delta N^2 \rangle \), \( \langle \delta E^2 \rangle \), ou croisées \( \langle \delta E\, \delta N \rangle \), sont données par les projections :
%\[
%\langle \delta Q_i\, \delta Q_j \rangle = \chi_{ij} = L^2 \iint d\theta\, d\theta'\, f_i(\theta)\, \chi(\theta, \theta')\, f_j(\theta').
%\]
%Cela généralise l’idée de la **formule de fluctuation-réponse** : la réponse à un potentiel conjugué est gouvernée par les corrélations spontanées du système au point d’équilibre.
%
%\paragraph{✅ Conclusion.}
%Pour toute charge \( \mathcal{Q}[f] = L \int f(\theta)\, \operator{\rho}(\theta)\, d\theta \), on a :
%\[
%\boxed{
%\chi_{f,f} = -\frac{\partial \langle \mathcal{Q}[f] \rangle}{\partial \lambda} = L^2 \iint d\theta\, d\theta'\, f(\theta)\, \chi(\theta, \theta')\, f(\theta'),
%}
%\]
%où \( \lambda \) est le coefficient dans \( w(\theta) = \lambda f(\theta) \).
%




%------------------------
%
%
%\section{Fonction correlation du nombre d'atomes et de l'énergie}
%
%
%Il est maintenant pertinent de tester notre expression des fluctuations. On fait l'hypothèset que le système est en équilibre thermique caractérisé par la températeur $T$ et le potentiel chimoque $\mu$.
%
%La valeur propre $\mathcal{N}[\rho]$ (resp $\mathcal{E}[\rho]$) de opérateur nombre d'atomes $\operator{\mathcal{N}}$ (resp énergie. $\operator{\mathcal{E}}$) et assiciés aux configuration liès à la distribution de rapidité $\rho$ s'écrit (avec comme combention la masse des atomes $k_B = \hbar = m =1$)
%
%\begin{eqnarray*}
%	\mathcal{N}[\rho] & = & L \int d \theta \rho(\theta),\\
%	\mathcal{E}[\rho] & = & 	\frac{L}2 \int d \theta  \theta^2 \rho(\theta).	
%\end{eqnarray*}
%
%Les corrélation associèes s'est 
%\begin{eqnarray*}
%	C_{\operator{\mathcal{N}},\operator{\mathcal{N}}} & = &  L^2 \int d\theta_a \int d\theta_b \, \langle \delta \rho(\theta_a) \delta \rho(\theta_b) \rangle, \\
%	C_{\operator{\mathcal{E}},\operator{\mathcal{E}}} & = &  	\left(\frac{L}2\right)^2 \int d\theta_a \int d\theta_b \,  \theta_a^2 \theta_b^2 \, \langle \delta \rho(\theta_a) \delta \rho(\theta_b) \rangle, \\	
%\end{eqnarray*}
%
%
%
%%Les fluctuations des observables "nombre d’atomes" \( \operator{\mathcal{N}} \) et "énergie" \( \operator{\mathcal{E}} \) peuvent être exprimées à l’aide des fluctuations de \( \rho \) :
%
%%\begin{eqnarray*}
%%    \left \langle  \left( \operator{\mathcal{N}} - \langle \operator{\mathcal{N}} \rangle \right)^2 \right \rangle &=& L^2 \int d\theta_a \int d\theta_b \, \langle \delta \rho(\theta_a) \delta \rho(\theta_b) \rangle, \\
%%    \left \langle \left( \operator{\mathcal{E}} - \mu \operator{\mathcal{N}} - \langle \operator{\mathcal{E}} - \mu \operator{\mathcal{N}} \rangle \right)^2 \right \rangle &=& L^2 \int d\theta_a \int d\theta_b \, \left( -\mu + \frac{1}{2} m \theta_a^2 \right) \left( -\mu + \frac{1}{2} m \theta_b^2 \right) \langle \delta \rho(\theta_a) \delta \rho(\theta_b) \rangle,
%%\end{eqnarray*}
%
%%où \( \langle \operator{\mathcal{O}}_i \rangle \) désigne la moyenne de l’observable \( \operator{\mathcal{O}}_i \), \( m \) la masse des atomes et \( \mu \) le potentiel chimique.\\
%
%Dans la section \{??\}, nous avons vu que la variance d’une observable \( \operator{\mathcal{O}}_i \), autrement dit ses fluctuations, peut également s’exprimer comme une dérivée thermodynamique de sa moyenne :
%
%\begin{eqnarray*}
%    C_{\operator{\mathcal{O}}_i,\operator{\mathcal{O}}_i} = \Delta_{\operator{\mathcal{O}}_i}^2 = - \left. \frac{\partial \langle \operator{\mathcal{O}}_i \rangle}{\partial \beta_i} \right|_{\beta_{j \neq i}},
%\end{eqnarray*}
%
%où \( \beta_i \) est la variable conjuguée à \( \langle \operator{\mathcal{O}}_i \rangle \). En particulier, les fluctuations du nombre d’atomes et de l’énergie peuvent s’écrire :
%
%\begin{eqnarray*}
%    \Delta_{\operator{\mathcal{N}}}^2 &=& \frac{1}{\beta} \left. \frac{\partial \langle \operator{\mathcal{N}} \rangle}{\partial \mu} \right|_T, \\
%    \Delta_{\operator{\mathcal{E}}}^2 &=& \Delta_{\operator{\mathcal{E}} - \mu \operator{\mathcal{N}}}^2 + \mu \Delta_{\operator{\mathcal{N}}}^2 =   - \left. \frac{\partial \langle \operator{\mathcal{E}} - \mu \operator{\mathcal{N}} \rangle}{\partial \beta} \right|_\mu - \frac{\mu}{\beta} \left. \frac{\partial \langle \operator{\mathcal{N}} \rangle}{\partial \mu} \right|_T,
%\end{eqnarray*}
%
%avec \( \beta =T^{-1} \).
%
%Les quantités \( C_{\operator{\mathcal{N}},\operator{\mathcal{N}}} \) et \( \Delta_{\operator{\mathcal{N}}}^2 \) sont analytiquement équivalentes, de même que \( C_{\operator{\mathcal{E}},\operator{\mathcal{E}}} \) et \( \Delta_{\operator{\mathcal{E}}}^2 \).
%
%Nous souhaitons maintenant effectuer une comparaison numérique entre ces deux approches. Pour ce faire, nous avons d'abord résolu numériquement l’équation \{??\} avec \( f(\theta) = \beta \epsilon(\theta) - \beta\mu \), ce qui nous a permis d’obtenir \( \rho(\theta) \) et \( \rho_s(\theta) \) (Il y auras plus de détaille dans le chapitre ??) .
%
%%Nous avons ensuite calculé les fluctuations de \( \rho \) pour une densité spatiale fixée à \( n = 10~\mu \mathrm{m}^{-1} \), et pour différentes températures \( T \) allant de \( 5.7~\mathrm{K} \) à \( 53.5~\mathrm{nK} \). Le potentiel chimique \( \mu \) est ici une fonction de \( T \) et de \( n \). Les points correspondants sont représentés en bleu sur un diagramme de phase du modèle de Lieb-Liniger. En abscisse, nous avons le logarithme décimal du paramètre d’interaction de Lieb-Liniger \( \gamma = \frac{m g}{\hbar^2 n} \), et en ordonnée le logarithme décimal de \( t = \frac{\hbar^2}{\beta m g^2} \) (voir Fig.~\ref{fig:diag}).
%
%On fais les calcule des correlation pour $\gamma = g/n $ fixé à ?? et $t = 1/(\beta g^2)$ entre ?? et ??.Les points correspondants sont représentés en bleu sur un diagramme de phase du modèle de Lieb-Liniger (voir Fig.~\ref{fig:diag}).
%
%%Ce même diagramme contient également un point rouge correspondant à \( T = 60~\mathrm{nK} \) et \( \mu = 27~\mathrm{nK} \). Les fluctuations associées à ce point sont représentées en niveaux de couleur sur un graphique 2D (voir Fig.~\ref{fig.fluctu.A}).
%
%Ce même diagramme contient également un point rouge correspondant à \( t = ?? \). Les fluctuations associées à ce point sont représentées en niveaux de couleur sur un graphique 2D (voir Fig.~\ref{fig.fluctu.A})
%
%\begin{figure}[H]
%	\centering
%	\begin{subfigure}[b]{0.45\textwidth}
%		\includegraphics[width=\textwidth]{Figures/diagram.png}
%		\caption{}
%		\label{fig:diag}
%	\end{subfigure}
%	\hfill
%	\begin{subfigure}[b]{0.45\textwidth}
%		\includegraphics[width=\textwidth]{Figures/fluctu.png}
%		\caption{Fluctuations mesurées}
%		\label{fig.fluctu.A}
%	\end{subfigure}
%	\caption{(a) Diagramme de phase du modèle de Lieb-Liniger à l’équilibre thermique. Différents régimes asymptotiques sont séparés par des transitions progressives. Les points bleus représentent les fluctuations calculées numériquement pour différentes températures. Les coordonnées sont données par \( \gamma = \frac{m g}{\hbar^2 n} \) et \( t = \frac{k_B T}{m g^2/\hbar^2} \). (b) Représentation en niveaux de couleur des fluctuations \( \delta \rho \) pour \( T = 60~\mathrm{nK} \) et \( \mu = 27~\mathrm{nK} \) (point rouge dans (a)).}
%	\label{fig:diag_fig}
%\end{figure}
%
%%Nous calculons ensuite les moyennes des observables :
%
%%\begin{eqnarray*}
%%    \langle \operator{\mathcal{N}} \rangle &=& L \int \rho(\theta) \, d\theta, \\
%%    \langle \operator{\mathcal{E}} - \mu \operator{\mathcal{N}} \rangle &=& L \int \left( - \mu + \frac{1}{2} m \theta^2 \right) \rho(\theta) \, d\theta,
%%\end{eqnarray*}
%
%%pour chaque point du diagramme (Fig.~\ref{fig:diag}). En faisant varier leur variable conjuguée, nous accédons alors aux fluctuations $\Delta_{\operator{\mathcal{N}}}^2 $ et $\Delta_{\operator{\mathcal{E}} - \mu \operator{\mathcal{N}}}^2$.
%
%%\[
%%\Delta_{\operator{\mathcal{N}}}^2 = \frac{1}{\beta} \left. \frac{\partial \langle \operator{\mathcal{N}} \rangle}{\partial \mu} \right|_T, \quad
%%\Delta_{\operator{\mathcal{E}} - \mu \operator{\mathcal{N}}}^2 = - \left. \frac{\partial \langle \operator{\mathcal{E}} - \mu \operator{\mathcal{N}} \rangle}{\partial \beta} \right|_\mu.
%%\]
%
%Les résultats obtenus à l’aide de cette méthode thermodynamique sont comparés à ceux issus du calcul direct des fluctuations de \( \rho \). Ces comparaisons sont représentées sur la Fig.~\ref{fig.fluctu.A_com}, et montrent une excellente concordance entre les deux approches.
%
%\begin{figure}[H]
%	\centering 
%	\includegraphics[width=1\textwidth]{Figures/fluctuations_plot_log_gamma=-1.342.png}	
%	%\includegraphics[width=1\textwidth]{Figures/fluctuations_relativ_plot_log_gamma=-1.342.png}	
%	\caption{Comparaison numérique entre les fluctuations calculées à partir de l’analyse quadratique de l’action (fluctuations de \( \rho \)) et celles obtenues par dérivées thermodynamiques des observables moyennes.}
%	\label{fig.fluctu.A_com}
%\end{figure}
%
%{\color{blue} 
%%Dans ce chapitre, nous nous intéressons aux fluctuations de la distribution de rapidité \( \delta \rho \) autour d'une distribution de référence \( \rho^c \), qui maximise la contribution à la fonction de partition des états, exprimée comme une fonctionnelle de la distribution \( \rho \) : 
%
%%La fonction de partition des états, s'exprime comme une fonctionnelle de la distribution \( \rho \) : 
%
%%\begin{eqnarray*}\Xi & = & \sum_\rho \exp \left( -\mathcal{A}(\rho) \right).\end{eqnarray*}  
%
%%Dans la section {\em \bf Entropie de Yang-Yang} (\ref{??}), l'action \( \mathcal{A}(\rho) \) s'écrit sous la forme :  
%
%%\begin{eqnarray*}\mathcal{A}(\rho) & \doteq & - L\mathcal{S}_{YY}(\rho) + L\int f(\theta) \rho (\theta) \, d\theta,		\end{eqnarray*}  
%
%%où \( \mathcal{S}_{YY} \) est la fonctionnelle d'entropie de Yang-Yang, définie dans (\ref{??}), et \( f \) est la fonction paramétrant les charges, introduite dans (\ref{??}).  
%}
%{\color{blue} 
%%Dans cette même section {\em \bf Entropie de Yang-Yang} (\ref{??}), nous avons établi un lien entre \( f \) et distribution de référence \( \rho^c \), qui maximise la contribution à la fonction de partition des états .\\
%
%}
%
%{\color{blue} 
%
%%L'hypothèse qui après relaxation le système est décrit pas un GGE, est fondamantale dans notre compréhention, et a énormément d'implication. Et donc il est ittéressent de tester cette hyphotèse experimentalement.La distribution de rapidité moyenne $\rho^c$ ne permet pas de verifier que le GGE est bien l'enssemeble statistique adequa. En effet plein d'autre enssemble statistique donne la meme distribution de rapidité moyenne. Il faut donc aller au dela. Il faut regarder les fluctuation de distribution de rapidité \( \delta \rho \) autour \( \rho^c \)
%
%%L'hypothèse selon laquelle, après relaxation, le système est décrit par un ensemble généralisé de Gibbs (GGE) constitue un pilier fondamental de notre compréhension des dynamiques hors équilibre dans les systèmes intégrables. Cette hypothèse a des implications théoriques majeures, et il est donc essentiel de la confronter à l'expérience. Cependant, la seule connaissance de la distribution de rapidité moyenne $\rho^c$ ne permet pas de valider cette description. En effet, plusieurs ensembles statistiques peuvent conduire à une même distribution moyenne. Pour distinguer le GGE des autres candidats, il est nécessaire d’aller au-delà et d’analyser les fluctuations de la distribution de rapidité, notées \( \delta \rho \), autour de la valeur moyenne \( \rho^c \).
%
%
%%On veux tester si nos experience est décrit pas un GGE. Pour cela nous nous intéressons aux fluctuations de la distribution de rapidité \( \delta \rho \) autour \( \rho^c \).
%
%%Nous poursuivons à présent avec cette définition de l'action de classe $\mathcal{C}^2$ et admetant une distribution critique $\rho^c$ tel que sa différentielle en ce point critique soit nulle $d\mathcal{A}_{\rho^c} = 0 $ (\ref{??}) de sorte que d'aprés la formule de Taylor-Youg %afin de déterminer les fluctuations autour de \( \Pi^c \). Pour cela, nous réécrivons l'action sous la forme :  
%
%%Nous poursuivons en développant l'action autour de la  distribution  $\rho^c$. La différentielle de l'action en ce point  est nulle ($d\mathcal{A}_{\rho^c} = 0 $ (\ref{??})).D'aprés la formule de Taylor-Youg , à l'ordre 2 en $\delta \rho$,  l'action s'écrit avec une forme quadratique : % tel que sa différentielle en ce point critique soit nulle ,  de sorte que d'aprés la formulle de Taylor-Youg %afin de déterminer les fluctuations autour de \( \Pi^c \). Pour cela, nous réécrivons l'action sous la forme :  
%
%%\begin{eqnarray*}  \mathcal{A}(\rho^c + \delta \rho) & \underset{ \delta \rho \to 0 }{=} & \mathcal{A}(\rho^c)  + \frac{1}{2} \left. \frac{\delta^2 \mathcal{A}}{\delta \rho^2} \right|_{\rho^c} (\delta \rho) + \mathcal{O}((\delta \rho)^3),  \end{eqnarray*}  
%
%%une expression quadratique pour l'action à l'ordre dominant en \( \delta \Pi \) avec $\left. \frac{\delta^2 \mathcal{A}}{\delta \rho^2} \right|_{\rho^c}$ la forme quadratique définie positive (Fig (\ref{fig.fluctu.A})).
%
%}
%
%{\color{blue}
%%On discrétise l'axe des rapidités en  petite cellule de rapidité $[\theta, \theta+\delta\theta]$, qui contient $L\rho(\theta) \delta \theta$ rapidités. 
%	
%
%
%
%%Avec ces petites tranches, la forme quadratique s’écrit :
%
%%\begin{eqnarray*}
%%    \left. \frac{\delta^2 \mathcal{A}}{{\delta \rho}^2} \right|_{\rho^c}(\delta \rho ) &=&  \sum_{a,b \mid \text{tranche}}  
%%    \delta \rho(\theta_a)  \frac{\partial^2 \mathcal{A}}{\partial \delta \rho(\theta_a) \partial \delta \rho(\theta_b) } (\rho^c)  \delta \rho(\theta_b).
%%\end{eqnarray*}
%%Les fluctuations s’écrivent donc :
%
%%\begin{eqnarray*}
%%    \langle \delta \rho ( \theta) \delta \rho ( \theta') \rangle &=&  
%%    \frac{ \int d\delta \rho \, \delta \rho(\theta) \delta \rho ( \theta') 
%%    \exp \left( - \frac{1}{2} \sum_{a,b \mid \text{tranche}}  
%%    \delta \rho(\theta_a) \frac{\partial^2 \mathcal{A}}{\partial \delta \rho(\theta_a) \partial \delta \rho(\theta_b) } (\rho^c)  \delta \rho(\theta_b) \right) }
%%    { \int d\delta \Pi  
%%    \exp \left( - \frac{1}{2} \sum_{a,b \mid \text{tranche}}  
%%    \delta \rho(\theta_a) \frac{\partial^2 \mathcal{A}}{\partial \delta \rho(\theta_a) \partial \delta \rho(\theta_b) } (\rho^c)  \delta \rho(\theta_b) \right) } \\
%%    &=& \left( \mathbf{A}^{-1} \right)_{\theta , \theta'}
%%\end{eqnarray*}
%
%
%%\begin{aff}
%
%%\begin{eqnarray*}\langle \delta \rho ( \theta) \delta \rho ( \theta') \rangle &=& 	\left( \mathbf{A}^{-1} \right)_{\theta , \theta'}\end{eqnarray*}
%
%	
%%avec la  {\em matrice hessienne} $\mathbf{A}_{\theta , \theta'} \equiv \frac{\partial^2 \mathcal{A}}{\partial \delta \rho(\theta) \partial \delta \rho(\theta') }(\rho^c)$, au point critique/ qui maximise la probabilité  $\rho^c=\rho^c_s \nu^c $, s'écrit
%
%%avec la  matrice $\mathbf{A}_{\theta , \theta'} \equiv \frac{\partial^2 \mathcal{A}}{\partial \delta \rho(\theta) \partial \delta \rho(\theta') }(\rho^c)$, qui maximise la probabilité  $\rho^c=\rho^c_s \nu^c $, s'écrit
%
%%\begin{eqnarray*}
%%	\operator{A} & = & \operator{A}^{(0)} + \delta \theta \operator{V}
%%\end{eqnarray*}
%
%%avec 
%
%%\begin{eqnarray*}
%%	A^{(0)}_{\theta , \theta'}  & = &  L\delta \theta \left ( \frac{ 1}{\rho^c_s ( 1  - \nu^c ) \nu^c } \right )(\theta)    \delta({\theta - \theta '})	,\\
%%	V_{\theta , \theta'}  &= & L \delta \theta \left \{ - \left [ \left ( \frac{1}{\rho^c_s( 1 - \nu^c) } \right ) ( \theta)  +  \left ( \frac{1}{\rho^c_s( 1 - \nu^c) } \right ) ( \theta' )\right ] \frac{ \Delta( \theta'- \theta )}{ 2 \pi } + \int d\theta''  \left ( \frac{\nu^c}{\rho^c_s( 1 - \nu^c) } \right )(\theta'') \frac{\Delta(\theta''- \theta)}{2 \pi}\frac{\Delta(\theta''- \theta')}{2 \pi}   \right \} 	
%%\end{eqnarray*}
%
%%\end{aff}
%
%}
%
%
%{\color{blue} 
%%Maintenant il est interressent de tester notre expression des fluctuation.\\
%%Dans la section {??} nous avons vus que la variance d'un observable $\operator{\mathcal{O}}_i$ s'écrit :
%
%%\begin{eqnarray*}
%%	\Delta_{\operator{\mathcal{O}}_i}^2 & = & -  \left . \frac{ \partial \langle \operator{\mathcal{O}}_i \rangle }{ \partial \beta_i } 	 \right )_{\beta_{j \neq i } }
%%\end{eqnarray*}
%
%%avec la moyenne $\langle \operator{\mathcal{O}}_i \rangle $ de  $\operator{\mathcal{O}}_i$ et $\beta_i$ la variable conjuguais de $\langle \operator{\mathcal{O}}_i \rangle $. Si on note les observables nombre d'atomes $\operator{\mathcal{N}}$ et énergie $\operator{\mathcal{E}}$, alors leur variance s'écrit 
% 
%
%%\begin{eqnarray*}
%%	\Delta_{\operator{\mathcal{N}}}^2  & = &  \frac{1}{\beta} \left . \frac{\partial \langle \operator{\mathcal{N}} \rangle}{\partial \mu} \right )_T \\
%%	\Delta_{\operator{\mathcal{E}}-\mu \operator{\mathcal{N}}}^2  & = &  - \left . \frac{\partial \langle \operator{\mathcal{E}}-\mu \operator{\mathcal{N}} \rangle}{\partial \beta} \right )_\mu 
%%\end{eqnarray*}
%
%%avec $\beta = (k_B T)^{-1}$, la temperature $T$ et  le potentielle chimique $\mu$.\\
%
%%Et écrit à l'aide des fluctuation de $\rho$
%
%%\begin{eqnarray*}
%%	\tilde{\Delta}_{\operator{\mathcal{N}}}^2  &= & L^2 \int d\theta_a \int d \theta_b \, \langle \delta \rho(\theta_a) \delta \rho(\theta_b) \rangle \\
%%	\tilde{\Delta}_{\operator{\mathcal{E}}-\mu \operator{\mathcal{N}}}^2  & = & L^2 \int d\theta_a \int d \theta_b \, \left ( - \mu + \frac{1}2 m \theta_a^2  \right  )\left ( - \mu + \frac{1}2 m \theta_b^2  \right  )  \langle \delta \rho(\theta_a) \delta \rho(\theta_b) \rangle
%%\end{eqnarray*}
%
%%On veux comparer $\Delta_{\operator{\mathcal{N}}}^2$ à $\tilde{\Delta}_{\operator{\mathcal{N}}}^2$ et $\Delta_{\operator{\mathcal{E}}-\mu \operator{\mathcal{N}}}^2$ à $\tilde{\Delta}_{\operator{\mathcal{E}}-\mu \operator{\mathcal{N}}}^2 $. Pour ce faire, nous avons d'abord résolu numériquement l'équation {??} avec \( f(\theta) = \frac{\epsilon(\theta) - \mu}{T} \),  ce qui nous a permis d'obtenir \( \rho(\theta) \) et \( \rho_s(\theta) \). Si on fixe le couple $T$ et $\mu$. On peux donc deterniner les moyenne 
%
%%\begin{eqnarray*}
%%	\langle \operator{\mathcal{N}} \rangle & = & L \int \, \rho(\theta) d \theta ,\\
%%	\langle \operator{\mathcal{E}} - \mu \operator{\mathcal{N}}\rangle & = & L \int \, \left (\frac{1}2 m \theta^2 - \mu  \right ) \rho(\theta) d\theta . 		
%%\end{eqnarray*}
%
%
%
%%Il est maintenant intéressant de tester notre expression des fluctuations.
%
%%Les fluctuations des observables nombre d'atomes $\operator{\mathcal{N}}$ et énergie $\operator{\mathcal{E}}$ peuvent s'écrire à l'aide de des fluctuation de $\rho$  :
%
%%\begin{eqnarray*}
%%    \left \langle  \left ( \operator{\mathcal{N}} - \langle \operator{\mathcal{N}}  \rangle  \right )^2 \right \rangle  &=& L^2 \int d\theta_a \int d\theta_b \, \langle \delta \rho(\theta_a) \delta \rho(\theta_b) \rangle, \\
%%    \left \langle  \left (  \operator{\mathcal{E}} - \mu \operator{\mathcal{N}}   -  \langle\operator{\mathcal{E}} - \mu \operator{\mathcal{N}}  \rangle  \right )^2  \right \rangle  &=& L^2 \int d\theta_a \int d\theta_b \, \left( - \mu + \frac{1}{2} m \theta_a^2 \right) \left( - \mu + \frac{1}{2} m \theta_b^2 \right) \langle \delta \rho(\theta_a) \delta \rho(\theta_b) \rangle,
%%\end{eqnarray*}
%
%%où \( \langle \operator{\mathcal{O}}_i \rangle \) est la moyenne de l'oservable  \( \operator{\mathcal{O}}_i \) ,  $m$ est la masse des atomes et $\mu$ est le potentielle chimique.\\
% 
%%Dans la section {??}, nous avons vu que la variance d'un observable \( \operator{\mathcal{O}}_i \) autrement dit les fluctuation de \( \operator{\mathcal{O}}_i \) peuvent d'ecrire avec  derievés de leur moyenne  :
%
%%\begin{eqnarray*}
%%    \Delta_{\operator{\mathcal{O}}_i}^2 &=& - \left. \frac{\partial \langle \operator{\mathcal{O}}_i \rangle}{\partial \beta_i} \right|_{\beta_{j \neq i}}.
%%\end{eqnarray*}
%
%% où \( \beta_i \) est la variable conjuguée de \( \langle \operator{\mathcal{O}}_i \rangle \). Soit les fluctuation des observables nombre d'atome et énergie peuvent aussi s'écrire avec une dérivé de leur moyenne :
%
%%\begin{eqnarray*}
%%    \Delta_{\operator{\mathcal{N}}}^2 &=& \frac{1}{\beta} \left. \frac{\partial \langle \operator{\mathcal{N}} \rangle}{\partial \mu} \right|_T, \\
%%    \Delta_{\operator{\mathcal{E}} - \mu \operator{\mathcal{N}}}^2 &=& - \left. \frac{\partial \langle \operator{\mathcal{E}} - \mu \operator{\mathcal{N}} \rangle}{\partial \beta} \right|_\mu.
%%\end{eqnarray*}
%
%%avec \( \beta = (k_B T)^{-1} \), la température \( T \).
%
%%Les fluctuation \( \left \langle  \left ( \operator{\mathcal{N}} - \langle \operator{\mathcal{N}}  \rangle  \right )^2 \right \rangle  \) et  \( \Delta_{\operator{\mathcal{N}}}^2 \) sont analitiquement égaux et de meme pour  \( \left \langle  \left (  \operator{\mathcal{E}} - \mu \operator{\mathcal{N}}   -  \langle\operator{\mathcal{E}} - \mu \operator{\mathcal{N}}  \rangle  \right )^2  \right \rangle\) et  \( \tilde{\Delta}_{\operator{\mathcal{E}} - \mu \operator{\mathcal{N}}}^2 \).
%
%%Nous souhaitons faire une comparaisont numérique . Pour ce faire, nous avons d'abord résolu numériquement l'équation {??} avec \( f(\theta) = \frac{\epsilon(\theta) - \mu}{T} \), ce qui nous a permis d'obtenir \( \rho(\theta) \) et \( \rho_s(\theta) \). \\
%
%%J'ai calculer les fluctuation de $\rho$, à densité spatial $n = 10 \mu m^{-1}$  fixées et pour different température $T$ , allant de $5.7 ~K $ à $53,5~ nK$. Le potentiel chimique est ici ine fonction de $T$ et de $n$. J'ai représenter en blue ces points sur un Diagramme de phase du modèle de LL. En abscises on a le logarythmes en base 10 du facteur de LL $\gamma$, qui je rappelle $\gamma = \frac{m g}{\hbar^2 n } $ et en ordonné le logarythme en base 10 de $t = \frac{\hbar^2  }{ \beta m g^2} $ (voir Fig \ref{fig:diag}) . Ce ce diagrammme se trouve aussi une point rouge pour $T = 60 ~nK$ et $\mu = 27~ nK$. J'ai representer les flctuations coresponds sur une graph 2D en niveau de couleur (voir Fig \ref{fig.fluctu.A}).
%
%%\begin{figure}[H]
%%	\centering
%%	\begin{subfigure}[b]{0.45\textwidth}
%%		\includegraphics[width=\textwidth]{Figures/diagram.png}
%%		\caption{}
%%		\label{fig:diag}
%%	\end{subfigure}
%%	\hfill
%%	\begin{subfigure}[b]{0.45\textwidth}
%%		\includegraphics[width=\textwidth]{Figures/fluctu.png}
%%		\caption{Fluctuations mesurées}
%%		\label{fig.fluctu.A}
%%	\end{subfigure}
%%	\caption{(a) Diagramme de phase du modèle de Lieb-Liniger à l'équilibre thermique. Différents régimes asymptotiques sont séparés par des transitions progressives. %Le passage entre le régime de gaz de Bose idéal et le régime de quasi-condensat a lieu pour \( t \sim \gamma^{-3/2} \), celui entre le régime de quasi-condensat et le régime de bosons impenetrables (hard-core) a lieu pour \( \gamma \sim 1 \), et celui entre le régime hard-core et le gaz de Bose idéal se produit pour \( t \sim 1 \). La ligne en pointillés représente la condition de dégénérescence quantique, qui s’écrit \( t \sim \gamma^{-2} \).% Il est à noter que l'équilibre thermique n’est pas garanti dans le modèle de Lieb-Liniger, en raison de son intégrabilité.
%%	Le point bleu reprensentent les fluctuation calculer avec $\gamma = m g/\hbar^2 n $ et $t = k_B T/(m g^2/\hbar^2)$. (b) Reprenstation en nuande de couleur des fluctuations $\delta \rho$ avec $T = 60 ~nK$ et $\mu = 27~ nK$ (point rouge dans (b)
%%}
%%	\label{fig:diag_fig}
%%\end{figure}
%
%
%%Maintenant on calcule les moyenne des observables : 
%
%%\begin{eqnarray*}
%%    \langle \operator{\mathcal{N}} \rangle &=& L \int \, \rho(\theta) \, d\theta, \\
%%    \langle \operator{\mathcal{E}} - \mu \operator{\mathcal{N}} \rangle &=& L \int \, \left( - \mu + \frac{1}{2} m \theta^2  \right) \rho(\theta) \, d\theta,
%%\end{eqnarray*}
%
%%pour chaque point du diagrame (\ref{fig:diag}) et en faisant variers leur variable conjugué arrive au fluctuation  $\frac{1}{\beta} \left. \frac{\partial \langle \operator{\mathcal{N}} \rangle}{\partial \mu} \right|_T$ et $- \left. \frac{\partial \langle \operator{\mathcal{E}} - \mu \operator{\mathcal{N}} \rangle}{\partial \beta} \right|_\mu.$
%
%%J'ai repésenter sur le graphe \ref{fig.fluctu.A_com} les different résultat eu acec des deux methode de calcule de fluctuation. Ils ont bien égaux numériquement. %Pour de petite valeur de la temperature les deux méthode donne des 
%%On veux comparer les deux calcule de simulation. On commence par fixer la densité spatial de particule $n$ et on fais fais fluctuer la temperature $T$ et on calcule les fluctuation de densité de rapidité (voir Fig \ref{fig:diag_fig}) 
%
%
%
%%Et on peux 
%
%
%%\begin{figure}[H]
%%%	\centering 
%%	\includegraphics[width=1\textwidth]{Figures/fluctuations_plot_log_gamma=-1.342.png}	
%%	\includegraphics[width=1\textwidth]{Figures/fluctuations_relativ_plot_log_gamma=-1.342.png}	
%%	\captionsetup{skip=10pt} % Ajoute de l’espace après la légende
%%	\label{fig.fluctu.A_com}
%%\end{figure}
%
%
%
%%\includegraphics[width=1\textwidth]{Figures/test}
%
%%\begin{aff}
%%Donc une a l'ordre un en $\delta \theta (\operator{A}^{(0)})^{-1} %\operator{V}$ 
%
%%\begin{eqnarray*}
%%	\langle \delta \Pi ( \theta) \delta \Pi ( \theta') \rangle & = &  ( (\Pi^c_s - \Pi^c)\Pi^c/\Pi^c_s ) ( \theta ) \delta_{\theta, \theta'}/\delta \theta + \mathscr{F}(\theta , \theta' ) ,	
%%\end{eqnarray*}
%
%%avec 
%
%%\begin{eqnarray*}
%%	\mathscr{F}(\theta , \theta' ) & = & \left [ (\Pi^c_s - \Pi^c )( \theta)  +  (\Pi^c_s - \Pi^c ) ( \theta' )\right ] \frac{\Pi^c}{\Pi^c_s}(\theta)\frac{\Pi^c}{\Pi^c_s}(\theta') \frac{ \Delta( \theta'- \theta )}{ 2 \pi }\\
%%	&&  - \left [ (\Pi^c_s - \Pi^c )( \theta)   (\Pi^c_s - \Pi^c ) ( \theta' )\right ] \frac{\Pi^c}{\Pi^c_s}(\theta)\frac{\Pi^c}{\Pi^c_s}(\theta')\int d\theta'' \left (   \frac{ \Pi^c/\Pi^c_s}{\Pi^c_s - \Pi^c} \right )(\theta'') \frac{\Delta(\theta''- \theta)}{2 \pi}\frac{\Delta(\theta''- \theta')}{2 \pi}  	
%%\end{eqnarray*}
%%\end{aff}
%
%
%
%
%}
%
%
%%\subsection{Approximation des fluctuation de $\rho$}
%
%%On est confient sur notre formule des fluctuation de $rho$. Mais là pour l'instant on a une formule analytique pour l'inverce des fluctuation. On aimerais avoir une formule analytique pour les fluctuation. On vas cherche une approxiamation. En voyant la forme de $\operator{A}$  l'inverce des fluctuaution (ref ??) il est tantant de d'aplique une théorie des perturtion. D'apres Neuman : 
%%\begin{eqnarray*}
%%	\operator{A}^{-1} & = & \sum_{k = 0 } (- \delta \theta )^{k} \left ( \left (  \operator{A}^{(0)} \right ) ^{-1}	 \operator{V} \right )^k  \operator{A}^{(0)} 
%%\end{eqnarray*}
%%avec $ \left \Vert  \delta \theta  \left (  \operator{A}^{(0)} \right ) ^{-1}	 \operator{V} \right  \Vert  < 1 $. Pour satisfère ce critais on peut se dire d'ajuster $\delta \theta$, puis que l'on veux de faire tendre $\delta \theta$ vers 0. Mais $\delta \theta  \left (  \operator{A}^{(0)} \right ) ^{-1}	 \operator{V}$ est indépendant de $\delta \theta$ et sa norme est superieur de 1 . On on ne peut pas utiliser de thèorie des pertubation.\\
%
%%Une autre idée est d'écrire 
%
%%\begin{eqnarray*}
%%	\langle \delta \rho(\theta) 	 \delta \rho(\theta') \rangle & = & \operator{B}_{\theta , \theta'} 
%%\end{eqnarray*}
%
%%avec $\operator{B}$ une matrice $2 \times 2$.\\
%
%%Si on note $E$ l'espace où se trouve les $\delta \rho (\theta)$ , et $F$ est sous espace de $F$ tel que $E= F \oplus F^\perp $ de sorte que l'on peut écrire la matrice $\operator{A}$ par bloc 
%%\begin{eqnarray*}
%%	\operator{A} & = & \left (  \begin{array}{cc}\operator{A}_{ \vert F } & \operator{A}_{ \vert F , F^\perp } \\ \operator{A}_{ \vert F^\perp  , F } & \operator{A}_{ \vert F^\perp  } \end{array}\right ) 	
%%\end{eqnarray*}
%
%%De plus $\operator{A}$ est inversible et symétrique donc d'apres le complement de Schur 
%
%%\begin{eqnarray*}
%%	\left (\operator{A}^{-1} \right )_{\vert F} & =& \left ( \operator{A}_{\vert F } - \operator{A}_{\vert F , F^\perp  } \left ( \operator{A}_{\vert F^\perp } \right )^{-1} \operator{A}_{\vert F^\perp , F  }\right )^{-1} 	,\\
%%	& = & 	\left ( \operator{A}_{\vert F }\right )^{-1} + \left ( \operator{A}_{\vert F }\right )^{-1} \operator{A}_{\vert F , F^\perp }\left (\operator{A}^{-1} \right )_{\vert F^\perp}  \operator{A}_{\vert F^\perp , F } \left ( \operator{A}_{\vert F }\right )^{-1}. 
%%\end{eqnarray*}
%
%%Quite a échanger $F$ et $F^\perp$, on peut écrire de la meme manière $\left (\operator{A}^{-1} \right )_{\vert F^\perp}$ et réinjecter , $\left (\operator{A}^{-1} \right )_{\vert F}$ est un point fixe tel que  
%
%%\begin{eqnarray*}
%%	\left (\operator{A}^{-1} \right )_{\vert F} & = & 	\left ( \operator{A}_{\vert F }\right )^{-1} + \left ( \operator{A}_{\vert F }\right )^{-1} \operator{A}_{\vert F , F^\perp }\left (\operator{A}_{\vert F^\perp} \right )^{-1}  \operator{A}_{\vert F^\perp , F } \left ( \operator{A}_{\vert F }\right )^{-1} + \\
%%	& +  & 	\left ( \operator{A}_{\vert F }\right )^{-1} \operator{A}_{\vert F , F^\perp }\left (\operator{A}_{\vert F^\perp} \right )^{-1}\operator{A}_{\vert F^\perp , F } \left (\operator{A}^{-1} \right )_{\vert F} \operator{A}_{\vert F , F^\perp }  \left (\operator{A}_{\vert F^\perp} \right )^{-1} \operator{A}_{\vert F^\perp , F } \left ( \operator{A}_{\vert F }\right )^{-1}.
%%\end{eqnarray*}
%
%%Et si on réinjecte de maniere iterative il vien que 
%
%%\begin{eqnarray*}
%%	\left (\operator{A}^{-1} \right )_{\vert F} & = & \sum_{k = 1 }^\infty \left [  \left ( \left ( \operator{A}_{\vert F }\right )^{-1} \operator{A}_{\vert F , F^\perp }\left (\operator{A}_{\vert F^\perp} \right )^{-1}\operator{A}_{\vert F^\perp , F } \right )^k  \right .\\
%%	&& \left . \times \left \{  \left ( \operator{A}_{\vert F }\right )^{-1} + \left ( \operator{A}_{\vert F }\right )^{-1} \operator{A}_{\vert F , F^\perp }\left (\operator{A}_{\vert F^\perp} \right )^{-1}  \operator{A}_{\vert F^\perp , F } \left ( \operator{A}_{\vert F }\right )^{-1} \right \} \left (  \operator{A}_{\vert F , F^\perp }  \left (\operator{A}_{\vert F^\perp} \right )^{-1} \operator{A}_{\vert F^\perp , F } \left ( \operator{A}_{\vert F }\right )^{-1} \right )^{k} \right ]  + \\
%%	 & + & \underset{ k \to \infty }{\lim} \left ( \left ( \operator{A}_{\vert F }\right )^{-1} \operator{A}_{\vert F , F^\perp }\left (\operator{A}_{\vert F^\perp} \right )^{-1}\operator{A}_{\vert F^\perp , F } \right )^k \left (\operator{A}^{-1} \right )_{\vert F}  \left (  \operator{A}_{\vert F , F^\perp }  \left (\operator{A}_{\vert F^\perp} \right )^{-1} \operator{A}_{\vert F^\perp , F } \left ( \operator{A}_{\vert F }\right )^{-1} \right )^{k}	 
%%\end{eqnarray*}
%





 







