Considérons l'ensemble canonique et calculons la fonction de partition $Z$ du modèle.

Ici, $H$ est le hamiltonien donné par (1.2) et $T$ est la température.  
L'énergie libre $F$ est donnée par (5.1). Rappelons que nous étudions  
la limite thermodynamique ($L \to \infty, N \to \infty$) tout en conservant  
la densité du gaz fixe :

Dans la limite thermodynamique, les lacunes, les particules et les trous  
(ceux définis dans la section 2, voir (2.29), (2.30)) ont des densités de  
distribution finies $\rho_t(\lambda)$, $\rho_p(\lambda)$ et $\rho_h(\lambda)$  
dans l'espace des moments, qui sont définies comme suit :

Le nombre de lacunes est simplement la somme du nombre de particules  
et de trous :

Par lacunes, nous entendons les positions potentielles, dans l'espace des moments,  
qui peuvent être occupées par des particules ou des trous. Dans la limite thermodynamique,  
la somme dans l'équation (2.31) se transforme en une intégrale impliquant la densité $\rho_p(\lambda)$  
et on a :

Il convient de souligner que nous passons maintenant d'une description microscopique  
du modèle (à l'aide de l'ensemble $\{n_j\}$) à une description macroscopique  
en termes de densités $\rho_p$, $\rho_h$ et $\rho_t$.  
La situation macroscopique donnée, décrite par des valeurs fixées de $\rho_p$, $\rho_h$ et $\rho_t$,  
correspond à de nombreux ensembles d'états microscopiques (les ensembles de nombres $\{n_j\}$).  
En effet, il existe de nombreuses façons de placer $L \rho_p(\lambda) d\lambda$ particules  
dans $L \rho_t(\lambda) d\lambda$ lacunes. Le nombre de possibilités est donné par :

Ce nombre est grand dans la limite thermodynamique.  
En utilisant la formule de Stirling pour l'asymptotique du factoriel, on obtient :

Plus tard, nous verrons que l'énergie et les autres observables du système 
ne dépendent que des variables macroscopiques \( \rho_p \) et \( \rho_h \). La quantité

est l'entropie. Passons maintenant à la fonction de partition (5.1). Elle peut être 
représentée sous la forme

où \( E_N = \sum \lambda^2 \} \) et les moments \( \lambda_j \) sont les solutions des équations de Bethe (2.13). 
Nous allons considérer la thermodynamique du gaz qui est au repos dans son ensemble, 
ainsi l'impulsion totale \( P_N \) (1.29) est égale à zéro ; cela signifie que \( \sum n_j = 0 \). 
En introduisant de nouvelles variables \( n_{j+1 , j} = n_{j+1} - n_j \), nous pouvons réécrire (5.11) :

Dans la limite thermodynamique, l'énergie d'un état

Dans la limite thermodynamique, l'énergie d'un état ne dépend que de la variable macroscopique \( rho_p(\lambda) \) ; cela permet de passer  
de la sommation sur les variables microscopiques dans (5.12) à une intégration par rapport aux variables macroscopiques.  
Pour ce faire, effectuons la substitution

En fait, nous pouvons calculer le rapport entre le nombre de vacantes et le nombre de particules (dans \( d\lambda \)) en termes de variables microscopiques et macroscopiques, puis comparer les résultats.  
La variable \( n_{j+1 , j} \) peut être considérée comme le nombre de vacantes pour la \( j \)-ème particule (c'est-à-dire qu'une seule vacante parmi \( n_{j+1,j} \) est occupée).  
Macroscopiquement, ce rapport est donné par le rapport des densités correspondantes (5.3), (5.5), c'est-à-dire \( \frac{\rho_t(\lambda)}{\rho_p(\lambda)} \).  
Ainsi, la somme dans (5.12) peut être remplacée par une intégrale fonctionnelle.  
Cependant, il faut se rappeler qu'un grand nombre de configurations microscopiques ((5.8) et (5.9)) correspond à des variables macroscopiques données.  
On obtient ainsi la représentation intégrale fonctionnelle de la fonction de partition \( Z_N \) :




