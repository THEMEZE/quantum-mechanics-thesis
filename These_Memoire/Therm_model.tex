Considérons l'ensemble canonique et calculons la fonction de partition $Z$ du modèle.

Ici, $H$ est le hamiltonien donné par (1.2) et $T$ est la température.  
L'énergie libre $F$ est donnée par (5.1). Rappelons que nous étudions  
la limite thermodynamique ($L \to \infty, N \to \infty$) tout en conservant  
la densité du gaz fixe :

Dans la limite thermodynamique, les lacunes, les particules et les trous  
(ceux définis dans la section 2, voir (2.29), (2.30)) ont des densités de  
distribution finies $\rho_t(\lambda)$, $\rho_p(\lambda)$ et $\rho_h(\lambda)$  
dans l'espace des moments, qui sont définies comme suit :

Le nombre de lacunes est simplement la somme du nombre de particules  
et de trous :

Par lacunes, nous entendons les positions potentielles, dans l'espace des moments,  
qui peuvent être occupées par des particules ou des trous. Dans la limite thermodynamique,  
la somme dans l'équation (2.31) se transforme en une intégrale impliquant la densité $\rho_p(\lambda)$  
et on a :
