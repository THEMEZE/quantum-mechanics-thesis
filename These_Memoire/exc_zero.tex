Nous allons d'abord considérer les excitations au-dessus du vide physique dans le secteur de charge physique nulle (c'est-à-dire les excitations où le nombre de particules \( N \) dans l'état excité est identique au nombre de particules dans l'état fondamental). Nous commencerons par des conditions aux limites périodiques (2.13).

L'état fondamental est décrit par un ensemble particulier d'entiers \( n_j \), voir (2.26) et (3.2). Tous les autres ensembles de \( n_j \) (avec la contrainte que \( n_j \neq n_k \)) correspondent à des états excités. Cela constitue une description complète de tous les états excités. Ces excitations sont obtenues en supprimant un certain nombre de particules ayant des moments \( -q < \lambda < q \) de la distribution du vide (c'est-à-dire en créant des trous avec des moments \( \lambda_h \)) et en ajoutant un nombre égal de particules ayant des moments \( \lambda_p > q \).

Nous allons d'abord construire l'état dans lequel une particule de moment \( \lambda_p > q \) diffuse avec un trou de moment \( -q < \lambda_h < q \). La présence simultanée de la particule et du trou modifie les valeurs permises des moments des particules du vide : \( \lambda_j \to \tilde{\lambda}_j \), de sorte que les équations de Bethe pour les particules du vide sont réécrites comme suit.

En soustrayant cette contribution de la distribution du vide (3.2) et en tenant compte du fait que \( \lambda_j - \lambda_j' = \mathcal{O}(1/L) \) et que \( \theta( \lambda + \Delta) - \theta(\lambda) = \mathcal{O}(\Delta) \), on obtient :

En utilisant les équations (2.31), (3.5) et (3.7), on obtient :

On introduit maintenant la "fonction de décalage" \( F \) :

Dans la limite thermodynamique, on peut remplacer la somme dans (4.3) par une intégrale, ce qui donne :
Ainsi, nous sommes en mesure de décrire la polarisation du vide causée par une particule et un trou. Cela permet le calcul des grandeurs observables (énergie, momenta, et matrice de diffusion) pour les excitations au-dessus de l'état fondamental. Ces grandeurs observables sont obtenues en ajoutant les contributions de la polarisation du vide aux quantités "pures" correspondantes. Nous commençons par calculer l'énergie observable \( E \), qui est égale à l'énergie de l'état excité moins l'énergie de l'état fondamental :\\

où \( E_0(\lambda) = \lambda2 - h \). De même, on a pour le moment observable (le moment "pur" est simplement égal à $\lambda$) :

Toutes les excitations dans le secteur à charge nulle peuvent être construites comme un état de diffusion constitué de nombres égaux de particules et de trous. L'énergie et le moment de ces excitations sont égaux à la somme des énergies et des moments des particules et des trous individuels. L'excitation à une particule et un trou construite ci-dessus est un état à deux corps. Dans l'ensemble canonique grand, nous pouvons changer le nombre de particules. Construisons une excitation à une particule avec énergie :

et le moment \( k(p) \) égal à :

(voir (3.7)). Il s'agit d'une excitation topologique (nous devons changer les conditions aux frontières en antipériodiques). La valeur \( \lambda_p \) doit être en dehors de la sphère de Fermi, \( |\lambda_p| > q \), \( \text{Im} \, \lambda_p = 0 \). On peut également construire une autre excitation topologique (trou élémentaire) avec une énergie égale à \( -e(h) \) et un moment égal à :



