Nous allons d'abord considérer les excitations au-dessus du vide physique dans le secteur de charge physique nulle (c'est-à-dire les excitations où le nombre de particules \( N \) dans l'état excité est identique au nombre de particules dans l'état fondamental). Nous commencerons par des conditions aux limites périodiques (2.13).

L'état fondamental est décrit par un ensemble particulier d'entiers \( n_j \), voir (2.26) et (3.2). Tous les autres ensembles de \( n_j \) (avec la contrainte que \( n_j \neq n_k \)) correspondent à des états excités. Cela constitue une description complète de tous les états excités. Ces excitations sont obtenues en supprimant un certain nombre de particules ayant des moments \( -q < p < q \) de la distribution du vide (c'est-à-dire en créant des trous avec des moments \( p_h \)) et en ajoutant un nombre égal de particules ayant des moments \( p_p > q \).

Nous allons d'abord construire l'état dans lequel une particule de moment \( p_p > q \) diffuse avec un trou de moment \( -q < p_h < q \). La présence simultanée de la particule et du trou modifie les valeurs permises des moments des particules du vide : \( p_j \to p_j' \), de sorte que les équations de Bethe pour les particules du vide sont réécrites comme suit.
