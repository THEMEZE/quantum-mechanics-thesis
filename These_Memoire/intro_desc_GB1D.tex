\section*{Pourquoi 1D ?}

{\em Explication classique  avec la un modele chaotique : la Thermalisation en 2D avec example de l'eau qui boue avec comme parametre T , E , N ; modelisation en spher dure ; et le modele ergodique : l'integrabiliter de modele de shere dure dans un expace en 1D , echange de vitesse : distribution de vitesse inchanger}

\section*{Pourquoi 1D Quantique ?}

{\em Le gaz de Bose unidimensionnel avec interaction ponctuelle des particules (la variante quantique de l’équation de Schrödinger non linéaire) est l’un des modèles principauxet les plus importants qui peut être résolu par la méthode de l’Ansatz de Bethe ({ref}). Ce modèle a été minutieusement étudié ({ref}). Nous commencerons par la construction des fonctions propres de l’Hamiltonien dans un volume fini. La construction des fonctions propres de l’Hamiltonien est expliquée dans la section 1. Leur forme explicite et, en particulier, la réductibilité à deux particules, sont des caractéristiques communes des modèles résolubles par la méthode de l’Ansatz de Bethe. Des conditions aux limites périodiques sont imposées à la fonction d’onde dans la section 2 ; les équations deBethepour lesmoments des particules sont introduites et analysées.}