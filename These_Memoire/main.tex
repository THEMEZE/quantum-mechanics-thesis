\input{preamble}

\begin{document}

\includepdf[pages=1]{./figures/official_template_phd_universite-paris_saclay_2/Modele_These_UParisSaclay_2022.pdf}

\printglossary[type=\acronymtype,title=Liste des acronymes]

\frontmatter

\begin{refsection}
\input{chapters/000_intro}
%\printbibliography[heading=bibliography,title={Bibliographie du l'Introduction}]
%\printbibliography[heading=subbibliography,title={Bibliographie du l'Introduction}]
%\printbibliography[heading=subsubbibliography,title={Bibliographie du l'Introduction}]
%\printbibliography[title={Bibliographie du l'Introduction}]
%\printbibliography[heading=subbibliography, section=\therefsection]
\printbibliography[heading=subbibliographyintro,title={Bibliographie de l'introduction}]
\end{refsection}

%\section*{Bibliographie de l'Introduction}
%\printbibliography[heading=none]
%\printbibliography[
%  heading=bibsectionnonum,
%  title={Bibliographie de l'Introduction}
%]

\tableofcontents
\mainmatter

\begin{refsection}
\input{chapters/010_LL_BA}
%\printbibliography[heading=bibintoc,title={Bibliographie du chapitre}]
%\printbibliography[heading=subbibliography, title={Bibliographie du chapitre}]
%\printbibliography[heading=subsubbibliography, title={Bibliographie du chapitre}]
%\printbibliography[title={Bibliographie du chapitre}]
%\printbibliography[heading=subbibliography, section=\therefsection]
\printbibliography[heading=subbibliography,title={Bibliographie du chapitre}]
\end{refsection}

\begin{refsection}
\input{chapters/020_GGE_TBA}
%\printbibliography[heading=bibintoc,title={Bibliographie du chapitre}]
%\printbibliography[heading=subbibliography, title={Bibliographie du chapitre}]
%\printbibliography[heading=subsubbibliography,title={Bibliographie du chapitre}]
%\printbibliography[heading=subbibliography, section=\therefsection]
\printbibliography[heading=subbibliography,title={Bibliographie du chapitre}]
\end{refsection}


\begin{refsection}
\input{chapters/030_GHD}
%\printbibliography[heading=bibintoc,title={Bibliographie du chapitre}]
%\printbibliography[heading=subbibliography, title={Bibliographie du chapitre}]
%\printbibliography[heading=subsubbibliography,title={Bibliographie du chapitre}]
%\printbibliography[heading=subbibliography, section=\therefsection]
\printbibliography[heading=subbibliography,title={Bibliographie du chapitre}]
\end{refsection}

%\input{chapters/030_GHD_1}
%\input{chapters/030_GHD_Isa}
%\input{chapters/97_GHD}

\begin{refsection}
\input{chapters/040_GGE_Fluctuation}
%\printbibliography[heading=bibintoc,title={Bibliographie du chapitre}]
%\printbibliography[heading=subbibliography, title={Bibliographie du chapitre}]
%\printbibliography[heading=subsubbibliography,title={Bibliographie du chapitre}]
%\printbibliography[heading=subbibliography, section=\therefsection]
\printbibliography[heading=subbibliography,title={Bibliographie du chapitre}]
\end{refsection}

\begin{refsection}
\input{chapters/050_Disp_Exp}
%\printbibliography[heading=bibintoc,title={Bibliographie du chapitre}]
%\printbibliography[heading=subbibliography, title={Bibliographie du chapitre}]
%\printbibliography[heading=subsubbibliography,title={Bibliographie du chapitre}]
%\printbibliography[heading=subbibliography, section=\therefsection]
\printbibliography[heading=subbibliography,title={Bibliographie du chapitre}]
\end{refsection}

\begin{refsection}
\input{chapters/060_Bipart}
%\printbibliography[heading=bibintoc,title={Bibliographie du chapitre}]
%\printbibliography[heading=subbibliography, title={Bibliographie du chapitre}]
%\printbibliography[heading=subsubbibliography,title={Bibliographie du chapitre}]
%\printbibliography[heading=subbibliography, section=\therefsection]
\printbibliography[heading=subbibliography,title={Bibliographie du chapitre}]
\end{refsection}

\begin{refsection}
\input{chapters/070_Dipolaire}
%\printbibliography[heading=bibintoc,title={Bibliographie du chapitre}]
%\printbibliography[heading=subbibliography, title={Bibliographie du chapitre}]
%\printbibliography[heading=subsubbibliography,title={Bibliographie du chapitre}]
%\printbibliography[heading=subbibliography, section=\therefsection]
\printbibliography[heading=subbibliography,title={Bibliographie du chapitre}]
\end{refsection}

\input{chapters/080_conclusion}



\appendix
%\input{chapters/99_annexes}
\input{chapters/011_PH}
\input{chapters/031_Euler}
%\input{chapters/032_Dressing}

\begin{refsection}
\input{chapters/041_Jerome}
%\printbibliography[heading=bibliography,title={Bibliographie de l'annex}]
%\printbibliography[heading=subbibliography,title={Bibliographie de l'annex}]
%\printbibliography[heading=subsubbibliography,title={Bibliographie de l'annex}]
%\printbibliography[heading=annexbibliography,section=\therefsection]
\printbibliography[heading=annexbibliography,title={Bibliographie de l'annexe}]
\end{refsection}


\input{chapters/051_scaling}
%
%

\begin{refsection}
\input{chapters/071_Starck}
%\printbibliography[heading=bibliography,title={Bibliographie de l'annex}]
%\printbibliography[heading=subbibliography,title={Bibliographie de l'annex}]
%\printbibliography[heading=subsubbibliography,title={Bibliographie de l'annex}]
%\printbibliography[heading=annexbibliography,section=\therefsection]
\printbibliography[heading=annexbibliography,title={Bibliographie de l'annexe}]
\end{refsection}


\input{chapters/072_no_tensor}


\cleardoublepage
%\bibliographystyle{abbrv}
%\bibliography{thesis}
%\printbibliography
% Bibliographie générale
\nocite{*}
\printbibliography[title={Bibliographie générale}]
%\printbibliography[title={Bibliographie générale}, resetnumbers=false, notcategory=cited]


% Fin de ton manuscrit
\cleardoublepage

% 4e de couverture (page paire)
\includepdf[pages=2,pagecommand={}]{./figures/official_template_phd_universite-paris_saclay_2/Modele_These_UParisSaclay_2022.pdf}



\end{document}

%| Style     | Description                                                             |
%| --------- | ----------------------------------------------------------------------- |
%| `plain`   | Tri alphabétique, numérotation croissante                               |
%| `unsrt`   | Même que `plain` mais sans tri, respecte l’ordre d’apparition           |
%| `abbrv`   | Comme `plain` mais avec prénoms et noms abrégés                         |
%| `alpha`   | Les références sont étiquetées par une combinaison du nom et de l’année |
%| `apalike` | Style APA simplifié                                                     |
%| `ieeetr`  | Style IEEE, tri par ordre d’apparition                                  |
%| `siam`    | Style SIAM (mathématiques appliquées)                                   |
%| `acm`     | Style ACM (informatique)                                                |
%


