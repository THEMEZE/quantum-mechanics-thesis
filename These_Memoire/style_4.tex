% !TEX encoding = IsoLatin

\documentclass[10pt, titlepage]{book} % Taille de base des caractères (12pt recommandée pour lecture)

% -------------------------------------
% Encodage et langue
% -------------------------------------
\usepackage[utf8]{inputenc}
\usepackage[T1]{fontenc}
\usepackage[french]{babel}

% -------------------------------------
% Marges et dimensions
% -------------------------------------
\usepackage[a4paper, top=1.0cm, bottom=1.0cm, left=1.0cm, right=1.0cm]{geometry} 
% Ajuste ici les marges selon tes préférences

% -------------------------------------
% Interligne
% -------------------------------------
\usepackage{setspace}
%\onehalfspacing  % Interligne 1.5 
%\doublespacing %(utilise \doublespacing pour double interligne)

% -------------------------------------
% Police (facultatif)
% -------------------------------------
%\usepackage{mathptmx} % Police Times (ancienne)
%\usepackage{libertine} % Police élégante
%\usepackage{newtxtext,newtxmath} % Times moderne pour texte et maths

% -------------------------------------
% Paquets utiles
% -------------------------------------
\usepackage{amsmath, amssymb, amsthm}
\usepackage{graphicx}
\usepackage{hyperref}
\usepackage{xcolor}
\usepackage{braket}
\usepackage{tikz}
\usepackage{pgfplots}
\usepackage{float}
\usepackage{enumitem}
\usepackage{caption}
\usepackage{subcaption}
\usepackage{algorithm2e}
\usepackage{cancel}
\usepackage{bm}
\usepackage{listings}
\usepackage{pdfpages}
\usepackage{mdframed}
\usepackage{braket}
\usepackage{stmaryrd} 
\usetikzlibrary {datavisualization}
\usetikzlibrary {arrows.meta,bending,positioning}
\usetikzlibrary {datavisualization.formats.functions}
%PREAMBULE pour schÃéma
\usepackage{pgfplots}
\usepackage{tikz}
\usepackage[european resistor, european voltage, european current]{circuitikz}
\usetikzlibrary{arrows,shapes,positioning}
\usetikzlibrary{decorations.markings,decorations.pathmorphing,
decorations.pathreplacing}
\usetikzlibrary{calc,patterns,shapes.geometric}
\usepackage{anyfontsize}


% -------------------------------------
% Pour les chapitres
% -------------------------------------
\usepackage[Glenn]{fncychap} % Style de chapitres

% -------------------------------------
% Pour les mini-tables des matières
% -------------------------------------
\usepackage{minitoc}
\dominitoc

% -------------------------------------
% Largeur du texte (évite de le redéfinir si tu utilises geometry)
% -------------------------------------
%\setlength\textwidth{20.5cm}
%\setlength\textheight{22cm}

% -------------------------------------
% Optionnel : si tu veux jouer avec les marges manuellement
% -------------------------------------
% \setlength\topmargin{-1cm}
% \setlength\evensidemargin{-2cm}
% \setlength\oddsidemargin{\evensidemargin}





% Figures flottantes:
% fraction maximale d'une page pouvant etre occupe par une figure:
\renewcommand{\topfraction}{0.8}
% fraction minimale d'une page reservee pour le texte:
\renewcommand{\textfraction}{0.2}
% fraction minimale d'occupation de la page par une figure pleine page:
\renewcommand{\floatpagefraction}{0.7}

%%%%%%%%%%%%%%%%%%%%%%%%%%%%%%%%%%%%%%%%
%         D\'ecoupage des mots           %
%%%%%%%%%%%%%%%%%%%%%%%%%%%%%%%%%%%%%%%%
\hyphenation{}

%%%%%%%%%%%%%%%%%%%%%%%%%%%%%%%%%%%%%%%%
%%%%  Th\'eor\`emes, d\'efinitions, etc.
%%%%%%%%%%%%%%%%%%%%%%%%%%%%%%%%%%%%%%%%


% Il y a diffÃérents types d'ÃénoncÃés qui mÃéritent un environnement spÃécifique, voici une liste assez exhaustive.
\theoremstyle{plain}
    \newtheorem{Theo}{Th\'eor\`eme}[section] %compteur commençant par le numÃéro de la section (on pourrait aussi faire commencer par le numÃéro de la sous-section - remplacer "section" par "subsection")
    \newtheorem{Prop}[Theo]{Proposition}        %mÃême compteur que pour les thÃéorÃèmes
    \newtheorem{Prob}[Theo]{Probl\`eme}        %idem
    \newtheorem{Lemm}[Theo]{Lemme}            %etc...
    \newtheorem{Coro}[Theo]{Corollaire}
    \newtheorem{Propr}[Theo]{Propri\'et\'e}
    \newtheorem{Conj}[Theo]{ Conjecture}
    \newtheorem{Aff}[Theo]{Affirmation}

    \newtheorem{TheoPrinc}{Th\'eor\`eme}     %compteur spÃécifique pour les thÃéorÃèmes les plus importants du papier
        
\theoremstyle{definition}
    \newtheorem{Defi}[Theo]{D\'efinition}
    \newtheorem{Exem}[Theo]{Exemple}
    \newtheorem{Nota}[Theo]{\Large Notation}

\theoremstyle{remark}
    \newtheorem{Rema}[Theo]{Remarque}
    \newtheorem{NB}[Theo]{N.B.}
    \newtheorem{Comm}[Theo]{Commentaire}
    \newtheorem{question}[Theo]{$\ast$ Question}
    \newtheorem{exer}[Theo]{Exercice}
    \newtheorem{Consequence}[Theo]{Conséquence}
    \newtheorem{Rap}[Theo]{Rappel}
    \newtheorem*{Merci}{Remerciements}
    
\usepackage{mdframed}

\mdfdefinestyle{propstyle}{%
linecolor=black,linewidth=2pt,%
hidealllines=true,
frametitlerule=true,%
frametitlebackgroundcolor=gray!20,
backgroundcolor=gray!10!white,
roundcorner=5pt,
innertopmargin=\topskip,
}

%\mdtheorem[style=propstyle]{prop}{Property}[chapter]
\mdtheorem[style=propstyle]{lemma}[prop]{Lemma}
\mdtheorem[style=propstyle]{TheoPrinc}{Th\'eor\`eme}[chapter]

% Définition d'un style personnalisé pour les Affirmations
\mdfdefinestyle{affirmestyle}{%
    linecolor=gray, % Couleur de la bordure
    linewidth=1pt, % Épaisseur de la bordure
    backgroundcolor=gray!10, % Couleur de fond (gris clair)
    roundcorner=5pt, % Coins arrondis
    innertopmargin=0pt, % Marge intérieure au-dessus du cadre
    innerbottommargin=10pt, % Marge intérieure en-dessous du cadre
    innerleftmargin=10pt, % Marge intérieure à gauche
    innerrightmargin=10pt, % Marge intérieure à droite
    skipabove=10pt, % Espace au-dessus du cadre
    skipbelow=10pt % Espace en-dessous du cadre
}

% Définition de l'environnement Affirmation
\theoremstyle{definition} % Style de théorème pour les affirmations
\newmdtheoremenv[style=affirmestyle]{aff}{Point clé n$^{\circ}$} % Environnement Affirmation avec le style personnalisé
    



\usepackage{scalerel}
\usepackage{xcolor}
\usepackage{stackengine}
\usepgflibrary {shadings}


\usetikzlibrary {decorations.pathmorphing}

\newcommand\dangersign[1]{%
    \renewcommand\stacktype{L}%
    \scaleto{\stackon[1.3pt]{\color{red}$\triangle$}{\tiny !}}{#1}%
}

\usepackage{tikz}
\tikzset{every picture/.style={execute at begin picture={\shorthandoff{:;!?};}}}
\tikzstyle{every picture}+=[remember picture]
\tikzstyle{na} = [shape=rectangle,inner sep=0pt]

% Commandes pour les flèches textuelles
\newcommand{\ptFleche}[2]{        % Déclaration d'une extrémité de flèche
    \tikz[baseline=(#1.base)]\node[na](#1){#2};
  }
%\newcommand{\Fleche}[5][thick]{    % Dessin de la flèche
%    \begin{tikzpicture}[overlay]
%        \path[->,#1](#2) edge [out=#4, in=#5] (#3);
%    \end{tikzpicture}
%  }
  
% \newcommand{\Flecheprim}[5][thick]{    % Dessin de la flèche
%    \begin{tikzpicture}[overlay]
%        \path[->,#1](#2) edge [out=#4, in=#5] (#3);
%    \end{tikzpicture}
%  }
%
\usepackage{marvosym}
\usepackage{changepage}

\usepackage{minitoc}
\usepackage{tocloft}
%\renewcommand{\cfttoctitle}{\hspace{-2em}}
% Nastaveni obsahu
% Nastaveni obsahu

\usepackage{imakeidx}
\usepackage{fancyhdr}

%\usepackage{makeidx}
\makeindex[intoc=true]
\makeindex[name=pers, title=Index of person names, intoc=true]

\usepackage{xcolor}
\definecolor{linkcolor}{RGB}{0,0,180}
\PassOptionsToPackage{
    colorlinks=true,
    linkcolor=linkcolor,
    citecolor=linkcolor,
    urlcolor=linkcolor
}{hyperref}
\usepackage{hyperref}

%%%%%%%%%%%%%%%%%%%%%
%\definecolor{linkcolor}{RGB}{0,0,180}
\usepackage{titlesec}

% Appliquer la couleur à tous les niveaux de titre
\titleformat{\section}{\normalfont\color{linkcolor!90!black}\Large\bfseries}{\thesection}{1em}{}
\titleformat{\subsection}{\normalfont\color{linkcolor!70!black}\large\bfseries}{\thesubsection}{1em}{}
\titleformat{\subsubsection}{\normalfont\color{linkcolor!50!black}\normalsize\bfseries}{\thesubsubsection}{1em}{}
\titleformat{\paragraph}[runin]{\normalfont\color{linkcolor!30!black}\bfseries}{\theparagraph}{1em}{}
\titleformat{\subparagraph}[runin]{\normalfont\color{linkcolor!10!black}\itshape}{\thesubparagraph}{1em}{}
%%%%%%%%%%%%%%%%%%%%%%%

%%Couleurs dans la table des matières

\usepackage{tocloft}

% Modifier la couleur des entrées de la TOC
\renewcommand{\cftsecfont}{\color{linkcolor!90!black}}
\renewcommand{\cftsubsecfont}{\color{linkcolor!70!black}}
\renewcommand{\cftsubsubsecfont}{\color{linkcolor!50!black}}
\renewcommand{\cftparafont}{\color{linkcolor!30!black}}
\renewcommand{\cftsubparafont}{\color{linkcolor!10!black}}
%%%%%%%%%%%%%%%%%%%%%%%%%%%%%
% Reglages:
%
%\pagestyle{fancyplain}
%\addtolength{\headwidth}{\marginparsep}
%\addtolength{\headwidth}{\marginparwidth}
%\renewcommand{\chaptermark}[1]{\markboth{#1}{}}
%\renewcommand{\sectionmark}[1]{\markright{\thesection\ #1}}
%\lhead[\fancyplain{}{\bfseries\thepage}]{}
%\rhead[]{\fancyplain{}{\bfseries\thepage}}
%\chead[\fancyplain{}{\bfseries\leftmark}]{\fancyplain{}{\bfseries\rightmark}}
%\cfoot{}
%

%usepackage{titlesec}
% Changer la couleur des paragraphes en rouge par exemple :
%\titleformat{\paragraph}[runin] % ou [block] selon ce que tu veux
%  {\normalfont\color{red}\bfseries}
%  {\theparagraph}{1em}{}

%%% ===== Index principal + index secondaire (noms propres) =====
\makeindex[intoc=true]
\makeindex[name=pers, title=Index des noms propres, intoc=true]

%%% ===== Couleur des liens =====
\definecolor{linkcolor}{RGB}{0,0,180}
\PassOptionsToPackage{
    colorlinks=true,
    linkcolor=linkcolor,
    citecolor=linkcolor,
    urlcolor=linkcolor
}{hyperref}
\usepackage{hyperref}

%%% ===== Réglages hyperref =====
\hypersetup{
  pdftitle={Étude de la dynamique hors équilibre de bosons unidimensionnels},
  pdfsubject={Quantum Physics},
  pdfauthor={Guillaume THEMEZE <guillaume.themeze@gmail.fr>},
  pdfkeywords={LaTeX, quantum, bosons, dynamique},
  colorlinks=true
}

%%% ===== Style des titres (colorés) =====
\titleformat{\chapter}[display]{\normalfont\sffamily\huge\bfseries\color{black}}{\chaptertitlename\ \thechapter}{20pt}{\Huge}
\titleformat{\section}{\normalfont\color{linkcolor!90!black}\Large\bfseries}{\thesection}{1em}{}
\titleformat{\subsection}{\normalfont\color{linkcolor!70!black}\large\bfseries}{\thesubsection}{1em}{}
\titleformat{\subsubsection}{\normalfont\color{linkcolor!50!black}\normalsize\bfseries}{\thesubsubsection}{1em}{}
\titleformat{\paragraph}[runin]{\normalfont\color{linkcolor!30!black}\bfseries}{\theparagraph}{1em}{}
\titleformat{\subparagraph}[runin]{\normalfont\color{linkcolor!10!black}\itshape}{\thesubparagraph}{1em}{}

%%% ===== Couleurs de la table des matières =====
\renewcommand{\cftsecfont}{\color{linkcolor!90!black}}
\renewcommand{\cftsubsecfont}{\color{linkcolor!70!black}}
\renewcommand{\cftsubsubsecfont}{\color{linkcolor!50!black}}
\renewcommand{\cftparafont}{\color{linkcolor!30!black}}
\renewcommand{\cftsubparafont}{\color{linkcolor!10!black}}

%%% ===== En-têtes et pieds de page =====
\pagestyle{fancy}
\fancyhf{}
\setlength{\headheight}{14pt}

\fancyhead[RO,LE]{\thepage}
\fancyhead[LO]{\scshape \nouppercase{\rightmark}}  % Section
\fancyhead[RE]{\scshape \nouppercase{\leftmark}}  % Chapitre
\renewcommand{\headrulewidth}{.4pt}

\usepackage{datetime} % Pour une date personnalisée
\newdateformat{mydate}{\THEDAY~\monthname[\THEMONTH]~\THEYEAR}
\newdateformat{mydatetime}{\THEDAY~\monthname[\THEMONTH]~\THEYEAR~à~\currenttime}
\usepackage[useregional]{datetime2}
%\DTMsetstyle{french} % ou autre style
%\DTMsetup{showtimezone=false}

\fancyfoot[L]{Thèse}
%\fancyfoot[R]{Paris, \mydatetime\today{} -- Période 2022--2025}
\fancyfoot[R]{Paris, \DTMnow -- Période 2022--2025}
%\fancyfoot[R]{Paris, le \DTMdate\today{} à \DTMcurrenttime -- Période 2022--2025}
\renewcommand{\footrulewidth}{.4pt}

% Supprimer les numéros sur la première page de chaque chapitre
\makeatletter
\let\ps@plain=\ps@empty
\makeatother

%%% ===== Réglages des titres de sections dans les en-têtes =====
\renewcommand{\chaptermark}[1]{\markboth{#1}{}}
\renewcommand{\sectionmark}[1]{\markright{\thesection\ #1}}

%%% ===== Notes de bas de page à la française =====
\usepackage[french]{babel}
%\usepackage[frenchfootnotes]{french}
%\FrenchFootnotes
%\AddThinSpaceBeforeFootnotes

%%%%%%%%%%%%%%%%%%%%%%%%%%%%%%%%%%
\newcommand{\operatorvec}[1]{\vec{{\bm{#1}}}} % pour les operateur
\newcommand{\operator}[1]{\hat{\bm{#1}}} % pour les operaeur vecteur
\newcommand{\operatormat}[1]{\operatorname{#1}} % pour les operaeur vecteur
\newcommand{\operatortilde}[1]{\tilde{\bm{#1}}} % pour les opetateur avec un tilde
\newcommand{\operatortildevec}[1]{\tilde{\bm{#1}}}% pour les opetateur avec un tilde et vecteur
%%%%%%%%%%%%%%%%%%%%%%%%%%%%%%%%%%

