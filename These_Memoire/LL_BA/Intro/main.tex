\section*{Pourquoi en 1D ?}

{\em 

Explication classique à l’aide d’un modèle chaotique : la thermalisation en 2D, illustrée par l’exemple de l’eau en ébullition, avec comme paramètres \(T\), \(E\), \(N\). Modélisation par des sphères dures et introduction du modèle ergodique : en 1D, l’intégrabilité du modèle de sphères dures dans un espace réduit entraîne un simple échange de vitesses, sans modifier la distribution des vitesses.

}

\section*{Pourquoi en 1D quantique ?}

{\em 
Le gaz de Bose unidimensionnel avec interactions ponctuelles (la version quantique de l’équation de Schrödinger non linéaire) est l’un des modèles intégrables les plus fondamentaux, pouvant être résolu par la méthode de l’Ansatz de Bethe ({ref}). Ce modèle a fait l’objet d’études approfondies ({ref}).  

Après avoir décrit le modèle de Lieb-Liniger et analysé ses asymptotiques ainsi que les théories linéarisées (Gross-Pitaevskii et Bogoliubov) dans le chapitre 1, nous poursuivons par la construction des fonctions propres de l’Hamiltonien dans un volume fini.  

Cette construction, détaillée dans le chapitre 2, met en évidence la forme explicite des fonctions propres et leur réductibilité au cas à deux particules, une caractéristique commune des modèles résolubles par l’Ansatz de Bethe. Enfin, dans la dernière section, nous imposons des conditions aux limites périodiques à %la fonction d’onde, ce qui nous conduit aux équations de Bethe pour les moments des particules, lesquelles seront introduites et analysées.
}


Dans ce chapitre, nous introduisons progressivement le modèle de Lieb-Liniger et l'Ansatz de Bethe, outils fondamentaux pour décrire un gaz de bosons unidimensionnel avec interactions delta. L'objectif est d'accompagner pas à pas le lecteur depuis la formulation du problème quantique en champ de bosons jusqu'aux solutions exactes obtenues par l'Ansatz de Bethe.

Nous commençons par écrire l'équation du champ de bosons, exprimée à l’aide des opérateurs de création et d’annihilation en représentation de position. Pour des raisons pédagogiques, nous abordons d’abord le cas d’une seule particule, sans interaction. Cela permet d’introduire naturellement les états de position et leur évolution sous l’action du Hamiltonien libre.

Ensuite, nous étudions le cas de deux particules, cette fois en tenant compte de l’interaction locale. Cela nous amène à considérer les états de position dans le cas général, y compris lorsque les deux particules peuvent occuper la même position. Cette situation, bien plus subtile qu’il n’y paraît, met en évidence la complexité introduite par l’interaction, et justifie que l’on commence par analyser les configurations où les particules sont à des positions distinctes.

En passant au référentiel du centre de masse, le problème à deux corps avec interaction devient équivalent à un problème à une seule particule en interaction avec une barrière delta au centre. Cette reformulation permet d’interpréter l’effet de l’interaction comme une condition de raccord sur la fonction d’onde, tout en respectant la symétrie bosonique.

Nous revenons ensuite aux coordonnées du laboratoire afin d’introduire naturellement la forme des solutions imposée par l’Ansatz de Bethe. Cela nous conduit aux équations dites de Bethe, qui relient les quasimoments des particules à travers des conditions de périodicité modifiées par l’interaction.

Une fois les notations bien établies, nous généralisons le raisonnement au cas de \(N\) particules, pour obtenir l’Hamiltonien de Lieb-Liniger complet ainsi que la forme générale de l’Ansatz de Bethe. Les solutions ainsi construites permettent non seulement de déterminer le spectre de l’Hamiltonien, mais aussi de calculer des observables physiques importantes, telles que l’impulsion totale ou le nombre de particules.

Enfin, nous introduisons la notion de distribution de rapidité, outil essentiel dans l’étude des états d’énergie minimale (états fondamentaux) et dans la description thermodynamique du système. Ce cadre servira de base aux développements ultérieurs sur les gaz intégrables à température finie et les états stationnaires après quench quantique.

