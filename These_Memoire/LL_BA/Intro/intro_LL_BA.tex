\section*{Pourquoi en 1D ?}

{\em 

Explication classique à l’aide d’un modèle chaotique : la thermalisation en 2D, illustrée par l’exemple de l’eau en ébullition, avec comme paramètres \(T\), \(E\), \(N\). Modélisation par des sphères dures et introduction du modèle ergodique : en 1D, l’intégrabilité du modèle de sphères dures dans un espace réduit entraîne un simple échange de vitesses, sans modifier la distribution des vitesses.

}

\section*{Pourquoi en 1D quantique ?}

{\em 
Le gaz de Bose unidimensionnel avec interactions ponctuelles (la version quantique de l’équation de Schrödinger non linéaire) est l’un des modèles intégrables les plus fondamentaux, pouvant être résolu par la méthode de l’Ansatz de Bethe ({ref}). Ce modèle a fait l’objet d’études approfondies ({ref}).  

Après avoir décrit le modèle de Lieb-Liniger et analysé ses asymptotiques ainsi que les théories linéarisées (Gross-Pitaevskii et Bogoliubov) dans le chapitre 1, nous poursuivons par la construction des fonctions propres de l’Hamiltonien dans un volume fini.  

Cette construction, détaillée dans le chapitre 2, met en évidence la forme explicite des fonctions propres et leur réductibilité au cas à deux particules, une caractéristique commune des modèles résolubles par l’Ansatz de Bethe. Enfin, dans la dernière section, nous imposons des conditions aux limites périodiques à %la fonction d’onde, ce qui nous conduit aux équations de Bethe pour les moments des particules, lesquelles seront introduites et analysées.
}  
