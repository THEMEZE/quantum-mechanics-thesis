Dans le domaine $z_1 < z_2 < \cdots < z_N$, la fonction d’onde pour un état de Bethe à $N$ particules s’écrit ({\color{blue}Gaudin 2014}, {\color{blue}Korepin et al. 1997}, {\color{blue}Lieb et Liniger 1963}) :
\begin{eqnarray}
	\varphi_{\{\theta_a\}} ( z_1 , \cdots , z_N ) & = & \langle 0 \vert \Psi ( z_1 ) \cdots \Psi (z_N ) \vert \{ \theta_a \} \rangle \notag\\
	& \propto & \sum_\sigma (-1)^{|\sigma|} \left( \prod_{1 \leq a < b \leq N} (\theta_{\sigma(b)} - \theta_{\sigma(a)} - i c) \right) e^{i \sum_{j=1}^{N} z_j \theta_{\sigma(j)}},\label{eq:I-2-17}
\end{eqnarray}
où la somme s'étend sur toutes les permutations $\sigma$ de $\{1,\dots,N\}$. Le facteur $(-1)^{|\sigma|}$ est la signature de la permutation, et les amplitudes dépendent des différences de quasi-moments $\theta_j$ ainsi que du couplage $c$.

Cette fonction d’onde est ensuite étendue par symétrie aux autres domaines du type $z_{\pi(1)} < z_{\pi(2)} < \cdots < z_{\pi(N)}$ via des propriétés d’échange symétriques.
