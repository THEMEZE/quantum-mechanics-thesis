\subsubsection{Opérateurs nombre de particules $\operator{Q}$ et moment $\operator{P}$}
L'opérateur du nombre de particules $\operator{Q}$ et l'opérateur de moment $\operator{P}$ sont définis comme 
\begin{eqnarray}
	\operator{Q} & = & \int \operator{\Psi}^\dag (x) \operator{\Psi} (x) \, d x \\
	\operator{P} & = & - \frac{i}2 \int \left \{  \operator{\Psi}^\dag(x) \operator{\partial}_x \operator{\Psi}(x) - \left [ \operator{\partial}_x \operator{\Psi}^\dag(x)\right ] \operator{\Psi}(x)\right \} dx \label{eq.1.7}
\end{eqnarray}

\subsubsection{Propriétés}

Ce sont des opérateurs hermitiens et ils constituent des intégrales du mouvement

\begin{eqnarray}
	[ \operator{H} , \operator{Q} ] = 	[ \operator{H} , \operator{P} ] = O. 
\end{eqnarray}

Il convient de noter que dans le chapitre 2 , nous allons construire un nombre infini d'intégrales du mouvement.

\subsubsection{L’état propre}

Les fonctions propres $\vert \{\theta_a\} \rangle$ de l'Hamiltonien $\operator{H}$, le sont aussi pour $\operator{Q}$ et $\operator{P}$ :

Ici, $\vert \{\theta_a\} \rangle$ est une fonction symétrique de toutes les variables $z_j$. L'équation aux valeurs propres 

\begin{eqnarray}
	\operator{H} |\psi_N\rangle = E_N |\psi_N\rangle, \quad \operator{P} |\psi_N\rangle = p_N |\psi_N\rangle, \quad \operator{Q} |\psi_n\rangle = q_N |\psi_N\rangle,	
\end{eqnarray}

conduit au fait que $\chi_N$ est une fonction propre à la fois de l'Hamiltonien mécanique quantique $\operator{\mathcal{H}}_N$ et de l'opérateur du moment mécanique quantique $\operator{\mathcal{P}}_N$ :

\begin{eqnarray}
	&& \left \{ \begin{array}{rcl} \operator{ \mathcal{H}}_N & = & \displaystyle \sum_{j=1}^N   - \operator{\partial}_{z_j}^2 + 2c \sum_{1 \leq k < j \leq N } \operator{\delta} ( z_j - z_k) \\ \operator{\mathcal{P}}_N & = & \displaystyle \sum_{j=1}^N -i \operator{\partial}_{z_j}\end{array} \right .  \label{eq.1.11}\\
	%&& ~~~\operator{ \mathcal{H}}_N \chi_N ~= ~	E_N \chi_N .\label{eq.1.12}
	&& \left \{ \begin{array}{rcl}\operator{ \mathcal{H}}_N \chi_N &=&	E_N \chi_N \\ \operator{ \mathcal{P}}_N \chi_N &=&	p_N \chi_N\end{array} \right . .\label{eq.1.12}
\end{eqnarray}