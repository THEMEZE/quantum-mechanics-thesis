\subsubsection{Opérateurs nombre de particules $\operator{Q}$ et moment $\operator{P}$}
L'opérateur du nombre de particules $\operator{Q}$ et l'opérateur de moment $\operator{P}$ sont définis comme 
\begin{eqnarray}
	\operator{Q} & = & \int \operator{\Psi}^\dag (x) \operator{\Psi} (x) \, d x \\
	\operator{P} & = & - \frac{i}2 \int \left \{  \operator{\Psi}^\dag(x) \operator{\partial}_x \operator{\Psi}(x) - \left [ \operator{\partial}_x \operator{\Psi}^\dag(x)\right ] \operator{\Psi}(x)\right \} dx \label{eq.1.7}
\end{eqnarray}

\subsubsection{Propriétés}

Ce sont des opérateurs hermitiens et ils constituent des intégrales du mouvement

\begin{eqnarray}
	[ \operator{H} , \operator{Q} ] = 	[ \operator{H} , \operator{P} ] = O. 
\end{eqnarray}

Il convient de noter que dans le chapitre 2 , nous allons construire un nombre infini d'intégrales du mouvement.

\subsubsection{L’état propre}

Les fonctions propres $\vert \{\theta_a\}\rangle$ de l'Hamiltonien $\operator{H}$, le sont aussi pour $\operator{Q}$ et $\operator{P}$ :

%Ici, $\vert \{\theta_a\}\rangle$ est une fonction symétrique de toutes les variables $z_j$. 
L'équation aux valeurs propres 

\begin{eqnarray}
	\operator{H} \vert \{\theta_a\} \rangle = E \vert \{\theta_a\}\rangle, \quad \operator{P} \vert \{\theta_a\}\rangle = P \vert \{\theta_a\}\rangle, \quad \operator{Q} \vert \{\theta_a\}\rangle = N \vert \{\theta_a\}\rangle,	
\end{eqnarray}

conduit au fait que $\vert \{\theta_a\} \rangle$ est une fonction propre à la fois de l'Hamiltonien mécanique quantique $\operator{\mathcal{H}}_N$, de l'opérateur du nombre de particule $\operator{\mathcal{N}}$ et de l'opérateur du moment mécanique quantique $\operator{\mathcal{P}}_N$ :

\begin{eqnarray}
	\operator{ \mathcal{N}} ~ = ~ \sum_{k = 1}^N 1,~\operator{ \mathcal{P}}_N  ~=~- i\sum_{k = 1}^N \operator{\partial}_{z_k}	
\end{eqnarray}

associés valeurs propres 

\begin{eqnarray}
	N = \sum_{a = 1}^N \theta_a^0 , ~ P = \sum_{a = 1}^N \theta_a^1 ,~E = \frac{1}{2}\sum_{a = 1}^N \theta_a^2.	
\end{eqnarray}
