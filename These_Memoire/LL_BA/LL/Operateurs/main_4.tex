\subsection{Opérateurs nombre de particules et moment dans la formulation quantique du gaz de Lieb-Liniger}

Dans cette section, nous nous intéressons aux opérateurs fondamentaux que sont le {\em nombre total de particules} $\operator{Q}$ et le {\em moment total} $\operator{P}$, dans le cadre du gaz de bosons unidimensionnel décrit par l’Hamiltonien de Lieb-Liniger. Après avoir introduit ces opérateurs dans le langage de la seconde quantification, nous montrons qu’ils sont {\em conservés} par la dynamique, et qu’ils admettent les {\em mêmes états propres} que l’Hamiltonien. Nous donnons ensuite leur expression dans la représentation à  $N$ particules, ainsi que la forme explicite de leurs valeurs propres en fonction des {\em rapidités} $\theta_a$ , illustrant la structure polynomiale typique des intégrales du mouvement dans les systèmes intégrables.

\subsubsection{Définition en seconde quantification}

Les opérateurs du nombre total de particules $\operator{Q}$ et du moment total $\operator{P}$ s’écrivent en seconde quantification comme suit :
\begin{eqnarray}
	\operator{Q}  =  \int \operator{\Psi}^\dag (x) \operator{\Psi} (x) \, d x, \quad 
	\operator{P}  =  - \frac{i}2 \int \left \{  \operator{\Psi}^\dag(x) \operator{\partial}_x \operator{\Psi}(x) - \left [ \operator{\partial}_x \operator{\Psi}^\dag(x)\right ] \operator{\Psi}(x)\right \} dx \label{eq.1.7}
\end{eqnarray}
Ces deux opérateurs sont {\em hermitiens}, et représentent des observables physiques fondamentales : le nombre de particules et la quantité de mouvement du système.

\subsubsection{Conservation et commutation}
Ces opérateurs commutent avec l’Hamiltonien $\operator{H}$ du modèle de Lieb-Liniger :
\begin{eqnarray}
[ \operator{H} , \operator{Q} ] = 0, \quad [ \operator{H} , \operator{P} ] = 0.
\end{eqnarray}
Ils constituent ainsi des intégrales du mouvement. Cette propriété est une manifestation de la symétrie translationnelle du système (pour $\operator{P}$) et de la conservation du nombre total de particules (pour $\operator{Q}$).

\begin{mdframed}[
	linewidth=0.5pt, 
	backgroundcolor=gray!5, 
	roundcorner=50pt,	
	innerleftmargin=5pt,
    innerrightmargin=5pt,
    innertopmargin=5pt,
    innerbottommargin=2pt,
    leftmargin=2pt,
    rightmargin=2pt
	]
	Nous verrons au chapitre 2 que cette situation s’étend à une {\bf \em infinité d’intégrales du mouvement} dans les systèmes intégrables, ce qui permettra de construire l’ensemble de Gibbs généralisé (GGE).
\end{mdframed}

\subsubsection{États propres et valeurs propres}
Les états propres $\ket{\{\theta_a\}}$, construits dans le cadre de la seconde quantification à partir de la solution du modèle de Lieb-Liniger, sont simultanément fonctions propres des opérateurs $\operator{Q}$, $\operator{P}$ et $\operator{H}$ :
\begin{eqnarray}
\operator{Q} \ket{\{\theta_a\}} = N \ket{\{\theta_a\}}, \quad
\operator{P} \ket{\{\theta_a\}} = \left( \sum_{a=1}^N \theta_a \right) \ket{\{\theta_a\}}, \
\operator{H} \ket{\{\theta_a\}} = \left( \frac{1}{2} \sum_{a=1}^N \theta_a^2 \right) \ket{\{\theta_a\}}.
\end{eqnarray}
Autrement dit, les valeurs propres associées à ces trois opérateurs sont données par :
\begin{eqnarray}
N = \sum_{a = 1}^N \theta_a^0, \quad p = \sum_{a = 1}^N \theta_a, \quad e = \frac{1}{2} \sum_{a = 1}^N \theta_a^2.
\end{eqnarray}
Cela illustre que les trois premières intégrales du mouvement du système — nombre, moment, énergie — peuvent être exprimées comme des {\bf \em moments successifs} des rapidités.	

\subsubsection{Forme en première quantification}
En utilisant la représentation en espace de configuration $\{z_a\} \equiv \{z_1 , \cdots , z_N \}$, les opérateurs $\operator{Q}$ et $\operator{P}$ agissent comme suit sur les fonctions d’onde $\varphi_{\{\theta_a\}}(\{z_a\})$ :
\begin{eqnarray}
	\operator{Q}\ket{\{\theta_a\}} =  \sqrt{N!}\int d^Nz \, \operator{\mathcal{N}} \varphi_{\{\theta_a\}}(\{z_a\} )\ket{\{z_a\}}, \, \operator{P}\ket{\{\theta_a\}} =  \sqrt{N!}\int d^Nz \, \operator{\mathcal{P}}_N \varphi_{\{\theta_a\}}(\{z_a\} )\ket{\{z_a\}} 
\end{eqnarray}
où les opérateurs associés agissant sur les fonctions d’onde à $N$ particules sont :
\begin{eqnarray}
	\operator{ \mathcal{N}}  =  \sum_{k = 1}^N 1 = N ,~\operator{ \mathcal{P}}_N  = -i \sum_{k = 1}^N k =- i\sum_{k = 1}^N \operator{\partial}_{z_k}	
\end{eqnarray}

Ces formes découlent directement des règles de commutation canonique (\ref{chap:1:com.1}) et de la définition des opérateurs en seconde quantification (\ref{chap:eq.vide.fock}) (cf. annexes \ref{annex:N.part}).

\subsubsection{Conclusion}
Ainsi, les opérateurs $\operator{Q}$ , $\operator{P}$ et $\operator{H}$ possèdent une structure diagonale commune dans la base des états propres $\ket{\{\theta_a\}}$, révélant la nature intégrable du modèle de Lieb-Liniger. Leurs valeurs propres sont respectivement les 0e, 1er et 2e moments des rapidités. Cette structure permet de généraliser la construction à une hiérarchie complète d’observables conservées, qui seront présentées au chapitre suivant.


















