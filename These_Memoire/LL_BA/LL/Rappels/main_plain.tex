\paragraph{Why Second Quantification ?}

\subparagraph{First Quantification (Single Particle)}
\begin{itemize}[label = $\bullet$]
	\item In first quantification, promote classical observable $x$, $p$ of a single particle to operators  $\operator{x}$ , $\operator{p}$  acting on wavefunctions $\psi(x)$.
\end{itemize}


\subparagraph{The Many-Body Problem}
\begin{itemize}[label = $\bullet$]
	\item When we try to describe $N$ identical particles, we write a wavefunction $\Psi(x_1 , x_2 , \cdots , x_N )$ of $N$ variables.
	\item Enforcing the fact that "particles are identical" means $Psi$ must be symmetric (for bosons) or antisymmetric (for fermions) under swapping any pair of $x_i \leftrightarrow x_j$.
	\item As $N$ grows, these symmetrized wavefunctions becomes hard to deal with.
\end{itemize}

\subparagraph{Second Quantification}
\begin{itemize}[label = $\bullet$]
	\item Instead of labeling each particle by coordinates $(x_1 , x_2 , \cdots )$, we switch to describing "how many particles live in each possible single-particle state."
	\item Think of having a list of available one-particle "modes" (e.g. energy levels. plane-wave states, etc.) We then say: "There are $n_1$ particles in mode $1$ , $n_2$ particles in mode $2$, and so on." 
	\item This viewpoint automatically takes care of (anti)symmetry for us. We only keep track of {bf occupation numbers} $n_i$, not which specific particle is where.	
\end{itemize}

\paragraph{Pourquoi la seconde quantification ?}
\subparagraph{Première quantification (particule unique)}
\begin{itemize}[label = $\bullet$]
\item Dans la première quantification, on promeut les observables classiques $x$ et $p$ d’une particule unique en opérateurs $\hat{x}$ et $\hat{p}$ agissant sur les fonctions d’onde $\psi(x)$.
\end{itemize}

\subparagraph{Le problème à plusieurs corps}
\begin{itemize}[label = $\bullet$]
\item Lorsqu’on cherche à décrire un système de $N$ particules identiques, on introduit une fonction d’onde $\Psi(x_1, x_2, \cdots, x_N)$ dépendant de $N$ variables.
\item Le fait que les particules soient indiscernables impose que $\Psi$ soit symétrique (pour des bosons) ou antisymétrique (pour des fermions) sous l’échange de deux coordonnées $x_i \leftrightarrow x_j$.
\item Lorsque $N$ devient grand, manipuler explicitement ces fonctions d’onde (anti)symétrisées devient rapidement incommode et complexe.
\end{itemize}

\subparagraph{Seconde quantification}
\begin{itemize}[label = $\bullet$]
\item Au lieu d’indexer chaque particule par ses coordonnées $(x_1, x_2, \cdots)$, on adopte un formalisme où l’on décrit combien de particules occupent chaque état à une particule.
\item On considère une base d’états à une particule (par exemple : niveaux d’énergie, états d’onde plane, etc.) et on énumère : "il y a $n_1$ particules dans le mode $1$, $n_2$ dans le mode $2$, etc."
\item Cette approche gère automatiquement les contraintes de symétrie : on ne suit plus chaque particule individuellement, mais uniquement les nombres d’occupation $n_i$.
\end{itemize}

