
\subsubsection{Champs de Bose et commutation canonique}

Le gaz de Bose unidimensionnel est décrit par les champs de Bose quantiques canoniques \( \operator{\Psi}(x, t) \) avec les relations de commutation canoniques à temps égal :


\begin{eqnarray}
	\left . \begin{array}{rcl}
		[ \operator{\Psi}(x, t),  \operator{\Psi}^\dagger(y, t) ]  &=&  \operator{\delta}(x - y) \\
		\left [ \operator{\Psi}(x, t),  \operator{\Psi}(y, t) \right ]   =  [ \operator{\Psi}^\dag(x, t),  \operator{\Psi}^\dag(y, t) ]  &=&  0 
	\end{array} \right . \label{eq.1.1}
\end{eqnarray}

Par la suite, l'argument \( t \) sera en général omis, car toutes les considérations de ce chapitre s'appliquent à un moment fixe dans le temps.