L'hamiltonien (\ref{chap:1:hal.mod}) s'écrit asuus en faisant  une integration par partie :


\begin{eqnarray}
	\operator{H} & = & \int dx \left [ -\left [\operator{\partial}_x^2 \operator{\Psi}^\dag(x)\right ] \operator{\Psi}(x) + c \operator{\Psi}^\dag (x) \operator{\Psi}^\dag (x) \operator{\Psi} (x) \operator{\Psi} (x) \right ] \label{chap:1:hal.mod.2}.
\end{eqnarray}

Agir avec cet opérateur sur la fonction propre (\ref{eq.1.9}) donne 
%\begin{eqnarray}
	%\operator{H}\vert \psi ( \theta_1 , \cdots , \theta_N ) \rangle & = & \left \{ \begin{array}{l}- \frac{1}{ \sqrt{N!}} \int dx \int d^N z \, \chi_N ( z_1 , \cdots , z_N  ~\vert ~ \theta_1 , \cdots , \theta _N ) \,	\left [\operator{\partial}_x^2 \operator{\Psi}^\dag(x)\right ] \operator{\Psi}(x)\,  \operator{\Psi}^\dag(z_1) \cdots \operator{\Psi}^\dag(z_N) \vert 0 \rangle \\ + \\ \frac{c}{ \sqrt{N!}} \int dx \int d^N z \, \chi_N ( z_1 , \cdots , z_N  ~\vert ~ \theta_1 , \cdots , \theta _N ) 	\, \operator{\Psi}^\dag(x)\operator{\Psi}^\dag(x) \operator{\Psi}(x)\operator{\Psi}(x)  \operator{\Psi}^\dag(z_1) \cdots \operator{\Psi}^\dag(z_N) \vert 0 \rangle  \end{array}\right.\label{chap:1:hal.mod.3}
%\end{eqnarray}

\begin{eqnarray}
	& & - \frac{1}{ \sqrt{N!}} \int dx \int d^N z \, \chi_N ( z_1 , \cdots , z_N  ~\vert ~ \theta_1 , \cdots , \theta _N ) \,	\left [\operator{\partial}_x^2 \operator{\Psi}^\dag(x)\right ] \operator{\Psi}(x)\,  \operator{\Psi}^\dag(z_1) \cdots \operator{\Psi}^\dag(z_N) \vert 0 \rangle \label{chap:1:hal.mod.3.1}\\
	\operator{H}\vert \psi ( \theta_1 , \cdots , \theta_N ) \rangle & = & \nonumber \\
	& & +\frac{c}{ \sqrt{N!}} \int dx \int d^N z \, \chi_N ( z_1 , \cdots , z_N  ~\vert ~ \theta_1 , \cdots , \theta _N ) 	\, \operator{\Psi}^\dag(x)\operator{\Psi}^\dag(x) \operator{\Psi}(x)\operator{\Psi}(x)  \operator{\Psi}^\dag(z_1) \cdots \operator{\Psi}^\dag(z_N) \vert 0 \rangle \label{chap:1:hal.mod.3.1} 
\end{eqnarray}

Les règles de commutations (\ref{eq.1.1}) impliquent que 

\begin{eqnarray}
	\left . \begin{array}{rcl}
		[ \operator{\Psi}(x),  \operator{\Psi}^\dag(z_1)\cdots \operator{\Psi}^\dag(z_N)  ]  &=&  \sum_{i = 0}^{N} \operator{\Psi}^\dag(z_1) \cdots \operator{\delta}(x-z_i) \cdots \operator{\Psi}^\dag(z_N)  \\
		\left [ \operator{\partial}_x\operator{\Psi}^\dag(x),  \operator{\Psi}^\dag(z) \right ]   & =  & 0 
	\end{array} \right . \label{chap:1:com.2}
\end{eqnarray}


En utilisant ces dernier règles de comutations et la définition d'état de Fock (\ref{eq.1.4}) , la premiers partie de l'application de l'hamiltonien sur l'état $\vert \psi  \rangle$ , (\ref{chap:1:hal.mod.3.1}) de simplifie en 

\begin{eqnarray}
	 - \frac{1}{ \sqrt{N!}} \int d^N z \, \chi_N ( z_1 , \cdots , z_N  ~\vert ~ \theta_1 , \cdots , \theta _N ) \,	\sum_{i=1}^N  \operator{\Psi}^\dag(z_1) \cdots \left [\operator{\partial}_{z_i}^2 \operator{\Psi}^\dag(z_i)\right ] \cdots\operator{\Psi}^\dag(z_N) \vert 0 \rangle 	
\end{eqnarray}

Et en faisant deux integration par partie selon la variable $z_i$ , cette premier partie devient

\begin{eqnarray}
	 - \frac{1}{ \sqrt{N!}} \int d^N z \, \sum_{i=1}^N \, \operator{\partial}_{z_i}^2\chi_N ( z_1 , \cdots , z_N  ~\vert ~ \theta_1 , \cdots , \theta _N ) \,	 \operator{\Psi}^\dag(z_1)  \cdots\operator{\Psi}^\dag(z_N) \vert 0 \rangle \label{chap:1:hal.mod.3.1.1}	
\end{eqnarray}

Pour la seconde partie  , en remarquant ques Les règles de commutations (\ref{eq.1.1}) impliquent que 

\begin{eqnarray}
	[ \operator{\Psi}(x) \operator{\Psi}(x),  \operator{\Psi}^\dag(z) ] & =& 2\operator{\Psi}(x)\operator{\delta}(x - z)\label{chap:1:com.3}  		
\end{eqnarray}

et en remplaçant $\operator{\Psi}(x)$ par $\operator{\Psi}(x)\operator{\Psi}(x)$ dans  (\ref{chap:1:com.2}) il vient que  

\begin{eqnarray}
	[ \operator{\Psi}(x)\operator{\Psi}(x),  \operator{\Psi}^\dag(z_1)\cdots \operator{\Psi}^\dag(z_N)  ]  &=&  2\sum_{i = 0}^{N} \operator{\Psi}^\dag(z_1) \cdots  \operator{\Psi}(x)\operator{\delta}(x-z_i) \cdots \operator{\Psi}^\dag(z_N) \label{chap:1:com.4}		
\end{eqnarray}

et en injectant (\ref{chap:1:com.2}),  (\ref{chap:1:com.4}) devient 

\begin{eqnarray}
	[ \operator{\Psi}(x)\operator{\Psi}(x),  \operator{\Psi}^\dag(z_1)\cdots \operator{\Psi}^\dag(z_N)  ]  &=& \left\{ \begin{array}{l} \displaystyle 2\sum_{i = 0}^{N} \operator{\Psi}^\dag(z_1) \cdots  \operator{\delta}(x-z_i) \cdots \operator{\Psi}^\dag(z_N) \operator{\Psi}(x) \\ + \\ \displaystyle 2\sum_{i = 0}^{N}\sum_{j = i+1}^{N} \operator{\Psi}^\dag(z_1) \cdots\operator{\delta}(x-z_i)\operator{\Psi}^\dag(z_{i+1})\cdots  \operator{\delta}(x-z_j) \cdots \operator{\Psi}^\dag(z_N) \end{array}\right. \label{chap:1:com.5}	.	
\end{eqnarray}

En utilisant la régle de comutation (\ref{chap:1:com.5}) et la définition de l'état de Fock (\ref{eq.1.4}) , la seconde partie de (\ref{chap:1:hal.mod.3.1}) devient

\begin{eqnarray}
	 \frac{1}{ \sqrt{N!}} \int d^N z \,  \,2	c \,\sum_{1\leq i<j\leq N} \operator{\delta}(z_i - z_j) \chi_N ( z_1 , \cdots , z_N  ~\vert ~ \theta_1 , \cdots , \theta _N ) \, \operator{\Psi}^\dag(z_1) \cdots  \cdots\operator{\Psi}^\dag(z_N) \vert 0 \rangle \label{chap:1:hal.mod.3.1.2}	
\end{eqnarray} 

en utilisant (\ref{chap:1:hal.mod.3.1.1}) et (\ref{chap:1:hal.mod.3.1.2}) , (\ref{chap:1:hal.mod.3.1.1}) devient 

\begin{eqnarray}
	\operator{H}\vert \psi ( \theta_1 , \cdots , \theta_N ) \rangle &= &  \frac{1}{ \sqrt{N!}} \int d^N z \,  	\left [ \operator{\mathcal{H}}_N \chi_N ( z_1 , \cdots , z_N  ~\vert ~ \theta_1 , \cdots , \theta _N ) \right ] \, \operator{\Psi}^\dag(z_1) \cdots  \cdots\operator{\Psi}^\dag(z_N) \vert 0 \rangle		
\end{eqnarray}

avec 

\begin{eqnarray}
	\operator{\mathcal{H}}_N & = &  - \sum_{i=1}^N \, \operator{\partial}_{z_i}^2 + 2	c \sum_{1\leq i<j\leq N} \operator{\delta}(z_i - z_j) 		
\end{eqnarray}





