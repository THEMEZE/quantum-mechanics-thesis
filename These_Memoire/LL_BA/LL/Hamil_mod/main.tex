
\subsubsection{Expression de l’Hamiltonien}
Le Hamiltonien du modèle est donné par

\begin{eqnarray}
	\operator{H} & = & \int dx \left [ \operator{\partial}_x \operator{\Psi}^\dag (x) \operator{\partial}_x \operator{\Psi}(x) + c \operator{\Psi}^\dag (x) \operator{\Psi}^\dag (x) \operator{\Psi} (x) \operator{\Psi} (x) \right ] \label{chap:1:hal.mod}
\end{eqnarray}


où \( c \) est la constante de couplage. 

\subsubsection{Équation du mouvement associée}

L'équation du mouvement correspondante
\begin{eqnarray}
	i \operator{\partial}_t \operator{\Psi}	 & = & - \operator{\partial}_x^2 \operator{\Psi} + 2 c \operator{\Psi}^\dag\operator{\Psi} \operator{\Psi}
\end{eqnarray}

est appelée l'équation de Schrödinger non linéaire (NS).

Pour $c > 0$, l'état fondamental à température nulle est une sphère de Fermi. Seul ce cas sera considéré par la suite. 
