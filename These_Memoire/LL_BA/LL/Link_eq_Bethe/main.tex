La formule (\ref{eq.1.26}) détermine la fonction d'onde de Bethe ; cette fonction est réductible à deux particules. Il convient de mentionner que les fonctions d'onde de tous les modèles résolubles par l'Ansatz de Bethe ont une forme similaire à (\ref{eq.1.26}). Discutons maintenant des propriétés de la fonction d'onde $\chi_N$. La fonction $\chi_N$ est une fonction symétrique des variables $z_j \ (j = 1, \dots, N)$ et une fonction continue de chacune d'elles. Ces propriétés deviennent évidentes si l'on réécrit la représentation (\ref{eq.1.26}) sous la forme suivante :

\begin{eqnarray}
	\chi_N & = &	\frac{  \displaystyle \prod_{k<j} ( \lambda_j - \lambda_k ) }{ \sqrt{N! \displaystyle \prod_{k<j}  [ ( \lambda_j - \lambda_k )^2 +c^2 ] } }  \sum_{\mathcal{P}}\exp \left \{ i \sum_{n = 1}^N  z_n \lambda_{ \mathcal{P}(n) } \right \} \notag \\
	&& \times \prod_{k<j}  	\left [ 1 - \frac{ic \epsilon(z_j - z_k ) }{( \lambda_{ \mathcal{P}(j) }  - \lambda_{ \mathcal{P}(k) }  ) }.  \right ] 
\end{eqnarray}

On peut également voir à partir de cette formule que $\chi_N$ est une fonction antisymétrique des $\lambda_j$ :

\begin{eqnarray}
	\chi_N ( z_1 , \cdots , z_N \vert \lambda_1 , \cdots , \lambda_j , \cdots , \lambda_k , \cdots , \lambda_N ) & =& - 	\chi_N ( z_1 , \cdots , z_N \vert \lambda_1 , \cdots , \lambda_k , \cdots , \lambda_j , \cdots , \lambda_N ).
\end{eqnarray}

Ainsi, $\chi_N = 0 \ \text{si} \ \lambda_j = \lambda_k$, $j \neq k$. C’est la base du \textbf{principe d'exclusion de Pauli} pour les bosons en interaction en une dimension, qui joue un rôle fondamental dans la construction de l'état fondamental, appelé \textbf{mer de Dirac}. La démonstration complète du principe de Pauli est donnée dans la section VII.4.

Il est connu que le théorème reliant le spin et les statistiques ne s'applique pas dans des dimensions 1 + 1 de l'espace-temps. Par conséquent, certains modèles de bosons sont équivalents à des modèles de fermions. Par exemple, le \textbf{modèle de sine-Gordon} est équivalent au \textbf{modèle de Thirring massif}, tandis que le gaz de bosons unidimensionnel pour $c = \infty$ est équivalent au modèle de fermions libres.

Nous avons déjà construit les fonctions propres communes $\chi_N$ (équations (\ref{eq.1.26}), (\ref{eq.1.9})) des opérateurs $\operator{H}$, $\operator{P}$ et $\operator{Q}$, les valeurs propres correspondantes étant données par

\begin{eqnarray}
	E_N = \sum_{j = 1 }^N \lambda_j^2 ; ~ 	P_N = \sum_{j = 1 }^N \lambda_j^1 ; Q_N = \sum_{j = 1 }^N \lambda_j^0 = N 
\end{eqnarray}

Considérons maintenant les fonctions propres dans tout l'espace des coordonnées $\mathbb{R}^N : -\infty < z_j < \infty$ (avec $j = 1, \cdots, N$). La normalisation dans ce cas a été calculée dans \cite{g} :

\begin{eqnarray}
	\int_{-\infty}^{+\infty} d^N z \, \chi_N^\ast ( z_1 , \cdots , z_N \vert \lambda_1 , \cdots , \lambda_N ) \chi_N ( z_1 , \cdots , z_N \vert \mu_1 , \cdots , \mu_N ) & = & ( 2 \pi )^N 	\prod_{ j = 1 }^N \delta ( \lambda_j - \mu _1 ).
\end{eqnarray}

Les moments \(\{ \lambda \}\) et \(\{\mu\}\) sont supposés être ordonnés :

\begin{eqnarray}
	\lambda_1 < \lambda_2 < \cdots < \lambda_N , && 		\mu_1 < \mu_2 < \cdots < \mu_N 
\end{eqnarray}

Dans le même livre \cite{9}, la complétude du système \(\chi_N\) est également prouvée :

\begin{eqnarray}
	\int_{-\infty}^{+\infty} d^N \lambda \, \chi_N^\ast ( z_1 , \cdots , z_N \vert \lambda_1 , \cdots , \lambda_N ) \chi_N ( y_1 , \cdots , y_N \vert \lambda_1 , \cdots , \lambda_N ) & = & ( 2 \pi )^N 	\prod_{ j = 1 }^N \delta ( z_j - y_1 ).
\end{eqnarray}

\begin{eqnarray}
	z_1 < z_2 < \cdots < z_N , && 		y_1 < y_2 < \cdots < y_N .
\end{eqnarray}