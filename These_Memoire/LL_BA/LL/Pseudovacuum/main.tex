\subsubsection{Le pseudovacuum $\vert 0 \rangle$}

Le vide de Fock $\vert 0 \rangle$, défini par 


\begin{eqnarray}
	\forall x \in \mathbb{R} & \colon & \operator{\Psi} (x) \vert 0 \rangle = 	0 \label{eq.1.4}
\end{eqnarray}

est d'une grande importance.

\subsubsection{Distinction avec le vide physique}
%It will be called the pseudovacum and it is to be distinguihed from the physical vacuum which is the ground state of the Hamiltonian (the Dirac sea). The dual pseudovacuum $\langle 0 \vert$ is defined as $\langle 0 \vert =  \vert 0 \rangle^\dag$ and satisfies the relations ... where the dagger denotes Hermitian conjugation. The number of particles aperator $\operator{Q}$ and momentum operator $\operator{P}$ are 

Il sera appelé le pseudovacuum et doit être distingué du vide physique, qui est l'état fondamental de l'Hamiltonien (la mer de Dirac). 

\subsubsection{Dual du pseudovacuum $\langle 0 \vert$}

Le pseudovacuum dual $\langle 0 \vert$ est défini comme $\langle 0 \vert = \vert 0 \rangle^\dag$ et satisfait les relations 

\begin{eqnarray}
	\forall x \in \mathbb{R} & \colon & \langle 0 \vert \operator{\Psi}^\dag (x)  = 	0, ~\langle 0 \vert 0 \rangle = 1 
\end{eqnarray}

où la dague ($^\dag$) désigne la conjugaison hermitienne.

