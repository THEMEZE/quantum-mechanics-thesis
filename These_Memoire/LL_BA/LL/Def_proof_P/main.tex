\subsubsection{Utilisation de la définition de $\operator{P}$ intégrée par parties }

Cela peut être expliqué en utilisant l'opérateur du moment $\operator{P}$ (\ref{eq.1.7}) comme exemple. Tout d'abord, nous intégrons (\ref{eq.1.7}) par parties pour représenter $\operator{P}$ sous la forme :

\begin{eqnarray*}
	\operator{P} & = & i \int \left [ \operator{\partial}_x \operator{\Psi}^\dag(x)\right ] \operator{\Psi}(x) dx 
\end{eqnarray*}

\subsubsection{Application à l’état à N particules}

Agir avec cet opérateur sur la fonction propre (\ref{eq.1.9}) donne 
\begin{eqnarray*}
	\operator{P}\vert \psi ( \lambda_1 , \cdots , \lambda_N ) \rangle & = & \frac{i}{ \sqrt{N!}} \int dx \int d^N z \, \chi_N ( z_1 , \cdots , z_N  ~\vert ~ \lambda_1 , \cdots , \lambda _N ) 	\left [\operator{\partial}_x \operator{\Psi}^\dag(x)\right ]  \\ & & \times \sum_{k = 1 }^N \operator{\Psi}^\dag(z_1) \cdots [\operator{\Psi}(x) , \operator{\Psi}^\dag(z_k)] \cdots \operator{\Psi}^\dag(z_N) \vert 0 \rangle
\end{eqnarray*}

où l'équation (\ref{eq.1.4}) a été utilisée. La formule (\ref{eq.1.1}) donne une fonction delta pour le commutateur, qui peut ensuite être intégrée pour donner

\begin{eqnarray*}
	\operator{P}\vert \psi ( \lambda_1 , \cdots , \lambda_N ) \rangle & = & \frac{i}{ \sqrt{N!}}  \int d^N z 	\, \chi_N ( z_1 , \cdots , z_N  ~\vert ~ \lambda_1 , \cdots , \lambda _N ) \\ & & \times  \sum_{k = 1 }^N  \operator{\Psi}^\dag(z_1) \cdots \left [\operator{\partial}_{z_k} \operator{\Psi}^\dag(z_k)\right ] \cdots \operator{\Psi}^\dag(z_N) \vert 0 \rangle. 
\end{eqnarray*}

Nous intégrons maintenant par parties par rapport à $z_k$ pour obtenir

\begin{eqnarray*}
	\operator{P}\vert \psi ( \lambda_1 , \cdots , \lambda_N ) \rangle & = & \frac{1}{ \sqrt{N!}}  \int d^N z 	 \left \{  -i \sum_{k = 1 }^N  \operator{\partial}_{z_k}\chi_N ( z_1 , \cdots , z_N  ~\vert ~ \lambda_1 , \cdots , \lambda _N ) \right \} \\ && \times \operator{\Psi}^\dag(z_1) \cdots \operator{\Psi}^\dag(z_N) \vert 0 \rangle. 
\end{eqnarray*}

Ainsi, nous avons prouvé que l'action de l'équation (\ref{eq.1.7}) sur l'équation (\ref{eq.1.9}) est équivalente à l'action de $\operator{\mathcal{P}}_N$ sur \(\chi_N\). La construction de l'Hamiltonien mécanique quantique est assez similaire.
