Ici, $\chi_N$ est une fonction symétrique de toutes les variables $z_j$. L'équation aux valeurs propres 

\begin{eqnarray}
	\operator{H} |\psi_N\rangle = E_N |\psi_N\rangle, \quad \operator{P} |\psi_N\rangle = p_N |\psi_N\rangle, \quad \operator{Q} |\psi_n\rangle = q_N |\psi_N\rangle,	
\end{eqnarray}

conduit au fait que $\chi_N$ est une fonction propre à la fois de l'Hamiltonien mécanique quantique $\operator{\mathcal{H}}_N$ et de l'opérateur du moment mécanique quantique $\operator{\mathcal{P}}_N$ :

\begin{eqnarray}
	&& \left \{ \begin{array}{rcl} \operator{ \mathcal{H}}_N & = & \displaystyle \sum_{j=1}^N   - \operator{\partial}_{z_j}^2 + 2c \sum_{1 \leq k < j \leq N } \operator{\delta} ( z_j - z_k) \\ \operator{\mathcal{P}}_N & = & \displaystyle \sum_{j=1}^N -i \operator{\partial}_{z_j}\end{array} \right .  \label{eq.1.11}\\
	%&& ~~~\operator{ \mathcal{H}}_N \chi_N ~= ~	E_N \chi_N .\label{eq.1.12}
	&& \left \{ \begin{array}{rcl}\operator{ \mathcal{H}}_N \chi_N &=&	E_N \chi_N \\ \operator{ \mathcal{P}}_N \chi_N &=&	p_N \chi_N\end{array} \right . .\label{eq.1.12}
\end{eqnarray}

