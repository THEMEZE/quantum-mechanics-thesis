%Nous considérons à présent le cas de deux bosons quantiques dans la même boîte unidimensionnelle de longueur \(L\), avec des conditions aux limites périodiques. Contrairement au cas à une particule, le terme d’interaction à contact intervient dans la dynamique. L'hamiltonien à 2 particule s'écrit :
%En première quantification, en utilisant les coordonnées du centre de masse et relatives $Z = (z_1 + z_2)/2$ et $Y = z_1 - z_2$, il vient que
%l'hamiltonien (\ref{chap:1:hal.mod.2.part.3}) se divise en une somme de deux problèmes indépendants à une seule particule.
%Les états propres de l'hamiltonien du centre de masse de masse $\overline{m}= 2$, $-\frac{1}{4} \partial_Z^2$, sont des ondes planes, et l'hamiltonien pour la coordonnée relative $Y$ correspond à celui d'une particule de masse réduite $\tilde{m} = 1/2$ en présence d'un potentiel delta en $Y = 0$. 
%\paragraph{Introduction au système à deux bosons avec interaction de contact.}
%Nous considérons à présent le cas de deux bosons quantiques dans une même boîte unidimensionnelle de longueur \(L\), avec des conditions aux limites périodiques. Contrairement au cas à une particule, un terme d’interaction de contact intervient ici dans la dynamique. L’Hamiltonien à deux particules s’écrit :
%\begin{eqnarray}
%	\operator{\mathcal{H}}_2  =  \operator{\mathcal{K}}_2 +\operator{\mathcal{V}}_2  & avec & \operator{\mathcal{K}}_2 =   - \frac{1}{2} \partial_{z_1}^2 - \frac{1}{2} \partial_{z_2}^2,  \quad \mbox{et} \quad  \operator{\mathcal{V}}_2  =  	g  \delta(z_1 - z_2). \label{chap:1:hal.mod.2.part.3} 		
%\end{eqnarray}
%On rappelle que l'énergies propres de  $\operator{\mathcal{K}}_2$ associées aux fonction d'ondes $\varphi_{\{ \theta_1 , \theta_2 \}}$ , la masse des particule étant égale à 1 (ie $\hbar= m=1$) s'écrit 
%\begin{eqnarray}
%	\varepsilon(\theta_1) + 	\varepsilon(\theta_2) & = & \frac{\theta_1^2}{2} + \frac{\theta_2^2}{2} 
%\end{eqnarray}
%On vas travailler dans le centre de masse.

%\paragraph{Changement de variables : coordonnées du centre de masse et relatives.}
 
%En première quantification, en introduisant les coordonnées du centre de masse \(Z = \frac{z_1 + z_2}{2}\) et relative \(Y = z_1 - z_2\), on obtient :
%\(
%	\partial_{z_1}^2 + \partial_{z_2}^2 = \frac{1}{2} \partial_Z^2 + 	2\partial_Y^2.  
%\)
%L’Hamiltonien~\eqref{chap:1:hal.mod.2.part.3} se décompose alors en une somme de deux problèmes indépendants à une seule variable :

%\begin{eqnarray}\label{chap:1:hal.mod.2.part.4}
%	\operator{\mathcal{H}}_2  =  	-\frac{1}{4} \partial_Z^2 + \operator{\mathcal{H}}_{rel} , \quad \mbox{avec}\quad  \operator{\mathcal{H}}_{rel} =  - 	\partial_Y^2 + g \delta ( Y ). 
%\end{eqnarray}

%\paragraph{Résolution du problème de centre de masse et de coordonnée relative.}

%Les états propres de l’Hamiltonien associé au centre de masse, \(-\frac{1}{4} \partial_Z^2\), correspondant à une particule de masse totale \(\bar{m} = 2\), sont des ondes planes associés à l'énergie $\overline{\theta}^2$ avec $\overline{\theta} = \frac{ \theta_1 + \theta_2}{2}$. L’Hamiltonien, $\operator{\mathcal{H}}_{rel}$, associé à la coordonnée relative \(Y\) correspond quant à lui à celui d’une particule de masse réduite \(\tilde{m} = \frac{1}{2}\), soumise à un potentiel delta en \(Y = 0\) :
%\begin{eqnarray}\label{chap:1:hal.mod.2.part.5}
%	- 	\partial_Y^2 \tilde{\varphi}(Y) + g \delta ( Y )\tilde{\varphi}(Y) & = & \tilde{\varepsilon}\,\tilde{\varphi}(Y),
%\end{eqnarray}
%où $\tilde{\varepsilon}$ est l’énergie propre du problème relatif.

%%%%%%
\paragraph{Introduction au système de deux bosons avec interaction de contact.}

Considérons maintenant un système de deux bosons quantiques confinés dans une boîte unidimensionnelle de longueur \(L\), avec des conditions aux limites périodiques. Contrairement au cas à une seule particule, une interaction de contact intervient ici dans la dynamique. L’Hamiltonien à deux particules s’écrit :
\begin{eqnarray}
	\operator{\mathcal{H}}_2 = \operator{\mathcal{K}}_2 + \operator{\mathcal{V}}_2, \quad \text{avec} \quad \operator{\mathcal{K}}_2 = - \frac{1}{2} \partial_{z_1}^2 - \frac{1}{2} \partial_{z_2}^2, \quad \text{et} \quad \operator{\mathcal{V}}_2 = g \, \delta(z_1 - z_2). \label{chap:1:hal.mod.2.part.3}
\end{eqnarray}

On rappelle que, pour des particules de masse unitaire (i.e., \(\hbar = m = 1\)), les énergies propres de l’opérateur cinétique \(\operator{\mathcal{K}}_2\), associées aux fonctions d’onde symétrisées \(\varphi_{\{ \theta_1 , \theta_2 \}}\), sont données par :
\begin{eqnarray}
	\varepsilon(\theta_1) + \varepsilon(\theta_2) = \frac{\theta_1^2}{2} + \frac{\theta_2^2}{2}.
\end{eqnarray}

Afin de simplifier le problème, nous nous plaçons dans le référentiel du centre de masse.

\paragraph{Changement de variables : coordonnées du centre de masse et relative.}

En première quantification, on introduit les nouvelles variables :
\(
Z = \frac{z_1 + z_2}{2} \, \text{(centre de masse)}, \qquad Y = z_1 - z_2 \, \text{(coordonnée relative)}.
\)
Dans ce changement de variables, l’opérateur laplacien total devient :
\(
\partial_{z_1}^2 + \partial_{z_2}^2 = \frac{1}{2} \partial_Z^2 + 2 \, \partial_Y^2.
\)
L’Hamiltonien~\eqref{chap:1:hal.mod.2.part.3} se décompose alors en la somme de deux Hamiltoniens agissant respectivement sur \(Z\) et \(Y\) :
\begin{eqnarray}\label{chap:1:hal.mod.2.part.4}
	\operator{\mathcal{H}}_2 = -\frac{1}{4} \partial_Z^2 + \operator{\mathcal{H}}_{\text{rel}}, \qquad \text{avec} \quad \operator{\mathcal{H}}_{\text{rel}} = - \partial_Y^2 + g \, \delta(Y).
\end{eqnarray}

\paragraph{Résolution du problème du centre de masse et de la coordonnée relative.}

L’Hamiltonien du centre de masse, \(-\frac{1}{4} \partial_Z^2\), décrit une particule de masse totale \(\bar{m} = 2\). Ses états propres sont des ondes planes associées à une énergie \(\overline{\theta}^2\), avec :
\(
\overline{\theta} = \frac{\theta_1 + \theta_2}{2},
\)
jouant ici un rôle analogue à celui d’un pseudo-moment associé dans le référentielle de laboratoire.
Le Hamiltonien relatif, \(\operator{\mathcal{H}}_{\text{rel}}\), correspond quant à lui à une particule de masse réduite \(\tilde{m} = \frac{1}{2}\) soumise à un potentiel delta centré en \(Y = 0\). Son équation propre s’écrit :
\begin{eqnarray}\label{chap:1:hal.mod.2.part.5}
	- \partial_Y^2 \, \tilde{\varphi}(Y) + g \, \delta(Y) \, \tilde{\varphi}(Y) = \tilde{\varepsilon} \, \tilde{\varphi}(Y),
\end{eqnarray}
où \(\tilde{\varepsilon}\) désigne l’énergie associée au mouvement relatif.
%%%%%%%%%%%%%%

\paragraph{Forme symétrique de la fonction d'onde pour bosons.}
Dans le référentiel du centre de masse. Le système est le même que que celuis d'un particules de masse $\tilde{m}= \frac{1}{2}$.
Le système étant composé de particules bosoniques, on cherche une solution symétrique que l’on écrit sous la forme  :
\begin{eqnarray}
	\tilde{\varphi}(Y) ~=~a~e^{i\frac{1}{2} \tilde{\theta} \vert Y \vert } + b~e^{-i\frac{1}{2} \tilde{\theta}\vert Y \vert } ~\propto~  \sin\left( \frac{1}{2} (\tilde{\theta} |Y| + \Phi ) \right). \label{eq:ansatz.boson}
\end{eqnarray}
Le paramètre \( \tilde{\theta} = \theta_1 - \theta_2 \) joue ici un rôle analogue à celui d’un pseudo-moment associé à la coordonnée relative,
est  la phase s'écrit
\begin{eqnarray}
	\Phi(\tilde{\theta}) &=& 2 \arctan\left (\frac{1}{i} \frac{a+b}{a-b}\right),	\label{chap:1:dif.mod.2.part.1} 
\end{eqnarray}
car $a\exp(ix)+b\exp(-ix) = 2\sqrt{ab}\sin(x+\arctan(-i (a+b)/(a-b))$. Pour $\tilde{\theta}<0$, les termes exponentiels \( \exp(i\tilde{\theta} \vert Y \vert/2 ) \) et \( \exp(-i\tilde{\theta} \vert Y \vert/2 ) \) correspondent aux paires de particules entrantes et sortantes d’un processus de diffusion à deux corps.\\


%En réinjectant l'équation \eqref{eq:ansatz.boson} dans l’équation \eqref{chap:1:hal.mod.2.part.5}, on obtient l’énergie propre du problème réduit $\tilde{\varepsilon}$ associé à l’état lié. Celle-ci prend la forme classique de l’énergie cinétique d’une particule, \( \frac{1}{2} \times \text{masse} \times \text{vitesse}^2 \), la masse réduite du problème étant ici \( \tilde{m} = \frac{1}{2} \), et où \( \tilde{\theta} \) joue un rôle analogue à celui d’une vitesse. On en déduit :
%\begin{eqnarray}\tilde{\varepsilon}(\tilde{\theta})  & = &  \frac{1}{2} \cdot \tilde{m} \cdot \tilde{\theta}^2 = \frac{1}{2} \cdot \frac{1}{2} \cdot \tilde{\theta}^2 = \frac{\tilde{\theta}^2}{4}.\end{eqnarray}
%\begin{eqnarray}
%	\tilde{\varepsilon}(\tilde{\theta})  & = &  \frac{\tilde{\theta}^2}{4}.
%\end{eqnarray}
% Il encode la décroissance exponentielle de la fonction d’onde liée dans l’espace relatif, et sa valeur est directement reliée à la profondeur de l’état lié. Une valeur plus grande de \( \tilde{\theta} \) correspond à un état plus fortement lié, c’est-à-dire plus localisé autour de \( Y = 0 \), ce qui reflète une interaction plus attractive entre les deux particules. $\overline{\theta}^2 +  \tilde{\varepsilon}(\tilde{\theta}) = \varepsilon{\theta_1} + \varepsilon{\theta_2}$.
En réinjectant l’ansatz~\eqref{eq:ansatz.boson} dans l’équation relative
\eqref{chap:1:hal.mod.2.part.5}, on obtient l’énergie propre
\(\tilde{\varepsilon}\) du problème réduit.  
Elle prend la forme cinétique usuelle
\(\tfrac{1}{2}\times\text{masse}\times\text{vitesse}^{2}\).  
La masse réduite vaut ici \(\tilde{m}= \frac{1}{2}\) et le paramètre
\(\tilde{\theta}\) joue le rôle d’une impulsion ; ainsi
\begin{equation}
   \tilde{\varepsilon}(\tilde{\theta})
   \;=\;
   \frac{1}{2}\,\tilde{m}\,\tilde{\theta}^{2}
   \;=\;
   \frac{1}{2}\times\frac{1}{2}\,\tilde{\theta}^{2}
   \;=\;
   \frac{\tilde{\theta}^{2}}{4}.
   \label{eq:energie_relative}
\end{equation}

Cette énergie gouverne la décroissance exponentielle de la fonction
d’onde dans la coordonnée relative : plus \(\tilde{\theta}\) est grand,
plus l’état est localisé autour de \(Y=0\), signe d’une interaction
attractive plus forte entre les deux bosons.

L’énergie totale se décompose enfin en la somme du mouvement du centre
de masse et du mouvement relatif :
\(
   \overline{\theta}^{2}
   \;+\;
   \tilde{\varepsilon}(\tilde{\theta})
   \;=\;
   \varepsilon(\theta_{1})
   \;+\;
   \varepsilon(\theta_{2}),
\)
où \(\overline{\theta}= \tfrac{\theta_{1}+\theta_{2}}{2}\) et
\(\varepsilon(\theta)=\theta^{2}/2\).






%%%%%%%%%%%%%%%%%%%%%%%%%%%
\paragraph{Condition de discontinuité à cause du potentiel delta.}
En raison de la présence du potentiel delta centré en $Y = 0$, la dérivée première de la fonction d’onde $\tilde{\varphi}(Y)$ présente une discontinuité en ce point. En effet, le potentiel étant infini en $Y = 0$, la phase $\Phi$ du régime symétrique est déterminée en intégrant l’équation du mouvement autour de la singularité. En intégrant entre $- \epsilon$ et $+ \epsilon$ et en faisant tendre $\epsilon \to 0$, on obtient la condition de saut de la dérivée :

%avec $\Phi$ une phase à déterminer. %\begin{equation}
%	E = \frac{\tilde{m} \theta^2}{2}.
%\end{equation}

%La dérivée de la fonction d’onde n’est pas continue en $Y = 0$. Le potentiel étant infini en $Y = 0$, la phase $\Phi$ est obtenue en intégrant l’équation du mouvement entre $- \epsilon$ et $+ \epsilon$ et en faisant tendre $\epsilon$ vers zéro :


%En raison de ce potentiel delta, la dérivée première de la fonction d'onde $\varphi(Y)$ doit avoir une discontinuité en $Y = 0$ : 

%{\color{lightgray} 
%\begin{eqnarray*}
%	\underset{ \epsilon \to 0 }{\lim} \int_{-\epsilon}^{+\epsilon}  	-\underbrace{\cancel{\frac{1}{4} \partial_Z^2\varphi(Y)}}_{0} - 	\partial_Y^2\varphi(Y) + c \delta ( Y )\varphi(Y) \, dY  & = & \underset{ \epsilon \to 0 }{\lim}  \int_{-\epsilon}^{+\epsilon}  E d Y , \\
%	\underset{ \epsilon \to 0 }{\lim}  \left [ \varphi'(\epsilon) - \varphi'(-\epsilon) \right ] - c \varphi (  0 ) & =  &  -\underset{ \epsilon \to 0 }{\lim}  \int_{-\epsilon}^{+\epsilon}  E d Y,\\
%	 \varphi'(0^+) - \varphi'(0^-) - c \varphi (  0 ) & = & 0 .
%\end{eqnarray*}


%}

\begin{eqnarray*}
	\underset{ \epsilon \to 0 }{\lim} \int_{-\epsilon}^{+\epsilon}  - 	\partial_Y^2\tilde{\varphi}(Y) + g \delta ( Y )\tilde{\varphi}(Y) \, dY  & = & \underset{ \epsilon \to 0 }{\lim}  \int_{-\epsilon}^{+\epsilon}  \tilde{\varepsilon}(\tilde{\theta})d Y ,\\
	\\
	\tilde{\varphi}'(0^+) - \tilde{\varphi}'(0^-) - g \tilde{\varphi} (  0 ) & = & 0 .
\end{eqnarray*}


%soit $\tilde{\varphi}'(0^+) - \tilde{\varphi}'(0^-) - c \tilde{\varphi} (  0 )  =  0 $ .

%%%%%%%%%%%%%%%
\paragraph{Détermination de la phase $\Phi$.}
Et en évaluant la discontinuité de sa dérivée au point $Y = 0$, on trouve que la phase $\Phi$ satisfait la condition :

%\begin{equation}
%	\tan\left( \frac{\Phi}{2} \right) = \frac{\tilde{\theta}}{c}.
%\end{equation}

\begin{eqnarray}\label{chap:1:dif.mod.2.part.2}
	\Phi(\tilde{\theta}) & = & 2 \arctan (\tilde{\theta}/g) \in [ - \pi , +\pi ].
\end{eqnarray}

%{\color{red}( à revoir)} Cette relation exprime l’impact de l’interaction delta sur le déphasage de la solution liée. On en déduit que plus le couplage $g$ est fort ($g \to \infty$), plus la phase $\Phi$ se rapproche de $0$, ce qui correspond à une fonction d’onde présentant s'annulant en $Y = 0$. En revanche, dans la limite d’interaction faible ($g \to 0$), la phase $\Phi$ tend vers $\pm \pi$ et la discontinué de la dérivé de la fonction d'onde devient négligeable.
%Cette relation exprime l’impact de l’interaction de type delta sur le déphasage de la fonction d’onde liée.On en déduit que plus le couplage $g$ est fort ($g \to \infty$), la phase $\Phi$ se rapproche de $0$, ce qui correspond à une fonction d’onde présentant s'annulant en $Y = 0$, à l’image du régime d’imperméabilité totale.
%À l’inverse, dans la limite d’interaction faible (\( g \to 0 \)), si bien que \( \Phi \) tend vers $\pi$ (ou \( -\pi \), selon le signe de \( \tilde{\theta} \)). Dans ce cas, la discontinuité de la dérivée de la fonction d’onde au point \( Y = 0 \) devient négligeable, ce qui traduit un couplage quasi inexistant entre les deux particules.
%Cette relation exprime l’impact de l’interaction de type delta sur le déphasage de la fonction d’onde liée. Lorsque le couplage \( g \) devient très fort (\( g \to \infty \)), la fraction \( \tilde{\theta}/g \to 0 \), et la phase \( \Phi(\tilde{\theta}) \to 0 \). Cela correspond à une situation dans laquelle la fonction d’onde est fortement contrainte à s’annuler en \( Y = 0 \), à l’image du régime d’imperméabilité totale.
%À l’inverse, dans la limite d’interaction faible (\( g \to 0 \)), la fraction \( \tilde{\theta}/g \to \infty \), si bien que \( \Phi(\tilde{\theta}) \to \pi \) (ou \( -\pi \), selon le signe de \( \tilde{\theta} \)). Dans ce cas, la discontinuité de la dérivée de la fonction d’onde au point \( Y = 0 \) devient négligeable, ce qui traduit un couplage quasi inexistant entre les deux particules.

Cette relation exprime l’impact de l’interaction de type delta sur le déphasage de la fonction d’onde liée. On en déduit que plus le couplage \( g \) est fort (\( g \to \infty \)), plus la phase \( \Phi \) se rapproche de zéro. Cela correspond à une fonction d’onde qui s’annule en \( Y = 0 \), caractéristique d’un régime d’imperméabilité totale.

À l’inverse, dans la limite d’une interaction faible (\( g \to 0 \)), la phase \( \Phi \) tend vers \( \pi \) (ou \( -\pi \), selon le signe de \( \tilde{\theta} \)). Dans ce cas, la discontinuité de la dérivée de la fonction d’onde au point \( Y = 0 \) devient négligeable, ce qui traduit une interaction presque absente entre les deux particules.


%%%%%%%%%%%%%%%%%%%%%%%%%%%%%%%%%%%
%\paragraph{Phase de diffusion à un corp.}
%Les équations \eqref{chap:1:dif.mod.2.part.1} et \eqref{chap:1:dif.mod.2.part.2}  et en remarquant que pour $z \in \mathbb{C} \backslash \{ \pm i \} 2\artan(z) = i \ln \left( \frac{ 1 - i z }{1+iz} \right ) $ soit $\exp(2i\arctan(x)) = (1 + ix)/(1 - ix)$ et $\Phi(\tilde{\theta}) = i \ln ( - b/a ) $  donne rapport entre les amplitudes $a$ et $b$ de la fonction d'onde \eqref{eq:ansatz.boson} définit la phase de diffusion / {\em matrice diffusion} $S( \tilde{\theta}) \doteq e^{i\Phi ( \tilde{\theta}  ) }$  :

%\begin{eqnarray}
%	e^{i\Phi ( \tilde{\theta}  ) } &=& -\frac{a}{b} ~=~\frac{1 +i\tilde{\theta}/g} { 1 - i\tilde{\theta}/g} .\label{chap:1:dif.mod.2.part.3}
%\end{eqnarray}

\paragraph{Phase de diffusion à deux corps.}

En combinant les équations~\eqref{chap:1:dif.mod.2.part.1} et~\eqref{chap:1:dif.mod.2.part.2} avec l’identité analytique valable pour tout
\(z \in \mathbb{C}\setminus\{\pm i\}\),
\(
2\arctan(z)=i\ln\!\left(\frac{1-iz}{1+iz}\right)
\Leftrightarrow
e^{2i\arctan(z)}=\frac{1+iz}{1-iz},
\)
on obtient que le rapport des amplitudes \(a\) et \(b\) de la fonction
d’onde relative~\eqref{eq:ansatz.boson} définit la {\em phase de diffusion }
\(
\Phi(\tilde{\theta}) = i\ln\!\left(-\frac{b}{a}\right).
\)
On introduit alors la {\em matrice de diffusion} (ou facteur de diffusion)
\begin{eqnarray}
	S(\tilde{\theta}) \;\doteq\; e^{i\Phi(\tilde{\theta})}= -\frac{a}{b}= \frac{1 + i\,\tilde{\theta}/g}{1 - i\,\tilde{\theta}/g}.%\tag{\ref{chap:1:dif.mod.2.part.3}}
\end{eqnarray}
%où \(g\) est le paramètre d’interaction et
%\(\tilde{\theta} = \theta_1 - \theta_2\) le pseudo‑moment relatif.  
Cette expression, unitaire et analytique, caractérise entièrement la diffusion élastique à deux corps dans le modèle considéré.



\paragraph{Lien entre phase de diffusion et décalage temporel : interprétation semi-classique. {\color{red}(à revoir)}}

Il a été souligné par {\color{black}Eisenbud (1948)} et {\color{black}Wigner (1955)} que la phase de diffusion peut être interprétée, de manière semi-classique, comme un {\em décalage temporel}. Esquissons brièvement l'argument de {\color{black}Wigner (1955)}.Tout d'abord, notons que, pour une particule unique, une approximation simple d’un paquet d’ondes peut être obtenue en superposant deux ondes planes avec des moments $\tilde{\theta}/2$ et $\tilde{\theta}/2 + \delta \tilde{\theta}$, respectivement :
\begin{eqnarray}
	\tilde{\varphi}_{\text{inc}}(Y) & \propto & e^{i\frac{1}{2}\tilde{\theta} \vert Y\vert} + e^{i\frac{1}{2}\left(\tilde{\theta} + 2\delta \tilde{\theta} \right) \vert Y\vert}.
\end{eqnarray}
Cette superposition évolue dans le temps comme :
\begin{eqnarray}
\tilde{\varphi}_{\text{inc}}(Y, t) &\propto &  e^{i\left( \frac{1}{2} \tilde{\theta}\vert Y\vert - t\,\tilde{\varepsilon}(\tilde{\theta}) \right)} + e^{i\left( \frac{1}{2}\left(  \tilde{\theta} + 2\delta \tilde{\theta} \right) \vert Y\vert - t\,\tilde{\varepsilon}(\tilde{\theta} + 2\delta \tilde{\theta}) \right)}.
\end{eqnarray}
%où l'on a utilisé l'expression de l'énergie réduite : $\tilde{\varepsilon}(\tilde{\theta}) = \tilde{\theta}^2 / 4$.
Le centre de ce 'paquet d'ondes' se situe à la position où les phases des deux termes coïncident, c'est-à-dire au point où $\vert Y\vert\delta \tilde{\theta}  - t[\tilde{\varepsilon}(\tilde{\theta} + 2\delta \tilde{\theta} ) - \tilde{\varepsilon}(\tilde{\theta})] = 0$, ce qui donne $\vert Y\vert \simeq \tilde{\theta} t$ avec la vitesse réduite $\tilde{\theta} = 1/2 \varepsilon'(\tilde{\theta}) $. %Ainsi, il s'agit effectivement d'un 'paquet d'ondes' se déplaçant à la vitesse $\theta$. Ensuite, considérons deux particules entrantes dans un état tel que le centre de masse $Z = (z_1 + z_2)/2$ ait une impulsion $\theta_1 - \theta_2$, tandis que la coordonnée relative $Y = z_1 - z_2$ se trouve dans un 'paquet d'ondes' se déplaçant à la vitesse $ (\theta_1 - \theta_2)/2$,
Selon les équations (\ref{eq:ansatz.boson}) et (\ref{chap:1:dif.mod.2.part.3}), l'état sortant de la diffusion correspondant serait :
\begin{eqnarray}
	\tilde{\varphi}_{outc} ( Y, t ) & \propto & -e^{i\Phi(\tilde{\theta})}e^{-i\frac{1}{2}\tilde{\theta} \vert Y\vert} - e^{i\Phi(\tilde{\theta} + 2 \delta \tilde{\theta} )}e^{-i\frac{1}{2}\left(\tilde{\theta} + 2\delta \tilde{\theta} \right) \vert Y\vert}. %\tag{2}
\end{eqnarray}
En répétant l'argument précédent de la stationnarité de phase, on trouve que la coordonnée relative est à la position $\vert Y \vert  \simeq \tilde{\theta} t - 2\Phi'( \tilde{\theta})$ au moment $t$. %Étant donné que le centre de masse n'est pas affecté par la collision et se déplace à la vitesse de groupe $\tilde{\theta} =(\theta_1 + \theta_2)/2$, nous constatons que la position des deux particules semiclassiques après la collision sera
\begin{eqnarray}
	\vert Y \vert & \simeq & 	\tilde{\theta} t  - 2 \Delta (\tilde{\theta} )
\end{eqnarray}
où le déplacement de diffusion $\Delta (\theta)$ est donné par la dérivée de la phase de diffusion,
\begin{eqnarray}\label{eq:I-1-16}
	\Delta ( \theta ) & \doteq & \frac{ d \Phi }{ d \theta } ( \theta )= \frac{ 2 g }{ g^2 + \theta^2} . 	
\end{eqnarray}


%\paragraph{Retour aux coordonnées du laboratoire.}
%En revenant aux coordonnées d'origine (celles du laboratoire), on constate que la fonction d'onde à deux corps 
%\(
%	\varphi_{\{\theta_1 , \theta_2\}} (z_1, z_2) = \langle \emptyset \vert \operator{\Psi} (z_1)\operator{\Psi} (z_2) \vert \{\theta_1, \theta_2\} \rangle,
%\)
%avec \(z_1 < z_2\) , (ie $Y>0$) . Et le centre de masse sur le mouvement
%\(
%	Z  =  \overline{\theta} t.	
%\)
%avec,  on rappelle , $\overline{\theta}$ la vitesse de groupe dans le référentielle de laboratoire.\\
%Nous constatons que la position des deux particules semiclassiques après la collision sera
%\begin{eqnarray}
%	z_1 ~=~ Z + \frac{Y}2 ~\simeq ~ \theta_1 t - \Delta(\theta_1 - \theta_2), & & 	z_2 ~=~ Z - \frac{Y}2 ~\simeq ~ \theta_2t + \Delta(\theta_1 - \theta_2),
%\end{eqnarray}

%avec  $\theta_1$ et $\theta_2$ on rappelle définie tel que 
%\(
%	\tilde{\theta} ~=~\theta_1 - \theta_2 , \,	\overline{\theta}~=~\frac{\theta_1 + \theta_2}{2}.	
%\)
%On remarquant que 
%\begin{eqnarray*}
%	z_1 \theta_1  + z_2  \theta_2 ~=~ 2Z\overline{\theta} + \frac{1}{2}Y\tilde{\theta}, & & z_1 \theta_2  + z_2  \theta_1 ~=~ 2Z\overline{\theta} - \frac{1}{2}Y\tilde{\theta}. 
%\end{eqnarray*}
%Ce qui est en accod avec la masse total $\overline{m} = 2$ et la masse résuite $\tilde{m} = \frac{1}{2}$ \\
%Ce qui nous motive à multiplier la fonction d'onde dans le référentiel du centre de masse \eqref{eq:ansatz.boson} par $\exp(2iZ\overline{\theta})$ pour obtenir 

%\begin{eqnarray}\label{eq:I-1-10}
%	\varphi_{\{\theta_1 , \theta_2\}}(z_1 , z_2) & \propto &  \left \{ \begin{array} { c cl} ( \theta_2 - \theta_1 - ic) e^{ i z_1 \theta_1 + iz_2 \theta_2 } - ( \theta_1 - \theta_2 - ic) e^{ i z_1 \theta_2 + iz_2 \theta_1} & \mbox{si} & z_1 < z_2 \\ (z_1 \leftrightarrow z_2) & \mbox{si} & z_1 > z_2 \end{array} \right.
%\end{eqnarray}

%correspondant aux valeurs propres

%\begin{eqnarray}
%	\varepsilon(\theta_1 , \theta_2) ~=~ \overbrace{ \overline{\theta}^2}^{\overline{\varepsilon}(\overline{\theta})}	 + \overbrace{\frac{1}{4} \tilde{\theta}^2}^{\tilde{\varepsilon}(\tilde{\theta})} ~=~ \frac{\theta_1}{2} + \frac{\theta_2}{2}.	
%\end{eqnarray}

%Pour $\theta_1 > \theta_2$, les deux termes $e^{iz_1 \theta_1 + iz_2 \theta_2 }$ et $e^{iz_1 \theta_2 + iz_2 \theta_1 }$ correspondent aux paires de particules entrantes et sortantes dans un processus de diffusion à deux corps. Le rapport de leurs amplitudes est la phase de diffusion à deux corps \eqref{chap:1:dif.mod.2.part.3} reste inchangé

%\begin{eqnarray}\label{chap:1:dif.mod.2.part.4}
%	e^{i\Phi ( \theta_1 - \theta_2  ) }~=~ -\frac{a}{b} ~=~\frac{\theta_1 - \theta_2  -ic} { \theta_2 - \theta_1  - ic}. 
%\end{eqnarray}


%%%%%%%%%%%%%%%%%%%%%%%%%%
\paragraph{Retour aux coordonnées du laboratoire.}

En revenant aux coordonnées du laboratoire, la fonction d’onde à deux corps s’écrit
\(
	\varphi_{\{\theta_1 , \theta_2\}} (z_1, z_2) 
	= \langle \emptyset \vert \operator{\Psi} (z_1)\operator{\Psi} (z_2) \vert \{\theta_1, \theta_2\} \rangle/\sqrt{2},
\)
dans le cas \(z_1 < z_2\), c’est-à-dire pour une séparation relative \(Y = z_1 - z_2 < 0\) (on pourra symétriser ultérieurement).  
Dans le référentiel du laboratoire, le centre de masse évolue selon
\(
	Z = \frac{z_1 + z_2}{2} = \overline{\theta}\,t.
\)
%où l’on rappelle que \(\overline{\theta} = \frac{\theta_1 + \theta_2}{2}\) est la vitesse de groupe du système dans le référentiel laboratoire.
Ainsi, la position semi-classique des deux particules après la collision s’écrit
\begin{eqnarray}
	z_1 = Z + \frac{Y}{2} \;\simeq\; \theta_1 t - \Delta(\theta_1 - \theta_2),\quad
	z_2 = Z - \frac{Y}{2} \;\simeq\; \theta_2 t + \Delta(\theta_1 - \theta_2),
\end{eqnarray}
%où \(\Delta(\theta_1 - \theta_2)\) représente le décalage dû à l’interaction entre les deux particules.
%On rappelle les définitions :
%\[
%	\tilde{\theta} = \theta_1 - \theta_2, 
%	\quad
%	\overline{\theta} = \frac{\theta_1 + \theta_2}{2}.
%\]
On peut vérifier les identités utiles suivantes :
\begin{eqnarray*}
	z_1 \theta_1 + z_2 \theta_2 = 2Z \overline{\theta} + \frac{1}{2} Y \tilde{\theta}, \quad
	z_1 \theta_2 + z_2 \theta_1 &=& 2Z \overline{\theta} - \frac{1}{2} Y \tilde{\theta},
\end{eqnarray*}
ce qui est en accord avec les masses associées : masse totale \(\overline{m} = 2\), masse réduite \(\tilde{m} = \frac{1}{2}\).

Cela nous motive à multiplier l’ansatz dans le référentiel du centre de masse (équation~\eqref{eq:ansatz.boson}) par un facteur de phase globale \(\exp(2iZ\overline{\theta})\) pour revenir à la représentation dans le laboratoire. On obtient alors l’expression de la fonction d’onde :
\begin{eqnarray}\label{eq:I-1-10}
	\varphi_{\{\theta_1 , \theta_2\}}(z_1 , z_2) & \propto &  \left \{ \begin{array} { c cl} ( \theta_2 - \theta_1 - ig) e^{ i z_1 \theta_1 + iz_2 \theta_2 } - ( \theta_1 - \theta_2 - ig) e^{ i z_1 \theta_2 + iz_2 \theta_1} & \mbox{si} & z_1 < z_2 \\ (z_1 \leftrightarrow z_2) & \mbox{si} & z_1 > z_2 \end{array} \right.
\end{eqnarray}

%Cette fonction d’onde correspond à une valeur propre d’énergie donnée par la somme des énergies associées aux deux degrés de liberté :

%\begin{equation}
%	\varepsilon(\theta_1 , \theta_2) 
%	= \underbrace{\overline{\theta}^2}_{\overline{\varepsilon}(\overline{\theta})}
%	+ \underbrace{\frac{1}{4} \tilde{\theta}^2}_{\tilde{\varepsilon}(\tilde{\theta})}
%	= \frac{\theta_1^2}{2} + \frac{\theta_2^2}{2}.
%\end{equation}

Pour \(\theta_1 > \theta_2\), les deux termes exponentiels 
\(e^{i z_1 \theta_1 + i z_2 \theta_2}\) et \(e^{i z_1 \theta_2 + i z_2 \theta_1}\)
correspondent respectivement aux ondes entrantes et sortantes dans le canal de diffusion à deux corps.  
Le rapport de leurs amplitudes définit la phase de diffusion / matrice diffusion $e^{i\Phi ( \tilde{\theta}  ) }$  à deux corps \eqref{chap:1:dif.mod.2.part.3} , reste inchangé :

\begin{equation}\label{chap:1:dif.mod.2.part.4}
	S(\theta_1- \theta_2) \doteq e^{i\Phi(\theta_1 - \theta_2)} 
	= \frac{\theta_1 - \theta_2 - ig}{\theta_2 - \theta_1 - ig}.
\end{equation}

Cette phase caractérise entièrement le processus de diffusion dans le modèle de Lieb-Liniger à deux particules.

\paragraph{Conditions périodiques et équations de Bethe pour deux bosons {\color{red}(à révoir)}.}

%La fonction d’onde obtenue par Bethe ansatz (voir
%\eqref{eq:I-1-10}) est, pour $z_{1}<z_{2}$,
%\[
%	\varphi_{\{\theta_{1},\theta_{2}\}}(z_{1},z_{2})
%		= a\,e^{i\theta_{1}z_{1}+i\theta_{2}z_{2}}
%		+b\,e^{i\theta_{2}z_{1}+i\theta_{1}z_{2}},
%	\quad
%	a=\theta_{2}-\theta_{1}-ic,\;
%	b=-(\theta_{1}-\theta_{2}-ic).
%\]

%\medskip
%\subparagraph{Périodicité sur $z_{2}$.}  
%On impose à la fonction d’onde obtenue par Bethe ansatz (voir
%\eqref{eq:I-1-10})
%\(
%	\varphi_{\{\theta_{1},\theta_{2}\}}(z_{1},z_{2}\!=\!L)
%	=
%	\varphi_{\{\theta_{1},\theta_{2}\}}(z_{1},z_{2}\!=\!0)
%\)
%avec $0<z_{1}<z_{2}=L$.  
%Au point $z_{2}=L$ on reste dans le secteur $z_{1}<z_{2}$, tandis qu’au point $z_{2}=0$ le domaine pertinent devient $z_{2}<z_{1}$;  la fonction d’onde y est obtenue en échangeant $z_{1}\leftrightarrow z_{2}$ , soit 
%\(
%	\varphi_{\{\theta_{1},\theta_{2}\}}(z_{1},\!L)
%	=
%	\varphi_{\{\theta_{1},\theta_{2}\}}(0 , z_{1})
%\)
%.  
%On obtient ainsi
%\begin{eqnarray*}
%	a\,e^{i\theta_{1}z_{1}+i\theta_{2}L}+b\,e^{i\theta_{2}z_{1}+i\theta_{1}L} & = &
%	a\,e^{i\theta_{2}z_{1}}\,e^{i\theta_{1}\! \cdot 0} + b \,e^{i\theta_{1}z_{1}}\,e^{i\theta_{2}\! \cdot 0},	
%\end{eqnarray*}
%avec la condition $z_1< z_2$, avec le rapport $a$ et $b$ vérifiant \eqref{chap:1:dif.mod.2.part.4} de la sorte $-b/a = e^{i\Phi(\theta_1 - \theta_2)}$ .

%%%%%%%%%%%%%%%%

\subparagraph{Périodicité en \( z_2 \).}  
On impose une condition de périodicité sur la fonction d’onde obtenue par ansatz de Bethe (voir équation~\eqref{eq:I-1-10}) :
\(
	\varphi_{\{\theta_1,\theta_2\}}(z_1, z_2 = L) = \varphi_{\{\theta_1,\theta_2\}}(z_1, z_2 = 0),
\)
avec \( 0 < z_1 < z_2 = L \).  
Au point \( z_2 = L \), la configuration reste dans le secteur \( z_1 < z_2 \), tandis qu’à \( z_2 = 0 \), on entre dans le secteur \( z_2 < z_1 \). La continuité de la fonction d’onde impose alors d’échanger les coordonnées \( z_1 \leftrightarrow z_2 \) :
\(
	\varphi_{\{\theta_1,\theta_2\}}(z_1, L) = \varphi_{\{\theta_1,\theta_2\}}(0, z_1).
\)
En utilisant l’expression explicite de l’ansatz dans les deux secteurs, on obtient l’égalité suivante :
\begin{eqnarray*}
	a\,e^{i\theta_1 z_1 + i\theta_2 L} + b\,e^{i\theta_2 z_1 + i\theta_1 L}
	&=& a\,e^{i\theta_2 z_1} + b\,e^{i\theta_1 z_1}.
\end{eqnarray*}
%où le second membre correspond à la fonction d’onde dans le secteur \( z_2 < z_1 \), évaluée en \( z_2 = 0 \) et \( z_1 = z_1 \).  
%La condition de périodicité impose donc :
%\[
%	a\,e^{i\theta_1 z_1 + i\theta_2 L} + b\,e^{i\theta_2 z_1 + i\theta_1 L}
%	= a\,e^{i\theta_2 z_1} + b\,e^{i\theta_1 z_1}.
%\]
Cette relation, valable pour tout \( z_1 \in (0,L) \), fixe une contrainte sur le rapport \( b/a \). En utilisant l’expression de la phase de diffusion introduite en \eqref{chap:1:dif.mod.2.part.4} pour $z_1<z_2$ :
\begin{eqnarray*}
	-\frac{b}{a} = e^{i\Phi(\theta_1 - \theta_2)},
\end{eqnarray*}
on obtient une condition quantique sur les phases \( \theta_1 \) et \( \theta_2 \), cœur de la quantification imposée par le formalisme de Bethe.

%\[
%	( \theta_2 - \theta_1 - ig)\,e^{i\theta_{1}z_{1}+i\theta_{2}L}
%	- ( \theta_1 - \theta_2 - ig)\,e^{i\theta_{2}z_{1}+i\theta_{1}L}
%	=
%	( \theta_2 - \theta_1 - ig)\,e^{i\theta_{2}z_{1}}\,e^{i\theta_{1}\! \cdot 0}
%	- ( \theta_1 - \theta_2 - ig)\,e^{i\theta_{1}z_{1}}\,e^{i\theta_{2}\! \cdot 0}.
%\]
En identifiant les coefficients de $e^{i\theta_{1}z_{1}}$ et
$e^{i\theta_{2}z_{1}}$ indépendamment, on obtient
\(
	e^{i\theta_{2}L}\;a = b, 
	\,
	e^{i\theta_{1}L}\;b = a,
\)
c’est‑à‑dire l'équations de Bethe
%\begin{equation}\label{eq:PC2}
%	e^{i\theta_{2}L} = \frac{b}{a}
%	= \frac{\theta_{1}-\theta_{2}+ic}{\theta_{2}-\theta_{1}+ic},
%\quad
%	e^{i\theta_{1}L} = \frac{a}{b}
%	= \frac{\theta_{2}-\theta_{1}+ic}{\theta_{1}-\theta_{2}+ic}.
%\end{equation}
\begin{eqnarray*}\label{eq:PC2}
	e^{i\theta_{1}L}\,e^{i\Phi(\theta_{1}-\theta_{2})} = -1,
	\qquad
	e^{i\theta_{2}L}\,e^{i\Phi(\theta_{2}-\theta_{1})} = -1.	
\end{eqnarray*}
En prenant le logarithme on obtient les \emph{équations de Bethe à deux
particules} :
\begin{equation}\label{eq:Bethe2}
	\theta_{1}L + \Phi(\theta_{1}-\theta_{2}) = 2\pi I_{1}, 
	\qquad
	\theta_{2}L + \Phi(\theta_{2}-\theta_{1}) = 2\pi I_{2},
\end{equation}
où $I_{1},I_{2}\in\mathbb{Z}$ sont les nombres quantiques entiers
(caractère bosonique). 

\subparagraph{Périodicité sur $z_{1}$.}  Le raisonnement symétrique conduit exactement aux mêmes égalités \eqref{eq:PC2}.  
%On vérifie donc que les deux conditions périodiques sont bien compatibles.

%\medskip
%\noindent\textbf{3. Équations de Bethe sous forme usuelle.}  
%On introduit la \emph{matrice de diffusion} à deux corps
%\[
%	S(\theta) \;=\; \frac{\theta-ic}{\theta+ic}\;=\;
%	e^{i\Phi(\theta)}, \quad
%	\Phi(\theta) = -\,2\arctan\!\bigl(\tfrac{\theta}{c}\bigr).
%\]
%Les relations \eqref{eq:PC2} se réécrivent
%\[
%	e^{i\theta_{1}L}\,S(\theta_{1}-\theta_{2}) = 1,
%	\qquad
%	e^{i\theta_{2}L}\,S(\theta_{2}-\theta_{1}) = 1.
%\]
%En prenant le logarithme on obtient les \emph{équations de Bethe à deux
%particules} :
%\begin{equation}\label{eq:Bethe2}
%	\theta_{1}L + \Phi(\theta_{1}-\theta_{2}) = 2\pi I_{1}, 
%	\qquad
%	\theta_{2}L + \Phi(\theta_{2}-\theta_{1}) = 2\pi I_{2},
%\end{equation}
%où $I_{1},I_{2}\in\mathbb{Z}$ sont les nombres quantiques entiers
%(caractère bosonique).  

%\medskip
%\noindent\textbf{4. Vérification de la cohérence énergétique.}  
%La solution \((\theta_{1},\theta_{2})\) de \eqref{eq:Bethe2} fournit des
%énergies
%\(
%	E = \frac{\theta_{1}^{2}}{2}+\frac{\theta_{2}^{2}}{2},
%\)
%qui coïncident avec la somme
%\(
%	E=\overline{\theta}^{2}+\frac{\tilde{\theta}^{2}}{4}
%\)
%déjà obtenue dans la décomposition centre‑de‑masse / relative.

\bigskip
Les équations \eqref{eq:Bethe2} constituent la quantification complète
du gaz de Lieb–Liniger à deux bosons sur un cercle de longueur $L$ et
seront le point de départ pour l’étude de l’état fondamental et des
excitations.







%.........................
 
%doit satisfaire une condition de raccord au point \(z_1 = z_2\), en lien avec la discontinuité de ses dérivées due à l’interaction locale. Cette équation peut être résolue en utilisant l’\textbf{Ansatz de Bethe}, en supposant que la fonction d’onde est une superposition symétrisée d’ondes planes dans les deux domaines \(z_1 < z_2\) et \(z_1 > z_2\) :



\begin{figure}[H]
	\centering
  %\includegraphics[width=0.5\textwidth]{}
  %\caption{Gauche : La fonction d'onde (\ref{eq:I-1-10}) sur la ligne infinie correspond à un processus de diffusion à deux corps. Semiclassiquement, la phase de diffusion dans ce processus à deux corps se reflète dans le décalage de diffusion (\ref{eq:I-1-16}) : après la collision, la position de la particule a été déplacée d'une distance $\Delta ( \theta_1 - \theta_2 )$ . Droite : La fonction d'onde de Bethe (\ref{eq:I-2-17}) sur la ligne infinie correspond à un processus de diffusion à $N$-corps qui se factorise en des processus à deux corps (le décalage de diffusion $\Delta$ est également présent ici, mais il n'est pas représenté dans la caricature). Dans ce processus à $N$-corps, les rapidités $\theta_j$ sont les moments asymptotiques des bosons.}
  \label{}	
\end{figure}

%Une telle superposition évolue dans le temps comme $\exp(i\vert Y \vert \tilde{\theta}/2 -it\tilde{\varepsilon}(\tilde{\theta}))) + \exp(i\vert Y \vert (\tilde{\theta}/2 + \delta \tilde{\theta}) -it\tilde{\varepsilon}(\tilde{\theta} + 2\delta\tilde{\theta} ))}$, où $\tilde{\varepsilon}(\tilde{\theta}) = \tilde{\theta}^2/4$ est l'énergie réduite. %Le centre de ce 'paquet d'ondes' se situe à la position où les phases des deux termes coïncident, c'est-à-dire au point où $z\delta \theta  - t[\varepsilon(\theta + \delta \theta ) - \varepsilon(\theta)] = 0$, ce qui donne $z \simeq vt$ avec la vitesse de groupe $v = d\varepsilon/d\theta = \theta$. Ainsi, il s'agit effectivement d'un 'paquet d'ondes' se déplaçant à la vitesse $\theta$. Ensuite, considérons deux particules entrantes dans un état tel que le centre de masse $Z = (z_1 + z_2)/2$ ait une impulsion $\theta_1 - \theta_2$, tandis que la coordonnée relative $Y = z_1 - z_2$ se trouve dans un 'paquet d'ondes' se déplaçant à la vitesse $ (\theta_1 - \theta_2)/2$,

%{\color{gray}
%\begin{eqnarray*}
%	\frac{ z_1 + z_2}{2} ( \theta_1 - \theta_2 ) + ( z_1 - z_2 ) \left ( \frac{ \theta_1 -\theta_2}{2} + \delta \theta  \right ) & = & z_1 ( \theta_1  + \delta \theta ) + z_2 ( \theta_2 - \delta \theta )  
%\end{eqnarray*}
%}

%\begin{eqnarray}
%	\psi_{inc} ( z_1 , z_2 ) & \propto & e^{i \frac{ z_1 + z_2 }{2} ( \theta_1 + \theta_2 ) }  \left ( e^{i ( z_1 - z_2 ) \frac{ \theta_1 - \theta_2}{2} } +  e^{i ( z_1 - z_2 )  \left ( \frac{ \theta_1 - \theta_2}{2}+ \delta \theta  \right )  } \right )\notag  \\
%	& \propto & e^{ i z_1 \theta_1 + i z_2 \theta_2 } + 	 e^{ i z_1 ( \theta_1 + \delta \theta )  + i z_2  ( \theta_2 - \delta \theta)   }%\tag{2}
%\end{eqnarray}

%Selon les équations (\ref{eq:I-1-10}) et (\ref{eq:I-1-11}), l'état sortant correspondant serait :

%\begin{eqnarray}
%	\psi_{outc} ( z_1 , z_2 ) & \propto & - e^{i\phi ( \theta_1 - \theta_2 ) }  e^{ iz_1 \theta_2 + i z_2 \theta_1 }  - e^{i\phi ( \theta_1 - \theta_2 + 2\delta \theta ) }  e^{ iz_1 (\theta_2- \delta \theta )  + i z_2 (\theta_1 + \delta \theta )  } \notag  \\
%	& \propto & e^{i \frac{ z_1 + z_2 }{2} ( \theta_1 + \theta_2 ) }  \left ( - e^{ i \phi ( \theta_1 - \theta_2 ) }e^{i(z_1 - z_2 ) \frac{ \theta_1 - \theta_2}{2} } -e^{ i \phi ( \theta_1 - \theta_2 + 2 \delta \theta ) }e^{i(z_1 - z_2 )  \left ( \frac{ \theta_1 - \theta_2}{2} + \delta \theta \right )  }  \right ). %\tag{2}
%\end{eqnarray}

%En répétant l'argument précédent de la stationnarité de phase, on trouve que la coordonnée relative est à la position $z_1 - z_2 \simeq (\theta_1 - \theta_2)t - 2d\Phi /d\theta$ au moment $t$. Étant donné que le centre de masse n'est pas affecté par la collision et se déplace à la vitesse de groupe $\tilde{\theta} =(\theta_1 + \theta_2)/2$, nous constatons que la position des deux particules semiclassiques après la collision sera

%\begin{eqnarray}
%	z_1 \simeq \theta _1 t - \Delta ( \theta_2 -\theta_1 ), & &	z_2 \simeq \theta _2 t - \Delta ( \theta_2 -\theta_1 ),
%\end{eqnarray}

%où le déplacement de diffusion $\Delta (\theta)$ est donné par la dérivée de la phase de diffusion,

%\begin{eqnarray}\label{eq:I-1-16}
%	\Delta ( \theta ) & \doteq & \frac{ d \Phi }{ d \theta } ( \theta )= \frac{ 2 c }{ c^2 + \theta^2}  	
%\end{eqnarray}

%Les deux particules sont retardées : leur position après la collision est la même que si elles étaient en retard respectivement de $\delta t_1 = \Delta ( \theta_2 - \theta_1 )/v_1 $ et $\delta t_2 = \Delta ( \theta_2 - \theta_1 )/v_2 $ .

%..................................;;

%%%%%%%%%%%%%%%%%%%%
%\paragraph{Retour aux coordonnées du laboratoire.}
%En revenant aux coordonnées d'origine (celles du laboratoire), on constate que la fonction d'onde à deux corps 
%\(
%\varphi_{\{\theta_1 , \theta_2\}} (z_1, z_2) = \langle 0 \vert \operator{\Psi} (z_1)\operator{\Psi} (z_2) \vert \{\theta_1, \theta_2\} \rangle
%\)
%doit satisfaire une condition de raccord au point \(z_1 = z_2\), en lien avec la discontinuité de ses dérivées due à l’interaction locale. Cette équation peut être résolue en utilisant l’\textbf{Ansatz de Bethe}, en supposant que la fonction d’onde est une superposition symétrisée d’ondes planes dans les deux domaines \(z_1 < z_2\) et \(z_1 > z_2\) :


%{\color{lightgray} 
%Dans notre systène ne compte que le difference des position $z_1$ et $z_2$. Donc je suis convaincu que  
%\begin{eqnarray*}
%	\underset{ \epsilon \to 0^+ }{\lim}	&=& \underset{ z_1 \to z_2^+ }{\lim} = \underset{ z_2 \to z_1^- }{\lim},\\
%	\underset{ \epsilon \to 0^- }{\lim}	&=& \underset{ z_1 \to z_2^- }{\lim} = \underset{ z_2 \to z_1^+ }{\lim},  \\
%	\mbox{soit   } \underset{ \epsilon \to 0^\pm  }{\lim}	&=& \underset{ z_1 \to z_2^\pm  }{\lim} = \underset{ z_2 \to z_1^\mp  }{\lim}.
%\end{eqnarray*}
%et par le même raisonnement, je suis convaincu que
%\begin{eqnarray*}
%	\underset{ z_2 \to z_1^\pm  }{\lim}	\partial_{z_1} \varphi (z_1, z_2) & = & \underset{ z_1 \to z_2^\mp  }{\lim}	\partial_{z_2} \varphi (z_1, z_2)
%\end{eqnarray*}

%Donc 

%\begin{eqnarray*}
%	\underset{ \epsilon \to 0 }{\lim}  \left [ \partial_Y \varphi(\epsilon) - \partial_Y \varphi(-\epsilon) \right ] & = & 	\underset{ \epsilon \to 0^+ }{\lim}   \partial_Y \varphi(\epsilon) 	 - \underset{ \epsilon \to 0^- }{\lim}  \partial_Y \varphi(\epsilon),\\
%	& = &  \frac{1}2 \underset{ z_2 \to z_1^+ }{\lim} \left [ \left ( \partial_{z_1} -\partial_{z_2} \right )\varphi ( z_1 , z_2 ) \right ] -  \frac{1}2 \underset{ z_2 \to z_1^- }{\lim} \left [ \left ( \partial_{z_1} -\partial_{z_2} \right )\varphi ( z_1 , z_2 ) \right ],\\
%	& = & \underset{ z_2 \to z_1^+ }{\lim} \left [ \left ( \partial_{z_1} -\partial_{z_2} \right )\varphi ( z_1 , z_2 ) \right ]
%\end{eqnarray*}

%Ainsi 
%}
%\begin{eqnarray}
%	\underset{ \vert z_1 - z_2 \vert  \to 0^+ }{\lim}  \left [  \operator{\partial}_{z_2}  \varphi _2( z_1 , z_2 ) - \operator{\partial}_{z_1}  \varphi_{\{\theta_1 , \theta_2\}} ( z_1 , z_2 ) - c  \varphi_{\{\theta_1 , \theta_2\}} ( z_1 , z_2 ) \right ] & = &0.  		
%\end{eqnarray}

%Les mêmes conditions s'appliquent lorsque $z_1$ est échangé avec $z_2$, puisque la fonction d'onde est symétrique. Ainsi, les états propres de l'équation (\ref{chap:eq.twobody}) sont

%%%%%%%%%%%%%%%%%%%%%%%%%%%%%%%%%%
%\paragraph{Condition de continuité et symétrie de la fonction d’onde et  Forme explicite de la fonction d’onde à deux corps. {\color{red}(à revoir)}}
%Comme la fonction d’onde est symétrique, l’échange des positions $z_1 \leftrightarrow z_2$ ne modifie pas la structure du problème. Les états propres de l’équation (\ref{chap:eq.twobody}) s’écrivent donc


%\begin{eqnarray}\label{eq:I-1-10}
%	\varphi_{\{\theta_1 , \theta_2\}}(z_1 , z_2) & \propto &  \left \{ \begin{array} { c cl} ( \theta_2 - \theta_1 - ic) e^{ i z_1 \theta_1 + iz_2 \theta_2 } - ( \theta_1 - \theta_2 - ic) e^{ i z_1 \theta_2 + iz_2 \theta_1} & \mbox{si} & z_1 < z_2 \\ (z_1 \leftrightarrow z_2) & \mbox{si} & z_1 > z_2 \end{array} \right.
%\end{eqnarray}

%correspondant aux valeurs propres $(\theta_1^2 + \theta_2^2)/2$. Pour $\theta_1 > \theta_2$, les deux termes $e^{iz_1 \theta_1 + iz_2 \theta_2 }$ et $e^{iz_1 \theta_2 + iz_2 \theta_1 }$ correspondent aux paires de particules entrantes et sortantes dans un processus de diffusion à deux corps. Le rapport de leurs amplitudes est la phase de diffusion à deux corps,

%et correspondent aux valeurs propres $E = (\theta_1^2 + \theta_2^2)/2$. Lorsque $\theta_1 > \theta_2$, les deux exponentielles $e^{i z_1 \theta_1 + i z_2 \theta_2}$ et $e^{i z_1 \theta_2 + i z_2 \theta_1}$ peuvent être interprétées comme les configurations entrante et sortante d’un processus de diffusion à deux corps. 

%%%%%%%%%%%%%%%%%%%%%%%%%%%%%%%%%%%
%\paragraph{Phase de diffusion à deux corps. {\color{red}(à revoir)}}
%Le rapport entre leurs amplitudes définit la phase de diffusion :

%\begin{eqnarray}\label{eq:I-1-11}
%	e^{i\Phi ( \theta_1 - \theta_2 ) } & \doteq & \frac{\theta_1 - \theta_2 - ic} { \theta_2 - \theta_1 - ic}. 
%\end{eqnarray}

%Une expression équivalente pour cette phase, souvent utilisée dans la littérature et que nous utilisons également ci-dessous, est 
%\begin{eqnarray}\label{chap:1:eq:Phi}
%	\Phi ( \theta )  & =  & 2 \arctan ( \theta/c) \in [ - \pi , \pi ].	
%\end{eqnarray}
%$\Phi ( \theta ) = 2 \arctan ( \theta/c) \in [ - \pi , \pi ] $.\\
%Une expression équivalente, fréquemment rencontrée dans la littérature et que nous adopterons par la suite, est donnée par :$\Phi ( \theta ) = 2 \arctan ( \theta/c) \in [ - \pi , \pi ] $.

%\paragraph{Lien entre phase de diffusion et décalage temporel : interprétation semi-classique. {\color{red}(à revoir)}}
%Il a été souligné par {\color{blue}Eisenbud (1948)} et {\color{blue}Wigner (1955)} que la phase de diffusion peut être interprétée de manière semi-classique comme un "décalage temporel". Esquissons brièvement l'argument de {\color{blue}Wigner (1955)}. Tout d'abord, notons que, pour une particule unique, une approximation simple d'un paquet d'ondes peut être obtenue en superposant deux ondes planes avec des moments $\theta$ et $\theta + \delta \theta $, respectivement,

%\begin{eqnarray}
%	e^{iz \theta } + e^{ i z ( \theta + \delta \theta )}.
%\end{eqnarray}

%\begin{figure}[H]
%	\centering
  %\includegraphics[width=0.5\textwidth]{}
  %\caption{Gauche : La fonction d'onde (\ref{eq:I-1-10}) sur la ligne infinie correspond à un processus de diffusion à deux corps. Semiclassiquement, la phase de diffusion dans ce processus à deux corps se reflète dans le décalage de diffusion (\ref{eq:I-1-16}) : après la collision, la position de la particule a été déplacée d'une distance $\Delta ( \theta_1 - \theta_2 )$ . Droite : La fonction d'onde de Bethe (\ref{eq:I-2-17}) sur la ligne infinie correspond à un processus de diffusion à $N$-corps qui se factorise en des processus à deux corps (le décalage de diffusion $\Delta$ est également présent ici, mais il n'est pas représenté dans la caricature). Dans ce processus à $N$-corps, les rapidités $\theta_j$ sont les moments asymptotiques des bosons.}
%  \label{}	
%\end{figure}

%Une telle superposition évolue dans le temps comme $e^{(iz\theta -it\varepsilon(\theta))} + e^{(iz(\theta + \delta \theta )  -it\varepsilon(\theta + \delta \theta ))}$, où $\varepsilon(\theta) = \theta ^2/2$ est l'énergie. Le centre de ce 'paquet d'ondes' se situe à la position où les phases des deux termes coïncident, c'est-à-dire au point où $z\delta \theta  - t[\varepsilon(\theta + \delta \theta ) - \varepsilon(\theta)] = 0$, ce qui donne $z \simeq vt$ avec la vitesse de groupe $v = d\varepsilon/d\theta = \theta$. Ainsi, il s'agit effectivement d'un 'paquet d'ondes' se déplaçant à la vitesse $\theta$. Ensuite, considérons deux particules entrantes dans un état tel que le centre de masse $Z = (z_1 + z_2)/2$ ait une impulsion $\theta_1 - \theta_2$, tandis que la coordonnée relative $Y = z_1 - z_2$ se trouve dans un 'paquet d'ondes' se déplaçant à la vitesse $ (\theta_1 - \theta_2)/2$,

%{\color{gray}
%\begin{eqnarray*}
%	\frac{ z_1 + z_2}{2} ( \theta_1 - \theta_2 ) + ( z_1 - z_2 ) \left ( \frac{ \theta_1 -\theta_2}{2} + \delta \theta  \right ) & = & z_1 ( \theta_1  + \delta \theta ) + z_2 ( \theta_2 - \delta \theta )  
%\end{eqnarray*}
%}

%\begin{eqnarray}
%	\psi_{inc} ( z_1 , z_2 ) & \propto & e^{i \frac{ z_1 + z_2 }{2} ( \theta_1 + \theta_2 ) }  \left ( e^{i ( z_1 - z_2 ) \frac{ \theta_1 - \theta_2}{2} } +  e^{i ( z_1 - z_2 )  \left ( \frac{ \theta_1 - \theta_2}{2}+ \delta \theta  \right )  } \right )\notag  \\
%	& \propto & e^{ i z_1 \theta_1 + i z_2 \theta_2 } + 	 e^{ i z_1 ( \theta_1 + \delta \theta )  + i z_2  ( \theta_2 - \delta \theta)   }%\tag{2}
%\end{eqnarray}

%Selon les équations (\ref{eq:I-1-10}) et (\ref{eq:I-1-11}), l'état sortant correspondant serait :

%\begin{eqnarray}
%	\psi_{outc} ( z_1 , z_2 ) & \propto & - e^{i\phi ( \theta_1 - \theta_2 ) }  e^{ iz_1 \theta_2 + i z_2 \theta_1 }  - e^{i\phi ( \theta_1 - \theta_2 + 2\delta \theta ) }  e^{ iz_1 (\theta_2- \delta \theta )  + i z_2 (\theta_1 + \delta \theta )  } \notag  \\
%	& \propto & e^{i \frac{ z_1 + z_2 }{2} ( \theta_1 + \theta_2 ) }  \left ( - e^{ i \phi ( \theta_1 - \theta_2 ) }e^{i(z_1 - z_2 ) \frac{ \theta_1 - \theta_2}{2} } -e^{ i \phi ( \theta_1 - \theta_2 + 2 \delta \theta ) }e^{i(z_1 - z_2 )  \left ( \frac{ \theta_1 - \theta_2}{2} + \delta \theta \right )  }  \right ). %\tag{2}
%\end{eqnarray}

%En répétant l'argument précédent de la stationnarité de phase, on trouve que la coordonnée relative est à la position $z_1 - z_2 \simeq (\theta_1 - \theta_2)t - 2d\Phi /d\theta$ au moment $t$. Étant donné que le centre de masse n'est pas affecté par la collision et se déplace à la vitesse de groupe $\tilde{\theta} =(\theta_1 + \theta_2)/2$, nous constatons que la position des deux particules semiclassiques après la collision sera

%\begin{eqnarray}
%	z_1 \simeq \theta _1 t - \Delta ( \theta_2 -\theta_1 ), & &	z_2 \simeq \theta _2 t - \Delta ( \theta_2 -\theta_1 ),
%\end{eqnarray}

%où le déplacement de diffusion $\Delta (\theta)$ est donné par la dérivée de la phase de diffusion,

%\begin{eqnarray}\label{eq:I-1-16}
%	\Delta ( \theta ) & \doteq & \frac{ d \Phi }{ d \theta } ( \theta )= \frac{ 2 c }{ c^2 + \theta^2}  	
%\end{eqnarray}

%Les deux particules sont retardées : leur position après la collision est la même que si elles étaient en retard respectivement de $\delta t_1 = \Delta ( \theta_2 - \theta_1 )/v_1 $ et $\delta t_2 = \Delta ( \theta_2 - \theta_1 )/v_2 $ .

%...........................................

%Cette équation décrit deux particules libres, sauf lorsqu'elles se rencontrent (i.e., \(z_1 = z_2\)), où l'interaction à contact produit un potentiel delta de Dirac.

%\paragraph{Méthode de Bethe}

%On peut résoudre cette équation en utilisant l’**Ansatz de Bethe**, en supposant que la fonction d’onde prend la forme d’une superposition symétrisée d’ondes planes dans les domaines \(z_1 < z_2\) et \(z_1 > z_2\) :

%\begin{eqnarray}
%	\varphi_{\{\theta_1 , \theta_2\}}(z_1, z_2) = \begin{cases}
%		A_{12} e^{i(k_1 z_1 + k_2 z_2)} + A_{21} e^{i(k_2 z_1 + k_1 z_2)} & \text{si } z_1 < z_2, \\
%		A_{12} e^{i(k_1 z_2 + k_2 z_1)} + A_{21} e^{i(k_2 z_2 + k_1 z_1)} & \text{si } z_1 > z_2.
%	\end{cases}
%\end{eqnarray}

%La continuité de \(\varphi_{\{\theta_1 , \theta_2\}}\) à \(z_1 = z_2\), ainsi que le saut de sa dérivée transverse imposé par le potentiel delta, permettent de relier les coefficients \(A_{12}\) et \(A_{21}\). On obtient la condition de diffusion bosonique :

%\begin{eqnarray}
%	\frac{A_{21}}{A_{12}} & = & - \frac{ic(k_1 - k_2) + (k_1 - k_2)^2}{ic(k_1 - k_2) - (k_1 - k_2)^2}.
%\end{eqnarray}

%L’énergie totale est simplement la somme des énergies individuelles :

%\begin{eqnarray}
%	E = \frac{k_1^2 + k_2^2}{2},
%\end{eqnarray}

%et les \(k_j\) doivent satisfaire les conditions aux limites périodiques, ce qui donne un système transcendantal pour les \(k_j\) dépendant du couplage \(c\) (voir chapitre sur l'équation de Bethe pour \(N\) particules).

%\vspace{1em}
%Ce cas à deux particules constitue une étape fondamentale vers la compréhension complète du système à \(N\) bosons, où les structures d’interférences entre ondes multiples et les effets d’interactions deviennent de plus en plus riches.
