Agir avec cet opérateur sur la fonction propre (\ref{chap:1:2.part}) donne 
%\begin{eqnarray}
	%\operator{H}\vert \psi ( \theta_1 , \cdots , \theta_N ) \rangle & = & \left \{ \begin{array}{l}- \frac{1}{ \sqrt{N!}} \int dx \int d^N z \, \chi_N ( z_1 , \cdots , z_N  ~\vert ~ \theta_1 , \cdots , \theta _N ) \,	\left [\operator{\partial}_x^2 \operator{\Psi}^\dag(x)\right ] \operator{\Psi}(x)\,  \operator{\Psi}^\dag(z_1) \cdots \operator{\Psi}^\dag(z_N) \vert 0 \rangle \\ + \\ \frac{c}{ \sqrt{N!}} \int dx \int d^N z \, \chi_N ( z_1 , \cdots , z_N  ~\vert ~ \theta_1 , \cdots , \theta _N ) 	\, \operator{\Psi}^\dag(x)\operator{\Psi}^\dag(x) \operator{\Psi}(x)\operator{\Psi}(x)  \operator{\Psi}^\dag(z_1) \cdots \operator{\Psi}^\dag(z_N) \vert 0 \rangle  \end{array}\right.\label{chap:1:hal.mod.3}
%\end{eqnarray}

\begin{eqnarray}
	& & - \frac{1}{ \sqrt{2}} \int dx \iint dz_1 dz_2 \, \varphi_2(z_1, z_2) \,	\left [\frac{1}{2}\operator{\partial}_x^2 \operator{\Psi}^\dag(x)\right ] \operator{\Psi}(x)\,  \operator{\Psi}^\dag(z_1)\operator{\Psi}^\dag(z_2) \vert 0 \rangle \label{chap:1:hal.mod.2.part.1}\\
	\operator{H}\vert 2 \rangle & = & \nonumber \\
	& & +\frac{c}{ 2\sqrt{2}} \int dx  \iint dz_1 dz_2 \,  \varphi_2(z_1, z_2) 	\, \operator{\Psi}^\dag(x)\operator{\Psi}^\dag(x) \operator{\Psi}(x)\operator{\Psi}(x)  \operator{\Psi}^\dag(z_1) \operator{\Psi}^\dag(z_2) \vert 0 \rangle \label{chap:1:hal.mod.2.part.2} 
\end{eqnarray}

Les règles de commutations (\ref{chap:1:com.1}) impliquent que 

\begin{eqnarray}
	\left . \begin{array}{rcl}
		[ \operator{\Psi}(x),  \operator{\Psi}^\dag(z_1) \operator{\Psi}^\dag(z_2)  ]  &=&  \operator{\Psi}^\dag(z_2)  \operator{\delta}(x-z_1) +  \operator{\Psi}^\dag(z_1) \operator{\delta}(x-z_2)  \\
		\left [ \operator{\partial}_x\operator{\Psi}^\dag(x),  \operator{\Psi}^\dag(z) \right ]   & =  & 0 
	\end{array} \right . \label{chap:1:2.part.com.2}
\end{eqnarray}


En utilisant ces dernier règles de commutations et la définition d'état de Fock (\ref{chap:eq.vide.fock}) , la premiers partie de l'application de l'hamiltonien sur l'état $\vert 2  \rangle$ , (\ref{chap:1:hal.mod.2.part.1}) de simplifie en 

\begin{eqnarray}
	 - \frac{1}{ \sqrt{2}} \iint dz_1 dz_2 \, \varphi_2(z_1, z_2) \,	 \left \{ \operator{\Psi}^\dag(z_1) \left [\frac{1}{2}\operator{\partial}_{z_2}^2 \operator{\Psi}^\dag(z_2) \right ] + \operator{\Psi}^\dag(z_2) \left [\frac{1}{2}\operator{\partial}_{z_1}^2 \operator{\Psi}^\dag(z_1)  \right]  \right \}  \vert 0 \rangle 	
\end{eqnarray}

Et en faisant deux integration par partie selon la variable $z_{1,2}$ , cette premier partie devient

\begin{eqnarray}
	 - \frac{1}{ \sqrt{2}}  \iint dz_1 dz_2  \,  \frac{1}{2} \left \{ \operator{\partial}_{z_1}^2 +  \operator{\partial}_{z_2}^2 \right\} \varphi_2(z_1, z_2) \,	 \operator{\Psi}^\dag(z_1) \operator{\Psi}^\dag(z_2) \vert 0 \rangle \label{chap:1:hal.mod.2.part.1.1}	
\end{eqnarray}

Pour la seconde partie  , en remarquant que Les règles de commutations (\ref{chap:1:com.1}) impliquent que 

\begin{eqnarray}
	[ \operator{\Psi}(x) \operator{\Psi}(x),  \operator{\Psi}^\dag(z) ] & =& 2\operator{\Psi}(x)\operator{\delta}(x - z)\label{chap:1:com.3}  		
\end{eqnarray}

et en remplaçant $\operator{\Psi}(x)$ par $\operator{\Psi}(x)\operator{\Psi}(x)$ dans  (\ref{chap:1:2.part.com.2}) il vient que  

\begin{eqnarray}
	[ \operator{\Psi}(x)\operator{\Psi}(x),  \operator{\Psi}^\dag(z_1) \operator{\Psi}^\dag(z_2)  ]  &=&  2 \left \{   \operator{\delta}(x-z_1)\operator{\Psi}(x)\operator{\Psi}^\dag(z_2)  + \operator{\delta}(x-z_2)\operator{\Psi}^\dag(z_1)  \operator{\Psi}(x)\right \} \label{chap:1:.2.part.com.4}		
\end{eqnarray}

et en injectant (\ref{chap:1:2.part.com.2}),  (\ref{chap:1:.2.part.com.4}) devient 

\begin{eqnarray}
	[ \operator{\Psi}(x)\operator{\Psi}(x),  \operator{\Psi}^\dag(z_1)\operator{\Psi}^\dag(z_2)  ]  &=& 2 \left \{\operator{\delta}(x-z_1)\operator{\delta}(x-z_2) +   \operator{\delta}(x-z_1)\operator{\Psi}^\dag(z_2)\operator{\Psi}(x)  + \operator{\delta}(x-z_2)\operator{\Psi}^\dag(z_1)  \operator{\Psi}(x)\right \} \label{chap:1:2.part.com.5}	.	
\end{eqnarray}

En utilisant la régle de commutation (\ref{chap:1:2.part.com.5}) et la définition de l'état de Fock (\ref{chap:eq.vide.fock}) , la seconde partie de (\ref{chap:1:hal.mod.2.part.2}) devient

\begin{eqnarray}
	 \frac{1}{ \sqrt{2}}  \iint dz_1 dz_2 \,  \,c \, \operator{\delta}(z_1 - z_2)  \varphi_2(z_1, z_2) \, \operator{\Psi}^\dag(z_1) \operator{\Psi}^\dag(z_2) \vert 0 \rangle \label{chap:1:hal.mod.2.part.2.2}	
\end{eqnarray} 

en utilisant (\ref{chap:1:hal.mod.2.part.1.1}) et (\ref{chap:1:hal.mod.2.part.2.2}) 


\begin{eqnarray}
	\operator{H}\vert 1 \rangle &= &  \frac{1}{ \sqrt{2}}  \iint dz_1 dz_2 \,  	\left [ \operator{\mathcal{H}}_2 \varphi_2(z_1, z_2) \right ] \, \operator{\Psi}^\dag(z_1) \operator{\Psi}^\dag(z_2) \vert 0 \rangle		
\end{eqnarray}

avec 

\begin{eqnarray}
	\operator{\mathcal{H}}_2 & = &  - \frac{1}{2} \operator{\partial}_{z_1}^2 - \frac{1}{2} \operator{\partial}_{z_2}^2+ 	c  \operator{\delta}(z_1 - z_2). \label{chap:1:hal.mod.2.part.3} 		
\end{eqnarray}
