Nous considérons à présent le cas de deux bosons quantiques dans la même boîte unidimensionnelle de longueur \(L\), avec des conditions aux limites périodiques. Contrairement au cas à une particule, le terme d’interaction à contact intervient dans la dynamique.

On cherche à déterminer les états propres \(\vert 2 \rangle\) de l’Hamiltonien \(\operator{H}\), satisfaisant l’équation de Schrödinger stationnaire :

\begin{eqnarray}
	\operator{H} \vert 2 \rangle & = & E \vert 2 \rangle.
\end{eqnarray}


L’état à deux particules peut être écrit dans la base positionnelle de Fock sous la forme :

\begin{eqnarray}
	\vert 2 \rangle & = & \frac{1}{\sqrt{2}}\iint dz_1dz_2 \, \vert z_1 , z_2 \rangle \langle z_1 , z_2  \vert 2  \rangle ~=~\frac{1}{\sqrt{2}} \iint dz_1\,dz_2 \, \varphi_2(z_1, z_2) \operator{\Psi}^\dag(z_1) \operator{\Psi}^\dag(z_2) \vert 0 \rangle, \ref{chap:1:2.part}
\end{eqnarray}

où la base spatiale à deux particule bosomique $\vert z_1 , z_2 \rangle$ s'écrit pour $z_1 \neq  z_2$

\begin{eqnarray}
	\vert z_1 , z_2 \rangle  & = &  \operator{\Psi}^\dag(z_1) \operator{\Psi}^\dag(z_2) \vert 0 \rangle ,
\end{eqnarray}

orthonormalisée car car avec en utilisant les relations de commutation (\ref{chap:1:com.1})  et la définition du vide de Fock (\ref{chap:eq.vide.fock}) il vient que 

\begin{eqnarray}
	\langle z_1 , z_2 \vert z_1' , z_2' \rangle  & = & \operator{\delta}(z_1 -z_1')\operator{\delta}(z_2 -z_2')  +   \operator{\delta}(z_1 -z_2')\operator{\delta}(z_2 -z_1'),
\end{eqnarray}

et si $z_1 =  z_2 = z$
\begin{eqnarray}
	\vert z , z \rangle  & = &  \frac{1}{\sqrt{2}} \left(\operator{\Psi}^\dag(z)\right)^2  \vert 0 \rangle ,
\end{eqnarray}

orthonormalisée car 

\begin{eqnarray}
	\langle z , z \vert z' , z' \rangle  & = & \operator{\delta}(z -z'),
\end{eqnarray}

et \(\varphi_2(z_1, z_2)\) tel que 

\begin{eqnarray}
	\varphi_2(z_1, z_2) & = & 	\langle 0 \vert  \operator{\Psi}(z_1) \operator{\Psi}(z_2)\vert  2  \rangle 
\end{eqnarray}


est la fonction d’onde symétrique :

\begin{eqnarray}
	\varphi_2(z_1, z_2) = \varphi_2(z_2, z_1),
\end{eqnarray}

et normalisée selon :

\begin{eqnarray}
	\langle 2 \vert 2 \rangle ~=~ \int dz_1\, dz_2\, |\varphi_2(z_1, z_2)|^2 = 1.
\end{eqnarray}

L'Hamiltonien (\ref{chap:1:ham.mod}) peut être réécrit en effectuant une intégration par parties. 

\begin{eqnarray}
	\operator{H} & = & \int dx \left \{ -\left [\frac{1}{2}\operator{\partial}_x^2 \operator{\Psi}^\dag(x)\right ] \operator{\Psi}(x) + \frac{c}{2} \operator{\Psi}^\dag (x) \operator{\Psi}^\dag (x) \operator{\Psi} (x) \operator{\Psi} (x) \right \} \label{chap:1:ham.mod.2}.
\end{eqnarray}

L’action de \(\operator{H}\) sur \(\vert 2 \rangle\) fournit une équation de Schrödinger pour la fonction d’onde à deux particules. 

Agir avec cet opérateur sur la fonction propre (\ref{chap:1:2.part}) donne 
%\begin{eqnarray}
	%\operator{H}\vert \psi ( \theta_1 , \cdots , \theta_N ) \rangle & = & \left \{ \begin{array}{l}- \frac{1}{ \sqrt{N!}} \int dx \int d^N z \, \chi_N ( z_1 , \cdots , z_N  ~\vert ~ \theta_1 , \cdots , \theta _N ) \,	\left [\operator{\partial}_x^2 \operator{\Psi}^\dag(x)\right ] \operator{\Psi}(x)\,  \operator{\Psi}^\dag(z_1) \cdots \operator{\Psi}^\dag(z_N) \vert 0 \rangle \\ + \\ \frac{c}{ \sqrt{N!}} \int dx \int d^N z \, \chi_N ( z_1 , \cdots , z_N  ~\vert ~ \theta_1 , \cdots , \theta _N ) 	\, \operator{\Psi}^\dag(x)\operator{\Psi}^\dag(x) \operator{\Psi}(x)\operator{\Psi}(x)  \operator{\Psi}^\dag(z_1) \cdots \operator{\Psi}^\dag(z_N) \vert 0 \rangle  \end{array}\right.\label{chap:1:hal.mod.3}
%\end{eqnarray}

\begin{eqnarray}
	& & - \frac{1}{ \sqrt{2}} \int dx \iint dz_1 dz_2 \, \varphi_2(z_1, z_2) \,	\left [\frac{1}{2}\operator{\partial}_x^2 \operator{\Psi}^\dag(x)\right ] \operator{\Psi}(x)\,  \operator{\Psi}^\dag(z_1)\operator{\Psi}^\dag(z_2) \vert 0 \rangle \label{chap:1:hal.mod.2.part.1}\\
	\operator{H}\vert 2 \rangle & = & \nonumber \\
	& & +\frac{c}{ 2\sqrt{2}} \int dx  \iint dz_1 dz_2 \,  \varphi_2(z_1, z_2) 	\, \operator{\Psi}^\dag(x)\operator{\Psi}^\dag(x) \operator{\Psi}(x)\operator{\Psi}(x)  \operator{\Psi}^\dag(z_1) \operator{\Psi}^\dag(z_2) \vert 0 \rangle \label{chap:1:hal.mod.2.part.2} 
\end{eqnarray}

Les règles de commutations (\ref{chap:1:com.1}) impliquent que 

\begin{eqnarray}
	\left . \begin{array}{rcl}
		[ \operator{\Psi}(x),  \operator{\Psi}^\dag(z_1) \operator{\Psi}^\dag(z_2)  ]  &=&  \operator{\Psi}^\dag(z_2)  \operator{\delta}(x-z_1) +  \operator{\Psi}^\dag(z_1) \operator{\delta}(x-z_2)  \\
		\left [ \operator{\partial}_x\operator{\Psi}^\dag(x),  \operator{\Psi}^\dag(z) \right ]   & =  & 0 
	\end{array} \right . \label{chap:1:2.part.com.2}
\end{eqnarray}


En utilisant ces dernier règles de commutations et la définition d'état de Fock (\ref{chap:eq.vide.fock}) , la premiers partie de l'application de l'hamiltonien sur l'état $\vert 2  \rangle$ , (\ref{chap:1:hal.mod.2.part.1}) de simplifie en 

\begin{eqnarray}
	 - \frac{1}{ \sqrt{2}} \iint dz_1 dz_2 \, \varphi_2(z_1, z_2) \,	 \left \{ \operator{\Psi}^\dag(z_1) \left [\frac{1}{2}\operator{\partial}_{z_2}^2 \operator{\Psi}^\dag(z_2) \right ] + \operator{\Psi}^\dag(z_2) \left [\frac{1}{2}\operator{\partial}_{z_1}^2 \operator{\Psi}^\dag(z_1)  \right]  \right \}  \vert 0 \rangle 	
\end{eqnarray}

Et en faisant deux integration par partie selon la variable $z_{1,2}$ , cette premier partie devient

\begin{eqnarray}
	 - \frac{1}{ \sqrt{2}}  \iint dz_1 dz_2  \,  \frac{1}{2} \left \{ \operator{\partial}_{z_1}^2 +  \operator{\partial}_{z_2}^2 \right\} \varphi_2(z_1, z_2) \,	 \operator{\Psi}^\dag(z_1) \operator{\Psi}^\dag(z_2) \vert 0 \rangle \label{chap:1:hal.mod.2.part.1.1}	
\end{eqnarray}

Pour la seconde partie  , en remarquant que Les règles de commutations (\ref{chap:1:com.1}) impliquent que 

\begin{eqnarray}
	[ \operator{\Psi}(x) \operator{\Psi}(x),  \operator{\Psi}^\dag(z) ] & =& 2\operator{\Psi}(x)\operator{\delta}(x - z)\label{chap:1:com.3}  		
\end{eqnarray}

et en remplaçant $\operator{\Psi}(x)$ par $\operator{\Psi}(x)\operator{\Psi}(x)$ dans  (\ref{chap:1:2.part.com.2}) il vient que  

\begin{eqnarray}
	[ \operator{\Psi}(x)\operator{\Psi}(x),  \operator{\Psi}^\dag(z_1) \operator{\Psi}^\dag(z_2)  ]  &=&  2 \left \{   \operator{\delta}(x-z_1)\operator{\Psi}(x)\operator{\Psi}^\dag(z_2)  + \operator{\delta}(x-z_2)\operator{\Psi}^\dag(z_1)  \operator{\Psi}(x)\right \} \label{chap:1:.2.part.com.4}		
\end{eqnarray}

et en injectant (\ref{chap:1:2.part.com.2}),  (\ref{chap:1:.2.part.com.4}) devient 

\begin{eqnarray}
	[ \operator{\Psi}(x)\operator{\Psi}(x),  \operator{\Psi}^\dag(z_1)\operator{\Psi}^\dag(z_2)  ]  &=& 2 \left \{\operator{\delta}(x-z_1)\operator{\delta}(x-z_2) +   \operator{\delta}(x-z_1)\operator{\Psi}^\dag(z_2)\operator{\Psi}(x)  + \operator{\delta}(x-z_2)\operator{\Psi}^\dag(z_1)  \operator{\Psi}(x)\right \} \label{chap:1:2.part.com.5}	.	
\end{eqnarray}

En utilisant la régle de commutation (\ref{chap:1:2.part.com.5}) et la définition de l'état de Fock (\ref{chap:eq.vide.fock}) , la seconde partie de (\ref{chap:1:hal.mod.2.part.2}) devient

\begin{eqnarray}
	 \frac{1}{ \sqrt{2}}  \iint dz_1 dz_2 \,  \,c \, \operator{\delta}(z_1 - z_2)  \varphi_2(z_1, z_2) \, \operator{\Psi}^\dag(z_1) \operator{\Psi}^\dag(z_2) \vert 0 \rangle \label{chap:1:hal.mod.2.part.2.2}	
\end{eqnarray} 

en utilisant (\ref{chap:1:hal.mod.2.part.1.1}) et (\ref{chap:1:hal.mod.2.part.2.2}) 

\begin{eqnarray}
	\operator{H}\vert 1 \rangle &= &  \frac{1}{ \sqrt{2}}  \iint dz_1 dz_2 \,  	\left [ \operator{\mathcal{H}}_2 \varphi_2(z_1, z_2) \right ] \, \operator{\Psi}^\dag(z_1) \operator{\Psi}^\dag(z_2) \vert 0 \rangle		
\end{eqnarray}

avec 

\begin{eqnarray}
	\operator{\mathcal{H}}_2 & = &  - \frac{1}{2} \operator{\partial}_{z_1}^2 - \frac{1}{2} \operator{\partial}_{z_2}^2+ 	c  \operator{\delta}(z_1 - z_2). \label{chap:1:hal.mod.2.part.3} 		
\end{eqnarray}


Après développement en utilisant les relations de commutation bosoniques et les propriétés du vide de Fock, on obtient l’équation suivante :

\begin{eqnarray}
	\operator{\mathcal{H}}_2 \varphi_2(z_1, z_2) = E \varphi_2(z_1, z_2). \label{chap:eq.twobody}
\end{eqnarray}


...........................

%Il est instructif de commencer par le cas de $N = 2$ particules sur une ligne infinie. 

En première quantification, en utilisant les coordonnées du centre de masse et relatives $Z = (z_1 + z_2)/2$ et $Y = z_1 - z_2$,

{\color{gray}  
\begin{eqnarray*}
	\left ( \begin{array}{c} Z \\ Y \end{array} \right ) = \left ( \begin{array}{cc} 1/2 &  1/2  \\ 1 & - 1  \end{array} \right ) \left ( \begin{array}{c} z_1 \\ z_2 \end{array} \right ), & & \left ( \begin{array}{c} z_1 \\ z_2 \end{array} \right ) = \left ( \begin{array}{cc} 1 &  1/2  \\ 1 & - 1/2  \end{array} \right ) \left ( \begin{array}{c} Z \\ Y \end{array} \right )
\end{eqnarray*}


\begin{Propr}
	Soient deux application $f \colon U \subset	 \mathbb{R}^n \rightarrow \mathbb{R} \mbox{ et } \varphi \colon V \subset \mathbb{R}^m \rightarrow \mathbb{R}$ ( où $U$ et $V$ sont ouverts) telles que $\varphi(V) \subset U$. On écrit $\varphi$ sous la forme $\varphi = ( \varphi_1 , \cdots , \varphi_n )$ où $\varphi_i \colon V \rightarrow \mathbb{R} $ pour tous $i$. Soit $a \in V $ tel que $\varphi$ est différentiable en $a$ et $f$ est différentiable en $\varphi(a)$. Alors l'application $F = f \circ \varphi \colon V \rightarrow \mathbb{R}$ est différentiable en $a$ et 
	\begin{eqnarray*}
		\forall j \in \llbracket 1 , m \rrbracket \colon \frac{\partial F}{\partial u_j} ( a ) = \sum_{i=1}^n \frac{\partial f}{ \partial x_i} (\varphi (a) ) \cdot \frac{\partial \varphi_i}{ \partial u_j} (a) 
	\end{eqnarray*} 
\end{Propr}

avec la régle de la chaine  $\partial_Z = \frac{\partial z_1}{\partial Z}\partial_{z_1} + \frac{\partial z_2}{\partial Z}\partial_{z_2}, \partial_Y = \frac{\partial z_1}{\partial Y}\partial_{z_1} + \frac{\partial z_2}{\partial Y}\partial_{z_2}$.

\begin{eqnarray*}
	\left ( \begin{array}{c} \partial_Z \\ \partial_Y \end{array} \right ) = \left ( \begin{array}{cc} 1 & 1  \\ 1/2 &  -1/2   \end{array} \right ) \left ( \begin{array}{c} \partial_{z_1} \\ \partial_{z_2} \end{array} \right ), & & \left ( \begin{array}{c} \partial_{z_1} \\ \partial_{z_2} \end{array} \right ) = \left ( \begin{array}{cc} 1/2 &  1  \\ 1/2 & - 1  \end{array} \right ) \left ( \begin{array}{c} \partial_Z \\ \partial_Y \end{array} \right )		
\end{eqnarray*}

\begin{eqnarray*}
	\partial_Z^2 & = & 	\partial_{z_1}^2 + \partial_{z_2}^2 + 2 \partial_{z_1} \partial_{z_2},\\
	4\partial_Y^2 & = & 	\partial_{z_1}^2 + \partial_{z_2}^2 - 2 \partial_{z_1} \partial_{z_2},
\end{eqnarray*}

Donc 


}

\begin{eqnarray*}
	\frac{1}{2} \operator{\partial}_Z^2 + 	2\operator{\partial}_Y^2 & = & \operator{\partial}_{z_1}^2 + \operator{\partial}_{z_2}^2
\end{eqnarray*}

l'hamiltonien (\ref{chap:1:hal.mod.2.part.3}) se divise en une somme de deux problèmes indépendants à une seule particule.

\begin{eqnarray}\label{chap:1:hal.mod.2.part.4}
	\operator{\mathcal{H}}_2 & = & 	-\frac{1}{4} \operator{\partial}_Z^2 - 	\operator{\partial}_Y^2 + c \operator{\delta} ( Y ) 
\end{eqnarray}

Les états propres de l'hamiltonien du centre de masse, $-\frac{1}{4} \partial_Z^2$, sont des ondes planes, et l'hamiltonien pour la coordonnée relative $Y$ correspond à celui d'une particule de masse 1/2 en présence d'un potentiel delta en $Y = 0$. 

\begin{eqnarray}\label{chap:1:hal.mod.2.part.5}
	- 	\operator{\partial}_Y^2 \tilde{\varphi}(Y) + c \operator{\delta} ( Y )\tilde{\varphi}(Y) & = & \tilde{E}\tilde{\varphi}_2(Y) 
\end{eqnarray}

Le système étant composé de particules bosoniques, on cherche une solution symétrique que l’on écrit sous la forme :


\begin{equation}
	\tilde{\varphi}(Y) = \sin\left( \frac{1}{2} (\tilde{\theta} |Y| + \Phi ) \right). \label{eq:ansatz.boson}
\end{equation}

En réinjectant l'équation \eqref{eq:ansatz.boson} dans l’équation \eqref{chap:1:hal.mod.2.part.5}, on obtient l’énergie propre de l’état lié $\tilde{E} = \frac{1}{2}\frac{\tilde{\theta}^2}{2}$.


En raison de la présence du potentiel delta centré en $Y = 0$, la dérivée première de la fonction d’onde $\tilde{\varphi}(Y)$ présente une discontinuité en ce point. En effet, le potentiel étant infini en $Y = 0$, la phase $\Phi$ du régime symétrique est déterminée en intégrant l’équation du mouvement autour de la singularité. En intégrant entre $- \epsilon$ et $+ \epsilon$ et en faisant tendre $\epsilon \to 0$, on obtient la condition de saut de la dérivée :

%avec $\Phi$ une phase à déterminer. %\begin{equation}
%	E = \frac{\tilde{m} \theta^2}{2}.
%\end{equation}

%La dérivée de la fonction d’onde n’est pas continue en $Y = 0$. Le potentiel étant infini en $Y = 0$, la phase $\Phi$ est obtenue en intégrant l’équation du mouvement entre $- \epsilon$ et $+ \epsilon$ et en faisant tendre $\epsilon$ vers zéro :


%En raison de ce potentiel delta, la dérivée première de la fonction d'onde $\varphi(Y)$ doit avoir une discontinuité en $Y = 0$ : 

{\color{gray} 
\begin{eqnarray*}
	\underset{ \epsilon \to 0 }{\lim} \int_{-\epsilon}^{+\epsilon}  	-\underbrace{\cancel{\frac{1}{4} \partial_Z^2\varphi(Y)}}_{0} - 	\partial_Y^2\varphi(Y) + c \delta ( Y )\varphi(Y) \, dY  & = & \underset{ \epsilon \to 0 }{\lim}  \int_{-\epsilon}^{+\epsilon}  E d Y , \\
	\underset{ \epsilon \to 0 }{\lim}  \left [ \varphi'(\epsilon) - \varphi'(-\epsilon) \right ] - c \varphi (  0 ) & =  &  -\underset{ \epsilon \to 0 }{\lim}  \int_{-\epsilon}^{+\epsilon}  E d Y,\\
	 \varphi'(0^+) - \varphi'(0^-) - c \varphi (  0 ) & = & 0 .
\end{eqnarray*}


}

\begin{eqnarray*}
	\underset{ \epsilon \to 0 }{\lim} \int_{-\epsilon}^{+\epsilon}  - 	\operator{\partial}_Y^2\tilde{\varphi}(Y) + c \operator{\delta} ( Y )\tilde{\varphi}(Y) \, dY  & = & \underset{ \epsilon \to 0 }{\lim}  \int_{-\epsilon}^{+\epsilon}  E d Y ,\\
	\\
	\tilde{\varphi}'(0^+) - \tilde{\varphi}'(0^-) - c \tilde{\varphi} (  0 ) & = & 0 .
\end{eqnarray*}


%soit $\tilde{\varphi}'(0^+) - \tilde{\varphi}'(0^-) - c \tilde{\varphi} (  0 )  =  0 $ .

Et en évaluant la discontinuité de sa dérivée au point $Y = 0$, on trouve que la phase $\Phi$ satisfait la condition :

\begin{equation}
	\tan\left( \frac{\Phi}{2} \right) = \frac{\tilde{\theta}}{c}.
\end{equation}

Cette relation exprime l’impact de l’interaction delta sur le déphasage de la solution liée. On en déduit que plus le couplage $c$ est fort, plus la phase $\Phi$ tend vers $\pi$, ce qui correspond à une fonction d’onde présentant un maximum en $Y = 0$. En revanche, dans la limite d’interaction faible ($c \to 0$), la phase $\Phi$ tend vers zéro et la fonction d’onde devient nulle en $Y = 0$.

En revenant aux coordonnées d'origine (les coordonnées du laboratoire) , on constate que la fonction d'onde à deux corps $\varphi_2 (z_1, z_2) = \langle 0 \vert \operator{\Psi} (z_1)\operator{\Psi} (z_2) \vert 2 \rangle$ satisfait à cette condition :\\

{\color{gray} 
Dans notre systène ne compte que le difference des position $z_1$ et $z_2$. Donc je suis convaincu que  
\begin{eqnarray*}
	\underset{ \epsilon \to 0^+ }{\lim}	&=& \underset{ z_1 \to z_2^+ }{\lim} = \underset{ z_2 \to z_1^- }{\lim},\\
	\underset{ \epsilon \to 0^- }{\lim}	&=& \underset{ z_1 \to z_2^- }{\lim} = \underset{ z_2 \to z_1^+ }{\lim},  \\
	\mbox{soit   } \underset{ \epsilon \to 0^\pm  }{\lim}	&=& \underset{ z_1 \to z_2^\pm  }{\lim} = \underset{ z_2 \to z_1^\mp  }{\lim}.
\end{eqnarray*}
et par le même raisonnement, je suis convaincu que
\begin{eqnarray*}
	\underset{ z_2 \to z_1^\pm  }{\lim}	\partial_{z_1} \varphi (z_1, z_2) & = & \underset{ z_1 \to z_2^\mp  }{\lim}	\partial_{z_2} \varphi (z_1, z_2)
\end{eqnarray*}

Donc 

\begin{eqnarray*}
	\underset{ \epsilon \to 0 }{\lim}  \left [ \partial_Y \varphi(\epsilon) - \partial_Y \varphi(-\epsilon) \right ] & = & 	\underset{ \epsilon \to 0^+ }{\lim}   \partial_Y \varphi(\epsilon) 	 - \underset{ \epsilon \to 0^- }{\lim}  \partial_Y \varphi(\epsilon),\\
	& = &  \frac{1}2 \underset{ z_2 \to z_1^+ }{\lim} \left [ \left ( \partial_{z_1} -\partial_{z_2} \right )\varphi ( z_1 , z_2 ) \right ] -  \frac{1}2 \underset{ z_2 \to z_1^- }{\lim} \left [ \left ( \partial_{z_1} -\partial_{z_2} \right )\varphi ( z_1 , z_2 ) \right ],\\
	& = & \underset{ z_2 \to z_1^+ }{\lim} \left [ \left ( \partial_{z_1} -\partial_{z_2} \right )\varphi ( z_1 , z_2 ) \right ]
\end{eqnarray*}

Ainsi 
}
\begin{eqnarray}
	\underset{ \vert z_1 - z_2 \vert  \to 0^+ }{\lim}  \left [  \operator{\partial}_{z_2}  \varphi _2( z_1 , z_2 ) - \operator{\partial}_{z_1}  \varphi_2 ( z_1 , z_2 ) - c  \varphi_2 ( z_1 , z_2 ) \right ] & = &0.  		
\end{eqnarray}

Les mêmes conditions s'appliquent lorsque $x_1$ est échangé avec $x_2$, puisque la fonction d'onde est symétrique. Ainsi, les états propres de l'équation (\ref{eq:I-1-8}) sont

\begin{eqnarray}\label{eq:I-1-10}
	\varphi(z_1 , z_2) & \propto &  \left \{ \begin{array} { c cl} ( \theta_2 - \theta_1 - ic) e^{ i z_1 \theta_1 + iz_2 \theta_2 } - ( \theta_1 - \theta_2 - ic) e^{ i z_1 \theta_2 + iz_2 \theta_1} & \mbox{si} & z_1 < z_2 \\ (z_1 \leftrightarrow z_2) & \mbox{si} & z_1 > z_2 \end{array} \right.
\end{eqnarray}

correspondant aux valeurs propres $(\theta_1^2 + \theta_2^2)/2$. Pour $\theta_1 > \theta_2$, les deux termes $e^{iz_1 \theta_1 + iz_2 \theta_2 }$ et $e^{iz_1 \theta_2 + iz_2 \theta_1 }$ correspondent aux paires de particules entrantes et sortantes dans un processus de diffusion à deux corps. Le rapport de leurs amplitudes est la phase de diffusion à deux corps,

\begin{eqnarray}\label{eq:I-1-11}
	e^{i\phi ( \theta_1 - \theta_2 ) } & \doteq & \frac{\theta_1 - \theta_2 - ic} { \theta_2 - \theta_1 - ic}. 
\end{eqnarray}

Une expression équivalente pour cette phase, souvent utilisée dans la littérature et que nous utilisons également ci-dessous, est $\phi ( \theta ) = 2 \arctan ( \theta/c) \in [ - \pi , \pi ] $.\\

Il a été souligné par {\color{blue}Eisenbud (1948)} et {\color{blue}Wigner (1955)} que la phase de diffusion peut être interprétée de manière semi-classique comme un "décalage temporel". Esquissons brièvement l'argument de {\color{blue}Wigner (1955)}. Tout d'abord, notons que, pour une particule unique, une approximation simple d'un paquet d'ondes peut être obtenue en superposant deux ondes planes avec des moments $\theta$ et $\theta + \delta \theta $, respectivement,

\begin{eqnarray}
	e^{iz \theta } + e^{ i z ( \theta + \delta \theta )}.
\end{eqnarray}

\begin{figure}[H]
	\centering
  %\includegraphics[width=0.5\textwidth]{}
  \caption{Gauche : La fonction d'onde (\ref{eq:I-1-10}) sur la ligne infinie correspond à un processus de diffusion à deux corps. Semiclassiquement, la phase de diffusion dans ce processus à deux corps se reflète dans le décalage de diffusion (\ref{eq:I-1-16}) : après la collision, la position de la particule a été déplacée d'une distance $\Delta ( \theta_1 - \theta_2 )$ . Droite : La fonction d'onde de Bethe (\ref{eq:I-2-17}) sur la ligne infinie correspond à un processus de diffusion à $N$-corps qui se factorise en des processus à deux corps (le décalage de diffusion $\Delta$ est également présent ici, mais il n'est pas représenté dans la caricature). Dans ce processus à $N$-corps, les rapidités $\theta_j$ sont les moments asymptotiques des bosons.}
  \label{}	
\end{figure}

Une telle superposition évolue dans le temps comme $e^{(iz\theta -it\varepsilon(\theta))} + e^{(iz(\theta + \delta \theta )  -it\varepsilon(\theta + \delta \theta ))}$, où $\varepsilon(\theta) = \theta ^2/2$ est l'énergie. Le centre de ce 'paquet d'ondes' se situe à la position où les phases des deux termes coïncident, c'est-à-dire au point où $z\delta \theta  - t[\varepsilon(\theta + \delta \theta ) - \varepsilon(\theta)] = 0$, ce qui donne $z \simeq vt$ avec la vitesse de groupe $v = d\varepsilon/d\theta = \theta$. Ainsi, il s'agit effectivement d'un 'paquet d'ondes' se déplaçant à la vitesse $\theta$. Ensuite, considérons deux particules entrantes dans un état tel que le centre de masse $(z_1 + z_2)/2$ ait une impulsion $\theta_1 - \theta_2$, tandis que la coordonnée relative $z_1 - z_2$ se trouve dans un 'paquet d'ondes' se déplaçant à la vitesse $(\theta_1 - \theta_2)/2$,

{\color{gray}
\begin{eqnarray*}
	\frac{ z_1 + z_2}{2} ( \theta_1 - \theta_2 ) + ( z_1 - z_2 ) \left ( \frac{ \theta_1 -\theta_2}{2} + \delta \theta  \right ) & = & z_1 ( \theta_1  + \delta \theta ) + z_2 ( \theta_2 - \delta \theta )  
\end{eqnarray*}
}

\begin{eqnarray}
	\psi_{inc} ( z_1 , z_2 ) & \propto & e^{i \frac{ z_1 + z_2 }{2} ( \theta_1 + \theta_2 ) }  \left ( e^{i ( z_1 - z_2 ) \frac{ \theta_1 - \theta_2}{2} } +  e^{i ( z_1 - z_2 )  \left ( \frac{ \theta_1 - \theta_2}{2}+ \delta \theta  \right )  } \right )\notag  \\
	& \propto & e^{ i z_1 \theta_1 + i z_2 \theta_2 } + 	 e^{ i z_1 ( \theta_1 + \delta \theta )  + i z_2  ( \theta_2 - \delta \theta)   }%\tag{2}
\end{eqnarray}

Selon les équations (\ref{eq:I-1-10}) et (\ref{eq:I-1-11}), l'état sortant correspondant serait :

\begin{eqnarray}
	\psi_{outc} ( z_1 , z_2 ) & \propto & - e^{i\phi ( \theta_1 - \theta_2 ) }  e^{ iz_1 \theta_2 + i z_2 \theta_1 }  - e^{i\phi ( \theta_1 - \theta_2 + 2\delta \theta ) }  e^{ iz_1 (\theta_2- \delta \theta )  + i z_2 (\theta_1 + \delta \theta )  } \notag  \\
	& \propto & e^{i \frac{ z_1 + z_2 }{2} ( \theta_1 + \theta_2 ) }  \left ( - e^{ i \phi ( \theta_1 - \theta_2 ) }e^{i(z_1 - z_2 ) \frac{ \theta_1 - \theta_2}{2} } -e^{ i \phi ( \theta_1 - \theta_2 + 2 \delta \theta ) }e^{i(z_1 - z_2 )  \left ( \frac{ \theta_1 - \theta_2}{2} + \delta \theta \right )  }  \right ). %\tag{2}
\end{eqnarray}

En répétant l'argument précédent de la stationnarité de phase, on trouve que la coordonnée relative est à la position $z_1 - z_2 \simeq (\theta_1 - \theta_2)t - 2d\phi /d\theta$ au moment $t$. Étant donné que le centre de masse n'est pas affecté par la collision et se déplace à la vitesse de groupe $(\theta_1 + \theta_2)/2$, nous constatons que la position des deux particules semiclassiques après la collision sera

\begin{eqnarray}
	z_1 \simeq \theta _1 t - \Delta ( \theta_2 -\theta_1 ), & &	z_2 \simeq \theta _2 t - \Delta ( \theta_2 -\theta_1 ),
\end{eqnarray}

où le déplacement de diffusion $\Delta (\theta)$ est donné par la dérivée de la phase de diffusion,

\begin{eqnarray}\label{eq:I-1-16}
	\Delta ( \theta ) & \doteq & \frac{ d \phi }{ d \theta } ( \theta )= \frac{ 2 c }{ c^2 + \theta^2}  	
\end{eqnarray}

Les deux particules sont retardées : leur position après la collision est la même que si elles étaient en retard respectivement de $\delta t_1 = \Delta ( \theta_2 - \theta_1 )/v_1 $ et $\delta t_2 = \Delta ( \theta_2 - \theta_1 )/v_2 $ .

...........................................

Cette équation décrit deux particules libres, sauf lorsqu'elles se rencontrent (i.e., \(z_1 = z_2\)), où l'interaction à contact produit un potentiel delta de Dirac.

\paragraph{Méthode de Bethe}

On peut résoudre cette équation en utilisant l’**Ansatz de Bethe**, en supposant que la fonction d’onde prend la forme d’une superposition symétrisée d’ondes planes dans les domaines \(z_1 < z_2\) et \(z_1 > z_2\) :

\begin{eqnarray}
	\varphi_2(z_1, z_2) = \begin{cases}
		A_{12} e^{i(k_1 z_1 + k_2 z_2)} + A_{21} e^{i(k_2 z_1 + k_1 z_2)} & \text{si } z_1 < z_2, \\
		A_{12} e^{i(k_1 z_2 + k_2 z_1)} + A_{21} e^{i(k_2 z_2 + k_1 z_1)} & \text{si } z_1 > z_2.
	\end{cases}
\end{eqnarray}

La continuité de \(\varphi_2\) à \(z_1 = z_2\), ainsi que le saut de sa dérivée transverse imposé par le potentiel delta, permettent de relier les coefficients \(A_{12}\) et \(A_{21}\). On obtient la condition de diffusion bosonique :

\begin{eqnarray}
	\frac{A_{21}}{A_{12}} & = & - \frac{ic(k_1 - k_2) + (k_1 - k_2)^2}{ic(k_1 - k_2) - (k_1 - k_2)^2}.
\end{eqnarray}

L’énergie totale est simplement la somme des énergies individuelles :

\begin{eqnarray}
	E = \frac{k_1^2 + k_2^2}{2},
\end{eqnarray}

et les \(k_j\) doivent satisfaire les conditions aux limites périodiques, ce qui donne un système transcendantal pour les \(k_j\) dépendant du couplage \(c\) (voir chapitre sur l'équation de Bethe pour \(N\) particules).

\vspace{1em}
Ce cas à deux particules constitue une étape fondamentale vers la compréhension complète du système à \(N\) bosons, où les structures d’interférences entre ondes multiples et les effets d’interactions deviennent de plus en plus riches.
