\paragraph{Contexte physique du système.}

Le système considéré est constitué d’une seule particule quantique confinée dans une boîte unidimensionnelle de taille finie \(L\), avec des conditions aux limites périodiques. Ce cadre sert de base pour l’étude ultérieure de systèmes à plusieurs corps. En absence d’interactions (i.e., \(c = 0\)), l’Hamiltonien se réduit à sa composante cinétique, et le problème devient celui d’une particule libre sur un cercle.


%%%%%%%%%%%%%%%%%%%%%%%%%%%%%%%%%%%%%%%
\paragraph{Détermination des états propres.}

Nous cherchons à déterminer les états propres \(\vert \theta \rangle\) de l’Hamiltonien \(\operator{H}\), associés au quasi-moment \(\theta\), qui satisfont l’équation de Schrödinger stationnaire.


\begin{eqnarray}
	\operator{H} \vert \theta \rangle & = & E \vert \theta \rangle.
\end{eqnarray}
%%%%%%%%%%%%%%%%%%%%%%%%%%%%%%%%%%%%%%%

\paragraph{État à une particule dans la base de Fock.}

L’état à une particule peut être exprimé dans la base positionnelle de Fock :

\begin{eqnarray}
	\vert \theta \rangle & = & \int dz\, \vert z \rangle \langle z \vert \theta \rangle ~= \int dz\, \varphi_{\{\theta\}}(z) \operator{\Psi}^\dag(z) \vert 0 \rangle,
\end{eqnarray}

où l’opérateur \(\operator{\Psi}^\dag(z)\) crée une particule à la position \(z\), la base spatial
\begin{eqnarray}
	\vert z \rangle  & = & \operator{\Psi}^\dag (z)\vert 0 \rangle ,
\end{eqnarray}

et \(\vert 0 \rangle\) est le vide de Fock défini par :

\begin{eqnarray}
	\forall x \in \mathbb{R}, \qquad \operator{\Psi}(x) \vert 0 \rangle = 0 ,\quad  \langle 0\vert 0 \rangle = 1. \label{chap:eq.vide.fock}
\end{eqnarray}

%%%%%%%%%%%%%%%%%%%%%%%%%%%%%%%%%%%%%%%
\paragraph{Orthonormalité de la base.}

La base $\vert z \rangle$ est orthonormale car avec en utilisant les relations de commutation (\ref{chap:1:com.1}) et et la définition du vide de Fock (\ref{chap:eq.vide.fock}) , il vient que 

\begin{eqnarray}
	\langle z\vert z' \rangle  & = & \delta_{z,z'}.
\end{eqnarray}


%%%%%%%%%%%%%%%%%%%%%%%%%%%%%%%%%%%%%%%
\paragraph{Fonction d’onde à une particule.}

La fonction d’onde associée \(\varphi_{\{\theta \}}(z)\) est définie comme la projection :

\begin{eqnarray}
	\varphi_{\{\theta\}}(z) & = & \langle 0 \vert \operator{\Psi}(z) \vert 1 \rangle,
\end{eqnarray}

et normalisée selon :

\begin{eqnarray}
	\langle \theta' \vert \theta \rangle ~=~ \int dz\,  \varphi_{\{\theta'\}}^\ast(z)\varphi_{\{\theta\}}(z) = \delta_{\theta, \theta'}.
\end{eqnarray}

%%%%%%%%%%%%%%%%%%%%%%%%%%%%%%%%%%%%%%%
\paragraph{Hamiltonien dans le cas à une particule.}

L'Hamiltonien (\ref{chap:1:ham.mod}) peut être réécrit en effectuant une intégration par parties. Dans le cas d'un seul boson, seul le terme cinétique contribue, car le terme d'interaction s'annule (ce dernier étant nul pour une seule particule) :


\begin{eqnarray}
	\operator{H} & = & -\int dx  \left [\frac{1}{2}\operator{\partial}_x^2 \operator{\Psi}^\dag(x)\right ] \operator{\Psi}(x)  \label{chap:1:ham.mod.1.part.2}.
\end{eqnarray}


%%%%%%%%%%%%%%%%%%%%%%%%%%%%%%%%%%%%%%%
\paragraph{Hamiltonien dans le cas à une particule et Action de l’Hamiltonien. }

Lorsque l’on agit avec l’Hamiltonien (\ref{chap:1:ham.mod.1.part.2}) sur \(\vert 1 \rangle\), et en utilisant les relations de commutation (\ref{chap:1:com.1}) et la définition du vide de Fock (\ref{chap:eq.vide.fock}), on obtient : 

\begin{eqnarray}
	\operator{H}\vert \theta \rangle & = & \int dz \left [\operator{H}_1  \varphi_{\{\theta\}}(z)  \right ] \operator{\Psi}^\dag (z)	 \vert 0 \rangle,	
\end{eqnarray}

avec 

\begin{eqnarray*}
		\operator{H}_1 & = & - \frac{1}{2} \operator{\partial}_z^2
\end{eqnarray*}


%%%%%%%%%%%%%%%%%%%%%%%%%%%%%%%%%%%%%%%
\paragraph{Équation de Schrödinger différentielle.}

En reportant cela dans l’équation de Schrödinger, on obtient une équation différentielle linéaire :

\begin{eqnarray}
	- \frac{1}{2} \operator{\partial}_z^2 \varphi_{\{\theta\}}(z) & = & E \varphi_{\{\theta\}}(z). \label{chap:eq.onebody.schrod}
\end{eqnarray}

%avec 

%\begin{eqnarray}
%	E & = & \frac{\theta ^2}{2}.
%\end{eqnarray}

%Il s'agit de l'équation de Schrödinger pour une particule libre sur un cercle. Les solutions normales sous conditions périodiques \(\varphi_1(z+L) = \varphi_1(z)\) sont les fonctions d’onde planes :

%\begin{eqnarray}
%	\varphi_1(z) & = & \frac{1}{\sqrt{L}} e^{i \theta z}, \quad \text{avec} \quad \theta  = \frac{2\pi n}{L}, \quad n \in \mathbb{Z},
%\end{eqnarray}

%et les énergies correspondantes sont quantifiées :

%\begin{eqnarray}
%	E_n & = & \frac{\theta ^2}{2} = \frac{2\pi^2 n^2}{L^2}.
%\end{eqnarray}

%......

%%%%%%%%%%%%%%%%%%%%%%%%%%%%%%%%%%%%%%%
\paragraph{Résolution avec conditions périodiques.}

Il s'agit de l’équation de Schrödinger stationnaire pour une particule libre en une dimension. Dans notre cas, la particule est confinée dans une boîte unidimensionnelle de longueur $L$, avec des conditions aux limites périodiques. Ce problème est donc équivalent à celui d’une particule libre sur un cercle de périmètre $L$.

La fonction d’onde $\varphi_{\{\theta\}}(z)$ doit alors satisfaire :
\begin{eqnarray}
	\varphi_{\{\theta\}}(z+L) = \varphi_{\{\theta\}}(z),
\end{eqnarray}
ce qui impose une **périodicité** stricte. Les solutions admissibles de \eqref{chap:eq.onebody.schrod} sont alors des ondes planes :
\begin{eqnarray}
	\varphi_{\{\theta\}}(z) = \frac{1}{\sqrt{L}} e^{i \theta z},
\end{eqnarray}
avec $\theta \in \frac{2\pi}{L} \mathbb{Z}$, imposé par la condition :
\begin{eqnarray}
	e^{i \theta(z+L)} = e^{i \theta z} \quad \Rightarrow \quad e^{i \theta L} = 1 \quad \Rightarrow \quad \theta = \frac{2\pi n}{L}, \quad n \in \mathbb{Z}.
\end{eqnarray}


%%%%%%%%%%%%%%%%%%%%%%%%%%%%%%%%%%%%%%%
\paragraph{Énergies quantifiées.}

L’énergie associée à une onde plane est donnée par :
\begin{eqnarray}
	E & = & \frac{\theta^2}{2} = \frac{2\pi^2 n^2}{L^2},
\end{eqnarray}

ce qui exprime l’énergie cinétique d’une particule libre de masse réduite $m = 1$, en fonction de son quasi-moment $\theta$.

%%%%%%%%%%%%%%%%%%%%%%%%%%%%%%%%%%%%%%%
\paragraph{Notation adoptée et interprétation.}

Dans le cas à une particule, l’état propre de l’Hamiltonien s’écrit naturellement dans la base des quasi-moment comme un état \(\vert \theta \rangle\), où \(\theta\) désigne la quantité de mouvement de la particule.  
%Nous adoptons donc la notation :
%\begin{eqnarray}
%	\vert \theta \rangle ~ \doteq ~  \vert 1 \rangle \quad, \mbox{et} \quad  \varphi_{\{\theta\}}(z) ~ \doteq ~ \varphi_1(z).
%\end{eqnarray}

Ces solutions correspondent à des **états non liés** (ou états de diffusion) : la particule est délocalisée sur tout l’espace (le cercle), sans structure particulière.

%Ces états forment une base orthonormée des états à une particule dans l’espace de Fock. Ce cadre sert de point de départ pour l’analyse de systèmes à \(N\) bosons en interaction.


%En revanche, dans d’autres contextes physiques — par exemple si un potentiel attractif est introduit — il peut exister des **états liés**, pour lesquels la fonction d’onde $\varphi_1(z)$ est **localisée** spatialement autour d’une certaine position. Ces états satisfont également une équation de Schrödinger similaire, mais avec un potentiel $V(z)$ :
%\begin{eqnarray}
%	\left[-\frac{1}{2} \partial_z^2 + V(z)\right] \varphi_1(z) = E \varphi_1(z),
%\end{eqnarray}
%et mènent alors à des solutions de type exponentiellement décroissantes en dehors d’une région de localisation. Ces états liés sont caractérisés par une énergie $E < 0$ (dans un cadre non périodique), mais dans le cas périodique sur un cercle, l’analogue des états liés correspondrait à des **paquets d’ondes localisés** ou à des états propres affectés par une interaction attractive forte (par exemple dans un gaz de Lieb-Liniger attractif).

%Dans le cas présent, sans potentiel et avec des interactions nulles, seule la famille des ondes planes constitue l’ensemble des états propres du système à une particule.

