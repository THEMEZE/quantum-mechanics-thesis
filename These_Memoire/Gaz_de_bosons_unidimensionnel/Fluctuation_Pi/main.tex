Dans ce chapitre, nous nous intéressons aux fluctuations de la distribution de rapidité \( \delta \Pi \) autour d'une distribution de référence \( \Pi^c \), qui maximise la contribution à la fonction de partition des états, exprimée comme une fonctionnelle de la distribution \( \Pi \) :  

\begin{eqnarray*}
	\mathcal{Z} & = & \sum_\Pi \exp \left( -\mathcal{A}(\Pi) \right).
\end{eqnarray*}  

Dans la section {\em \bf Entropie de Yang-Yang} (\ref{??}), l'action \( \mathcal{A}(\Pi) \) s'écrit sous la forme :  

\begin{eqnarray*}
	\mathcal{A}(\Pi) & \doteq & - \mathcal{S}_{YY}(\Pi) + \int f(\theta) \Pi (\theta) \, d\theta,		
\end{eqnarray*}  

où \( \mathcal{S}_{YY} \) est la fonctionnelle d'entropie de Yang-Yang, définie dans (\ref{??}), et \( f \) est la fonction paramétrant les charges, introduite dans (\ref{??}).  

Dans cette même section {\em \bf Entropie de Yang-Yang} (\ref{??}), nous avons établi un lien entre \( f \) et \( \Pi^c \).\\
  

Nous poursuivons à présent avec cette définition de l'action de classe $\mathcal{C}^2$ et admetant une distribution critique $\Pi^c$ tel que sa différentielle en ce point critique soit nulle $d\mathcal{A}_{\Pi^c} = 0 $ (\ref{??}) de sorte que d'aprés la formule de Taylor-Youg %afin de déterminer les fluctuations autour de \( \Pi^c \). Pour cela, nous réécrivons l'action sous la forme :  

\begin{eqnarray*}  
	\mathcal{A}(\Pi^c + \delta \Pi) & \underset{ \delta \Pi \to 0 }{=} & \mathcal{A}(\Pi^c)  + \frac{1}{2} \left. \frac{\delta^2 \mathcal{A}}{\delta \Pi^2} \right|_{\Pi^c} (\delta \Pi)^2 + \mathcal{O}((\delta \Pi)^3),  
\end{eqnarray*}  

une expression quadratique pour l'action à l'ordre dominant en \( \delta \Pi \) avec $\left. \frac{\delta^2 \mathcal{A}}{\delta \Pi^2} \right|_{\Pi^c}$ la forme quadratique définie positive (Fig (\ref{fig.fluctu.A})).

\begin{figure}[H]
	\centering 
	\begin{tikzpicture}
		\begin{scope}[shift={(0,0)}]
			\begin{scope}[transform canvas={scale=0.6}]
				% Définition des couleurs avec les codes HTML
\definecolor{colorOne}{HTML}{443E46}
\definecolor{colorTwo}{HTML}{F6DEB8}
\definecolor{colorThree}{HTML}{908CA4}
\definecolor{colorFour}{HTML}{57659E}
\definecolor{colorFive}{HTML}{C57284}
\definecolor{colorSix}{HTML}{FF5B69}

% Raccourcis pour les couleurs
\def\colorOne{colorOne}
\def\colorTwo{colorTwo}
\def\colorThree{colorThree}
\def\colorFour{colorFour}
\def\colorFive{colorFive}
\def\colorSix{colorSix}

\def\colorslide{blue!50!black}



\begin{scope}
	% Tracer une courbe lisse entre des points
	\draw[shift={(0,0)} ,\colorOne]
		(-1 , 0 ) edge [thick,line width=0.8ex , ->,>=triangle 45  , \colorOne] node [pos = 1 , below ]{\huge$\rho$}( 5  , 0 )
	;
	\draw[shift={(0,0)}, color=\colorOne]
		(0, -1.0 ) edge [thick,line width=0.8ex , ->,>=triangle 45  ]node [pos=0.9,left=0.2cm ]{\huge$\mathcal{A}(\rho)$}( 0  , 5 )
	;
	\draw[]
		(2.5, 0.12 ) edge [thick,line width=0.8ex ,\colorThree ]node [pos=1,below  ]{\huge$\rho^c$} (2.5, -0.12 )	
	;
	
	\draw[]
		(2.5, -0.12 ) edge [thick,line width=0.4ex , dashed, \colorThree ] (2.5, 5.5 )
		(1.5, 1 ) edge [thick,line width=0.4ex , <->,>=triangle 45  , \colorThree ] (3.5, 1 )
		(-0.3,1) edge [thick,line width=0.4ex  , \colorThree ] node [pos=0,left ]{\huge$\mathcal{A}(\rho^c)$} (0.3, 1 )	
	;
    \draw[thick, line width=0.8ex , \colorFour] plot[smooth, tension=0.7] coordinates {
        (1, 5) (1.6 , 3 ) (2.5, 1) (3.5 , 3 )  (4, 5)
    };		
	
\end{scope}

	
			
			\end{scope}
			
			\draw[color = red , scale = 0.5 , draw = none  ] (-2 , -1) rectangle (5, 6) ; 	
		\end{scope}
		
		\begin{scope}[shift={(19,-1)}]
			\begin{scope}[transform canvas={scale=0.6}]
				% Définition des couleurs avec les codes HTML
\definecolor{colorOne}{HTML}{443E46}
\definecolor{colorTwo}{HTML}{F6DEB8}
\definecolor{colorThree}{HTML}{908CA4}
\definecolor{colorFour}{HTML}{57659E}
\definecolor{colorFive}{HTML}{C57284}
\definecolor{colorSix}{HTML}{FF5B69}

% Raccourcis pour les couleurs
\def\colorOne{colorOne}
\def\colorTwo{colorTwo}
\def\colorThree{colorThree}
\def\colorFour{colorFour}
\def\colorFive{colorFive}
\def\colorSix{colorSix}

\def\colorslide{blue!50!black}

\def\Occupation{
	\def\traitx{0.3}
	\def\traity{0.5}
	\draw[shift={(0,0)}]
		(-13.5 , 0 ) edge [thick,line width=0.8ex ]( -3.2  , 0 )
		( -3.2 - \traitx  , 0 - \traity ) edge [thick,line width=0.8ex ]( -3.2 + \traitx  , 0 + \traity  )
		( -2.8 - \traitx  , 0 - \traity ) edge [thick,line width=0.8ex ]( -2.8 + \traitx  , 0 + \traity  )
		(-2.8 , 0 ) edge [thick,line width=0.8ex ](2.8  , 0 )
		( 2.8 - \traitx  , 0 - \traity ) edge [thick,line width=0.8ex ]( 2.8 + \traitx  , 0 + \traity  )
		( 3.2 - \traitx  , 0 - \traity ) edge [thick,line width=0.8ex ]( 3.2 + \traitx  , 0 + \traity  )
		(3.2, 0 ) edge [thick,line width=0.8ex,->,>=triangle 45 , color = black ]node [pos=1.01,below  ]{\huge$\theta$}	( 13  , 0 )
	;
	\draw[shift={(0,0)}, color=\colorOne]
		(-10.5 , -1.5 ) edge [thick,line width=0.8ex , ->,>=triangle 45  ]( -10.5  , 4.5 )
	;
		
	\foreach \r in {1 , ... , 3 } {
%		\draw[
%		decoration={
%		markings,
%    	mark connection node=my node,
%    	mark=at position 0 with{\node [blue,transform shape] (my node) {\large \r};}},
%		color=gray, thick, 
%		line width=0.5ex] decorate { 
%            (-11.0, \r) -- (-10.1, \r )}
%        ;
        \draw[
			color=\colorOne,
			] 
            (-11.0, \r) edge[color=\colorThree , thick,line width=0.5ex] node [pos=-0.5 ]{\large\color{\colorFour} $\frac{\r}{\delta \theta}$ } (-10.3, \r )
        	;
	
	}
	

	
	% Graduation abcsisse 
	% Définitions des listes
% Definitions of the lists
\def\listetuple{-9/\theta_{1}, -8/\theta_{2} , -5/\theta_{3} , -2/\theta_{a-1} , 0/\theta_{a} , 1/\theta_{a+1} , 2/\theta_{a+2} ,  5/\theta_{N-4} , 7/\theta_{N-3},8/\theta_{N-1},9/\theta_{N} }
\def\listetrais{-12 , -11, -10, -9 , -8 , -7 ,  -6 , -5, -4.5,-4, -2 , -1, 0 , 0.5, 1, 2, 4 , 5 ,  6 , 7 , 8 ,8.5, 9 ,  10 , 11, 12 }

% Loop over listetrais
\foreach \r in \listetrais {
    % Initialize found variable to zero
    % Initialize found variable to zero
    %\pgfmathsetmacro\found{0}
    \global\def\found{0}
    \xdef\nomtheta{}
    
    % Check if \r is in listetuple
    \foreach \x/\y in \listetuple { 
        \ifdim \r pt=\x pt % If \r matches any \x in listetuple
            \global\def\found{1} ;
            \xdef\nomtheta{\y} % Set \nomtheta to the corresponding \y
            %\pgfmathsetmacro\found{1} % Set found to 1            
            %\global\pgfmathsetmacro\found{1}
        \fi
    }
    
    %\node [circle, draw, red] (A) at (\r, 2) {\found , $\nomtheta$};
    
    % Draw the line and display \nomtheta if found
    \ifnum\found=1
        \draw[color=\colorOne, thick, line width=0.5ex] 
            (\r, -0.3) -- (\r, 0.3) node[red , pos=-0.5] {\large $\nomtheta$};
         \filldraw[line width=0.5ex, color=\colorSix, outer color=\colorSix, inner color=\colorSix] 
            (\r, 0) circle (4pt);
    \else 
        % Draw without \nomtheta and add a blue circle if not found
        \draw[color=\colorOne, thick, line width=0.5ex] 
            (\r, -0.3) -- (\r, 0.3);
        \filldraw[line width=0.5ex, color=\colorSix, outer color=\colorTwo, inner color=\colorTwo] 
            (\r, 0) circle (4pt); 
    \fi
}

\def\listetrais{-9.5/\theta_{i-1}/2/3, -6.5/\theta_{i}/1/4  ,   -1.5/\theta_{j}/2/4 , 1.5/\theta_{j+1}/-1/3 , 3.5/\theta_{\ell-1}/1/3 , 6.5/\theta_{\ell}/3/4 , 9.5/\theta(\theta_{\ell+1})/-1/3 };



\foreach \r/\nomx/\y/\ys in \listetrais {
	\draw[
		decoration={
		markings,
    	mark connection node=my node,
    	mark=at position .5 with{\node [blue,transform shape] (my node) {\large \color{\colorFour} $\nomx$};}},
		color=\colorThree , thick, 
		line width=0.5ex] decorate { 
            (\r, 0.12) -- (\r, -1.2)}
        ;
     
     \ifdim \y pt > -1 pt 
     	\draw[
			decoration={
			markings,
    		mark connection node=my node,
    		mark=at position .5 with{\node [blue,transform shape] (my node) {\large \color{\colorFour} $\Pi(\nomx) $};}},
			color=\colorThree, thick, 
			line width=0.5ex] decorate { 
            (\r, \y) -- (\r +3, \y)}
        ;
        \draw[
			decoration={
			markings,
    		mark connection node=my node,
    		mark=at position .5 with{\node [blue,transform shape] (my node) {\large \color{\colorFive} $\Pi_s(\nomx) $};}},
			color=\colorFive, thick, 
			line width=0.5ex] decorate { 
            (\r, \ys) -- (\r +3, \ys)}
        ;
     \fi 
     \ifdim \r pt= -1.5 pt
     	\draw[
     		decoration={
			markings,
    		mark connection node=my node,
    		mark=at position .5 with{\node [blue,transform shape] (my node) {\large \color{\colorFour}  $\delta \theta $};},
    		%mark=at position 0.1  with {\arrow[blue, line width=0.5ex]{<}},
    		%mark=at position 1  with {\arrow[blue, line width=0.5ex]{>}}
    		},
        	color=\colorThree,
        	thick,
        	line width=0.5ex,
        	%arrows={Computer Modern Rightarrow[line cap=round]-Computer Modern Rightarrow[line cap=round]}
   			](\r, -1.2) edge[arrows={Computer Modern Rightarrow[line cap=round]-}] (\r + 0.4, -1.2)decorate {
    		(\r, -1.2) -- (\r + 3, -1.2)}(\r + 2, -1.2) edge[arrows={-Computer Modern Rightarrow[line cap=round]}] (\r + 3, -1.2)
    		;
    \fi
			
	
}


			
}


\begin{scope}
	%\draw[help lines , width=1.5ex] (-8,-3) grid (8,3);\draw[help lines ,width=0.5ex , opacity = 0.5] (-3,-3) grid[step=0.1] (3,3));
	
	%\draw[help lines] 
	%	(-3,-3) edge[width=1.5ex] grid (3,3)	
	%	(-3,-3) edge[width=0.5ex , opacity = 0.5] grid (3,3)	
	%;
	\begin{scope}[shift={(0,1)},rotate=0,opacity=1,color=black]
		\Occupation	
		
		%\node[anchor=east, font=\bfseries] at (-11, 0) {\color{red}\large (T = 0 )} ;	
	\end{scope}
	
	
	
	
	\begin{scope}[shift={(-10.5,7)},rotate=0,opacity=1,color=black]
	
	\begin{scope}[shift={(-0,0)},rotate=0,opacity=1,color=black]
	
		\draw[shift={(0,0)} ,line width=1ex,rounded corners = 1ex,color=\colorOne , opacity =1 ,fill=\colorOne!00 , pattern={north east lines} , pattern color=\colorOne!00 ]
			(0 , -1 ) rectangle (5,1)
		;
		

		\begin{scope}[shift={(0.5,0.5)}]
			\draw[color=\colorOne, thick, line width=0.5ex] 
            (0, -0.3) -- (0, 0.3) ;
            \filldraw[line width=0.5ex, color=\colorSix, outer color=\colorSix, inner color=\colorSix] 
            (0, 0) circle (4pt);
            
            \node[anchor=west, font=\bfseries] at (0.2, 0) {\color{\colorSix}\large : quasi-particule};
		\end{scope}
		
		\begin{scope}[shift={(0.5,-0.5)}]
			\draw[color=\colorOne, thick, line width=0.5ex] 
            (0, -0.3) -- (0, 0.3) ;
            \filldraw[line width=0.5ex, color=\colorSix, outer color=\colorTwo, inner color=\colorTwo] 
            (0, 0) circle (4pt);
            
            \node[anchor=west, font=\bfseries] at (0.2, 0) {\color{\colorSix}\large : hole};
		\end{scope}

	\end{scope}
	
	\begin{scope}[shift={(6,0)},rotate=0,opacity=1,color=black]	
		
		\draw[shift={(0,0)} ,line width=1ex,rounded corners = 1ex,color=\colorOne , opacity =1 ,fill=\colorOne!00 , pattern={north east lines} , pattern color=\colorOne!00 ]
			(0 , -1 ) rectangle (7.5,1)
		;
		
		\node[anchor=west] at (0.5, 0.5) {\color{\colorFour}\large $\Pi$ };\node[anchor=west, font=\bfseries] at (1, 0.5) {\color{\colorFour}\large : quasi-particule distribution};
		
		\node[anchor=west] at (0.5, -0.5) {\color{\colorFour}\large $\Pi_h$ };\node[anchor=west, font=\bfseries] at (1, -0.5) {\color{\colorFour}\large  : hole distribution};
		
	\end{scope}
	
	\begin{scope}[shift={(14.5,0)},rotate=0,opacity=1,color=black]	
		
		\draw[shift={(0,0)} ,line width=1ex,rounded corners = 1ex,color=\colorOne , opacity =1 ,fill=\colorOne!00 , pattern={north east lines} , pattern color=\colorOne!00 ]
			(0 , -0.5 ) rectangle (7.0,0.5)
		;
		
		\node[anchor=west] at (0.2, 0) {\color{\colorFour}\large ${\color{\colorFive}\Pi_s} = \Pi + \Pi_h $ } node[anchor=west , font=\bfseries] at (3.1 , 0 )  {\color{\colorFour}\large {\color{\colorFive} : density of states}};
		
	\end{scope}
	
	
	\end{scope}


		
	
\end{scope}

	
			
			\end{scope}
			\begin{scope}[scale=1]
				\draw[color = red , scale = 1 , draw = none  ] (-1 , -1) rectangle (5, 5) ; 
			\end{scope}	
		\end{scope}

		
				
			
	\end{tikzpicture}	
	\captionsetup{skip=10pt} % Ajoute de l’espace après la légende
	\label{fig.fluctu.A}
\end{figure}


On discrétise l'axe des rapidités en  petite cellule de rapidité $[\theta, \theta+\delta\theta]$, qui contient $\Pi(\theta) \delta \theta$ rapidités. 
	



Avec ces petites tranches, la forme quadratique s’écrit :

\begin{eqnarray*}
    \left. \frac{\delta^2 \mathcal{A}}{\delta \Pi^2} \right|_{\Pi^c}(\delta \Pi ) &=&  \sum_{a,b \mid \text{tranche}}  
    \delta \Pi(\theta_a)  \frac{\partial^2 \mathcal{A}}{\partial \delta \Pi(\theta_a) \partial \delta \Pi(\theta_b) } (\Pi^c)  \delta \Pi(\theta_b).
\end{eqnarray*}
Les fluctuations s’écrivent donc :

\begin{eqnarray*}
    \langle \delta \Pi ( \theta) \delta \Pi ( \theta') \rangle &=&  
    \frac{ \int d\delta \Pi \, \delta \Pi(\theta) \delta \Pi ( \theta') 
    \exp \left( - \frac{1}{2} \sum_{a,b \mid \text{tranche}}  
    \delta \Pi(\theta_a) \frac{\partial^2 \mathcal{A}}{\partial \delta \Pi(\theta_a) \partial \delta \Pi(\theta_b) } (\Pi^c)  \delta \Pi(\theta_b) \right) }
    { \int d\delta \Pi  
    \exp \left( - \frac{1}{2} \sum_{a,b \mid \text{tranche}}  
    \delta \Pi(\theta_a) \frac{\partial^2 \mathcal{A}}{\partial \delta \Pi(\theta_a) \partial \delta \Pi(\theta_b) } (\Pi^c)  \delta \Pi(\theta_b) \right) } \\
    &=& \left( \mathbf{A}^{-1} \right)_{\theta , \theta'}
\end{eqnarray*}

La {\em matrice hessienne} $\mathbf{A}_{\theta , \theta'} \equiv \frac{\partial^2 \mathcal{A}}{\partial \delta \Pi(\theta) \partial \delta \Pi(\theta') }(\Pi^c)$, au point critique $\Pi^c$, s'écrit

\begin{eqnarray*}
	\operator{A} & = & \operator{A}^{(0)} + \delta \theta \operator{V}
\end{eqnarray*}

avec 

\begin{eqnarray*}
	A^{(0)}_{\theta , \theta'}  & = &  \left ( \frac{ ( \Pi^c/\Pi^c_s)^{-1}}{\Pi^c_s - \Pi^c} \right )(\theta) \delta \theta   \delta_{\theta,\theta '}	,\\
	V_{\theta , \theta'}  &= & \left \{ - \left [ \left ( \frac{1}{\Pi^c_s - \Pi^c } \right ) ( \theta)  + \left ( \frac{1}{\Pi^c_s - \Pi^c } \right ) ( \theta' )\right ] \frac{ \Delta( \theta'- \theta )}{ 2 \pi } + \int d\theta'' \left (   \frac{ \Pi^c/\Pi^c_s}{\Pi^c_s - \Pi^c} \right )(\theta'') \frac{\Delta(\theta''- \theta)}{2 \pi}\frac{\Delta(\theta''- \theta')}{2 \pi}   \right \} \delta \theta	
\end{eqnarray*}

\begin{aff}
Donc une a l'ordre un en $\delta \theta (\operator{A}^{(0)})^{-1} \operator{V}$ 

\begin{eqnarray*}
	\langle \delta \Pi ( \theta) \delta \Pi ( \theta') \rangle & = &  ( (\Pi^c_s - \Pi^c)\Pi^c/\Pi^c_s ) ( \theta ) \delta_{\theta, \theta'}/\delta \theta + \mathscr{F}(\theta , \theta' ) ,	
\end{eqnarray*}

avec 

\begin{eqnarray*}
	\mathscr{F}(\theta , \theta' ) & = & \left [ (\Pi^c_s - \Pi^c )( \theta)  +  (\Pi^c_s - \Pi^c ) ( \theta' )\right ] \frac{\Pi^c}{\Pi^c_s}(\theta)\frac{\Pi^c}{\Pi^c_s}(\theta') \frac{ \Delta( \theta'- \theta )}{ 2 \pi }\\
	&&  - \left [ (\Pi^c_s - \Pi^c )( \theta)   (\Pi^c_s - \Pi^c ) ( \theta' )\right ] \frac{\Pi^c}{\Pi^c_s}(\theta)\frac{\Pi^c}{\Pi^c_s}(\theta')\int d\theta'' \left (   \frac{ \Pi^c/\Pi^c_s}{\Pi^c_s - \Pi^c} \right )(\theta'') \frac{\Delta(\theta''- \theta)}{2 \pi}\frac{\Delta(\theta''- \theta')}{2 \pi}  	
\end{eqnarray*}
\end{aff}



 







