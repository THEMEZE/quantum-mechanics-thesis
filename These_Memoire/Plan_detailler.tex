
Chapitre 1 : présente le modèle de Lieb-Liniger, sa solution par Bethe Ansatz, la densité de rapidité à température nulle, et les excitations élémentaires.

Chapitre 2 : développe la thermodynamique du système, introduit le Thermodynamic Bethe Ansatz, les principes statistiques, le GGE, les charges conservées et l’entropie de Yang-Yang.

\chapter{Modèle de Lieb-Liniger et approche Bethe Ansatz}

\section*{Introduction}
Contexte historique et physique du modèle de Lieb-Liniger. Réalisations expérimentales en 1D et rôle de l'intégrabilité. Objectifs du chapitre.

\section{Le modèle de Lieb-Liniger}

\subsection{Hamiltonien du modèle}
\[
\hat{H} = -\sum_{j=1}^N \frac{\partial^2}{\partial x_j^2} + 2c \sum_{1 \leq i < j \leq N} \delta(x_i - x_j)
\]
Présentation des paramètres physiques : interaction $c > 0$, densité $n$, couplage $\gamma$. Conditions aux bords périodiques. Symétrie bosonique.

\subsection{Domaines physiques}
Limite faible interaction (quasi-condensat) et forte interaction (Tonks-Girardeau).

\section{Résolution exacte par Bethe Ansatz}

\subsection{Construction des états propres}
Forme de l’onde : superposition d’ondes planes avec phases. Conditions de raccord au contact.

\subsection{Équations de Bethe}
Conditions aux bords périodiques $\Rightarrow$ quantification des moments $\{k_j\}$.

\subsection{Limite thermodynamique}
Définition des densités $\rho(\lambda)$ et $\rho^h(\lambda)$. Équation intégrale à $T=0$ :
\[
\rho(\lambda) + \rho^h(\lambda) = \frac{1}{2\pi} + \int_{-\infty}^{\infty} K(\lambda - \mu) \rho(\mu) \, d\mu
\]

\section{Propriétés physiques à température nulle}

\subsection{Énergie du fondamental}
Énergie par unité de longueur :
\[
e = \int \lambda^2 \rho(\lambda) \, d\lambda
\]

\subsection{Densité de particules et impulsion}
Formules intégrales. Dépendance en $\gamma$.

\section{Excitations élémentaires à température nulle}

\subsection{Types d’excitations}
Excitations particule-trou. Interprétation via la distribution $\rho$.

\subsection{Relation énergie–impulsion}
Spectre d’excitation (type I et II), vitesse du son $v_s$.

\subsection{Structure de quasi-particules}
Luttinger liquid. Connexions avec la théorie des perturbations.

\section*{Conclusion}
Récapitulatif : solution exacte, densité de rapidité, excitations. Préparation à l’étude thermodynamique.

%%%%%%%%%%%%%%%%%%%%%%%%%%%%%%%%%%%%%%%%%%%%%%%%%%%%%%%%%%%%%%%%%%%%%

\chapter{Thermodynamique du gaz de Lieb-Liniger}

\section*{Introduction}
Problématique : comment définir la thermodynamique d’un système intégrable ? Objectifs du chapitre : TBA, GGE, entropie.

\section{Thermodynamique à température finie : le TBA}

\subsection{Idée générale}
Motivation : accès aux propriétés thermiques. Notion de pseudo-énergie, rôle de l’entropie.

\subsection{Équation de Yang–Yang}
Définition :
\[
\varepsilon(\lambda) = \frac{\lambda^2 - \mu}{T} - \int K(\lambda - \mu') \log\left(1 + e^{-\varepsilon(\mu')} \right) d\mu'
\]

\subsection{Résolution et interprétation}
Accès à $\rho(\lambda)$ à $T>0$. Calculs thermodynamiques.

\section{Statistique quantique et ensemble de Gibbs généralisé (GGE)}

\subsection{Échec de l'ensemble canonique}
Intégrabilité $\Rightarrow$ non thermalisation standard.

\subsection{Définition du GGE}
\[
\rho_{\mathrm{GGE}} = \frac{1}{Z_{\mathrm{GGE}}} \exp\left(-\sum_n \beta_n Q_n\right)
\]

\subsection{Cas du Lieb-Liniger}
Structure des charges $Q_n$, rôle des multiplicateurs $\beta_n$.

\section{Intégrabilité et charges conservées}

\subsection{Définition formelle}
Infinité de charges locales. Lien avec la matrice de transfert.

\subsection{Structure hiérarchique}
Charges de rang élevé, interprétation physique (courants, énergies...).

\subsection{Rôle thermodynamique}
Contraintes fortes sur les états accessibles, impact sur le TBA et le GGE.

\section{Entropie de Yang–Yang}

\subsection{Formule de l’entropie}
\[
s[\rho] = \int d\lambda \left[ (\rho + \rho^h)\log(\rho + \rho^h) - \rho \log \rho - \rho^h \log \rho^h \right]
\]

\subsection{Principe variationnel}
Équilibre thermodynamique = minimisation de
\[
f[\rho] = e[\rho] - T s[\rho]
\]

\subsection{Applications}
Calculs explicites et interprétations physiques.

\section*{Conclusion}
Résumé : TBA, GGE, entropie de Yang–Yang. Perspectives : dynamique hors équilibre, hydrodynamique généralisée.


-----------------------------------

\chapter{Quantification canonique du gaz de Bose unidimensionnel à interaction delta}

\section{Introduction}
Dans ce chapitre, nous présentons la quantification canonique d’un gaz de Bose en une dimension, décrit par un champ de Bose soumis à une interaction de type delta. Ce modèle est fondamental en physique théorique, notamment pour l’étude des systèmes intégrables.

\section{Champs de Bose et relations de commutation}
On introduit le champ de Bose quantique $\Psi(x,t)$ et son adjoint hermitien $\Psi^\dagger(x,t)$, qui satisfont les relations de commutation canoniques à temps égal :
\begin{align}
[\Psi(x), \Psi^\dagger(y)] &= \delta(x - y) \\
[\Psi(x), \Psi(y)] &= [\Psi^\dagger(x), \Psi^\dagger(y)] = 0
\end{align}
Dans ce qui suit, on omettra l’argument temporel $t$ pour alléger les notations.

\section{Hamiltonien du modèle}
Le système est régi par l’Hamiltonien :
\begin{equation}
H = \int dx \left[ (\partial_x \Psi^\dagger)(\partial_x \Psi) + c\, \Psi^\dagger \Psi^\dagger \Psi \Psi \right]
\end{equation}
où $c$ est une constante de couplage.

L’équation du mouvement associée est l’équation de Schrödinger non linéaire :
\begin{equation}
i \partial_t \Psi = -\partial_x^2 \Psi + 2c\, \Psi^\dagger \Psi \Psi
\end{equation}

\section{Le pseudovacuum}
On considère l’état $|0\rangle$ tel que :
\begin{equation}
\Psi(x) |0\rangle = 0, \quad \forall x
\end{equation}
Le dual est défini par $\langle 0| = (|0\rangle)^\dagger$ et vérifie :
\begin{equation}
\langle 0| \Psi^\dagger(x) = 0, \quad \langle 0|0\rangle = 1
\end{equation}
Remarque : $|0\rangle$ n’est pas le vrai vide du système pour $c > 0$ (le vide est un état de type Fermi), mais sert de pseudovacuum pour construire les états excités.

\section{Observables conservées}
Trois opérateurs importants commutent avec le Hamiltonien :
\begin{itemize}
  \item Nombre de particules :
  \[
  Q = \int dx\, \Psi^\dagger(x) \Psi(x)
  \]
  \item Moment total :
  \[
  P = -\frac{i}{2} \int dx\, \left[ \Psi^\dagger \partial_x \Psi - (\partial_x \Psi^\dagger) \Psi \right]
  \]
  \item Hamiltonien $H$
\end{itemize}
Tous ces opérateurs sont hermitiens et satisfont $[H, Q] = [H, P] = 0$.

\section{États propres à N particules}
Un état propre à $N$ particules s’écrit :
\begin{equation}
|\psi(\lambda_1, ..., \lambda_N)\rangle = \frac{1}{\sqrt{N!}} \int d^N z\, \chi_N(z_1,...,z_N|\lambda_1,...,\lambda_N)\, \Psi^\dagger(z_1)\cdots \Psi^\dagger(z_N) |0\rangle
\end{equation}
où $\chi_N$ est une fonction d’onde symétrique des $z_j$, fonction propre du Hamiltonien et du moment total à $N$ corps.

\section{Hamiltonien et moment à N corps}
On définit les opérateurs :
\begin{align}
\mathcal{H}_N &= \sum_{j=1}^N -\partial_{z_j}^2 + 2c \sum_{1 \leq k < j \leq N} \delta(z_j - z_k) \\
\mathcal{P}_N &= \sum_{j=1}^N -i \partial_{z_j}
\end{align}
et les équations aux valeurs propres :
\begin{align}
\mathcal{H}_N \chi_N &= E_N \chi_N \\
\mathcal{P}_N \chi_N &= p_N \chi_N
\end{align}

\section{Action de l’opérateur moment}
L’action de l’opérateur $P$ sur $|\psi_N\rangle$ est :
\[
P |\psi_N\rangle = i \int dx\, (\partial_x \Psi^\dagger) \Psi\, |\psi_N\rangle
\]
On utilise les relations de commutation et des intégrations par parties pour obtenir :
\[
P |\psi_N\rangle = \frac{1}{\sqrt{N!}} \int d^N z\, \left( \sum_{k=1}^N -i \partial_{z_k} \right) \chi_N(z_1,...,z_N)\, \Psi^\dagger(z_1)\cdots \Psi^\dagger(z_N)|0\rangle
\]
ce qui confirme que l’état est bien un vecteur propre de l’opérateur moment avec valeur propre $p_N = \sum_k \lambda_k$ si $\chi_N$ est une onde plane symétrisée.


