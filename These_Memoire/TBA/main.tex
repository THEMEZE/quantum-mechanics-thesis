%------------------------------------------------------------------
\section{Introduction générale}

Dans les modèles quantiques intégrables, l’évolution vers l’équilibre, à partir d’un état initial arbitraire (et typiquement hors d’équilibre), ne conduit pas à une thermique de Gibbs classique.  
En effet, du fait de l’existence d’une infinité de charges conservées en involution, les systèmes intégrables n’explorent qu’une sous-partie contrainte de l’espace des états accessibles.  
Ils relaxent alors vers un état stationnaire décrit par une \emph{Ensemble Thermodynamique Généralisé} (GGE), qui encode la conservation de toutes ces quantités.

Cette section pose les fondations nécessaires à la description de ces états stationnaires dans le cadre de la \textbf{thermodynamique de Bethe} (TBA), qui généralise l’analyse au-delà de l’état fondamental.  
Nous considérons ici un régime macroscopique à température (ou entropie) finie, correspondant à des états hautement excités du spectre, mais toujours décrits dans le formalisme intégrable exact.

Notre point de départ est la relation constitutive entre la densité de \emph{quasi-particules} (ou rapidités) $\rho(\theta)$ et la densité d’états disponibles $\rho_s(\theta)$, qui encode le spectre accessible en présence d’interactions.  
Nous introduisons ensuite une opération clé de la TBA, appelée \emph{habillage} (\emph{dressing}), qui intervient systématiquement dans le calcul des observables physiques et permet de prendre en compte de manière non perturbative les effets des interactions.  
Cette construction sera illustrée dans le cadre du modèle intégrable de Lieb–Liniger, qui décrit un gaz unidimensionnel de bosons avec interaction delta répulsive.

Les outils développés ici seront fondamentaux pour formuler dans la section suivante le concept d’ensemble généralisé (GGE), et pour décrire la dynamique de relaxation des systèmes intégrables.


%------------------------------------------------------------------
\subsection{Limite thermodynamique}

Dans la limite thermodynamique $N,L\to\infty$ à densité finie
$N/L=\text{cst}$, les équations de Bethe discrètes se condensent en une
équation intégrale pour les densités de rapidités :
\begin{equation}
2\pi \rho_s \;=\; 1 \;+\;\Delta \star \rho,
\label{eq:TBA-rhos}
\end{equation}
où le symbole 
\(
[\Delta \star \rho](\theta)
\)
désigne la \textbf{convolution}
\[
[\Delta \star \rho](\theta)\;=\; \int_{-\infty}^{\infty}
      \!d\theta'\; \Delta(\theta-\theta')\,\rho(\theta').
\]
Pour le modèle de \textbf{Lieb–Liniger} de couplage $g>0$.  
%le noyau
%\(
%\Delta(\theta)=\dfrac{2g}{\theta^{2}+g^{2}}
%\)
%provient de la dérivée de la phase de diffusion
%$S(\theta)=\dfrac{\theta-ic}{\theta+ic}$.

%------------------------------------------------------------------
\subsection{Opération de \emph{dressing}}
\label{sec:dressing}

\subsubsection{Définition}

À toute fonction  $f(\theta)$ on associe sa version \emph{habillée}
$f^{\mathrm{dr}}(\theta)$, solution de l’équation intégrale
%\begin{align}
%f^{\mathrm{dr}}(\theta) &= f(\theta) \;+\;
%       \int_{-\infty}^{\infty}\!\frac{d\theta'}{2\pi}\;
%             \Delta(\theta-\theta')\,\nu(\theta')\,f^{\mathrm{dr}}(\theta')
%\nonumber\\
%&= f(\theta)\;+\;\Bigl[\tfrac{\Delta}{2\pi}\star\bigl(\nu\,f^{\mathrm{dr}}\bigr)
%     \Bigr](\theta),
%\label{eq:dressing}
%\end{align}
\begin{eqnarray}
	f^{\mathrm{dr}} & = & f  \;+\;\Bigl[\tfrac{\Delta}{2\pi}\star\bigl(\nu\,f^{\mathrm{dr}}\bigr)\Bigr] \label{eq:dressing}	
\end{eqnarray}
où $\Delta/2\pi$ est un noyau de convolution spécifique et 
\(
\nu=\dfrac{\rho}{\rho_s}
\)
est le \emph{facteur d’occupation}.

\subsubsection{Interprétation physique}

Le dressing incorpore \emph{à tous ordres} les rétrodiffusions entre
quasi‑particules ; il encode ainsi les corrections
d’interaction aux grandeurs physiques initiales $f(\theta)$.
Dans le Lieb–Liniger, on l’utilise pour obtenir :

* l’énergie habillée $\varepsilon^{\mathrm{dr}}(\theta)$,
* la charge impulsion $p^{\mathrm{dr}}(\theta)$,
* les susceptibilités (voir chap.\,\ref{chap:GGE}).

\subsubsection{Exemple\,: densité de sites}

En posant $f(\theta)=1$, l’équation~\eqref{eq:dressing} donne
\(
1^{\mathrm{dr}}=1+\frac{\Delta}{2\pi}\star\bigl(\nu\,1^{\mathrm{dr}}\bigr)
\) soit 
\[
2\pi\rho_s=1^{\mathrm{dr}},
\]
ce qui n’est autre que la
relation constitutive~\eqref{eq:TBA-rhos}.

\vspace{1ex}
La formalisation ci‑dessus fournit la brique de base pour la
\textbf{hydrodynamique généralisée} et, dans la section suivante, pour la
définition précise de l’\textbf{Ensemble d’Équilibre Généralised (GGE)}
assurant la relaxation des systèmes intégrables hors‑équilibre.





%\subsubsection{Opérateurs nombre de particules $\operator{Q}$ et moment $\operator{P}$}
L'opérateur du nombre de particules $\operator{Q}$ et l'opérateur de moment $\operator{P}$ sont définis comme 
\begin{eqnarray}
	\operator{Q} & = & \int \operator{\Psi}^\dag (x) \operator{\Psi} (x) \, d x \\
	\operator{P} & = & - \frac{i}2 \int \left \{  \operator{\Psi}^\dag(x) \operator{\partial}_x \operator{\Psi}(x) - \left [ \operator{\partial}_x \operator{\Psi}^\dag(x)\right ] \operator{\Psi}(x)\right \} dx \label{eq.1.7}
\end{eqnarray}

\subsubsection{Propriétés}

Ce sont des opérateurs hermitiens et ils constituent des intégrales du mouvement

\begin{eqnarray}
	[ \operator{H} , \operator{Q} ] = 	[ \operator{H} , \operator{P} ] = O. 
\end{eqnarray}

Il convient de noter que dans le chapitre 2 , nous allons construire un nombre infini d'intégrales du mouvement.
 
\subsection{États propres du système à N particules}

\subsubsection{Construction de l’état propre}

Nous pouvons maintenant chercher les fonctions propres communes $\vert \psi_N\rangle$ des opérateurs $\operator{H}$, $\operator{Q}$, et $\operator{P}$ :

\begin{eqnarray}
	\vert \psi_N ( \theta_1 , \cdots , \theta_N ) \rangle & = & \frac{1}{\sqrt{N!}} \int d^N z \, \chi_N ( z_1 , \cdots , z_N  ~\vert ~ \theta_1 , \cdots , \theta _N ) \operator{\Psi}^\dag (z_1 ) \cdots \operator{\Psi}^\dag (z_N )	 \vert 0 \rangle. \label{eq.1.9}
\end{eqnarray}

%\subsubsection{Fonction d’onde $\chi_N$}

\subsection{Notion d’état d’équilibre généralisé (GGE)}
\subsubsection{Opérateurs nombre de particules $\operator{Q}$ et moment $\operator{P}$}
L'opérateur du nombre de particules $\operator{Q}$ et l'opérateur de moment $\operator{P}$ sont définis comme 
\begin{eqnarray}
	\operator{Q} & = & \int \operator{\Psi}^\dag (x) \operator{\Psi} (x) \, d x \\
	\operator{P} & = & - \frac{i}2 \int \left \{  \operator{\Psi}^\dag(x) \operator{\partial}_x \operator{\Psi}(x) - \left [ \operator{\partial}_x \operator{\Psi}^\dag(x)\right ] \operator{\Psi}(x)\right \} dx \label{eq.1.7}
\end{eqnarray}

\subsubsection{Propriétés}

Ce sont des opérateurs hermitiens et ils constituent des intégrales du mouvement

\begin{eqnarray}
	[ \operator{H} , \operator{Q} ] = 	[ \operator{H} , \operator{P} ] = O. 
\end{eqnarray}

Il convient de noter que dans le chapitre 2 , nous allons construire un nombre infini d'intégrales du mouvement.
 
\subsection{États propres du système à N particules}

\subsubsection{Construction de l’état propre}

Nous pouvons maintenant chercher les fonctions propres communes $\vert \psi_N\rangle$ des opérateurs $\operator{H}$, $\operator{Q}$, et $\operator{P}$ :

\begin{eqnarray}
	\vert \psi_N ( \theta_1 , \cdots , \theta_N ) \rangle & = & \frac{1}{\sqrt{N!}} \int d^N z \, \chi_N ( z_1 , \cdots , z_N  ~\vert ~ \theta_1 , \cdots , \theta _N ) \operator{\Psi}^\dag (z_1 ) \cdots \operator{\Psi}^\dag (z_N )	 \vert 0 \rangle. \label{eq.1.9}
\end{eqnarray}

%\subsubsection{Fonction d’onde $\chi_N$}

\subsection{Rôle des charges conservées extensives et quasi-locales}
\subsubsection{Opérateurs nombre de particules $\operator{Q}$ et moment $\operator{P}$}
L'opérateur du nombre de particules $\operator{Q}$ et l'opérateur de moment $\operator{P}$ sont définis comme 
\begin{eqnarray}
	\operator{Q} & = & \int \operator{\Psi}^\dag (x) \operator{\Psi} (x) \, d x \\
	\operator{P} & = & - \frac{i}2 \int \left \{  \operator{\Psi}^\dag(x) \operator{\partial}_x \operator{\Psi}(x) - \left [ \operator{\partial}_x \operator{\Psi}^\dag(x)\right ] \operator{\Psi}(x)\right \} dx \label{eq.1.7}
\end{eqnarray}

\subsubsection{Propriétés}

Ce sont des opérateurs hermitiens et ils constituent des intégrales du mouvement

\begin{eqnarray}
	[ \operator{H} , \operator{Q} ] = 	[ \operator{H} , \operator{P} ] = O. 
\end{eqnarray}

Il convient de noter que dans le chapitre 2 , nous allons construire un nombre infini d'intégrales du mouvement.
 
\subsection{États propres du système à N particules}

\subsubsection{Construction de l’état propre}

Nous pouvons maintenant chercher les fonctions propres communes $\vert \psi_N\rangle$ des opérateurs $\operator{H}$, $\operator{Q}$, et $\operator{P}$ :

\begin{eqnarray}
	\vert \psi_N ( \theta_1 , \cdots , \theta_N ) \rangle & = & \frac{1}{\sqrt{N!}} \int d^N z \, \chi_N ( z_1 , \cdots , z_N  ~\vert ~ \theta_1 , \cdots , \theta _N ) \operator{\Psi}^\dag (z_1 ) \cdots \operator{\Psi}^\dag (z_N )	 \vert 0 \rangle. \label{eq.1.9}
\end{eqnarray}

%\subsubsection{Fonction d’onde $\chi_N$}


\section{Thermodynamique de Bethe et relaxation}
\subsection{Statistique des macro-états : entropie de Yang-Yang et moyennes dans le GGE}
\subsubsection{Opérateurs nombre de particules $\operator{Q}$ et moment $\operator{P}$}
L'opérateur du nombre de particules $\operator{Q}$ et l'opérateur de moment $\operator{P}$ sont définis comme 
\begin{eqnarray}
	\operator{Q} & = & \int \operator{\Psi}^\dag (x) \operator{\Psi} (x) \, d x \\
	\operator{P} & = & - \frac{i}2 \int \left \{  \operator{\Psi}^\dag(x) \operator{\partial}_x \operator{\Psi}(x) - \left [ \operator{\partial}_x \operator{\Psi}^\dag(x)\right ] \operator{\Psi}(x)\right \} dx \label{eq.1.7}
\end{eqnarray}

\subsubsection{Propriétés}

Ce sont des opérateurs hermitiens et ils constituent des intégrales du mouvement

\begin{eqnarray}
	[ \operator{H} , \operator{Q} ] = 	[ \operator{H} , \operator{P} ] = O. 
\end{eqnarray}

Il convient de noter que dans le chapitre 2 , nous allons construire un nombre infini d'intégrales du mouvement.
 
\subsection{États propres du système à N particules}

\subsubsection{Construction de l’état propre}

Nous pouvons maintenant chercher les fonctions propres communes $\vert \psi_N\rangle$ des opérateurs $\operator{H}$, $\operator{Q}$, et $\operator{P}$ :

\begin{eqnarray}
	\vert \psi_N ( \theta_1 , \cdots , \theta_N ) \rangle & = & \frac{1}{\sqrt{N!}} \int d^N z \, \chi_N ( z_1 , \cdots , z_N  ~\vert ~ \theta_1 , \cdots , \theta _N ) \operator{\Psi}^\dag (z_1 ) \cdots \operator{\Psi}^\dag (z_N )	 \vert 0 \rangle. \label{eq.1.9}
\end{eqnarray}

%\subsubsection{Fonction d’onde $\chi_N$}

\subsection{Équations intégrales de la TBA}
\subsubsection{Opérateurs nombre de particules $\operator{Q}$ et moment $\operator{P}$}
L'opérateur du nombre de particules $\operator{Q}$ et l'opérateur de moment $\operator{P}$ sont définis comme 
\begin{eqnarray}
	\operator{Q} & = & \int \operator{\Psi}^\dag (x) \operator{\Psi} (x) \, d x \\
	\operator{P} & = & - \frac{i}2 \int \left \{  \operator{\Psi}^\dag(x) \operator{\partial}_x \operator{\Psi}(x) - \left [ \operator{\partial}_x \operator{\Psi}^\dag(x)\right ] \operator{\Psi}(x)\right \} dx \label{eq.1.7}
\end{eqnarray}

\subsubsection{Propriétés}

Ce sont des opérateurs hermitiens et ils constituent des intégrales du mouvement

\begin{eqnarray}
	[ \operator{H} , \operator{Q} ] = 	[ \operator{H} , \operator{P} ] = O. 
\end{eqnarray}

Il convient de noter que dans le chapitre 2 , nous allons construire un nombre infini d'intégrales du mouvement.
 
\subsection{États propres du système à N particules}

\subsubsection{Construction de l’état propre}

Nous pouvons maintenant chercher les fonctions propres communes $\vert \psi_N\rangle$ des opérateurs $\operator{H}$, $\operator{Q}$, et $\operator{P}$ :

\begin{eqnarray}
	\vert \psi_N ( \theta_1 , \cdots , \theta_N ) \rangle & = & \frac{1}{\sqrt{N!}} \int d^N z \, \chi_N ( z_1 , \cdots , z_N  ~\vert ~ \theta_1 , \cdots , \theta _N ) \operator{\Psi}^\dag (z_1 ) \cdots \operator{\Psi}^\dag (z_N )	 \vert 0 \rangle. \label{eq.1.9}
\end{eqnarray}

%\subsubsection{Fonction d’onde $\chi_N$}



