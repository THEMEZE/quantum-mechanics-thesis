\paragraph{I. Macro-états et entropie dans la TBA.}

%Dans la sous-sous-section précédente, nous avons expliqué comment les observables physiques, telles que les valeurs d'attente des charges et des courants, deviennent des fonctionnelles de la distribution de rapidité $\rho(\theta)$ dans la limite thermodynamique. Cependant, nous n'avons pas expliqué comment construire des distributions de rapidité physiquement significatives (sauf pour l'état fondamental de l'hamiltonien de Lieb-Liniger, pour lequel $\nu(\theta)$) est une fonction rectangulaire, voir la sous-section (??)). {\em Par exemple, quelle est la distribution de rapidité correspondant à un état d'équilibre thermique à une température non nulle ?}\\

Dans la sous-section précédente, nous avons vu que, dans la limite thermodynamique, les observables physiques deviennent des fonctionnelles de la distribution de rapidité $\rho(\theta)$. Cette description est efficace car elle permet d’échapper au détail de chaque état propre. Toutefois, cette simplification laisse en suspens une question cruciale : {\bf quelle est la distribution de rapidité d’un système à l'équilibre thermique à température finie ?} Pour répondre à cette question, nous devons comprendre la {\bf structure statistique des états propres} associés à une même distribution $\rho(\theta)$.

\paragraph{Distribution de rapidité comme macro-état.}

Chaque distribution de rapidité $\rho(\theta)$ ne correspond pas à un état propre unique, mais à un grand {\bf ensemble de micro-états} : différents choix des ensembles de quasi-moments ${\theta_a}$ peuvent conduire à la même densité de distribution à l’échelle macroscopique. Ainsi, $\rho(\theta)$ doit être interprétée comme un {\bf macro-état}, qui agrège un très grand nombre d’états propres microscopiques.

La question thermodynamique devient alors : {\bf Combien de micro-états microscopiquement distincts sont compatibles avec un même macro-état $\rho(\theta)$ ?} 

\paragraph{Dénombrement local des configurations microcanoniques.}
Pour répondre à cette question, on subdivise l’axe des rapidités en petites tranches ou cellules de largeur $\delta \theta$, chacune centrée en un point $\theta_a$. Dans une tranche $[\theta_a, \theta_a + \delta\theta]$, on suppose que la densité $\rho(\theta)$ est à peu près constante. Le nombre de quasi-particules dans cette tranche est alors approximativement :

\begin{eqnarray*}
	N_a = L\rho(\theta_a) \delta \theta,
\end{eqnarray*}

et le nombre total d'états disponibles (i.e., le nombre d’états possibles si toutes les positions en moment étaient disponibles) est donné par la densité totale de niveaux 

\begin{eqnarray*}
	M_a = L\rho_s(\theta_a) \delta \theta,
\end{eqnarray*}

La densité de niveaux $\rho_s(\theta)$ tient compte du fait que les moments sont quantifiés de manière discrète, en raison des équations de Bethe (voir équation (??)).

Sous l'hypothèse que les particules occupent ces niveaux de manière analogue à des fermions libres (principe d’exclusion de Pauli), le nombre de manières différentes de choisir $N_a$ niveaux parmi $M_a$ est donné par :
	
	%La question a été répondue dans les travaux pionniers de Yang et Yang (1969), que nous allons maintenant examiner brièvement. Tout d'abord, nous observons qu'il existe de nombreuses choix différents de séquences d'états propres $(\{\theta_a\}_{ a \in \llbracket 1 , N \rrbracket} )_{ N \in \mathbb{Z}}$ qui conduisent à la même distribution de rapidité thermodynamique (??). La description du système en termes de distribution de rapidité $\rho( \theta ) $ est seulement une description grossière : on devrait considérer la distribution de rapidité $\rho ( \theta ) $ comme caractérisant un macro-état du système, correspondant à un très grand nombre de micro-états possibles $\vert \{ \theta_a \} \rangle $. {\em Pour faire de la thermodynamique, il faut estimer le nombre de ces micro-états.}\\
	
	%Pour estimer ce nombre, on se concentre sur une petite cellule de rapidité $[\theta, \theta+\delta\theta]$, qui contient $L\rho(\theta) \delta \theta$ rapidités. 
	
	\begin{figure}[H]
		\centering 
		\begin{tikzpicture}
			%\def\Occupation{
	\def\traitx{0.3}
	\def\traity{0.5}
	\draw
		(-10.5 , 0 ) edge [thick,line width=0.8ex ]( -3.2  , 0 )
		( -3.2 - \traitx  , 0 - \traity ) edge [thick,line width=0.8ex ]( -3.2 + \traitx  , 0 + \traity  )
		( -2.8 - \traitx  , 0 - \traity ) edge [thick,line width=0.8ex ]( -2.8 + \traitx  , 0 + \traity  )
		(-2.8 , 0 ) edge [thick,line width=0.8ex ](2.8  , 0 )
		( 2.8 - \traitx  , 0 - \traity ) edge [thick,line width=0.8ex ]( 2.8 + \traitx  , 0 + \traity  )
		( 3.2 - \traitx  , 0 - \traity ) edge [thick,line width=0.8ex ]( 3.2 + \traitx  , 0 + \traity  )
		(3.2, 0 ) edge [thick,line width=0.8ex,->,>=triangle 45 , color = black ]node [pos=1.01,below  ]{\huge$a$}	( 11  , 0 )
	;
	
	
	% Graduation abcsisse 
	\foreach \r in { 0 ,... , 2   } {
		\ifnum\r=0
			\draw [color = gray  ]
				( \r  , -0.3 ) edge [thick,line width=0.5ex] node [pos=-0.5 ]{\large \r }	( \r  , 0.3 )
			;
			\filldraw [line width=0.5ex , color = red ,outer color=red,inner color=red ] 
				( \r  , 0 )  circle (4pt)
			;
		\else\ifnum\r>0
			\draw [color = gray  ]
				( \r  , -0.3 ) edge [thick,line width=0.5ex] node [pos=-0.5 ,right]{\large \r }	( \r  , 0.3 )
				( -\r  , -0.3 ) edge [thick,line width=0.5ex]node [pos=-0.5, left ]{\large -\r }	( -\r  , 0.3 ) 
			;
			\filldraw [ line width=0.5ex , color = red ,outer color=red,inner color=red ]
				( -\r  , 0 )  circle (5pt)
			;
			\filldraw [ line width=0.5ex , color = red ,outer color=red,inner color=red ] 
				( \r  , 0 )  circle (5pt)
			;
		\fi\fi
	}

	
	\def\nmax{6};
	\def\nsmax{10}; % les trous
	%\foreach \r in {\nmax + 1 ,...,\nsmax} { % proble dans le -\r 
	\foreach \r in {4 ,...,\nsmax} {
		\draw [color=gray] 
			(\r,-0.3) edge [thick,line width=0.5ex]  (\r,0.3)
			(-\r,-0.3) edge [thick,line width=0.5ex]  (-\r,0.3)
		;
		\filldraw [line width=0.5ex , color = red ,outer color=white,inner color=white ] (\r,0) circle (4pt);
		\filldraw [line width=0.5ex , color = red ,outer color=white,inner color=white] (-\r,0) circle (4pt);
	}
	
	
	\foreach \r in {4,...,\nmax} {
		\ifnum\r=\nmax
			\draw [color=gray] 
				(\r,-0.3) edge [thick,line width=0.5ex] node [pos=-0.5]{ $\frac{N}{2}$ } (\r,0.3)
				(-\r,-0.3) edge [thick,line width=0.5ex] node [pos=-0.5]{ $-\frac{N}{2}$ } (-\r,0.3)
			;
			\filldraw [line width=0.5ex , color = red ,outer color=red,inner color=red ] (\r,0) circle (4pt);
			\filldraw [line width=0.5ex , color = red ,outer color=red,inner color=red ] (-\r,0) circle (4pt);
		\else\ifnum\r>4
			%\def\nreste{\nmax-\r}
			\pgfmathtruncatemacro{\nreste}{\nmax-\r}

			\draw [color=gray] 
				(\r,-0.3) edge [thick,line width=0.5ex] node [pos=-0.5]{ $\frac{N}{2} - \nreste$ } (\r,0.3)
				(-\r,-0.3) edge [thick,line width=0.5ex] node [pos=-0.5]{$-\frac{N}{2}+\nreste$ } (-\r,0.3)
			;
			\filldraw [line width=0.5ex , color = red ,outer color=red,inner color=red ] (\r,0) circle (4pt);
			\filldraw [line width=0.5ex , color = red ,outer color=red,inner color=red ] (-\r,0) circle (4pt);
		\else
			\draw [color=gray] 
				(\r,-0.3) edge [thick,line width=0.5ex] (\r,0.3)
				(-\r,-0.3) edge [thick,line width=0.5ex] (-\r,0.3)
			;
			\filldraw [line width=0.5ex , color = red ,outer color=red,inner color=red ] (\r,0) circle (4pt);
			\filldraw [line width=0.5ex , color = red ,outer color=red,inner color=red ] (-\r,0) circle (4pt);
		\fi\fi
	}
	
	
			
}


\begin{scope}
	%\draw[help lines , width=1.5ex] (-8,-3) grid (8,3);\draw[help lines ,width=0.5ex , opacity = 0.5] (-3,-3) grid[step=0.1] (3,3));
	
	%\draw[help lines] 
	%	(-3,-3) edge[width=1.5ex] grid (3,3)	
	%	(-3,-3) edge[width=0.5ex , opacity = 0.5] grid (3,3)	
	%;
	\begin{scope}[shift={(0,1)},rotate=0,opacity=1,color=black]
		\Occupation	
		
		\node[anchor=east, font=\bfseries] at (-11, 0) {\color{red}\large (T = 0 )} ;	
	\end{scope}
	
	
	\begin{scope}[shift={(0,-1.0)},rotate=0,opacity=1,color=black]
		\Occupation
		
		\node[anchor=east, font=\bfseries] at (-11, 0) {\color{red}\large (T > 0 )} ;
		%\def\nmax{8};
	
		\def\r{1};
		\draw [color=gray] 
			(\nmax-\r-1,-0.3) edge [thick,line width=0.5ex]  (\nmax-\r-1,0.3)
			(-\nmax+\r,-0.3) edge [thick,line width=0.5ex]  (-\nmax+\r,0.3)
			;
		\filldraw [line width=0.5ex , color = red ,outer color=white,inner color=white ] (\nmax-\r-1,0) circle (4pt);
		\filldraw [line width=0.5ex , color = red ,outer color=white,inner color=white] (-\nmax+\r,0) circle (4pt);
	
		\def\r{-1};
		\draw [color=gray] 
			(\nmax-\r,-0.3) edge [thick,line width=0.5ex]  (\nmax-\r,0.3)
			(-\nmax+\r,-0.3) edge [thick,line width=0.5ex]  (-\nmax+\r,0.3)
			;
		\filldraw [line width=0.5ex , color = red ,outer color=red,inner color=red ] (\nmax-\r,0) circle (4pt);
		\filldraw [line width=0.5ex , color = red ,outer color=red,inner color=red] (-\nmax+\r,0) circle (4pt);
	
		\draw[line width=0.8ex , color = red ,->,>=latex] ( \nmax - 1.8  ,0.7) to[out=55,in=125] (\nmax + 0.8 ,0.7) ;
		\draw[line width=0.8ex , color = red ,->,>=latex] (-\nmax + 0.8 ,0.7) to[out=125,in=55] (-\nmax - 0.8 ,0.7);
	
	
	\end{scope}
	
	\begin{scope}[shift={(-10.5,3)},rotate=0,opacity=1,color=black]
	
	\begin{scope}[shift={(-0,0)},rotate=0,opacity=1,color=black]
	
		\draw[shift={(0,0)} ,line width=1ex,rounded corners = 1ex,color=\colorslide , opacity =1 ,fill=\colorslide!00 , pattern={north east lines} , pattern color=\colorslide!00 ]
			(0 , -1 ) rectangle (5,1)
		;
		
		\filldraw [line width=0.5ex , color = red ,outer color=red,inner color=red ] (0.5 , 0.5) circle (4pt); 
		\node[anchor=west, font=\bfseries] at (0.7, 0.5) {\color{\colorslide}\large : quasi-particule};
		
		\filldraw [line width=-0.5ex , color = red ,outer color=white,inner color=white ] (0.5 , -0.5) circle (4pt); 
		\node[anchor=west, font=\bfseries] at (0.7, -0.5) {\color{\colorslide}\large : hole} ;
	\end{scope}
	
	\begin{scope}[shift={(6,0)},rotate=0,opacity=1,color=black]	
		
		\draw[shift={(0,0)} ,line width=1ex,rounded corners = 1ex,color=\colorslide , opacity =1 ,fill=\colorslide!00 , pattern={north east lines} , pattern color=\colorslide!00 ]
			(0 , -1 ) rectangle (7.5,1)
		;
		
		\node[anchor=west] at (0.5, 0.5) {\color{\colorslide}\large $\rho$ };\node[anchor=west, font=\bfseries] at (0.9, 0.5) {\color{\colorslide}\large : quasi-particule distribution};
		
		\node[anchor=west] at (0.5, -0.5) {\color{\colorslide}\large $\rho_h$ };\node[anchor=west, font=\bfseries] at (0.9, -0.5) {\color{\colorslide}\large  : hole distribution};
		
	\end{scope}
	
	\begin{scope}[shift={(14.5,0)},rotate=0,opacity=1,color=black]	
		
		\draw[shift={(0,0)} ,line width=1ex,rounded corners = 1ex,color=\colorslide , opacity =1 ,fill=\colorslide!00 , pattern={north east lines} , pattern color=\colorslide!00 ]
			(0 , -0.5 ) rectangle (7.0,0.5)
		;
		
		\node[anchor=west] at (0.5, 0) {\color{\colorslide}\large $\rho_s = \rho + \rho_h $ };\node[anchor=west, font=\bfseries] at (2.5, 0) {\color{\colorslide}\large : density of states};
		
	\end{scope}
	
	
	\end{scope}


		
	
\end{scope}

	
			\begin{scope}[transform canvas={scale=0.6}]
			%% Définition des couleurs avec les codes HTML
\definecolor{colorOne}{HTML}{443E46}
\definecolor{colorTwo}{HTML}{F6DEB8}
\definecolor{colorThree}{HTML}{908CA4}
\definecolor{colorFour}{HTML}{57659E}
\definecolor{colorFive}{HTML}{C57284}
\definecolor{colorSix}{HTML}{FF5B69}

% Raccourcis pour les couleurs
\def\colorOne{colorOne}
\def\colorTwo{colorTwo}
\def\colorThree{colorThree}
\def\colorFour{colorFour}
\def\colorFive{colorFive}
\def\colorSix{colorSix}

\def\colorslide{blue!50!black}

\def\Occupation{
	\def\traitx{0.3}
	\def\traity{0.5}
	\draw[shift={(0,0)}]
		(-13.5 , 0 ) edge [thick,line width=0.8ex ]( -3.2  , 0 )
		( -3.2 - \traitx  , 0 - \traity ) edge [thick,line width=0.8ex ]( -3.2 + \traitx  , 0 + \traity  )
		( -2.8 - \traitx  , 0 - \traity ) edge [thick,line width=0.8ex ]( -2.8 + \traitx  , 0 + \traity  )
		(-2.8 , 0 ) edge [thick,line width=0.8ex ](2.8  , 0 )
		( 2.8 - \traitx  , 0 - \traity ) edge [thick,line width=0.8ex ]( 2.8 + \traitx  , 0 + \traity  )
		( 3.2 - \traitx  , 0 - \traity ) edge [thick,line width=0.8ex ]( 3.2 + \traitx  , 0 + \traity  )
		(3.2, 0 ) edge [thick,line width=0.8ex,->,>=triangle 45 , color = black ]node [pos=1.01,below  ]{\huge$\theta$}	( 13  , 0 )
	;
	\draw[shift={(0,0)}, color=\colorOne]
		(-10.5 , -1.5 ) edge [thick,line width=0.8ex , ->,>=triangle 45  ]( -10.5  , 4.5 )
	;
		
	\foreach \r in {1 , ... , 3 } {
%		\draw[
%		decoration={
%		markings,
%    	mark connection node=my node,
%    	mark=at position 0 with{\node [blue,transform shape] (my node) {\large \r};}},
%		color=gray, thick, 
%		line width=0.5ex] decorate { 
%            (-11.0, \r) -- (-10.1, \r )}
%        ;
        \draw[
			color=\colorOne,
			] 
            (-11.0, \r) edge[color=\colorThree , thick,line width=0.5ex] node [pos=-0.5 ]{\large\color{\colorFour} $\frac{\r}{\delta \theta}$ } (-10.3, \r )
        	;
	
	}
	

	
	% Graduation abcsisse 
	% Définitions des listes
% Definitions of the lists
\def\listetuple{-9/\theta_{1}, -8/\theta_{2} , -5/\theta_{3} , -2/\theta_{a-1} , 0/\theta_{a} , 1/\theta_{a+1} , 2/\theta_{a+2} ,  5/\theta_{N-4} , 7/\theta_{N-3},8/\theta_{N-1},9/\theta_{N} }
\def\listetrais{-12 , -11, -10, -9 , -8 , -7 ,  -6 , -5, -4.5,-4, -2 , -1, 0 , 0.5, 1, 2, 4 , 5 ,  6 , 7 , 8 ,8.5, 9 ,  10 , 11, 12 }

% Loop over listetrais
\foreach \r in \listetrais {
    % Initialize found variable to zero
    % Initialize found variable to zero
    %\pgfmathsetmacro\found{0}
    \global\def\found{0}
    \xdef\nomtheta{}
    
    % Check if \r is in listetuple
    \foreach \x/\y in \listetuple { 
        \ifdim \r pt=\x pt % If \r matches any \x in listetuple
            \global\def\found{1} ;
            \xdef\nomtheta{\y} % Set \nomtheta to the corresponding \y
            %\pgfmathsetmacro\found{1} % Set found to 1            
            %\global\pgfmathsetmacro\found{1}
        \fi
    }
    
    %\node [circle, draw, red] (A) at (\r, 2) {\found , $\nomtheta$};
    
    % Draw the line and display \nomtheta if found
    \ifnum\found=1
        \draw[color=\colorOne, thick, line width=0.5ex] 
            (\r, -0.3) -- (\r, 0.3) node[red , pos=-0.5] {\large $\nomtheta$};
         \filldraw[line width=0.5ex, color=\colorSix, outer color=\colorSix, inner color=\colorSix] 
            (\r, 0) circle (4pt);
    \else 
        % Draw without \nomtheta and add a blue circle if not found
        \draw[color=\colorOne, thick, line width=0.5ex] 
            (\r, -0.3) -- (\r, 0.3);
        \filldraw[line width=0.5ex, color=\colorSix, outer color=\colorTwo, inner color=\colorTwo] 
            (\r, 0) circle (4pt); 
    \fi
}

\def\listetrais{-9.5/\theta_{i-1}/2/3, -6.5/\theta_{i}/1/4  ,   -1.5/\theta_{j}/2/4 , 1.5/\theta_{j+1}/-1/3 , 3.5/\theta_{\ell-1}/1/3 , 6.5/\theta_{\ell}/3/4 , 9.5/\theta(\theta_{\ell+1})/-1/3 };



\foreach \r/\nomx/\y/\ys in \listetrais {
	\draw[
		decoration={
		markings,
    	mark connection node=my node,
    	mark=at position .5 with{\node [blue,transform shape] (my node) {\large \color{\colorFour} $\nomx$};}},
		color=\colorThree , thick, 
		line width=0.5ex] decorate { 
            (\r, 0.12) -- (\r, -1.2)}
        ;
     
     \ifdim \y pt > -1 pt 
     	\draw[
			decoration={
			markings,
    		mark connection node=my node,
    		mark=at position .5 with{\node [blue,transform shape] (my node) {\large \color{\colorFour} $\Pi(\nomx) $};}},
			color=\colorThree, thick, 
			line width=0.5ex] decorate { 
            (\r, \y) -- (\r +3, \y)}
        ;
        \draw[
			decoration={
			markings,
    		mark connection node=my node,
    		mark=at position .5 with{\node [blue,transform shape] (my node) {\large \color{\colorFive} $\Pi_s(\nomx) $};}},
			color=\colorFive, thick, 
			line width=0.5ex] decorate { 
            (\r, \ys) -- (\r +3, \ys)}
        ;
     \fi 
     \ifdim \r pt= -1.5 pt
     	\draw[
     		decoration={
			markings,
    		mark connection node=my node,
    		mark=at position .5 with{\node [blue,transform shape] (my node) {\large \color{\colorFour}  $\delta \theta $};},
    		%mark=at position 0.1  with {\arrow[blue, line width=0.5ex]{<}},
    		%mark=at position 1  with {\arrow[blue, line width=0.5ex]{>}}
    		},
        	color=\colorThree,
        	thick,
        	line width=0.5ex,
        	%arrows={Computer Modern Rightarrow[line cap=round]-Computer Modern Rightarrow[line cap=round]}
   			](\r, -1.2) edge[arrows={Computer Modern Rightarrow[line cap=round]-}] (\r + 0.4, -1.2)decorate {
    		(\r, -1.2) -- (\r + 3, -1.2)}(\r + 2, -1.2) edge[arrows={-Computer Modern Rightarrow[line cap=round]}] (\r + 3, -1.2)
    		;
    \fi
			
	
}


			
}


\begin{scope}
	%\draw[help lines , width=1.5ex] (-8,-3) grid (8,3);\draw[help lines ,width=0.5ex , opacity = 0.5] (-3,-3) grid[step=0.1] (3,3));
	
	%\draw[help lines] 
	%	(-3,-3) edge[width=1.5ex] grid (3,3)	
	%	(-3,-3) edge[width=0.5ex , opacity = 0.5] grid (3,3)	
	%;
	\begin{scope}[shift={(0,1)},rotate=0,opacity=1,color=black]
		\Occupation	
		
		%\node[anchor=east, font=\bfseries] at (-11, 0) {\color{red}\large (T = 0 )} ;	
	\end{scope}
	
	
	
	
	\begin{scope}[shift={(-10.5,7)},rotate=0,opacity=1,color=black]
	
	\begin{scope}[shift={(-0,0)},rotate=0,opacity=1,color=black]
	
		\draw[shift={(0,0)} ,line width=1ex,rounded corners = 1ex,color=\colorOne , opacity =1 ,fill=\colorOne!00 , pattern={north east lines} , pattern color=\colorOne!00 ]
			(0 , -1 ) rectangle (5,1)
		;
		

		\begin{scope}[shift={(0.5,0.5)}]
			\draw[color=\colorOne, thick, line width=0.5ex] 
            (0, -0.3) -- (0, 0.3) ;
            \filldraw[line width=0.5ex, color=\colorSix, outer color=\colorSix, inner color=\colorSix] 
            (0, 0) circle (4pt);
            
            \node[anchor=west, font=\bfseries] at (0.2, 0) {\color{\colorSix}\large : quasi-particule};
		\end{scope}
		
		\begin{scope}[shift={(0.5,-0.5)}]
			\draw[color=\colorOne, thick, line width=0.5ex] 
            (0, -0.3) -- (0, 0.3) ;
            \filldraw[line width=0.5ex, color=\colorSix, outer color=\colorTwo, inner color=\colorTwo] 
            (0, 0) circle (4pt);
            
            \node[anchor=west, font=\bfseries] at (0.2, 0) {\color{\colorSix}\large : hole};
		\end{scope}

	\end{scope}
	
	\begin{scope}[shift={(6,0)},rotate=0,opacity=1,color=black]	
		
		\draw[shift={(0,0)} ,line width=1ex,rounded corners = 1ex,color=\colorOne , opacity =1 ,fill=\colorOne!00 , pattern={north east lines} , pattern color=\colorOne!00 ]
			(0 , -1 ) rectangle (7.5,1)
		;
		
		\node[anchor=west] at (0.5, 0.5) {\color{\colorFour}\large $\Pi$ };\node[anchor=west, font=\bfseries] at (1, 0.5) {\color{\colorFour}\large : quasi-particule distribution};
		
		\node[anchor=west] at (0.5, -0.5) {\color{\colorFour}\large $\Pi_h$ };\node[anchor=west, font=\bfseries] at (1, -0.5) {\color{\colorFour}\large  : hole distribution};
		
	\end{scope}
	
	\begin{scope}[shift={(14.5,0)},rotate=0,opacity=1,color=black]	
		
		\draw[shift={(0,0)} ,line width=1ex,rounded corners = 1ex,color=\colorOne , opacity =1 ,fill=\colorOne!00 , pattern={north east lines} , pattern color=\colorOne!00 ]
			(0 , -0.5 ) rectangle (7.0,0.5)
		;
		
		\node[anchor=west] at (0.2, 0) {\color{\colorFour}\large ${\color{\colorFive}\Pi_s} = \Pi + \Pi_h $ } node[anchor=west , font=\bfseries] at (3.1 , 0 )  {\color{\colorFour}\large {\color{\colorFive} : density of states}};
		
	\end{scope}
	
	
	\end{scope}


		
	
\end{scope}

	
			\end{scope}
			
			\draw[color = red , scale = 0.5 , draw = none ] (-13.5 , -1) rectangle (13 , 10) ; 
				
			
		\end{tikzpicture}	
		\captionsetup{skip=10pt} % Ajoute de l’espace après la légende
	\end{figure}
	
	
	
%Les équations de Bethe (??) relient ces rapidités aux moments de fermions pa dans une cellule de moment $[p, p+\delta p]$, où $\delta \rho/\delta \theta$ est d'environ $2\pi \rho_s(\theta)$, voir l'équation (??). Il est important de noter que les moments de fermions $p_a$ sont soumis au principe d'exclusion de Pauli. Le nombre de micro-états est alors évalué en comptant le nombre de configurations de moments de fermions mutuellement distincts, équivalent à $\rho (\theta)\delta \theta$, qui peuvent être placées dans la boîte $[p, p + \delta p]$. Comme l'espacement minimal entre deux moments est de $2\pi /L$, la réponse est :
	
\begin{eqnarray}
	\# \mbox{conf.}(\theta_a) & \approx  & \left (\begin{array}{c} M_a \\ N_a\end{array} \right ) ~= ~   \frac{[ L\rho_s ( \theta ) \delta \theta ] ! }{ [ L\rho ( \theta ) \delta \theta ] ! [( L\rho_s ( \theta ) - L\rho ( \theta ) )  \delta \theta ] ! }. 	
\end{eqnarray}

\paragraph{Estimation asymptotique à l’aide de Stirling.}

En utilisant la formule de Stirling :
	%or avec la formule de Sterling :  
\begin{eqnarray}
	n! & \underset{n \to \infty}{\sim} n^n e^{-n} \sqrt{2\pi n}.,
\end{eqnarray}
	
composé du fonction logarithmique, il vient cette équivalence : 
\begin{eqnarray}
	\ln n! & \underset{n \to \infty}{\rightarrow} & n \ln n \underbrace{- n + \ln \sqrt{2 \pi n }}_{o \left ( n \ln n \right ) } ,\\
	&  \underset{n \to \infty}{\sim} & n \ln n  
\end{eqnarray}
	
$\# \mbox{conf.}$ est jamais null donc on peut approximer, pour de grandes valeurs de $L$ et de $\delta\theta$  : 
\begin{eqnarray}
    \ln \# \mbox{conf.}(\theta) & \underset{\underset{\rho (\theta )\leq  \rho_s (\theta )}{\rho \delta \theta  \to \infty}}{\sim}   & L [ \rho_s\ln \rho_s - \rho \ln \rho - (\rho_s - \rho ) \ln ( \rho_s - \rho) ] (\theta )\delta \theta .
\end{eqnarray}

Cette expression donne la contribution à l’{\bf entropie locale} (par unité de $\theta$) associée à la cellule autour de $\theta_a$.

\paragraph{Entropie totale de Yang-Yang.}
%L'entropie totale du macro-état $\rho(\theta)$, notée $\mathcal{S}_{YY}[\rho]$, est obtenue en sommant sur toutes les tranches. Pour alléger la notation, nous écrivons cette somme comme :
Le nombre total de micro-états est le produit de toutes ces configurations pour toutes les cellules de rapidité $[\theta, \theta + \delta \theta]$. %En prenant le logarithme et en remplaçant la somme par une intégrale sur $ \theta$, nous obtenons l'entropie de Yang-Yang :

L'entropie totale du macro-état $\rho(\theta)$, notée $\mathcal{S}_{YY}[\rho]$, est obtenue en sommant sur toutes les tranches. Pour alléger la notation, nous écrivons cette somme comme :
\begin{eqnarray}
	\sum_a^{\theta-\mbox{\tiny tranches}}	
\end{eqnarray}

où chaque $a$ indexe une cellule de rapidité $[\theta_a, \theta_a + \delta\theta]$.
On obtient :

\begin{eqnarray}
    \ln \# \mbox{micro-états.} & = & \sum_a^{\theta-\mbox{\tiny tranches}} \ln \# \mbox{conf.}(\theta_a), \\
    & \approx &   L\mathcal{S}_{YY} [ \rho ] , 	
\end{eqnarray}

avec 

\begin{eqnarray}
    \mathcal{S}_{YY}[\rho] & \doteq & \sum_a^{\theta-\mbox{\tiny tranches}} \, [ \rho_s\ln \rho_s - \rho \ln \rho - ( \rho_s - \rho ) \ln ( \rho_s - \rho ) ] (\theta_a) \delta \theta .
\end{eqnarray}
	
Les variation de $w$ sont négligeables sur $\delta \theta $. Alors o  $\sum_{a = 1}^N  f(\theta_a) = \sum_{a \vert tranche } f(\theta_a) \Pi( \theta_a)\delta \theta$.

\begin{eqnarray}
	 \mathcal{W}[w] & = & \sum_{a = 1}^N  w(\theta_a)	 ~ \sim ~ L\mathcal{W}[w,\rho] ~=~ L \sum_a^{\theta-\mbox{\tiny tranches}}	 w(\theta_a) \rho(\theta_a) \delta \theta.
\end{eqnarray}

	
	
\begin{eqnarray}
	\langle \operator{\mathcal{O}} \rangle_{GGE} & =  & \frac{  \displaystyle \sum_{\{\theta_a \} \vert \rho } \langle  \{ \theta_a\}  \vert   \operator{\mathcal{O}} \vert \{ \theta_a\} \rangle \# \mbox{micro-états.} e^{- L\mathcal{W}[w,\rho]    }}{ \displaystyle \sum_{\{\theta_a \} \vert \rho }  \# \mbox{micro-états.}e^{- L\mathcal{W}[w,\rho] } } ~ \approx  ~ \frac{  \displaystyle \sum_{\{\theta_a \} \vert \rho } \langle  \{ \theta_a\}  \vert   \operator{\mathcal{O}} \vert \{ \theta_a\} \rangle e^{L(\mathcal{S}_{YY}[\rho] -  \mathcal{W}[w,\rho]) }}{ \displaystyle \sum_{\{\theta_a \} \vert \rho } e^{L(\mathcal{S}_{YY}[\rho] -  \mathcal{W}[w,\rho]) } },
\end{eqnarray}
où $ \sum_{\{\theta_a \} \vert \rho }$ est une sommes sur les configuration à distr
	
Lorsque l'observable $\operator{\mathcal{O}}$ est suffisamment local, on croit que la valeur d'attente $\langle  \{ \theta_a\}  \vert   \mathcal{O} \vert \{ \theta_a\} \rangle$ ne dépend pas de l'état microscopique spécifique du système, de sorte qu'elle devient une fonctionnelle de $\Pi$ dans la limite thermodynamique.
\begin{eqnarray}
	\underset{\mbox{\tiny therm.}}{\lim} \langle  \{ \theta_a\}  \vert   \operator{\mathcal{O}} \vert \{ \theta_a\} \rangle & = & \langle \operator{\mathcal{O}}\rangle_{[\Pi]},
\end{eqnarray}
	
	et 
	
	\begin{eqnarray}
		\underset{\mbox{\tiny therm.}}{\lim} \langle \operator{\mathcal{O}} \rangle_{GGE} & =  & \frac{ \sum_{\Pi} \langle \operator{\mathcal{O}}\rangle_{[\Pi]} \# \mbox{microstates.} e^{- \sum_{a \vert tranche} f(\theta_a) \Pi ( \theta_a )  \delta \theta}    }{ \sum_{\Pi} \# \mbox{microstates.}  e^{- \sum_{a \vert tranche }  f(\theta_a) \Pi ( \theta_a ) \delta \theta } },\\
		& = & \frac{ \sum_{\Pi} \langle \operator{\mathcal{O}}\rangle_{[\Pi]}  e^{ \mathcal{S}_{YY}[\Pi]- \sum_{a \vert tranche} f(\theta_a) \Pi ( \theta_a )  \delta \theta}    }{ \sum_{\Pi}  e^{\mathcal{S}_{YY}[\Pi] - \sum_{a \vert tranche }  f(\theta_a) \Pi ( \theta_a ) \delta \theta } }\\
		& = &  \sum_{\Pi} \langle \operator{\mathcal{O}}\rangle_{[\Pi]} P_{\{\Pi\}}
	\end{eqnarray}
	
	avec la probabilité de la configuration $\{ \Pi \}$  : 
	
	\begin{eqnarray}
		P_{\{\Pi\}} & = & \frac{e^{\mathcal{S}_{YY}[\Pi] - \sum_{a \vert tranche }  f(\theta_a) \Pi ( \theta_a ) \delta \theta }}{ \sum_{\Pi}  e^{\mathcal{S}_{YY}[\Pi] - \sum_{a \vert tranche }  f(\theta_a) \Pi ( \theta_a ) \delta \theta } }
	\end{eqnarray}
	
	Il est tentant de définir une action  et une fonction de partition.