\paragraph{Moyenne des observables dans l’ensemble généralisé de Gibbs.}

%Considérons un système homogène de taille \( L \). Faisons l’hypothèse qu’après relaxation, ce système est décrit par un \textbf{ensemble généralisé de Gibbs} (GGE). Comme nous l’avons vu dans le chapitre précédent, la valeur moyenne d’une observable \( \langle\operator{\mathcal{O}\rangle \) dans un ensemble statistique peut s’écrire sous la forme d’une \textbf{intégrale fonctionnelle} :
%où :
%\begin{itemize}
%    \item \( \rho(\theta) \) est la densité de rapidité définissant un macro-état,
%    \item \( \mathcal{S}_{YY}[\rho] \) est l’entropie de Yang-Yang,
%    \item \( \mathcal{W}[\rho] \) est l’énergie généralisée,
%    \item \( \mathcal{O}[\rho] \) est la valeur de l’observable dans un état propre associé à la densité \( \rho \).
%\end{itemize}
%Considérons un système homogène de taille $L$. Faisons l'hypothèse qu'après relaxation, le système est décrite par un ensemble généralisé de Gibbs (GGE). On a vus dans le chapire d'avant que la valeur moyenne des observables $\langle\operator{\mathcal{O}}\rangle$ dans un ensemble statistique s'écrit comme une intégrale formelle

%\begin{eqnarray*}
%	\langle \operator{\mathcal{O}} \rangle & = & \frac{\int \mathcal{D} \rho \; e^{L (\mathcal{S}_{YY}[\rho] - \mathcal{W}[\rho])} \, \mathcal{O}[\rho]}{\int \mathcal{D} \rho \; e^{L (\mathcal{S}_{YY}[\rho] - \mathcal{W}[\rho])}}, %\label{chap:fluctu:eq:ensemble_average}
%\end{eqnarray*}

%avec $\mathcal{O}[\rho]$ la valeur de l’observable dans un état propre caractérisé par la distribution de rapidité $\rho$, $\mathcal{S}_{YY}$ l'entropie de Yang-Yang et $\mathcal{W}$ l'énergie généralisé.

\paragraph{Approximation au point selle.}

%Cette moyenne fonctionelle sur tous les profils de rapidité $\rho$, avec un poids $\propto e^{(\mathcal{S}_{YY}[\rho] - \mathcal{W}[\rho])}$. Dans la limite thermodynamique $L \to \infty$ , cette intégrale est dominée par la configuration $\langle \rho \rangle$ qui maximise le poids exponentiel, c’est-à-dire la distribution de rapidité la plus probable. Ainsi, la moyenne de l’observable s’écrit à première approximation :

Dans la limite thermodynamique \( L \to \infty \), cette intégrale est dominée par la configuration \( \langle \rho \rangle \) qui maximise le poids exponentiel. Il s’agit de la densité de rapidité la plus probable, solution d’un problème de maximisation :

 %est domminé par la la configuration caractérisé par la distribution de rapité le plus probable $\langle \rho \rangle$. De ce fait la moyenne  des observables $\langle\operator{\mathcal{O}}\rangle $ s'écrit 

\begin{eqnarray}
	\langle \operator{\mathcal{O}} \rangle & \approx & \mathcal{O}[\langle \rho \rangle ].	
	\label{chap:TBA:eq:ensemble_average:approx}
\end{eqnarray}

Cette approximation correspond à une méthode de \textit{selle statique}, où l’on développe l’action effective au voisinage de la distribution dominante.


\paragraph{Développement fonctionnel au premier ordre.}

On effectue un développement de Taylor fonctionnel de l'action à l’ordre linéaire :

\begin{eqnarray*}
	\mathcal{S}_{YY}[\rho] - \mathcal{W}[\rho] & \approx & \mathcal{S}_{YY}[\langle\rho\rangle] - \mathcal{W}[\langle\rho\rangle] +  \left. \frac{\delta (\mathcal{S}_{YY}[\rho] - \mathcal{W}[\rho]) }{\delta \rho} \right|_{\rho = \langle \rho \rangle }	(\delta \rho),
	\label{chap:TBA:eq:action}	
\end{eqnarray*}	


La condition de stationnarité au point selle impose :

\begin{eqnarray*}
	\delta (\mathcal{S}_{YY} - \mathcal{W}) & = & 0  	
\end{eqnarray*}

soit 

\begin{equation}
\left. \frac{\delta \mathcal{S}_{YY}}{\delta \rho} \right|_{\rho = \langle \rho \rangle} = \left. \frac{\delta \mathcal{W}}{\delta \rho} \right|_{\rho = \langle \rho \rangle}.
\end{equation}

\paragraph{Équation intégrale de la TBA.}

Cette égalité donne naissance à une équation intégrale pour le poids effectif \( w \), défini comme la dérivée fonctionnelle de l'énergie généralisée :

\begin{eqnarray*}
	w ~=~ \left. \frac{\delta \mathcal{W}[\rho]}{\delta \rho} \right|_{\rho = \langle \rho \rangle } ~= ~\left. \frac{\delta \mathcal{S}_{YY}[\rho]}{\delta \rho} \right|_{\rho = \langle \rho \rangle } ~=~ \ln ( \langle \nu \rangle^{-1}  - 1 ) - \frac{\Delta}{2\pi} \star \ln ( 1 -  \langle \nu \rangle )
\end{eqnarray*}

où \( \star \) désigne la \textit{convolution} sur l’espace des rapidités. La densité de rapidité moyenne et la fonction d’occupation vérifient :

\begin{equation}
\langle \rho \rangle = \langle \nu \rangle \cdot \langle \rho_s \rangle
\end{equation}
et
\begin{equation}
\langle \nu \rangle = \frac{1}{1 + e^{\beta \epsilon(\theta)}}
\end{equation}

Cette dernière équation relie la fonction d’occupation \( \langle \nu \rangle \) à une pseudo-énergie \( \epsilon(\theta) \), caractéristique de la théorie thermodynamique de Bethe (TBA).