\begin{eqnarray*}
	2\pi \rho_s & = & 1 + \Delta \star \rho 
\end{eqnarray*}

\subsubsection{The dressing}

Dans les manipulations thermodynamiques, il s'avère que l'opération suivante est omniprésente : à une fonction $f(\theta)$, on associe sa contrepartie « habillée » $f^{dr}(\theta)$
, définie par l'équation intégrale :
\begin{eqnarray}
	f^{\mathrm{dr}}(\theta) & = & f(\theta) + \int \frac{1}{2\pi} \Delta( \theta - \theta' ) \nu ( \theta' ) f^{\mathrm{dr}}(\theta') d \theta ' \\
	& = & 	f(\theta) + \left \{ \frac{\Delta}{2\pi} \star (\nu \ast f^{\mathrm{dr}}) \right \} ( \theta ) 		
\end{eqnarray}


où $\Delta ( \theta - \theta' )/2\pi$ est un noyau de convolution spécifique qui dépend du problème considéré. Cette équation permet de capturer les effets de l'interaction entre les particules et de rendre compte des corrections « habillées » à la fonction initiale $f(\theta)$. En résolvant cette équation, on obtient la fonction habillée $f^{\mathrm{dr}}(\theta)$ associée à la fonction donnée $f(\theta)$. Cette opération est couramment utilisée dans les calculs thermodynamiques pour prendre en compte les interactions et les corrections aux propriétés des systèmes physiques.\\


Bien que cela ne soit pas explicite dans la notation, $f^{\mathrm{dr}}(\theta)$ est toujours fonctionnelle de la distribution de rapidité, à travers sa dépendance vis-à-vis du rapport d'occupation de Fermi. Par exemple, avec cette définition, l'équation constitutive (\ref{eq:therm.rho_s_2}) est reformulée comme suit :
\begin{eqnarray}
	2\pi \rho_s ( \theta ) & = & \mathrm{1}^{\mathrm{dr}}(\theta)		
\end{eqnarray}

où $\mathrm{1}(\theta) = 1$ est la fonction constante.
