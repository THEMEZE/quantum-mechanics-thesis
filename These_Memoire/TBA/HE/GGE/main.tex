On note la configuration $\{ \theta_a \}_{a\in \llbracket 1 , N \rrbracket}$ des rapidité à nombre $N$ de particule fixes. Dans le la chapitre précédant (??) on vus que l'état $\vert \{ \theta_a \} \rangle$ associé à la configuration $\{ \theta_a \}_{a\in \llbracket 1 , N \rrbracket}$ est fonction propre des observable nombre et énergie (??) . Ces observable sont diagonales dans la base des fonctions propre :

\begin{eqnarray}
	\operator{\mathcal{N}} & = & \sum_{ \{\theta_a\} }   \left ( \sum_{a = 1}^N  1 \right )  \vert \{ \theta_a\}\rangle	\langle \{ \theta_a \}\vert,\\
	\operator{\mathcal{H}}_N & = & \sum_{\{ \theta_a\}}  \left ( \sum_{a = 1}^N  \varepsilon ( \theta_a ) \right )   \vert \{ \theta_a\}\rangle	\langle \{ \theta_a \}\vert,		
\end{eqnarray}

avec $ \sum_{\{ \theta_a\}}$ une somme sur tous les configurations.\\

On introduis $\operator{\mathcal{O}}_i$ des observables dont les fonction  $\vert \{ \theta_a \} \rangle$  sont fonction propres associés au valeurs propres $\langle \operator{\mathcal{O}}_i \rangle_{\{\theta_a \}} = \langle \{ \theta_a \} \vert\operator{\mathcal{O}}_i \vert \{ \theta_a \} \rangle$ :\\

\begin{eqnarray}
	\operator{\mathcal{O}}_i & = & \sum_{ \{\theta_a\} }   \langle \operator{\mathcal{O}}_i \rangle_{\{\theta_a \}}  \vert \{ \theta_a\}\rangle	\langle \{ \theta_a \}\vert.,		
\end{eqnarray}

On cherche à maximiser l'entropie :
\begin{eqnarray}
	\mathcal{S} & = & -\mathrm{Tr} ( \operator{\rho} \ln \operator{\rho} ) , 	
\end{eqnarray}

sous la contrainte 

\begin{eqnarray}
	\langle \operator{\mathcal{O}}_i \rangle & = & 	\mathrm{Tr} ( \operator{\rho} \operator{\mathcal{O}}_i ) ,
\end{eqnarray}

où $\langle \operator{\mathcal{O}}_i \rangle$ est la moyenne de l'observable $\operator{\mathcal{O}}_i$.

Pour définir la matrice densité, on introduit les multiplicateur de Lagrange $\beta_i$ pour chaque contrainte, ce qui donne  
\begin{eqnarray}
	 \operator{\rho} &= & Z^{-1} \exp \left ( - \sum_i \beta_i \operator{\mathcal{O}}_i \right ) , 		
\end{eqnarray}

avec la fonction de partition 
\begin{eqnarray}
	Z & = & \mathrm{Tr} \left [ \exp \left ( - \sum_i \beta_i \operator{\mathcal{O}}_i \right ) \right ]. 	
\end{eqnarray}

Les multiplicateurs de Lagranges $\beta_i$ apparaissent naturellement lors de l'optimisation sous contraintes, par exemple dans le formalisme de l'ensemble de Gibbs généralisé (GGE), oû il imposent la conservation des valeurs moyennes des charges $\langle \operator{\mathcal{O}}_i \rangle $.

On peut ainsi définie la probabilité d'etre dans l'état $\vert \{ \theta_a \} \rangle$ :
\begin{eqnarray}
	\mathbb{P} ( \{ \theta_a \} ) & = & \mathrm{Tr} \left [\operator{\rho} \vert \{ \theta_a \} \rangle \langle \{ \theta_a \} \vert \rangle \right ] ~=~  \langle \{ \theta_a \} \vert	\operator{\rho} \vert \{ \theta_a \} \rangle ~=~ Z^{-1} \exp \left (- \sum_i \beta_i \langle\operator{\mathcal{O}}_i\rangle_{\{\theta_a\}} \right ) .
\end{eqnarray}

Une autre manière d'écrire la moyenne de $\operator{\mathcal{O}}_i$ est 

\begin{eqnarray}
	\langle \operator{\mathcal{O}}_i \rangle &= & \sum_{\{ \theta_a\}} \langle\operator{\mathcal{O}}_i\rangle_{\{\theta_a\}} \mathbb{P} ( \{ \theta_a \} ) ~=~- \left. \frac{1}{Z} \frac{\partial Z}{\partial \beta_i} \right )_{\beta_{j \neq i }} ~=~ - 	\left . \frac{\partial  \ln Z}{\partial \beta_i} \right )_{\beta_{j \neq i }}	
\end{eqnarray}

Par le même raisonnement la moyenne de $\operator{\mathcal{O}}_i^n$ s'écrit :

\begin{eqnarray}
	\langle \operator{\mathcal{O}}_i^n \rangle &= & \sum_{\{ \theta_a\}} \langle\operator{\mathcal{O}}_i\rangle_{\{\theta_a\}}^n \mathbb{P} ( \{ \theta_a \} ) ~=~ (-1)^n \left. \frac{1}{Z} \frac{\partial^n Z}{{(\partial \beta_i)}^n} \right )_{\beta_{j \neq i }} .	
\end{eqnarray}

On a besoins que $n=2$ pour écrire de la variance de l'observable $\operator{\mathcal{O}}_i$ 
\begin{eqnarray}
	\Delta_{\operator{\mathcal{O}}_i}^2 ~=~  	\left \langle \left (\operator{\mathcal{O}}_i - \langle\operator{\mathcal{O}}_i \rangle \right )^2  \right \rangle  = 	\langle\operator{\mathcal{O}}_i^2 \rangle  -  \langle\operator{\mathcal{O}}_i \rangle^2 = \left . \frac{1}{Z} \frac{ \partial^2 Z }{ {\partial \beta_i}^2 }  \right )_{\beta_{j\neq i}} - \left ( \left . \frac{1}{Z}\frac{ \partial Z }{ \partial \beta_i }  \right )_{\beta_{j\neq i}}\right )^2  \\
		& = & \frac{\partial}{\partial \beta_i } \left ( \left . \frac{1}{Z} \frac{\partial Z}{\partial \beta_i }  \right )_{\beta_{j\neq i}}  \right )_{\beta_{j\neq i}} =  \left . \frac{\partial^2 \ln Z  }{{\partial \beta_i}^2 }  \right )_{\beta_{j\neq i}}  = - \left . 	\frac{\partial \langle\operator{\mathcal{O}}_i \rangle }{\partial \beta_i } \right )_{\beta_{j\neq i}}		
\end{eqnarray}





 



	
	avec $\sum_{a = 1}^N 1 \equiv \langle \operator{\mathcal{N}} \rangle_{ \{\theta_a \} }  \doteq  \langle \{ \theta_a \}\vert  \operator{\mathcal{N}} \vert\{ \theta_a\}\rangle  $ et $  \sum_{a = 1}^N  \varepsilon ( \theta_a ) \equiv \langle \operator{\mathcal{E}} \rangle_{\{\theta_a \}}  \doteq  \langle \{ \theta_a \}\vert  \operator{ \mathcal{E}}  \vert\{ \theta_a\}\rangle $.

	
	La probabilité que le système soit dans configuration $\{ \theta_a \}$  est 
	\begin{eqnarray}
		P_{\{\theta_a \}} & = & \frac{e^{- \beta \left ( \langle \operator{\mathcal{E}} \rangle_{\{\theta_a \}}   - \mu \langle \operator{\mathcal{N}} \rangle_{\{\theta_a \}} \right )}}{Z_{thermal}} = \frac{e^{- \beta \sum_{a=1}^N  ( \varepsilon( \theta_a )   - \mu  )}}{Z_{thermal}}	
	\end{eqnarray}
	
	avec la fonction de partition $Z_{thermal} = \sum_{\{ \theta_a\}}e^{- \beta \left ( \langle \operator{\mathcal{E}} \rangle_{\{\theta_a \}}   - \mu \langle \operator{\mathcal{N}} \rangle_{\{\theta_a \}} \right )} = \sum_{\{ \theta_a\}} e^{- \beta \sum_{a=1}^N  ( \varepsilon( \theta_a )   - \mu  )}$
	
	\begin{eqnarray}
		\langle \operator{\mathcal{N}} \rangle  = \left .\frac{1}{\beta} \frac{ \partial \ln Z}{\partial \mu } \right )_T,  & & \Delta^2_{\operator{\mathcal{N}}} = \left . \frac{1}{\beta^2} \frac{ \partial^2 \ln Z}{{\partial \mu}^2 } \right )_T =  \left . \frac{1}{\beta} \frac{ \partial \langle \operator{\mathcal{N}} \rangle}{\partial \mu } \right )_T\\
		\langle \operator{\mathcal{E}} - \mu\operator{\mathcal{N}}  \rangle  = -\left . \frac{ \partial \ln Z}{\partial \beta } \right )_\mu,  & & \Delta^2_{\operator{\mathcal{E}} - \mu\operator{\mathcal{N}}} = \left .  \frac{ \partial^2 \ln Z}{{\partial \beta}^2 } \right )_\mu =  -\left .  \frac{ \partial \langle \operator{\mathcal{E}} - \mu\operator{\mathcal{N}} \rangle}{\partial \beta } \right )_\mu	\\
		\langle \operator{\mathcal{E}} \rangle  = \left [ \left .\frac{\mu}{\beta} \frac{ \partial}{\partial \mu } \right )_T -\left . \frac{ \partial }{\partial \beta } \right )_\mu  \right ]\ln Z,  & & \Delta^2_{\operator{\mathcal{E}} } = \left [ \left .\frac{\mu}{\beta} \frac{ \partial}{\partial \mu } \right )_T -\left . \frac{ \partial }{\partial \beta } \right )_\mu  \right ]^2\ln Z=  \left [ \left .\frac{\mu}{\beta} \frac{ \partial}{\partial \mu } \right )_T -\left . \frac{ \partial }{\partial \beta } \right )_\mu  \right ]\langle \operator{\mathcal{E}} \rangle	
	\end{eqnarray}

	
	
	La matrice densité thermique est :
	\begin{eqnarray}
		\operator{\rho}_{thermal} & = & \frac{e^{- \beta \operator{H}}}{Z_{thermal}}, \\
		e^{-\beta \operator{H}} & = & 	\sum_{\{\theta_a \}} e^{- \beta \sum_{a=1}^N ( \varepsilon(\theta_a)- \mu ) } \vert \{ \theta_a\} \rangle \langle  \{ \theta_a\}  \vert 
	\end{eqnarray}