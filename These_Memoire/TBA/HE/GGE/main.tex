\paragraph{Configuration des états à N particules.}
On introduit la configuration ${ \theta_a }_{a\in \llbracket 1 , N \rrbracket}$ des rapidités pour un nombre fixe $N$ de particules, et les états propres associés.

%%%%%%%%%%%%%%%%%%%%%%%%%%%%%%%%%%%%%%%%%%%%%%%%%%
\paragraph{Observables diagonales dans la base des états propres.}
Dans le chapitre précédent (??), on a vu que l'état $\vert { \theta_a } \rangle$ associé à cette configuration est une fonction propre des observables nombre et énergie (??). Ces observables sont diagonales dans la base des états propres :

\begin{eqnarray}
	\operator{\mathcal{N}} & = & \sum_{ \{\theta_a\} }   \left ( \sum_{a = 1}^N  1 \right )  \vert \{ \theta_a\}\rangle	\langle \{ \theta_a \}\vert,\\
	\operator{\mathcal{H}}_N & = & \sum_{\{ \theta_a\}}  \left ( \sum_{a = 1}^N  \varepsilon ( \theta_a ) \right )   \vert \{ \theta_a\}\rangle	\langle \{ \theta_a \}\vert,		
\end{eqnarray}

avec $ \sum_{\{ \theta_a\}}$ une somme sur tous les configurations.\\


%%%%%%%%%%%%%%%%%%%%%%%%%%%%%%%%%%%%%%%%%%%%%%%%%%
\paragraph{Définition générale d'observables conservées.}
On introduit une famille d'observables $\operator{\mathcal{O}}_i$ telles que les états $\vert { \theta_a } \rangle$ soient aussi fonctions propres de chacune de ces observables, avec pour valeurs propres $\langle \operator{\mathcal{O}}_i \rangle_{\{\theta_a \}} = \langle \{ \theta_a \} \vert\operator{\mathcal{O}}_i \vert \{ \theta_a \} \rangle$ :\\

\begin{eqnarray}
	\operator{\mathcal{O}}_i & = & \sum_{ \{\theta_a\} }   \langle \operator{\mathcal{O}}_i \rangle_{\{\theta_a \}}  \vert \{ \theta_a\}\rangle	\langle \{ \theta_a \}\vert.,		
\end{eqnarray}


%%%%%%%%%%%%%%%%%%%%%%%%%%%%%%%%%%%%%%%%%%%%%
\paragraph{Principe de maximisation de l'entropie.}
On cherche à maximiser l'entropie de Shannon/ von Neuman de l’état statistique :
\begin{eqnarray}
	\mathcal{S} & = & -\mathrm{Tr} ( \operator{\rho} \ln \operator{\rho} ) , 	
\end{eqnarray}

sous la contrainte que les moyennes des observables soient fixées :

\begin{eqnarray}
	\langle \operator{\mathcal{O}}_i \rangle & = & 	\mathrm{Tr} ( \operator{\rho} \operator{\mathcal{O}}_i ) ,
\end{eqnarray}

où $\langle \operator{\mathcal{O}}_i \rangle$ est la moyenne de l'observable $\operator{\mathcal{O}}_i$.

%%%%%%%%%%%%%%%%%%%%%%%%%%%%%%%%%%%%%%%%%%%%%%%%
\paragraph{Définition de la matrice densité et de la fonction de partition.}
Pour résoudre ce problème, on introduit les multiplicateurs de Lagrange $\beta_i$ pour chaque contrainte. Cela conduit à une matrice densité de la forme : 
\begin{eqnarray}
	 \operator{\rho} &= & Z^{-1} \exp \left ( - \sum_i \beta_i \operator{\mathcal{O}}_i \right ) , 		
\end{eqnarray}

avec la fonction de partition 
\begin{eqnarray}
	Z & = & \mathrm{Tr} \left [ \exp \left ( - \sum_i \beta_i \operator{\mathcal{O}}_i \right ) \right ]. 	
\end{eqnarray}
 %%%%%%%%%%%%%%%%%%%%%%%%%%%%%%%%%%%%%%%%%%%%%%%%
 \paragraph{Interprétation physique des multiplicateurs de Lagrange.}
Les multiplicateurs de Lagranges $\beta_i$ apparaissent naturellement lors de l'optimisation sous contraintes, par exemple dans le formalisme de l'ensemble de Gibbs généralisé (GGE), oû il imposent la conservation des valeurs moyennes des charges $\langle \operator{\mathcal{O}}_i \rangle $.

%%%%%%%%%%%%%%%%%%%%%%%%%%%%%%%%%%%%%%%%
\paragraph{Probabilité d’un état à rapidités fixées.}
On peut alors définir la probabilité d’occurrence d’un état $\vert { \theta_a } \rangle$ :
\begin{eqnarray}
	\mathbb{P} ( \{ \theta_a \} ) & = & \mathrm{Tr} \left [\operator{\rho} \vert \{ \theta_a \} \rangle \langle \{ \theta_a \} \vert \rangle \right ] ~=~  \langle \{ \theta_a \} \vert	\operator{\rho} \vert \{ \theta_a \} \rangle ~=~ Z^{-1} \exp \left (- \sum_i \beta_i \langle\operator{\mathcal{O}}_i\rangle_{\{\theta_a\}} \right ) .
\end{eqnarray}

%%%%%%%%%%%%%%%%%%%%%%%%%%%
\paragraph{Moyenne d’un observable et dérivées de $Z$.}
On peut écrire la moyenne d’une observable comme une somme pondérée par cette probabilité, ou encore comme une dérivée de la fonction de partition :

\begin{eqnarray}
	\langle \operator{\mathcal{O}}_i \rangle &= & \sum_{\{ \theta_a\}} \langle\operator{\mathcal{O}}_i\rangle_{\{\theta_a\}} \mathbb{P} ( \{ \theta_a \} ) ~=~- \left. \frac{1}{Z} \frac{\partial Z}{\partial \beta_i} \right )_{\beta_{j \neq i }} ~=~ - 	\left . \frac{\partial  \ln Z}{\partial \beta_i} \right )_{\beta_{j \neq i }}	
\end{eqnarray}

Par le même raisonnement la moyenne de $\operator{\mathcal{O}}_i^n$ s'écrit :

\begin{eqnarray}
	\langle \operator{\mathcal{O}}_i^n \rangle &= & \sum_{\{ \theta_a\}} \langle\operator{\mathcal{O}}_i\rangle_{\{\theta_a\}}^n \mathbb{P} ( \{ \theta_a \} ) ~=~ (-1)^n \left. \frac{1}{Z} \frac{\partial^n Z}{{(\partial \beta_i)}^n} \right )_{\beta_{j \neq i }} .	
\end{eqnarray}

%%%%%%%%%%%%%%%%%%%%%%%%%%%%%%%
\paragraph{Moments d’ordre supérieur et fluctuations.}
Le premier et second moments permettent d’accéder à la variance de l’observable :
\begin{eqnarray}
	\Delta_{\operator{\mathcal{O}}_i}^2 &=&  	\left \langle \left (\operator{\mathcal{O}}_i - \langle\operator{\mathcal{O}}_i \rangle \right )^2  \right \rangle  = 	\langle\operator{\mathcal{O}}_i^2 \rangle  -  \langle\operator{\mathcal{O}}_i \rangle^2 = \left . \frac{1}{Z} \frac{ \partial^2 Z }{ {\partial \beta_i}^2 }  \right )_{\beta_{j\neq i}} - \left ( \left . \frac{1}{Z}\frac{ \partial Z }{ \partial \beta_i }  \right )_{\beta_{j\neq i}}\right )^2  \\
		& = & \frac{\partial}{\partial \beta_i } \left ( \left . \frac{1}{Z} \frac{\partial Z}{\partial \beta_i }  \right )_{\beta_{j\neq i}}  \right )_{\beta_{j\neq i}} =  \left . \frac{\partial^2 \ln Z  }{{\partial \beta_i}^2 }  \right )_{\beta_{j\neq i}}  = - \left . 	\frac{\partial \langle\operator{\mathcal{O}}_i \rangle }{\partial \beta_i } \right )_{\beta_{j\neq i}}	.	
\end{eqnarray}

%%%%%%%%%%%%%%%%%%%%%%%%%%%%%%
\paragraph{Cas particulier de l’équilibre thermique.}
-- Si $\operator{\mathcal{O}}_i = \operator{\mathcal{N}}$ alors $\beta_i = - \beta \mu $ et si $\operator{\mathcal{O}}_i = \operator{\mathcal{H}}_N - \mu \operator{\mathcal{N}} $ alors $\beta_i = \beta = T^{-1}$. Avec $\mu$ le potentielle chimique et $T$ la temperatures pour un équilibre thermique--.\\
Dans le cas d’un équilibre thermique, on identifie certains multiplicateurs à des grandeurs thermodynamiques classiques comme l’inverse de la température  et le potentiel chimique :
	
	%avec $\sum_{a = 1}^N 1 \equiv \langle \operator{\mathcal{N}} \rangle_{ \{\theta_a \} }  \doteq  \langle \{ \theta_a \}\vert  \operator{\mathcal{N}} \vert\{ \theta_a\}\rangle  $ et $  \sum_{a = 1}^N  \varepsilon ( \theta_a ) \equiv \langle \operator{\mathcal{E}} \rangle_{\{\theta_a \}}  \doteq  \langle \{ \theta_a \}\vert  \operator{ \mathcal{E}}  \vert\{ \theta_a\}\rangle $.

	
	%La probabilité que le système soit dans configuration $\{ \theta_a \}$  est 
	%\begin{eqnarray}
	%	P_{\{\theta_a \}} & = & \frac{e^{- \beta \left ( \langle \operator{\mathcal{E}} \rangle_{\{\theta_a \}}   - \mu \langle \operator{\mathcal{N}} \rangle_{\{\theta_a \}} \right )}}{Z_{thermal}} = \frac{e^{- \beta \sum_{a=1}^N  ( \varepsilon( \theta_a )   - \mu  )}}{Z_{thermal}}	
	%\end{eqnarray}
	
	%avec la fonction de partition $Z_{thermal} = \sum_{\{ \theta_a\}}e^{- \beta \left ( \langle \operator{\mathcal{E}} \rangle_{\{\theta_a \}}   - \mu \langle \operator{\mathcal{N}} \rangle_{\{\theta_a \}} \right )} = \sum_{\{ \theta_a\}} e^{- \beta \sum_{a=1}^N  ( \varepsilon( \theta_a )   - \mu  )}$
	

	
\begin{eqnarray}
	\langle \operator{\mathcal{N}} \rangle  = \left .\frac{1}{\beta} \frac{ \partial \ln Z}{\partial \mu } \right )_{T,\cdots},  & & \Delta^2_{\operator{\mathcal{N}}} = \left . \frac{1}{\beta^2} \frac{ \partial^2 \ln Z}{{\partial \mu}^2 } \right )_{T,\cdots} =  \left . \frac{1}{\beta} \frac{ \partial \langle \operator{\mathcal{N}} \rangle}{\partial \mu } \right )_{T,\cdots}\\
	\langle \operator{\mathcal{H}}_N - \mu\operator{\mathcal{N}}  \rangle  = -\left . \frac{ \partial \ln Z}{\partial \beta } \right )_{\mu , \cdots} ,  & & \Delta^2_{\operator{\mathcal{H}}_N - \mu\operator{\mathcal{N}}} = \left .  \frac{ \partial^2 \ln Z}{{\partial \beta}^2 } \right )_{\mu , \cdots} =  -\left .  \frac{ \partial \langle \operator{\mathcal{H}}_N - \mu\operator{\mathcal{N}} \rangle}{\partial \beta } \right )_{\mu , \cdots}		
\end{eqnarray}

Soit pour l'énergie, 

\begin{eqnarray}
	\langle \operator{\mathcal{H}}_N \rangle  = \left [ \left .\frac{\mu}{\beta} \frac{ \partial}{\partial \mu } \right )_{T,\cdots} -\left . \frac{ \partial }{\partial \beta } \right )_\mu  \right ]\ln Z,  & & \Delta^2_{\operator{\mathcal{H}}_N } = \left [ \left .\frac{\mu}{\beta} \frac{ \partial}{\partial \mu } \right )_{T,\cdots} -\left . \frac{ \partial }{\partial \beta } \right )_{\mu,\cdots}  \right ]^2\ln Z		
\end{eqnarray}


	
	
	%La matrice densité thermique est :
	%\begin{eqnarray}
	%	\operator{\rho}_{thermal} & = & \frac{e^{- \beta \operator{H}}}{Z_{thermal}}, \\
	%	e^{-\beta \operator{H}} & = & 	\sum_{\{\theta_a \}} e^{- \beta \sum_{a=1}^N ( \varepsilon(\theta_a)- \mu ) } \vert \{ \theta_a\} \rangle \langle  \{ \theta_a\}  \vert 
	%\end{eqnarray}