\textbf{Plan un peut trop matheux ... :Relaxation et Équilibre dans les Systèmes Quantiques Intégrables : Approche par la Thermodynamique de Bethe}

\paragraph{Introduction Générale}
\begin{itemize}
    \item Contexte physique : systèmes quantiques isolés, dynamique hors équilibre
    \item Problématique : absence de thermalisation conventionnelle dans les systèmes intégrables
    \item Présentation des outils : TBA, GGE, charges conservées, fonction de Yang-Yang
    \item Objectifs et contributions de la thèse
    \item Structure du manuscrit
\end{itemize}

\paragraph{Partie I – Fondements et cadre théorique}

\subparagraph{Chapitre 1 – Systèmes quantiques intégrables}
\begin{itemize}
    \item Définitions : intégrabilité classique vs quantique
    \item Propriétés caractéristiques : matrices de diffusion factorisées, absence de chaos
    \item Exemples : modèle de Lieb-Liniger, XXZ, Hubbard 1D
    \item Forme générale des hamiltoniens intégrables et structure des états propres
\end{itemize}

\subparagraph{Chapitre 2 – Dynamique hors équilibre et notions d’équilibre}
\begin{itemize}
    \item Quench quantique : définitions, protocoles typiques
    \item Problème de la thermalisation : ETH vs intégrabilité
    \item Notion d’état d’équilibre généralisé (GGE)
    \item Rôle des charges conservées extensives et quasi-locales
\end{itemize}

\paragraph{Partie II – Thermodynamique de Bethe et relaxation}

\subparagraph{Chapitre 3 – Thermodynamique de Bethe (TBA)}
\begin{itemize}
    \item Rappels sur les équations de Bethe
    \item Limite thermodynamique et distribution de rapidité
    \item Équations intégrales de la TBA
    \item Fonction de Yang-Yang : entropie, énergie libre, pression
\end{itemize}

\subparagraph{Chapitre 4 – Relaxation vers la GGE et interprétation TBA}
\begin{itemize}
    \item Construction de la GGE à partir de la densité de rapidité
    \item Lien entre rapidités et observables physiques
    \item Cas du modèle de Lieb-Liniger : TBA et GGE explicites
    \item Fonctionnelle de Yang-Yang comme générateur d’équilibre
    \item Cas stationnaire après quench : structure des équations
\end{itemize}

\paragraph{Partie III – Études spécifiques et développements mathématiques}

\subparagraph{Chapitre 5 – Analyse mathématique de la fonction de Yang-Yang}
\begin{itemize}
    \item Propriétés variationnelles : convexité, unicité du minimum
    \item Lien avec le principe de maximum d’entropie
    \item Existence et unicité de la solution des équations TBA
    \item Aspects fonctionnels : cadre rigoureux (espaces de fonctions, opérateurs)
\end{itemize}

\subparagraph{Chapitre 6 – Études de cas et applications}
\begin{itemize}
    \item Quench dans le modèle de Lieb-Liniger : relaxation effective
    \item Comparaison numérique et analytique : données issues de l’expérience (le cas échéant)
    \item Généralisation à d’autres modèles intégrables
    \item Rôle des quasi-particules et interprétation hydrodynamique (GHD si souhaité)
\end{itemize}

\paragraph{Conclusion Générale}
\begin{itemize}
    \item Bilan des résultats
    \item Discussion sur les limites et généralisations possibles
    \item Perspectives : GHD, corrections à la GGE, chaos partiel et non-intégrabilité
\end{itemize}

\paragraph{Annexes}
\begin{itemize}
    \item Détails techniques sur les équations de Bethe
    \item Résultats analytiques ou numériques complémentaires
    \item Outils mathématiques : théorèmes d’existence, opérateurs intégraux
\end{itemize}

\paragraph{Bibliographie}
Références principales : Lieb, Korepin, Caux, Ilievski, Doyon, Essler, etc.