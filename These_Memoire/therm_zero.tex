Dans la limite thermodynamique, le nombre de particules \( N \) et le volume 
(la longueur de la boîte) \( L \) tendent vers l'infini de sorte que leur rapport 
\( D = \frac{N}{L} \) reste fini :

\begin{eqnarray*}
	\lim_{N, L \to \infty} \frac{N}{L} = D = \mbox{const} < \infty	
\end{eqnarray*}

Considérons le système à température nulle. Rappelons que l'état 
d'énergie minimale dans le secteur avec un nombre fixe de particules 
correspond aux solutions \( j \) des équations de Bethe suivantes :

\begin{eqnarray*}
	L \theta_a + \sum_{b = 1}^N \Phi ( \theta_a - \theta_b ) & = & 2\pi I_a ,	
\end{eqnarray*}

avec 

où les nombres fermionique $I_a = a - (N+1)/2$ et $a \in \llbracket 1 , N  \rrbracket$ . Dans la limite thermodynamique, les valeurs de \( \theta_a \) se condensent (\(\theta_{a+1} - \theta_a = \mathcal{O}(1/L)\)), et remplissent l'intervalle symétrique , la mer de Dirac/ sphère de Fermi  \(\llbracket-q, q\rrbracket\) où $q = \theta_N$ (ici $n_a \geqq n_b \Rightarrow \theta_a \geqq \theta_b$)
 


