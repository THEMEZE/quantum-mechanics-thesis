Dans la limite thermodynamique, le nombre de particules \( N \) et le volume 
(la longueur de la boîte) \( L \) tendent vers l'infini de sorte que leur rapport 
\( D = \frac{N}{L} \) reste fini :

\begin{eqnarray*}
	\lim_{N, L \to \infty} \frac{N}{L} = D = \mbox{const} < \infty	
\end{eqnarray*}

Considérons le système à température nulle. Rappelons que l'état 
d'énergie minimale dans le secteur avec un nombre fixe de particules 
correspond aux solutions \( j \) des équations de Bethe suivantes :

\begin{eqnarray*}
	L \theta_a + \sum_{b = 1}^N \Phi ( \theta_a - \theta_b ) & = & 2\pi I_a ,	
\end{eqnarray*}

où les nombres fermionique $I_a = a - (N+1)/2$ et $a \in \llbracket 1 , N  \rrbracket$ . Dans la limite thermodynamique, les valeurs de \( \theta_a \) se condensent (\(\theta_{a+1} - \theta_a = \mathcal{O}(1/L)\)), et remplissent l'intervalle symétrique , la mer de Dirac/sphère de Fermi  \(\llbracket-K, K\rrbracket\) où $K = \theta_N$ (ici $I_a \geqq I_b \Rightarrow \theta_a \geqq \theta_b$)

La quantité $\rho_s$ tel que (manque une intro avant ) 

\begin{eqnarray*}
	2\pi \rho_s (\theta_a) & = & \frac{2\pi}{L}\underset{\tiny \mbox{therm}}{\lim} \frac{ \vert I_{a+1} - I_{a} \vert}{ \vert \theta_{a+1} - \theta_a \vert} = \frac{2\pi}{L} \frac{ \partial I}{\partial \theta } ( \theta_a ) = 1 	+ \frac{1}{L}\sum_{b = 1}^N \Delta (\theta_a - \theta_b)
\end{eqnarray*}

représente la densité de vacances que l'on appellera densité d'état avec $I(\theta_a) = I_a$ et  $\Delta(\theta) = \Phi'(\theta)  = \frac{2c}{c^2 + \theta^2}$

Maintenant interessons nons à la densité de particules dans l'espace des moments \( \rho(\theta) \) , définie de la manière suivante :

\begin{eqnarray*}
	\rho(\theta_a)  &=  &\lim_{L \to \infty} \frac{1}{L} \frac{1}{\theta_{a+1} - \theta_a} > 0.	
\end{eqnarray*}

À l'états fondamentale tous les vacances dans $[-K , K ]$  sont occupés donc $\rho = \rho_s$. La quantité $L\rho(\theta)d\theta$ est le nombre de rapidité dans la celule $[ \theta , \theta + d \theta ] $. La quantité $L \int_{-K}^K \rho (\theta ) \, d\theta $ est le nombre de particule $N$.On remplace la somme par une intégrale :
\begin{eqnarray*}
	2\pi \rho  = 1 + \Delta \star \rho 	
\end{eqnarray*}



 


