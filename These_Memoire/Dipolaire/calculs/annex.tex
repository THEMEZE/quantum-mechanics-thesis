%\subsubsection{Polarisation linéaire  et $d_0 \equiv \langle nP , m_L = 0 \vert \operatorvec{D} \cdot \operatorvec{u} \vert nS , m_L = 0 \rangle $}
\subsubsection{Polarisation linéaire}

On se limide à une polarisation laser linéaire. Par consequence dans la base $\vert  n , L , m_L \rangle$, il y a uniquement la transition $\vert n S , m_L = 0 \rangle \equiv \vert n , L = 0 , m_L = 0 \rangle \to \vert n P , m_L = 0 \rangle \equiv \vert n , L = 1 , m_L = 0 \rangle $.
Pour la suite on note %$d_0 \equiv \langle nP , m_L = 0 \vert \operatorvec{D} \cdot \operatorvec{u} \vert nS , m_L = 0 \rangle$ 
\begin{eqnarray*}
	d_0 & \equiv & \langle nP , m_L = 0 \vert \operatorvec{D} \cdot \operatorvec{u} \vert nS , m_L = 0 \rangle.	
\end{eqnarray*}


%\subsubsection{Dans la base fine et $ \langle n{~}^2P_{1/2} , m_J = 1/2 \vert \operatorvec{D} \cdot \operatorvec{u} \vert n{~}^2S_{1/2} , m_J = 1/2 \rangle = d_0 / \sqrt{3}$}
\subsubsection{Dans la base fine}

Le moment angulaire électronique total , $J = S + L$ , est la somme du nombre quantique de moment angulaire totale $L$ et du spin totale des électrons. Ici on ne considère juste la couche de valence et on se limete au alcalin donc $S = 1/2 $.\\

Décomposons $\vert n{~}^2P_{3/2} , m_J = 1/2 \rangle \equiv \vert n , J = 3/2 , m_J = 1/2 \rangle $ dans la base  $  \vert n , m_S  , m_L \rangle  \equiv \vert  n{~}^2P , m_S  ,  m_L \rangle \equiv \vert n  , S=1/2 , m_S , L=1 , m_L \rangle $.\\
On sait que 
\begin{eqnarray}
	\vert n , J = {3}/{2} , m_J = {3}/{2} \rangle & = &  \vert n , m_S = 1/2   , m_L = 1 \rangle
\end{eqnarray}
or
\begin{eqnarray}
	\operator{A}_\pm \vert n , A , m	_A \rangle & = & \hbar \sqrt{A(A+1) - m_A(m_A\pm 1 ) } \vert n , A , m	_A \pm 1  \rangle
\end{eqnarray}

avec l'opérateur $\operator{A} \in \{ \operator{S} , \operator{L} , \operator{J} , \cdots \}$. Donc  

\begin{eqnarray}
	\operator{J}_- \vert n , J = 3/2  , m_J = 3/2 \rangle & = & \hbar \sqrt{\frac{3}{2}\left(\frac{5}{2} \right ) - \frac{3}{2}\left(\frac{1}{2} \right )  } \vert n , J = 3/2  , m_J = 1/2  \rangle, \notag \\
	& = & \hbar \sqrt{3} \vert n , J = 3/2  , m_J = 1/2  \rangle.
\end{eqnarray}

or 

\begin{eqnarray}
	\operator{J}_- & = & 	\operator{S}_- + \operator{L}_-	
\end{eqnarray}

\begin{eqnarray}
	\operator{J}_- \vert n , J = 3/2  , m_J = 3/2 \rangle & = & (\operator{S}_- + \operator{L}_-) \vert n , m_S = 1/2   , m_L = 1 \rangle,\\
	& = & \hbar \sqrt{\frac{1}{2}\left(\frac{3}{2} \right ) - \frac{1}{2}\left(-\frac{1}{2} \right ) }	\vert n , m_S = -1/2   , m_L = 1 \rangle \notag \\
	& + & \hbar \sqrt{1\left(2 \right ) - 1\left(0\right ) }	\vert n , m_S = 1/2   , m_L = 0 \rangle \notag \\
	& = & \hbar \vert n , m_S = -1/2   , m_L = 1 \rangle \\ & + & \hbar \sqrt{2}	\vert n , m_S = 1/2   , m_L = 0 \rangle  
\end{eqnarray}

Donc d'après les équations ???

\begin{eqnarray}
	\vert nP_{3/2}  , m_J = 1/2  \rangle & = & \frac{1}{\sqrt{3} } \vert n , m_S=-1/2  , m_L=1 \rangle + \sqrt{\frac{2}{3}} \vert n , m_S=1/2  , m_L=0 \rangle			
\end{eqnarray}

avec $\vert nP_J  , m_J   \rangle \equiv \vert n , J  , m_J  \rangle$ et $\vert n , m_S  , m_L \rangle	 \equiv \vert n   , S=1/2 , m_S , L=1 , m_L \rangle = \vert n {~}^2P  , m_S , m_L \rangle $.\\

Pour aussi 
\begin{eqnarray}
	\vert nP_{1/2}  , m_J = 1/2   \rangle  & \in & 	\bm{Vect} \{ \vert n , m_S = -1/2  , m_L = 1  \rangle , \vert n , m_S = 1/2  , m_L = 0  \rangle  \} ,\\
	\langle  nP_{3/2}  , m_J = 1/2 \vert nP_{1/2}  , m_J = 1/2   \rangle & = & 0 		
\end{eqnarray}
Donc 

\begin{eqnarray}
	\vert nP_{1/2}  , m_J = 1/2  \rangle & = & \mp \sqrt{\frac{2}{3}} \vert n , m_S=-1/2  , m_L=1 \rangle \pm \frac{1}{\sqrt{3} }   \vert n , m_S=1/2  , m_L=0 \rangle		
\end{eqnarray}

et 
\begin{eqnarray}
	\vert nP_{1/2}  , m_J = -1/2   \rangle  & \in & 	\mathrm{Vect} \{ \vert n , m_S = 1/2  , m_L = -1  \rangle , \vert n , m_S = -1/2  , m_L = -0  \rangle  \} ,\\
	\vert nP_{3/2}  , m_J = -3/2   \rangle & = & \vert n , m_S = -1/2  , m_L = -1  \rangle		
\end{eqnarray}


et 

\begin{eqnarray}
	\hbar \sqrt{3} \vert n , J = 3/2  , m_J = -1/2 \rangle & \leftrightharpoons & \notag \\
	\hbar {\textstyle \sqrt{\frac{3}{2} \left ( \frac{5}2 \right ) -  \left ( -\frac{3}{2} \right ) \left ( -\frac{1}{2} \right )}} \vert n , J = 3/2  , m_J = -1/2 \rangle & \leftrightharpoons & \notag \\
	\operator{J}_+ \vert n , J = 3/2  , m_J = -3/2 \rangle & \leftrightharpoons & \notag \\
	\operator{J}_+ \vert n {~}^2P_{3/2}  , m_J = -3/2 \rangle & = & (\operator{S}_+ + \operator{L}_+)	\vert n {~}^2 P , m_S = -1/2  , m_L = -1  \rangle, \notag\\
	& \rightleftharpoons &	(\operator{S}_+ + \operator{L}_+)	\vert n , m_S = -1/2  , m_L = -1  \rangle, \notag\\
	& \rightleftharpoons &	\hbar 	{\textstyle \sqrt{\frac{1}{2} \left ( \frac{3}2 \right ) -  \left ( -\frac{1}{2} \right ) \left ( \frac{1}{2} \right )}}\vert n , m_S = 1/2  , m_L = -1  \rangle, \notag\\
	& + & \hbar 	{\textstyle \sqrt{1 \left ( 2 \right ) -  \left ( -1 \right ) \left ( 0 \right )}}\vert n , m_S = -1/2  , m_L = 0  \rangle, \notag\\
	& \rightleftharpoons & \hbar \vert n , m_S = 1/2  , m_L = -1  \rangle \notag \\
	& + & \hbar \sqrt{2}\vert n , m_S = -1/2  , m_L = 0  \rangle \notag 
\end{eqnarray}

soit 

\begin{eqnarray}
	\vert n {~}^2P_{3/2}  , m_J = -1/2 \rangle	 & = & 	 \sqrt{\frac{1}{3} }\vert n {~}^2P  , m_S = 1/2  , m_L = -1 \rangle + \sqrt{\frac{2}{3} }\vert n {~}^2P  , m_S =-1/2  , m_L = 0 \rangle \notag
\end{eqnarray}

Or 

\begin{eqnarray}
	\vert n {~}^2P_{1/2}  , m_J = -1/2 \rangle & \in & \mathrm{Vect}	 \{ \vert n {~}^2P  , m_S = 1/2  , m_L = -1 \rangle , \vert n {~}^2P  , m_S = -1/2  , m_L = 0 \rangle \} \notag,\\
	\langle n {~}^2 P_{3/2} , m_J = -1/2 \vert n {~}^2P_{1/2}  , m_J = -1/2 \rangle & = & 0 \notag 			
\end{eqnarray}

Soit 

\begin{eqnarray}
	\vert n {~}^2P_{1/2}  , m_J = -1/2 \rangle & = & \mp \sqrt{ \frac{2}{3}} \vert n {~}^2P  , m_S = 1/2  , m_L = -1 \rangle \pm  \sqrt{ \frac{1}{3}} \vert n {~}^2P  , m_S = -1/2  , m_L = 0 \rangle  \notag,		
\end{eqnarray}

\begin{eqnarray}
	\vert n {~}^2S_{1/2} , m_J = \pm 1/2 \rangle & = & 	\vert n {~}^2S , m_S = \pm 1/2 , m_L = 0 \rangle  \equiv  \vert n {~}^2S , m_S = \pm 1/2  \rangle , \notag \\
	\vert n {~}^2P_{1/2}  , m_J = \pm 1/2 \rangle & = & \epsilon \sqrt{ \frac{2}{3}} \vert n {~}^2P  , m_S = \mp 1/2  , m_L = \pm 1 \rangle -\epsilon  \sqrt{ \frac{1}{3}} \vert n {~}^2P  , m_S = \pm 1/2  , m_L = 0 \rangle  \notag,	\\
	\vert n {~}^2P_{3/2}  , m_J = \pm 1/2 \rangle & = & + \sqrt{ \frac{1}{3}} \vert n {~}^2P  , m_S = \mp 1/2  , m_L = \pm 1 \rangle +  \sqrt{ \frac{2}{3}} \vert n {~}^2P  , m_S = \pm 1/2  , m_L = 0 \rangle  \notag,	\\
	\vert n {~}^2P_{3/2} , m_J = \pm 3/2 \rangle & = & 	\vert n {~}^2P , m_S = \pm 1/2 , m_L = \pm 1  \rangle , \notag 
\end{eqnarray}

avec $\epsilon \in \{ +1 , -1 \} $.\\

\begin{aligned} \ket{n\,^2P_{1/2},+\tfrac12} &= -\sqrt{\tfrac{1}{3}}\,\ket{m_L=0,\,m_S=+\tfrac12} \;+\;\sqrt{\tfrac{2}{3}}\,\ket{m_L=+1,\,m_S=-\tfrac12},\\ \ket{n\,^2P_{1/2},-\tfrac12} &= +\sqrt{\tfrac{1}{3}}\,\ket{m_L=0,\,m_S=-\tfrac12} \;-\;\sqrt{\tfrac{2}{3}}\,\ket{m_L=-1,\,m_S=+\tfrac12}. \end{aligned}

\begin{aligned} \ket{n\,^2P_{3/2},+\tfrac12} &= +\sqrt{\tfrac{2}{3}}\,\ket{m_L=0,\,m_S=+\tfrac12} \;+\;\sqrt{\tfrac{1}{3}}\,\ket{m_L=+1,\,m_S=-\tfrac12},\\ \ket{n\,^2P_{3/2},-\tfrac12} &= +\sqrt{\tfrac{2}{3}}\,\ket{m_L=0,\,m_S=-\tfrac12} \;+\;\sqrt{\tfrac{1}{3}}\,\ket{m_L=-1,\,m_S=+\tfrac12}. \end{aligned}


Si la polarisation de la lumière est linéaire, cela signifie que les oscillations des champs électriques et magnétiques se produisent dans un seul plan. Dans ce cas, l'opérateur dipôle électronique $\operatorvec{D} \cdot \operatorvec{u}$ n'affectera pas directement les nombres quantiques magnétiques $m_L$ ou $m_J$, car la polarisation linéaire ne change pas l'orientation du moment angulaire ou du spin $m_S$ de la particule. 



\begin{eqnarray*}
	\langle n {~}^2P_{1/2} , m_J = \pm  1/2 \vert \operatorvec{D} \cdot \operatorvec{u}\vert n {~}^2S_{1/2} , m_J = \pm  1/2 \rangle	& & =  \\  \epsilon \sqrt{\frac{2}{3}} \underbrace{\langle n {~}^2P , m_S = \mp  1/2  , m_L = \pm 1 \vert \operatorvec{D} \cdot \operatorvec{u}\vert n {~}^2S , m_S = \pm  1/2 , m_L = 0  \rangle}_{0}  & + & \\
	- \epsilon  \sqrt{\frac{1}{3}}\underbrace{\langle n {~}^2P , m_S = \pm  1/2  , m_L = 0 \vert \operatorvec{D} \cdot \operatorvec{u}\vert n {~}^2S , m_S = \pm  1/2 , m_L = 0  \rangle	}_{d_0},\\
	\langle n {~}^2P_{1/2} , m_J = \pm  1/2 \vert \operatorvec{D} \cdot \operatorvec{u}\vert n {~}^2S_{1/2} , m_J = \pm  1/2 \rangle	 &=& - \epsilon \sqrt{\frac{1}{3}} d_0
\end{eqnarray*}

\begin{eqnarray*}
	\langle n {~}^2P_{3/2} , m_J = \pm  1/2 \vert \operatorvec{D} \cdot \operatorvec{u}\vert n {~}^2S_{1/2} , m_J = \pm  1/2 \rangle	& & =  \\  \sqrt{\frac{1}{3}} \underbrace{\langle n {~}^2P , m_S = \pm  1/2  , m_L = \mp 1 \vert \operatorvec{D} \cdot \operatorvec{u}\vert n {~}^2S , m_S = \pm  1/2 , m_L = 0  \rangle}_{0}  & + & \\
	 \sqrt{\frac{2}{3}}\underbrace{\langle n {~}^2P , m_S = \pm  1/2  , m_L = 0 \vert \operatorvec{D} \cdot \operatorvec{u}\vert n {~}^2S , m_S = \pm  1/2 , m_L = 0  \rangle	}_{d_0},\\
	\langle n {~}^2P_{3/2} , m_J = \pm  1/2 \vert \operatorvec{D} \cdot \operatorvec{u}\vert n {~}^2S_{1/2} , m_J = \pm  1/2 \rangle	 &=& \sqrt{\frac{2}{3}} d_0
\end{eqnarray*}

\subsubsection{Potentiel dipolaire}

En négligeant le temps non-resonant et avec l'hipothèse $\vert \omega_b - \omega_a -\omega  \vert \gg \gamma_{ab} $ ( $V_{AR} = \delta E$) l'équation ???? devient :

\begin{eqnarray*}
	V_{AR} & = & - \frac{ \vert \mathcal{E} \vert^2}{4 \hbar} \left( \frac{ \vert \langle n {~}^2P_{1/2}  \vert \operatorvec{D} \cdot \operatorvec{u}\vert n {~}^2S_{1/2}  \rangle \vert^2}{ \Delta_1} + \frac{ \vert\langle n {~}^2P_{3/2}  \vert \operatorvec{D} \cdot \operatorvec{u}\vert n {~}^2S_{1/2}  \rangle \vert^2}{ \Delta_2}  \right ) , \\
	&  = &  - \frac{  d_0^2 \vert \mathcal{E} \vert^2 }{4 \hbar} \left( \frac{ 1}{ 3 \Delta_1} + \frac{2 }{ 3 \Delta_2}  \right ) ,		
\end{eqnarray*}
avec $\Delta_i = \omega_i - \omega $ avec $ i \in \{ 1 , 2 \}$ avec $\omega_1 = \omega_{n {~}^2P_{1/2}, m_J = \pm  1/2}- \omega_{n {~}^2S_{1/2}, m_J = \pm  1/2}$ et $\omega_2 = \omega_{n {~}^2P_{3/2}, m_J = \pm  1/2} - \omega_{n {~}^2S_{1/2}, m_J = \pm  1/2}$

\subsubsection{Relier Intensité laser et $\mathcal{E}$}

La densité d'énergie électromagnétique:

\begin{eqnarray}
	u_{em} & = &  \frac{1}2 \left ( \varepsilon_0  \Re ( \vec{E} ) ^2 + \frac{1}{\mu_0} \Re(\vec{B})^2 \right ) \\
	\langle u_{em} \rangle & =& \frac{\varepsilon_0 \langle \Re(\vec{E})^2 \rangle}{2} + \frac{\langle \Re(\vec{B})^2 \rangle }{2 \mu_0} 
\end{eqnarray}



Soit $ T$ une application bilinaire ...

\begin{eqnarray}
	\left \langle T \left (  f(t),  g(t) \right ) \right \rangle & = & \left \langle T \left ( \Re \left  ( \underline{f} e^{i \omega t } \right ),  \Re \left ( \underline{g} e^{i \omega t } \right ) \right ) \right \rangle , \\
		& = &  \left \langle T \left ( \frac{1}{2} ( \underline{f} e^{i \omega t } + \underline{f}^\ast e^{-i \omega t } ) ,  \frac{1}{2} ( \underline{g} e^{i \omega t } + \underline{g}^\ast  e^{-i \omega t }  ) \right ) \right \rangle,\\
	&  = & \frac{1}{4} \left (\underline{f} \cdot \underline{g} \left \langle T \left ( e^{i \omega t } , e^{i \omega t }\right ) \right \rangle  + \underline{f}^\ast  \cdot \underline{g}^\ast  \left \langle T \left ( e^{-i \omega t } , e^{-i \omega t }\right ) \right \rangle + ( \underline{f}^\ast  \cdot \underline{g} + \underline{f}  \cdot \underline{g}^\ast)\left \langle T \left ( e^{i \omega t } , e^{-i \omega t }\right ) \right \rangle   \right ),\\
	& = & \frac{1}{2} \Re (\underline{f}  \cdot \underline{g}^\ast  )   
\end{eqnarray}


L'équipartition de l'énergie nous donne 
\begin{eqnarray}
	\varepsilon_0 	\vert \vec{E}\vert^2 & = & \frac{1}{\mu_0} \vert\vec{B}\vert^2 
\end{eqnarray}

\begin{eqnarray}
	\langle u_{em} \rangle & = & \frac{\varepsilon_0 \vert \mathcal{E} \vert ^2 }{2} 
\end{eqnarray}

De même pour le vecteur de poyting :

\begin{eqnarray}
	\vec{\Pi} & = &  \frac{ \vec{E} \times \vec{B}} { \mu_0} = \frac{1}{\mu_0 } \vec{E} \times ( \vec{n}/c \times \vec{E} ) \\
	\langle \vec{\Pi} \rangle & = & \left \langle \frac{ \vec{E} \times \vec{B}} { \mu_0} \right \rangle =  c \langle u_{em} \rangle \vec{n} 	
\end{eqnarray}

Donc 

\begin{eqnarray*}
	V_{AR} &  = &  - d_0^2 \frac{   I }{2 \varepsilon_0 c  \hbar} \left( \frac{ 1}{ 3 \Delta_1} + \frac{2 }{ 3 \Delta_2}  \right ) ,		
\end{eqnarray*}

avec $I = \frac{c \varepsilon_0 \vert \mathcal{E} \vert^2}{2}$